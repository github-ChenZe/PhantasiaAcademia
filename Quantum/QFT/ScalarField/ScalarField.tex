\documentclass[hidelinks]{article}

\usepackage[sensei=C.Z.,gakka=QFT,section=Quantum,gakkabbr=QM]{styles/kurisuen}
\usepackage{sidenotes}
\usepackage{van-de-la-sehen-en}
\usepackage{van-de-environnement-en}
\usepackage{boite/van-de-boite-en}
\usepackage{van-de-abbreviation}
\usepackage{van-de-neko}
\usepackage{van-le-trompe-loeil}
\usepackage{cyanide/van-de-cyanide}
\setlength{\parindent}{0pt}
\usepackage{enumitem}
\newlist{citemize}{itemize}{3}
\setlist[citemize,1]{noitemsep,topsep=0pt,label={-},leftmargin=1em}

\usepackage{mathtools}
\usepackage{ragged2e}

\DeclarePairedDelimiter\abs{\lvert}{\rvert}%
\DeclarePairedDelimiter\norm{\lVert}{\rVert}%

% Swap the definition of \abs* and \norm*, so that \abs
% and \norm resizes the size of the brackets, and the 
% starred version does not.
\makeatletter
\let\oldabs\abs
\def\abs{\@ifstar{\oldabs}{\oldabs*}}
%
\let\oldnorm\norm
\def\norm{\@ifstar{\oldnorm}{\oldnorm*}}
\makeatother

\newcommand*{\Value}{\frac{1}{2}x^2}%

\usepackage{fancyhdr}
\usepackage{lastpage}

\fancypagestyle{plain}{%
\fancyhf{} % clear all header and footer fields
\fancyhead[R]{\smash{\raisebox{2.75em}{{\hspace{1cm}\color{lightgray}\textsf{\rightmark\quad Page \thepage/\pageref{LastPage}}}}}} %RO=right odd, RE=right even
\renewcommand{\headrulewidth}{0pt}
\renewcommand{\footrulewidth}{0pt}}
\pagestyle{plain}

\newtheorem*{experiment*}{Measurement}
\newtheorem{example}{Example}
\newtheorem{remark}{Remark}

\def\elementcell#1#2#3#4#5#6#7{%
    \draw node[draw, regular polygon, regular polygon sides=4, minimum height=2cm, draw=cyan, line width=0.4mm, fill=cyan!15!white, #1, inner sep=-2mm](#3) {\Large\textbf{\textsf{\color{cyan!50!black}#4}}};
    \draw (#3.corner 1) node[below left] {\footnotesize{\phantom{Hj}#5}};
    \draw (#3.corner 2) node[below right] {\small{\textsf{#6}}};
    \draw (#3.side 3) node[above] {\footnotesize #7};
    \draw (#3.corner 2) ++ (0,-0.4mm) node(nw#3) {};
    \tcbsetmacrotowidthofnode{\elementcellwidth}{#3}
    \node [fill=cyan, line width=0mm, rectangle, rounded corners=1.8mm, rectangle round south east=false, rectangle round south west=false, anchor=south west, minimum width=\elementcellwidth] at (nw#3) {\small\textsf{\color{white}#2}};
}

\DeclareSIUnit\Dq{Dq}
\usepackage{physics}
\usepackage{bbm}
\newtheorem{lemma}{Lemma}
\newtheorem{proposition}{Proposition}

\DeclareMathOperator{\Pfaffian}{Pf}
\DeclareMathOperator{\sign}{sign}

\usetikzlibrary{decorations.markings}
\usetikzlibrary{arrows.meta}

\def\kurisueniconwidth{3.2cm}
\def\kurisueniconpath{img/CMSncFS.png}
\newcommand{\hermitianconj}{\mathrm{h.c.}}

\usepackage{tensor}
\let\grad\nabla

\usepackage{biblatex}
\bibliography{ScalarField.bib}

\begin{document}

\section{Quantization of the Scalar Field} % (fold)
\label{sec:quantization_of_scalar_field}

\subsection{The Klein Gordon Equation} % (fold)
\label{sub:the_klein_gordon_equation}

\subsubsection{Lorentz Invariance} % (fold)
\label{ssub:lorentz_invariance}

\begin{marginwarns}
    Inconsistent signature of $\pare{+,-,-,-}$ and $\pare{-,+,+,+}$ may occur.
\end{marginwarns}
Coordinates in different inertial frames are related by
\[ \conj{x}^\mu = \tensor{\Lambda}{^\mu_\nu} x^\nu + a^\mu. \]
The minkowski metric obeys
\[ g_{\mu\nu} \tensor{\Lambda}{^\mu_\rho}\tensor{\Lambda}{^\nu_\sigma} = g_{\rho\sigma} \]
in order to maintain the invariance of the proper distance. The partial derivatives on coordinates are
\[ \partial^\mu x^\nu = g^{\mu\nu}, \]
therefore the $\partial$-operator transforms like \begin{marginwarns}
    Inconsistent systems of units may occur.
\end{marginwarns}
\[ \conj{\partial}^\mu = \tensor{\Lambda}{^\mu_\nu} \partial^\nu. \]

% subsubsection lorentz_invariance (end)

\subsubsection{The Klein-Gordon Equation} % (fold)
\label{ssub:the_klein_gordon_equation}

\begin{finaleq}{The Klein-Gordon Equation}
    \begin{equation}
        \label{eq:the_klein-gordon-equation}
        \pare{-\partial^2 + \frac{m^2c^2}{\hbar^2}}\psi\pare{x} = 0.
    \end{equation}
\end{finaleq}
The Klein-Gordon equation is Lorentz invariant.

% subsubsection the_klein_gordon_equation (end)

% subsection the_klein_gordon_equation (end)

\subsection{The Klein Gordon Field} % (fold)
\label{sub:the_klein_gordon_field}

\subsubsection{Classical Field Theory} % (fold)
\label{ssub:classical_field_theory}

\paragraph{Lagrangian Field Theory} % (fold)
\label{par:lagrangian_field_theory}

In the Lagrangian formulation, the stationary point of the action
\[ S = \int \+cL\pare{\phi,\partial_\mu}\,\rd{^4x} \]
is evaluated, leading to
\begin{finaleq}{The Euler-Lagrange Equation of Fields}
    \[ \partial_\mu\pare{\+D{\pare{\partial_\mu \phi}}D{\+cL}} - \+D\phi D{\+cL} = 0. \]
\end{finaleq}

% paragraph lagrangian_field_theory (end)

\paragraph{Hamiltonian Field Theory} % (fold)
\label{par:hamiltonian_field_theory}

The \gloss{momentum density} conjugate to $\phi\pare{\+vx}$ is defined by
\[ \pi\pare{\+vx} = \+D{\dot{\phi}\pare{\+vx}}D{\+cL}, \]
and the \gloss{Hamiltonian density} is written
\[ \+cH = \pi\pare{\+vx} \dot{\phi}\pare{\+vx} - \+cL, \]
and the Hamiltonian
\[ H = \int \rd{^3x}\, \+cH. \]
\begin{sample}
    \begin{example}
        $\displaystyle \+cL = \half \dot{\phi}^2 - \half \pare{\grad\phi}^2 - \half m\phi^2$ yields the Klein-Gordon Equation \eqref{eq:the_klein-gordon-equation}. The Hamiltonian is given by
        \begin{equation}
            \label{eq:hamiltonian_of_the_klein_gordon_field}
            H = \int \rd{^3 x}\, \+cH = \int \rd{^3 x}\,\brac{\half \pi^2 + \half \pare{\grad \phi}^2 + \half m^2\phi^2}.
        \end{equation}
    \end{example}
\end{sample}

% paragraph hamiltonian_field_theory (end)

\paragraph{Noether's Theorem} % (fold)
\label{par:noether_s_theorem}

If $S$ is invariant under the infinitesimal transformation
\[ \phi\pare{x} \rightarrow \phi\pare{x} + \alpha\Delta\phi\pare{x}, \]
the Lagrangian is then invariant up to a 4-divergence, i.e.
\[ \+cL\pare{x} \rightarrow \+cL\pare{x} + \alpha\partial_\mu \+cJ^\mu\pare{x}. \]
Introducing
\[ j^\mu \pare{x} = \+D{\pare{\partial_\mu \phi}}D{\+cL}\Delta \phi - \+cJ^\mu, \]
we find
\begin{finaleq}{Conservation of Charge}
    \[ \partial_\mu j^\mu = 0, \]
    which may be written as
    \[ \dot{Q} = 0,\quad \text{where}\quad Q = \int \rd{^3x}\, j^0. \]
\end{finaleq}
\begin{marginwarns}
    Note that the $\phi^*$ is placed before $\pare{\partial^\mu \phi}$.
\end{marginwarns}
\begin{sample}
    \begin{example}
        Treating $\phi$ and $\phi^*$ as independent fields, the Lagrangian
        \[ \+cL = \abs{\partial_\mu \phi}^2 - m^2\abs{\phi}^2 \]
        is invariant under $\phi \rightarrow e^{i\alpha}\phi$, from which we find the Noether current
        \[ j^\mu = i\brac{\pare{\partial^\mu \phi^*}\phi - \phi^*\pare{\partial^\mu \phi}} \]
        conserved.
    \end{example}
\end{sample}
\begin{sample}
    \begin{example}
        Under the infinitesimal translation $x^\mu \rightarrow x^\mu - a^\mu$, we have
        \[ \phi\pare{x} \rightarrow \phi\pare{x} + a^\mu \partial_\mu \phi\pare{x},\quad \text{and}\quad \+cL = \+cL + a^\nu \partial_\mu \pare{\tensor{\delta}{^\mu_\nu}\+cL}, \]
        which yields a nonzero $\+cJ^\mu$ and conserved currents
        \[ \tensor{T}{^\mu_\nu} = \+D{\pare{\partial_\mu \phi}}D{\+cL}\partial_\nu \phi - \+cL \tensor{\delta}{^\mu_\nu}. \]
        The conserved charges are
        \begin{equation}
            \label{eq:conserved_momentum}
            P^\mu = \int \rd{^3x}\, T^{0\mu},
        \end{equation}
        where $T^{00} = \+cH$ and $P^0 = H$.
    \end{example}
\end{sample}

% paragraph noether_s_theorem (end)

% subsubsection classical_field_theory (end)

\subsubsection{The Klein-Gordon Field as Harmonic Oscillators} % (fold)
\label{ssub:the_klein_gordon_field_as_harmonic_oscillators}

\begin{finaleq}{Fundamental Commutation Relations}
    \[ \brac{\phi\pare{\+vx},\pi\pare{\+vy}} = i\delta^{\pare{3}}\pare{\+vx - \+vy},\quad \text{and}\quad \brac{\phi\pare{\+vx},\phi\pare{\+vy}} = \brac{\pi\pare{\+vx},\pi\pare{\+vy}} = 0, \]
\end{finaleq}
where $\phi$ and $\pi$ are promoted to operators now. \begin{margintips}
    To obtain the expression of $\pi\pare{\+vx}$, convert the $\+vp\cdot \+vx$ into its covariant form $-i\omega_{\+vp} + \+vp\cdot \+vx$ and then evaluate the $\partial_t$ as a shortcut since $\pi = \dot{\phi}$.
\end{margintips} Expanding the classical Klein-Gordon equation with its Fourier transformation, we found that $\phi$ in the momentum space evolves like the position operator of a classical harmonic oscillator of frequency $\omega_{\+vp} = \sqrt{\+vp^2 + m^2}$. We therefore write, in analog with the classical creation and annihilation operators, that
\begin{align}
    \label{eq:phi_at_time_init}
    \phi\pare{\+vx} &= \int \frac{\rd{^3 p}}{\pare{2\pi}^3} \rec{\sqrt{2\omega_{\+vp}}}\pare{a_{\+vp} + a^\dagger_{-\+vp}}e^{i\+vp\cdot \+vx}; \\
    \label{eq:pi_at_time_init}
    \pi\pare{\+vx} &= \int \frac{\rd{^3 p}}{\pare{2\pi}^3}\pare{-i}\sqrt{\frac{\omega_{\+vp}}{2}}\pare{a_{\+vp} - a_{-\+vp}^\dagger} e^{i\+vp\cdot \+vx},
\end{align}
where the commutation relation
\begin{finaleq}{The Commutation Relation of Creation and Annihilation Operators}
    \[ \brac{a_{\+vp},a^\dagger_{\+vp'}} = \pare{2\pi}^3 \delta^{\pare{3}}\pare{\+vp - \+vp'} \]
\end{finaleq}
holds, whence we could easily verify that \begin{margintips}
    Carry out $\int \rd{x}$ in $H$ first. Then apply $p \rightarrow -p$ if necessary.
\end{margintips}
\[ \brac{\phi\pare{\+vx},\pi\pare{\+vx'}} = i\delta^{\pare{3}}\pare{\+vx - \+vx'}, \]
and that the $H$ in equation \eqref{eq:hamiltonian_of_the_klein_gordon_field} may be expanded as
\begin{equation}
    \label{eq:h_expanded}
    H = \int \frac{\rd{^3 p}}{\pare{2\pi}^3}\omega_{\+vp}\pare{a_{\+vp}^\dagger a_{\+vp} + \half \brac{a_{\+vp}, a_{\+vp}^\dagger}}.
\end{equation}
The last term, proportional to $\delta\pare{0}$, is the zero-point energy and could be discarded. Therefore, we have the following commutation relations,
\begin{equation}
    \label{eq:commutation_relations_of_H_and_a}
    \inlinefinaleq{\brac{H,a_{\+vp}^\dagger} = \omega_{\+vp}a_{\+vp}^\dagger;\quad \brac{H,a_{\+vp}} = -\omega_{\+vp}a_{\+vp}.}
\end{equation}
From equation \eqref{eq:conserved_momentum} we find
\[ \+vP = -\int \rd{^3 x}\,\pi\pare{\+vx}\grad \phi\pare{\+vx} = \int \frac{\rd{^3 p}}{\pare{2\pi}^3} \+vp a_{\+vp}^\dagger a_{\+vp}. \]
So the operator $a_{\+vp}^\dagger$ creates a particle of momentum $\+vp$ and energy $\omega_{\+vp} = \sqrt{\+vp^2 + m^2}$. We define the normalization
\[ \inlinefinaleq{\ket{\+vp} = \sqrt{2E_{\+vp}}a_{\+vp}^\dagger\ket{0}} \]
so that
\[ \bra{\+vp}\ket{\+vq} = 2E_{\+vp}\pare{2\pi}^3 \delta^{\pare{3}}\pare{\+vp - \+vq} \]
is Lorentz invariant. Under a Lorentz transformation $\Lambda$ we have
\[ U\pare{\Lambda}\ket{\+vp} = \ket{\Lambda \+vp},\quad \text{and}\quad U\pare{\Lambda} a_{\+vp}^\dagger U^{-1}\pare{\Lambda} = \sqrt{\frac{E_{\Lambda \+vp}}{E_{\+vp}}}a^\dagger_{\Lambda \+vp}. \]
With this normalization, the completeness relation for the one-particle states is
\[ \pare{\mathbbm{1}}\+_1-particle_ = \int \frac{\rd{^3 p}}{\pare{2\pi}^3}\ket{\+vp}\rec{2E_{\+vp}}\bra{\+vp}. \]
\begin{sample}
    \begin{example}
        The integral
        \[ \int \frac{\rd{^3 p}}{\pare{2\pi}^3} \frac{f\pare{p}}{2E_{\+vp}} = \int \left.\frac{\rd{^4 p}}{\pare{2\pi}^4}\pare{2\pi}\delta\pare{p^2 - m^2} f\pare{p}\right\vert_{p^0 > 0} \]
        is Lorentz invariant of $f\pare{p}$ is.
    \end{example}
\end{sample}
A particle at position $\+vx$ is created by
\[ \phi\pare{\+vx}\ket{0} = \int \frac{\rd{^3 \+vp}}{\pare{2\pi}^3}\rec{2E_{\+vp}} e^{-i\+vp\cdot \+vx}\ket{\+vp}. \]
The position-space representation of $\ket{\+vp}$ is given by
\[ \bra{0}\phi\pare{\+vx}\ket{\+vp} = \bra{0} \int \frac{\rd{^3 p}}{\pare{2\pi}^3} \rec{\sqrt{2E_{\+vp'}}}\pare{a_{\+vp'} e^{i\+vp' \cdot \+vx} + a^\dagger_{\+vp'} e^{-i\+vp'\cdot \+vx}}\sqrt{2E_{\+vp}}a_{\+vp}^\dagger\ket{0} = e^{i\+vp\cdot \+vx}. \]

% subsubsection the_klein_gordon_field_as_harmonic_oscillators (end)

\subsubsection{The Complex Scalar Field} % (fold)
\label{ssub:the_complex_scalar_field}

The Lagrangian is given by \marginnote{
    \inlinehardlink{Problem 3.5, \cite{Srednicki:1019751}.} \\
    \inlinehardlink{Problem 2.2, \cite{Peskin:257493}.}
}
\[ \partial_\mu \phi^* \partial^\mu \phi - m^2\phi^* \phi, \]
and the Hamiltonian is given by
\begin{align*}
    H &= \int \rd{^3 x}\,\pare{\pi^* \pi + \grad \phi^* \cdot \grad \phi + m^2\phi^* \phi} \\
    &= \int \frac{\rd{^3 p}}{\pare{2\pi}^3}\omega_{\+vp}\brac{a^\dagger_{\+vp}a_{\+vp} + b^\dagger_{\+vp}b_{\+vp} + \frac{\brac{a_{\+vp}, a^\dagger_{\+vp}}}{2} + \frac{\brac{b_{\+vp}, b^\dagger_{\+vp}}}{2}},
\end{align*}
where two kinds of annihilation operators $a$ and $b$ have been introduced, corresponding to two kinds of particles involved.

% subsubsection the_complex_scalar_field (end)

% subsection the_klein_gordon_field (end)

\subsection{The Klein-Gordon Field in Space-Time} % (fold)
\label{sub:the_klein_gordon_field_in_space_time}

\subsubsection{Time Evolution} % (fold)
\label{ssub:time_evolution}

With the Heisenberg equation of motion we identify the time evolution of $\phi$ and $\pi$ as \begin{margintips}
    For $\partial_t{\pi}$, notice that the $\grad$ acts only on $x'$, and use the Green's identity to swap the items on the left and right of $\grad^2$.
\end{margintips}
\begin{align*}
    i\+DtD{}\phi\pare{\+vx,t} &= \brac{\phi\pare{\+vx,t}, \int \rd{^3 x'} \,\curb{ \half \pi^2\pare{\+vx',t} + \half \pare{\grad \phi\pare{\+vx',t}}^2 + \half m^2 \phi^2\pare{\+vx',t} }}\\ &= i\pi\pare{\+vx,t}; \\
    i\+DtD{}\pi\pare{\+vx,t} &= \brac{\pi\pare{\+vx,t},\int \rd{^3 x'}\,\curb{\half \pi^2\pare{\+vx',t} \pare{-\grad^2 + m^2} \phi\pare{\+vx',t}}}\\ &= -i\pare{-\grad^2 + m^2}\phi\pare{\+vx,t}.
\end{align*}
Combining the two results yields the Klein-Gordon equation \eqref{eq:the_klein-gordon-equation}. Using commutation relations \eqref{eq:commutation_relations_of_H_and_a} we find
\[ e^{iHt}a_{\+vp}e^{-iHt} = a_{\+vp}e^{-iE_{\+vp}t},\quad \text{and}\quad e^{iHt}a^\dagger_{\+vp}e^{-iHt} = a^\dagger_{\+vp} e^{iE_{\+vp}t}, \]
where $a_{\+vp}$ are time-independ operators in the Schr\"odinger picture. Combining $\phi\pare{\+vx} = e^{iHt} \phi\pare{\+vx} e^{-iHt}$ with equations \eqref{eq:phi_at_time_init} and \eqref{eq:pi_at_time_init} yields
\begin{equation}
    \label{eq:phi_pi_at_time_any}
    \inlinefinaleq{\phi\pare{\+vx,t} = \int \left.\frac{\rd{^3 p}}{\pare{2\pi}^3} \rec{\sqrt{2E_{\+vp}}} \pare{a_{\+vp} e^{-ip\cdot x} + a^\dagger_{\+vp} e^{ip\cdot x}}\right\vert_{p^0 = E_p};\quad \pi\pare{\+vx,t} = \+DtD{}\phi\pare{\+vx,t}.}
\end{equation}
With the identities \begin{margintips}
    One could also start with $\brac{\phi\pare{\+vx},\+vP} = i\partial_x \phi\pare{\+vx}$.
\end{margintips}
\[ e^{-i\+vP\cdot \+vx} a_{\+vp} e^{i\+vP\cdot \+vx} = a_{\+vp} e^{i\+vp\cdot \+vx},\quad \text{and} \quad e^{-i\+vP\cdot \+vx} a^\dagger_{\+vp} e^{i\+vP\cdot \+vx} = a^\dagger_{\+vp} e^{-i\+vp\cdot \+vx}, \]
we found (note that $\brac{\+vP,H} = 0$)
\[ \phi\pare{x} = e^{i\pare{Ht - \+vP\cdot \+vx}} \phi\pare{0} e^{-i\pare{Ht - \+vP\cdot \+vx}} = e^{iP\cdot x}\phi\pare{0} e^{-iP\cdot x}. \]

% subsubsection time_evolution (end)

\subsubsection{Causality} % (fold)
\label{ssub:causality}

The amplitude for a particle to propagate from $y$ to $x$ is \begin{margintips}
    \raggedright
    Only $\bra{0} a_{\+vp} a^\dagger_{\+vq}\ket{0}$ $=$ $\pare{2\pi}^3 \delta^{\pare{3}}\pare{\+vp-\+vq}$ survives.
\end{margintips}
\[ D\pare{x-y} = \bra{0}\phi\pare{x}\phi\pare{y}\ket{0} = \int \left.\frac{\rd{^3 p}}{\pare{2\pi}^3} \rec{2E_{\+vp}} e^{-ip\cdot \pare{x-y}}\right\vert_{p^0 = E_{\+vp}}. \]
\begin{sample}
    \begin{example}
    \label{ex:time_like_propagator}
        For $x^0 - y^0 = t$ and $\+vx - \+vy = 0$, we have
        \begin{align*}
            D\pare{x-y} &= \frac{4\pi}{\pare{2\pi}^3} \int_0^\infty \frac{p^2}{2\sqrt{p^2 + m^2}} e^{-i\sqrt{p^2 + m^2}t} \\
            &= \rec{4\pi^2} \int_m^\infty \rd{E}\, \sqrt{E^2 - m^2} e^{-iEt} \\
            & \underset{t\rightarrow \infty}{\sim} e^{-imt}.
        \end{align*}
    \end{example}
\end{sample}
\begin{sample}
    \begin{example}
    \label{ex:space_like_propagator}
        For $x^0 - y^0 = 0$, $\+vx - \+vy = \+vr$, we have
        \begin{align*}
            D\pare{x-y} = \frac{-i}{2\pare{2\pi}^2 r} \int_{-\infty}^{\infty}\rd{p}\, \frac{pe^{ipr}}{\sqrt{p^2 + m^2}} = \rec{4\pi^2 r} \int_m^\infty \frac{\rho e^{-\rho r}}{\sqrt{\rho^2 - m^2}} \underset{r\rightarrow \infty}{\sim} e^{-mr}.
        \end{align*}
    \end{example}
\end{sample}
To see if the measurements on two points could affect one another, we calculate
\begin{equation}
    \label{eq:commutator_of_position_field}
    \brac{\phi\pare{x},\phi\pare{y}} = \int \frac{\rd{^3 p}}{\pare{2\pi}^3} \rec{2E_{\+vp}}\pare{e^{-ip\cdot \pare{x-y}} - e^{ip\cdot \pare{x-y}}} = D\pare{x-y} - D\pare{y-x}.
\end{equation}
If $\pare{x-y}$ is space-like, a Lorentz transformation may carry it into the form in \cref{ex:space_like_propagator}, and the two terms cancels each other. If $\pare{x-y}$ is time-like, it may be carried into the form in \cref{ex:time_like_propagator} and the amplitude is roughly $\pare{e^{-imt} - e^{imt}}$, which is nonzero.

% subsubsection causality (end)

\subsubsection{The Klein-Gordon Propagator} % (fold)
\label{ssub:the_klein_gordon_propagator}

\begin{figure}[ht]
    \centering
    \begin{tikzpicture}
        \begin{scope}[thick,decoration={
            markings,
            mark=at position 0.5 with {\arrow{latex}}}
            ] 
            \draw[postaction={decorate}] (-3,0) -- (-1.7,0);
            \draw[postaction={decorate}] (-1.3,0) -- (1.3,0);
            \draw[postaction={decorate}] (1.7,0) -- (3,0);
        \end{scope}
        \draw (0,-1) -- (0,1);
        \draw (-3,0) -- (3,0);
        \draw (-1.7,0) arc (180:0:0.2);
        \draw (1.3,0) arc (180:0:0.2);
        \draw[fill=black] (-1.5,0) circle (0.05) node[below] {$-E_{\+vp}$};
        \draw[fill=black] (1.5,0) circle (0.05) node[below] {$E_{\+vp}$};
    \end{tikzpicture}
    \caption{Contour of the integral in equation \eqref{eq:retarded_propagator_total}.}
    \label{fig:retarded_contour}
\end{figure}
\begin{figure}[ht]
    \centering
    \begin{tikzpicture}
        \begin{scope}[thick,decoration={
            markings,
            mark=at position 0.5 with {\arrow{latex}}}
            ] 
            \draw[postaction={decorate}] (-3,0) -- (-1.7,0);
            \draw[postaction={decorate}] (-1.3,0) -- (1.3,0);
            \draw[postaction={decorate}] (1.7,0) -- (3,0);
        \end{scope}
        \draw (0,-1) -- (0,1);
        \draw (-3,0) -- (3,0);
        \draw (-1.7,0) arc (180:360:0.2);
        \draw (1.3,0) arc (180:0:0.2);
        \draw[fill=black] (-1.5,0) circle (0.05) node[above] {$-E_{\+vp}$};
        \draw[fill=black] (1.5,0) circle (0.05) node[below] {$E_{\+vp}$};
    \end{tikzpicture}
    \caption{Contour of the integral in equation \eqref{eq:feynman_propagator}.}
    \label{fig:feynman_contour}
\end{figure}
Evaluating \eqref{eq:commutator_of_position_field} for $x^0 > y^0$, we find \begin{margintips}
    \raggedright
    The propagator could also be found via the Fourier transform. See \href{https://en.wikipedia.org/wiki/Propagator}{propagator}.
\end{margintips}
\begin{align}
    \bra{0}\brac{\phi\pare{x},\phi\pare{y}}\ket{0} &= \int \frac{\rd{^3 p}}{\pare{2\pi}^3} \curb{\left. \rec{2E_{\+vp}}e^{-ip\cdot\pare{x-y}} \right\vert_{p^0 = E_{\+vp}} + \left. \rec{-2E_{\+vp}}e^{-ip\cdot\pare{x-y}} \right\vert_{p^0 = -E_{\+vp}}} \notag \\
    &\label{eq:retarded_propagator_total}\underset{x^0 > y^0}{=} \int \frac{\rd{^3 p}}{\pare{2\pi}^3}\int \frac{\rd{p^0}}{2\pi i} \frac{-1}{p^2 - m^2} e^{-ip\cdot \pare{x-y}},
\end{align}
where the integral in $p^0$ is along the contour shown in \cref{fig:retarded_contour}.
\par
We define the \gloss{retarded propagator} as
\[ D_R\pare{x-y} = \Theta\pare{x^0 - y^0}\bra{0}\brac{\phi\pare{x},\phi\pare{y}}\ket{0}, \]
which is a Green's function of the Klein-Gordon operator, i.e. \begin{margintips}[-2\baselineskip]
    If $f\pare{0} = 0$, then $\delta'\cdot f = -\delta\cdot f'$. And $ D_R\pare{x-y} = 0 $ if $x-y$ is space-like.
\end{margintips}
\[ \inlinefinaleq{\pare{\partial^2 + m^2}D_R\pare{x-y} = -i\delta^{\pare{4}}\pare{x-y}.} \]
\par
The \gloss{Feynman propagator}, which adopts a different contour shown in \cref{fig:feynman_contour}, is defined by
\begin{equation}
    \label{eq:feynman_propagator}
    D_F\pare{x-y} = \int \frac{\rd{^4 p}}{\pare{2\pi}^4} \frac{i}{p^2 - m^2 + i\epsilon} e^{-ip\cdot \pare{x-y}}.
\end{equation}
Closing the contour above and below for $x^0 < y^0$ and $x^0 > y^0$, respectively, yields
\begin{align*}
    D_F\pare{x-y} &= \begin{cases}
        D\pare{x-y}, & \mathrm{if\ } x^0 > y^0; \\
        D\pare{y-x}, & \mathrm{if\ } x^0 < y^0
    \end{cases} \\
    &= \Theta\pare{x^0 - y^0}\bra{0}\phi\pare{x}\phi\pare{y}\ket{0} + \theta\pare{y^0 - x^0}\bra{0}\phi\pare{y}\phi\pare{x}\ket{0},
\end{align*}
which is another Green's function of the Klein-Gordon operator.
\begin{finaleq}{Feynman Propagator as Time-Ordered Transition Amplitude}
    \[ D_F\pare{x-y} = \bra{0}T\phi\pare{x}\phi\pare{y}\ket{0}. \]
\end{finaleq}

% subsubsection the_klein_gordon_propagator (end)

\subsubsection{Particle Creation by a Classical Source} % (fold)
\label{ssub:particle_creation_by_a_classical_source}

For a source $j\pare{x}$ that is nonzero only for a finite time interval, the Lagranian is
\[ \+cL = \half\pare{\partial_\mu \phi}^2 - \half m^2 \phi^2 + j\pare{x}\phi\pare{x}, \]
which yields the field equation
\[ \pare{\partial^2 + m^2}\phi\pare{x} = j\pare{x}. \]
The $\phi\pare{x} = \phi_0\pare{x}$ before the source is turned on is given by equation \eqref{eq:phi_pi_at_time_any}, while the $\phi\pare{x}$ when all of $j$ is in the past is
\begin{align*}
    \phi\pare{x} &= \phi_0\pare{x} + i\int \rd{^4 y}\,D_R\pare{x-y}j\pare{y} \\
    &= \int \frac{\rd{^3 p}}{\pare{2\pi}^3}\rec{\sqrt{2E_{\+vp}}}\curb{\pare{a_{\+vp} + \frac{i}{\sqrt{2E_{\+vp}}}\tilde{j}\pare{p}}e^{-ip\cdot x} + \hermitianconj},
\end{align*}
where
\[ \tilde{j}\pare{p} = \int \rd{^4 y}\, e^{ip\cdot y}j\pare{y}. \]
The Hamiltonian after $j\pare{x}$ is acted is obtained, by replacing the $a_{\+vp}$ in equation \eqref{eq:h_expanded} with $\pare{a_{\+vp} + i\tilde{j}\pare{p}/\sqrt{2E_{\+vp}}}$, as
\[ H = \int \frac{\rd{^3 p}}{\pare{2\pi}^3}\, E_{\+vp}\pare{a^\dagger_{\+vp} - \frac{i}{\sqrt{2E_{\+vp}}}\tilde{j}^*\pare{p}}\pare{a_{\+vp} + \frac{i}{\sqrt{2E_{\+vp}}}\tilde{j}\pare{p}}. \]
The energy of the system after the source has been turned off is
\[ \bra{0}H\ket{0} = \int \frac{\rd{^3 p}}{\pare{2\pi}^3} E_{\+vp}\cdot \rec{2E_{\+vp}}\abs{\tilde{j}\pare{p}}^2, \]
where the integrand may be understood as the probability density for creating a particle in mode $p$, giving
\[ \int \rd{N} = \int \frac{\rd{^3 p}}{\pare{2\pi}^3}\rec{2E_{\+vp}}\abs{\tilde{j}\pare{p}}^2. \]

% subsubsection particle_creation_by_a_classical_source (end)

% subsection the_klein_gordon_field_in_space_time (end)

% section quantization_of_scalar_field (end)

\subsection{Bibliography} % (fold)
\label{sub:bibliography}

\printbibliography[heading=none]

% subsection bibliography (end)

\end{document}