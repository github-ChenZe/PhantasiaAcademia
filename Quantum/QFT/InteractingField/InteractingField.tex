\documentclass[hidelinks]{article}

\usepackage[sensei=C.Z.,gakka=QFT,section=Quantum,gakkabbr=QM]{styles/kurisuen}
\usepackage{sidenotes}
\usepackage{van-de-la-sehen-en}
\usepackage{van-de-environnement-en}
\usepackage{boite/van-de-boite-en}
\usepackage{van-de-abbreviation}
\usepackage{van-de-neko}
\usepackage{van-le-trompe-loeil}
\usepackage{cyanide/van-de-cyanide}
\setlength{\parindent}{0pt}
\usepackage{enumitem}
\newlist{citemize}{itemize}{3}
\setlist[citemize,1]{noitemsep,topsep=0pt,label={-},leftmargin=1em}

\usepackage{mathtools}
\usepackage{ragged2e}

\DeclarePairedDelimiter\abs{\lvert}{\rvert}%
\DeclarePairedDelimiter\norm{\lVert}{\rVert}%

% Swap the definition of \abs* and \norm*, so that \abs
% and \norm resizes the size of the brackets, and the 
% starred version does not.
\makeatletter
\let\oldabs\abs
\def\abs{\@ifstar{\oldabs}{\oldabs*}}
%
\let\oldnorm\norm
\def\norm{\@ifstar{\oldnorm}{\oldnorm*}}
\makeatother

\newcommand*{\Value}{\frac{1}{2}x^2}%

\usepackage{fancyhdr}
\usepackage{lastpage}

\fancypagestyle{plain}{%
\fancyhf{} % clear all header and footer fields
\fancyhead[R]{\smash{\raisebox{2.75em}{{\hspace{1cm}\color{lightgray}\textsf{\rightmark\quad Page \thepage/\pageref{LastPage}}}}}} %RO=right odd, RE=right even
\renewcommand{\headrulewidth}{0pt}
\renewcommand{\footrulewidth}{0pt}}
\pagestyle{plain}

\newtheorem*{experiment*}{Measurement}
\newtheorem{example}{Example}
\newtheorem{remark}{Remark}

\def\elementcell#1#2#3#4#5#6#7{%
    \draw node[draw, regular polygon, regular polygon sides=4, minimum height=2cm, draw=cyan, line width=0.4mm, fill=cyan!15!white, #1, inner sep=-2mm](#3) {\Large\textbf{\textsf{\color{cyan!50!black}#4}}};
    \draw (#3.corner 1) node[below left] {\footnotesize{\phantom{Hj}#5}};
    \draw (#3.corner 2) node[below right] {\small{\textsf{#6}}};
    \draw (#3.side 3) node[above] {\footnotesize #7};
    \draw (#3.corner 2) ++ (0,-0.4mm) node(nw#3) {};
    \tcbsetmacrotowidthofnode{\elementcellwidth}{#3}
    \node [fill=cyan, line width=0mm, rectangle, rounded corners=1.8mm, rectangle round south east=false, rectangle round south west=false, anchor=south west, minimum width=\elementcellwidth] at (nw#3) {\small\textsf{\color{white}#2}};
}

\DeclareSIUnit\Dq{Dq}
\usepackage{physics}
\usepackage{bbm}
\newtheorem{lemma}{Lemma}
\newtheorem{proposition}{Proposition}

\DeclareMathOperator{\Pfaffian}{Pf}
\DeclareMathOperator{\sign}{sign}

\usetikzlibrary{decorations.markings}
\usetikzlibrary{arrows.meta}

\def\kurisueniconwidth{3.2cm}
\def\kurisueniconpath{img/CMSncFS.png}
\newcommand{\hermitianconj}{\mathrm{h.c.}}

\usetikzlibrary{decorations.pathmorphing}

\tikzset{snake it/.style={decorate, decoration=snake}}

\usepackage{tensor}
\usepackage{slashed}
\let\grad\nabla

\begin{document}

\section{Interacting Fields} % (fold)
\label{sec:interacting_fields_and_feynman_diagrams}

\subsection{The Feynman Rules} % (fold)
\label{sub:the_feynman_rules}

\subsubsection{Rules} % (fold)
\label{ssub:rules}

To calculate the Amplitude $\+sM$, associated with a particular Feynman diagram, proceed as follows:
\begin{cenum}
    \item Notation.
    \begin{cenum}
        \item Label the incoming the outgoing four-momenta $p_1,p_2,\cdots$, and the corresponding spins $s_1,s_2,\cdots$.
        \item Label the internal four-momenta $q_1,q_2,\cdots$.
        \item The arrows on external fermion lines indicate whether it is an electron or a positron.
        \item Arrows on internal fermion lines are assigned so that the ``direction of flow'' through the diagram is preserved.
        \item The arrows on external photon lines point ``forward''.
        \item For internal photon lines the choice is arbitrary.
    \end{cenum}
    \item External lines. External lines contribute factors as follows:
    \begin{cenum}
        \item Electrons: $\begin{cases}
            Incoming\pare{\smash{\begin{tikzpicture}
                \draw[-latex](0,0) -- (0.5,0.1);
                \draw (0.5,0.1) -- (1,0.2);
                \draw[fill=black] (1,0.2) circle (0.05);
            \end{tikzpicture}}}: & u,\\
            Outgoing\pare{\smash{\begin{tikzpicture}
                \draw[-latex](0,0) -- (0.5,0.1);
                \draw (0.5,0.1) -- (1,0.2);
                \draw[fill=black] (0,0) circle (0.05);
            \end{tikzpicture}}}: & \conj{u}.
        \end{cases}$
        \item Positrons: $\begin{cases}
            Incoming\pare{\smash{\begin{tikzpicture}
                \draw(0,0) -- (0.5,0.1);
                \draw[latex-] (0.5,0.1) -- (1,0.2);
                \draw[fill=black] (1,0.2) circle (0.05);
            \end{tikzpicture}}}: & v,\\
            Outgoing\pare{\smash{\begin{tikzpicture}
                \draw(0,0) -- (0.5,0.1);
                \draw[latex-] (0.5,0.1) -- (1,0.2);
                \draw[fill=black] (0,0) circle (0.05);
            \end{tikzpicture}}}: & \conj{v}.
        \end{cases}$
        \item Photons: $\begin{cases}
            Incoming\pare{\smash{\begin{tikzpicture}
                \draw[snake it](0,0) -- (1,0.2);
                \draw[fill=black] (1,0.2) circle (0.05);
            \end{tikzpicture}}}: & \+v\epsilon_\mu,\\
            Outgoing\pare{\smash{\begin{tikzpicture}
                \draw[snake it](0,0) -- (1,0.2);
                \draw[fill=black] (0,0) circle (0.05);
            \end{tikzpicture}}}: & \conj{\+v\epsilon}^*_\mu.
        \end{cases}$
    \end{cenum}
    \item Vectex Factors. Each vertex contributes a factor
    \[ ig_e \gamma^\mu, \]
    where $g_e = e\sqrt{4\pi/\hbar c}$.
    \item Propagators. Each internal line contributes a factor as follows:
    \begin{cenum}
        \item Electrons and positrons: $\displaystyle \frac{i\pare{\slashed{q} + mc}}{q^2 - m^2c^2}$.
        \item Photons: $\displaystyle \frac{-ig_{\mu\nu}}{q^2}$.
    \end{cenum}
    \item Conservation of Energy and Momentum. For each vertex, write a delta function of the form
    \[ \pare{2\pi}^4 \delta^4\pare{k_1 + k_2 + k_3}, \]
    where the $k$'s are the three four-momenta coming into the vertex.
    \item Integrate Over Internal Momenta. For each internal momentum $q$, write a factor $\displaystyle \frac{\rd{^4 q}}{\pare{2\pi}^4}$ and integrate.
    \item Cancel the Delta Function. The result will include a factor
    \[ \pare{2\pi^4}\delta^4\pare{p_1 + p_2 + \cdots - p_n} \]
    corresponding to overall energy-momentum conservation. Cancel this factor, and what remains is $-i\+sM$.
    \item Antisymmetrization. Include a minus sign between diagrams that differ only in the interchange of two incoming (or outgoing) electrons (or positrons), or of an incoming electron with an outgoint positron (or vice versa).
\end{cenum}

% subsubsection rules (end)

\begin{figure}[ht]
    \centering
    \includegraphics{src/e-mu.pdf}
    \caption{Electron-muon scattering.}
    \label{fig:electron_muon_scattering}
\end{figure}
\begin{sample}
    \begin{example}[Electron-Muon Scattering]
        We proceed ``backward'' along each fermion line in \cref{fig:electron_muon_scattering}:
        \begin{gather*}
            \pare{2\pi}^4 \int \brac{\conj{u}^{\pare{s_3}}\pare{p_3}\pare{ig_e\gamma^\mu}u^{\pare{s_1}}\pare{p_1}}\times \\
            \frac{-ig_{\mu\nu}}{q^2}\brac{\conj{u}^{\pare{s_4}}\pare{p_4}\pare{ig_e \gamma^\nu}u^{\pare{s_2}}\pare{p_2}}\times \\
            \delta^4\pare{p_1 - p_3 - q}\delta^4\pare{p_2 + q - p_r}\,\rd{^4 q}.
        \end{gather*}
        Carrying out the $q$ integration and dropping the overall delta function, we find
        \[ \+sM = -\frac{g_e^2}{\pare{p_1 - p_3}^2}\brac{\conj{u}^{\pare{s_3}}\pare{p_3} \gamma^\mu u^{\pare{s_1}}\pare{p_1}}\brac{\conj{u}^{\pare{s_4}}\pare{p_4}\gamma_\mu u^{\pare{s_2}}\pare{p_2}}. \]
    \end{example}
\end{sample}

% subsection the_feynman_rules (end)

% section interacting_fields_and_feynman_diagrams (end)

\end{document}