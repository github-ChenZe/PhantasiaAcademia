\documentclass[hidelinks]{article}

\usepackage[sensei=Nakahara,gakka=Computational\ Physics,section=Quantum,gakkabbr=QM]{styles/kurisuen}
\usepackage{sidenotes}
\usepackage{van-de-la-sehen-en}
\usepackage{van-de-environnement-en}
\usepackage{boite/van-de-boite-en}
\usepackage{van-de-abbreviation}
\usepackage{van-de-neko}
\usepackage{van-le-trompe-loeil}
\usepackage{cyanide/van-de-cyanide}
\setlength{\parindent}{0pt}
\usepackage{enumitem}
\newlist{citemize}{itemize}{3}
\setlist[citemize,1]{noitemsep,topsep=0pt,label={-},leftmargin=1em}

\usepackage{mathtools}
\usepackage{ragged2e}

\DeclarePairedDelimiter\abs{\lvert}{\rvert}%
\DeclarePairedDelimiter\norm{\lVert}{\rVert}%

% Swap the definition of \abs* and \norm*, so that \abs
% and \norm resizes the size of the brackets, and the 
% starred version does not.
\makeatletter
\let\oldabs\abs
\def\abs{\@ifstar{\oldabs}{\oldabs*}}
%
\let\oldnorm\norm
\def\norm{\@ifstar{\oldnorm}{\oldnorm*}}
\makeatother

\newcommand*{\Value}{\frac{1}{2}x^2}%

\usepackage{fancyhdr}
\usepackage{lastpage}

\fancypagestyle{plain}{%
\fancyhf{} % clear all header and footer fields
\fancyhead[R]{\smash{\raisebox{2.75em}{{\hspace{1cm}\color{lightgray}\textsf{\rightmark\quad Page \thepage/\pageref{LastPage}}}}}} %RO=right odd, RE=right even
\renewcommand{\headrulewidth}{0pt}
\renewcommand{\footrulewidth}{0pt}}
\pagestyle{plain}

\newtheorem*{experiment*}{Measurement}
\newtheorem{example}{Example}
\newtheorem{remark}{Remark}

\def\elementcell#1#2#3#4#5#6#7{%
    \draw node[draw, regular polygon, regular polygon sides=4, minimum height=2cm, draw=cyan, line width=0.4mm, fill=cyan!15!white, #1, inner sep=-2mm](#3) {\Large\textbf{\textsf{\color{cyan!50!black}#4}}};
    \draw (#3.corner 1) node[below left] {\footnotesize{\phantom{Hj}#5}};
    \draw (#3.corner 2) node[below right] {\small{\textsf{#6}}};
    \draw (#3.side 3) node[above] {\footnotesize #7};
    \draw (#3.corner 2) ++ (0,-0.4mm) node(nw#3) {};
    \tcbsetmacrotowidthofnode{\elementcellwidth}{#3}
    \node [fill=cyan, line width=0mm, rectangle, rounded corners=1.8mm, rectangle round south east=false, rectangle round south west=false, anchor=south west, minimum width=\elementcellwidth] at (nw#3) {\small\textsf{\color{white}#2}};
}

\DeclareSIUnit\Dq{Dq}
\usepackage{physics}
\usepackage{bbm}
\newtheorem{lemma}{Lemma}
\newtheorem{proposition}{Proposition}

\DeclareMathOperator{\Pfaffian}{Pf}
\DeclareMathOperator{\sign}{sign}

\begin{document}

\section{Variational Method} % (fold)
\label{sec:variational_method}

\subsection{Variational Calculus} % (fold)
\label{sub:variational_calculus}

The solution of the Schr\"odinger equation demands that
\[ E\brac{\psi} = \frac{\bra{\psi}H\ket{\psi}}{\bra{\psi}\ket{\psi}} \]
reach its minimum. Expanding
\[ \ket{\psi} = \sum_p C_p\ket{\chi_p} \]
with an orthonormal basis $\ket{\chi_p}$ such that $\bra{\chi_p}\ket{\chi_q} = \delta_{pq}$ yields
\[ E = \frac{\sum_{p,q=1}^N C^*_p C_q H_{pq}}{\sum_{p,q=1}^N C^*_p C_q \delta_{pq}}, \]
where
\[ H_{pq} = \bra{\chi_p}H\ket{\chi_q}. \]
The stationary conditional requires
\[ \sum_{q=1}^N \pare{H_{pq} - E\delta_{pq}} C_q = 0,\quad \mathrm{for}\quad p=1,\cdots,N, \]
or, in matrix notation,
\[ \inlinefinaleq{\+vH\+vC = E \+vC.} \]
\vspace{-\baselineskip}
\begin{finaleq}{Gerneralised Eigenvalue Problem}
    If a non-orthonormal basis is used, with $S_{eq} = \bra{\chi_p}\ket{\chi_q}$, the sitationary conditional becomes
    \[ \+vH\+vC = E\+vS\+vC. \]
\end{finaleq}

% subsection variational_calculus (end)

\subsection{Solution of the Generalised Eigenvalue Problem} % (fold)
\label{sub:solution_of_the_generalised_eigenvalue_problem}

With $\+vV^\dagger \+vS \+vV = \+vI$, $\+vC' = \+vV^{-1}\+vC$, and $\+vH' = \+vV^\dagger \+vH \+vV$, we obtain
\[ \+vH'\+vC' = E \+vC'. \]
After diagonalizsing $\+vS$ by, for example, Givens-Householder $QR$ procedure,
\[ \+vU^\dagger \+vS \+vU = \+vs, \]
we have $\+vV = \+vU \+vs^{-1/2}$.

% subsection solution_of_the_generalised_eigenvalue_problem (end)

\subsection{Perturbation Theory} % (fold)
\label{sub:perturbation_theory}

Starting with an orthonormal basis, for example a set of plane waves. The basis is partitioned into two sets $A$ and $B$, where $A$ contain those of short wavelengths and $B$ of those long. Defining
\[ H'_{pq} = H_{pq}\pare{1-\delta_{pq}},\quad h'_{pn} = \frac{H'_{pn}}{E-H_{pp}} \]
yields
\[ C_p = \sum_{n\in A}h'_{pn} C_n + \sum_{n\in B} h'_{p\alpha}C_\alpha. \]
The $C_n$ and \begin{marginwarns}
    L\"owdin's method requires $E$ be known before the calculation.
\end{marginwarns} $C_\alpha$ in the above equation could be expanded using again the equation itself, which yields
\[ C_p = \sum_{n\in A} \frac{U_{pn}^A - H_{pn}\delta_{pn}}{E - H_{pp}} C_n, \]
where
\[ U_{pn}^A = H_{pn} + \sum_{\alpha\in B} \frac{H'_{p\alpha}H'_{\alpha n}}{E-H_{\alpha\alpha}} + \sum_{\alpha\beta\in B} \frac{H'_{p\alpha} H'_{\alpha\beta} H'_{\beta n}}{\pare{E - H_{\alpha\alpha}} \pare{E - H_{\beta\beta}}} + \cdots. \]
The equation of $C_p$ could be simplified as
\[ \+vU \+vC = E \+vC, \]
where $U$ is the $U^A_{pn}$ defined above and may be truncated.

% subsection perturbation_theory (end)

% section variational_method (end)

\end{document}
