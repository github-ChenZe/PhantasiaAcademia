\documentclass[hidelinks]{ctexart}

\usepackage[margintoc, singleton, nova]{van-de-la-sehen}

\newtcbox{\mylib}{enhanced,nobeforeafter,tcbox raise base,boxrule=0.4pt,top=0mm,bottom=0mm,
  right=0mm,left=0.15mm,arc=1pt,boxsep=2pt,fontupper=\ttfamily,
  colframe=gray,coltext=gray!25!black,colback=gray!10!white
}

\robustify{\mylib}
\newcommand{\mcode}[1]{\ \mylib{#1} \ }

\usepackage{makecell}
\usepackage{float}

\begin{document}

\showtitle{Unix网络编程}

\section{理论} % (fold)
\label{sec:理论}

\subsection{OSI模型} % (fold)
\label{sub:osi模型}

\subsubsection{七层协议} % (fold)
\label{ssub:七层协议}

\begin{table}[H]
\centering
\begin{tabular}{ccc}
\hline
 7 & 应用层 & \+:r3{应用层} \\
\cline{1-2}
 6 & 表示层 & \\
\cline{1-2}
 5 & 会话层 & \\
\Xhline{4\arrayrulewidth}
 4 & 传输层 & TCP, UDP \\
\hline
 3 & 网络层 & IPv4, IPv6 \\
\hline
 2 & 数据链路层 & 设备驱动 \\
\cline{1-2}
 1 & 物理层 & 程序和硬件 \\
\hline
\end{tabular}
\end{table}

% subsubsection 七层协议 (end)

% subsection osi模型 (end)

% section 理论 (end)

\section{套接字} % (fold)
\label{sec:套接字}

\subsection{一般步骤} % (fold)
\label{sub:一般步骤}

\subsubsection{客户端} % (fold)
\label{ssub:客户端}

\begin{cenum}
    \item 创建套接字\mcode{sockfd = socket(...)};
    \item 设定服务器地址
    \begin{cenum}
        \item 置零\mcode{bzero(...)};
        \item 设定协议\mcode{serveraddr.sin\_family = ...,};
        \item 设定端口\mcode{servaddr.sin\_port = ...};
        \item 设定服务器IP\mcode{inet\_pton(...)};
    \end{cenum}
    \item 连接服务器\mcode{connect(...)};
    \item 读取\mcode{n = read(...)};
\end{cenum}

% subsubsection 客户端 (end)

\subsubsection{服务端} % (fold)
\label{ssub:服务端}

\begin{cenum}
    \item 创建套接字\mcode{sockfd = socket(...)};
    \item 设定服务器地址
    \begin{cenum}
        \item 置零\mcode{bzero(...)};
        \item 设定协议\mcode{serveraddr.sin\_family = ...,};
        \item 设定端口\mcode{servaddr.sin\_port = ...};
        \item 设定接受IP\mcode{servaddr.sin\_addr.s\_addr = ...};
    \end{cenum}
    \item 配置为服务端
    \begin{cenum}
        \item 绑定\mcode{bind(...)};
        \item 监听\mcode{listen(...)};
    \end{cenum}
    \item 接受\mcode{connfd = accept(...)};
    \item 写\mcode{n = write(...)};
\end{cenum}

% subsubsection 服务端 (end)

% subsection 一般步骤 (end)

\subsection{套接字IO} % (fold)
\label{sub:套接字io}

\subsubsection{创建} % (fold)
\label{ssub:创建}

\begin{clst}
/@\cinclude{sys/types.h}@/ /* 通常无需加入此头文件 */
/@\cinclude{sys/socket.h}@/
/* 创建一个套接字文件 */
int socket(int domain, int type, int protocol);
/* 成功时返回fd, 失败时返回-1 */
/@\lhend @/
if ((socket_fd = socket(AF_INET, SOCK_STREAM, 0)) < 0)
    /* error */;
\end{clst}

% subsubsection 创建 (end)

\subsubsection{连接} % (fold)
\label{ssub:连接}

\begin{clst}
/@\cinclude{sys/types.h}@/ /* 通常无需加入此头文件 */
/@\cinclude{sys/socket.h}@/
/* 创建一个套接字文件 */
int connect(int sockfd, const struct /@\+c{sockaddr}@/ *addr,
    /@\+c{socklen\_t}@/ addrlen);
/* 成功时返回0, 失败时返回-1 */
/@\lhend @/
#define SA struct /@\+c{sockaddr} @/
if (connect(sockfd, (SA*) &servaddr, sizeof(servaddr)) < 0)
    /* error */;
\end{clst}

% subsubsection 连接 (end)

\subsubsection{绑定} % (fold)
\label{ssub:绑定}

\begin{clst}
/@\cinclude{sys/types.h}@/ /* 通常无需加入此头文件 */
/@\cinclude{sys/socket.h}@/
/* 将fd绑定到地址 */
int bind(int sockfd, const struct /@\+c{sockaddr}@/ *addr,
    /@\+c{socklen\_t}@/ addrlen);
/* 成功时返回0, 失败时返回-1 */
/@\lhend @/
#define SA struct /@\+c{sockaddr} @/
if (bind(listenfd, (SA*)servaddr, sizeof(servaddr)) < 0)
    /* error */;
\end{clst}

% subsubsection 绑定 (end)

\subsubsection{监听} % (fold)
\label{ssub:监听}

\begin{clst}
/@\cinclude{sys/types.h}@/ /* 通常无需加入此头文件 */
/@\cinclude{sys/socket.h}@/
/* 将fd转化为监听套接字 */
int listen(int sockfd, int backlog);
/* 成功时返回0, 失败时返回-1 */
/@\lhend @/
if (listen(listenfd, LISTENQ) < 0)
    /* error */;
\end{clst}

% subsubsection 监听 (end)

\subsubsection{接受} % (fold)
\label{ssub:接受}

\begin{clst}
/@\cinclude{sys/types.h}@/ /* 通常无需加入此头文件 */
/@\cinclude{sys/socket.h}@/
/* 将fd转化为监听套接字 */
int accept(int sockfd, struct /@\+c{sockaddr}@/ *addr, /@\+c{socklen\_t}@/ *addrlen);
/* 成功时返回fd, 失败时返回-1 */
/@\lhend @/
#define SA struct /@\+c{sockaddr} @/
if ((fd = accept(listenfd, (SA*) NULL, NULL)) < 0)
    /* error */;
\end{clst}

% subsubsection 接受 (end)

% subsection 套接字io (end)

\subsection{格式转化} % (fold)
\label{sub:格式转化}

\subsubsection{置零} % (fold)
\label{ssub:置零}

\begin{clst}
/@\cinclude{string.h}@/
/* 置零*s指向的位置, 长度为n */
void bzero(void *s, /@\+c{size\_t}@/ n);
/@\lhend @/
struct /@\+c{sockaddr\_in}@/ servaddr;
bzero(servaddr, sizeof(servaddr));
\end{clst}

% subsubsection 置零 (end)

\subsubsection{整数字节序} % (fold)
\label{ssub:整数字节序}

\begin{clst}
/@\cinclude{arpa/inet.h}@/
/* 主机到短整数: 修改大/小端顺序使适合网络顺序 */
/@\+c{uint16\_t}@/ htons(/@\+c{uint16\_t}@/ hostshort);
/@\lhend @/
struct /@\+c{sockaddr\_in}@/ servaddr;
servaddr.sin_port = htons(13);
\end{clst}

\begin{clst}
/@\cinclude{arpa/inet.h}@/
/* 主机到长整数: 修改大/小端顺序使适合网络顺序 */
/@\+c{uint32\_t}@/ htonl(/@\+c{uint32\_t}@/ hostlong);
/@\lhend @/
servaddr.sin_addr.s_addr = htonl(INADDR_ANY);
\end{clst}

% subsubsection 整数字节序 (end)

\subsubsection{字符串IP} % (fold)
\label{ssub:字符串ip}

\begin{clst}
/@\cinclude{arpa/inet.h}@/
/* 呈现形式到数值: 字符串到IP地址格式 */
int inet_pton(int af, const char *src, void *dst);
/* 成功时返回1, 出错时返回0或-1 */
/@\lhend @/
struct /@\+c{sockaddr\_in}@/ servaddr;
if (inet_pton(AF_INET, argv[1], &servaddr) <= 0) /* error */;
\end{clst}

% subsubsection 字符串ip (end)

% subsection 格式转化 (end)

% section 套接字 (end)

\end{document}
