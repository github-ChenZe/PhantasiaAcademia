\documentclass[hidelinks]{ctexart}

\usepackage[sensei=ない,gakka=計算化学,section=Kiso,gakkabbr=CC]{styles/kurisu}
\usepackage{van-de-la-illinoise}
\usepackage{chemfig}

\DeclareSIUnit\bohr{Bohr}

\begin{document}

\section{基础} % (fold)
\label{sec:基础}

\mathsubsection{Huckel}{H\"uckel的相关工作}{H\"uckel的相关工作}{Huckel的相关工作} % (fold)
\label{sub:huckel的相关工作}

\subsubsection{杂化} % (fold)
\label{ssub:杂化}

$\mathrm{2p}$轨道和$\mathrm{2s}$轨道可以按如下方式发生$\mathrm{sp^3}$杂化(\gloss{\mbox{$\mathrm{sp^3}$ hybridization}}):
\begin{align*}
    \phi_1 &= \half \pare{\mathrm{s} + \mathrm{p}_x + \mathrm{p}_y + \mathrm{p}_z}, \\
    \phi_2 &= \half \pare{\mathrm{s} + \mathrm{p}_x + \mathrm{p}_y - \mathrm{p}_z}, \\
    \phi_3 &= \half \pare{\mathrm{s} + \mathrm{p}_x - \mathrm{p}_y - \mathrm{p}_z}, \\
    \phi_4 &= \half \pare{\mathrm{s} + \mathrm{p}_x - \mathrm{p}_y + \mathrm{p}_z}.
\end{align*}
或者
\begin{align*}
    \phi_1 &= \half \pare{\mathrm{s} + \mathrm{p}_x + \sqrt{2}\mathrm{p}_z}, \\
    \phi_2 &= \half \pare{\mathrm{s} + \mathrm{p}_x - \sqrt{2}\mathrm{p}_z}, \\
    \phi_3 &= \half \pare{\mathrm{s} - \mathrm{p}_x + \sqrt{2}\mathrm{p}_y}, \\
    \phi_4 &= \half \pare{\mathrm{s} - \mathrm{p}_x - \sqrt{2}\mathrm{p}_y}.
\end{align*}
两种情形下$\mathrm{s}$和$\mathrm{p}$轨道的有效成分比皆为$1:3$. 实际上, 尽管甲烷具有正四面体对称性, 其能量仍非四重简并, 而是一重和三重简并.

% subsubsection 杂化 (end)

\mathsubsubsection{SHM}{简单H\"uckel理论}{简单H\"uckel理论}{简单Huckel理论} % (fold)
\label{ssub:简单Huckel理论}

简单H\"uckel理论(\gloss[-\baselineskip]{\mbox{SHM, SHT, HMO}})将$\psi$视为原子轨道的线性组合(\gloss[\baselineskip]{LCAO}). 在这一假设下分子轨道(\gloss[1\baselineskip]{MO})可以视为基函数(\gloss[1\baselineskip]{\mbox{basis function}})的线性组合. 基函数构成一组基(\gloss[2\baselineskip]{basis set}). 例如$\mathrm{p}$AO可以结合成$\pi$AO.
\par
\gloss[2\baselineskip]{HOMO}表示Highest Occupied Molecular Orbital. \gloss[3\baselineskip]{LUMO}表示Lowest Unoccupied Molecular Orbital.
\begin{remark}
    实际上基不一定是相应原子的原子轨道.
\end{remark}
$n$个基函数可结合成$n$个分子轨道. 对于两个分子轨道的情形, 设
\[ \psi = c_1 \phi_1 + c_2 \phi_2, \]
即有
\[ E = \frac{\bra{c_1\phi_1 + c_2 \phi_2}H\ket{c_1 \phi_1 + c_2 \phi_2}}{\bra{c_1\phi_1 + c_2 \phi_2}\ket{c_1 \phi_1 + c_2 \phi_2}} = \frac{c_1^2 H_{11} + 2c_1c_2 H_{12} + c_2^2 H_{22}}{c_1^2 S_{11} + 2c_1c_2 S_{12} + c_2^2 S_{22}}. \]
取对于$c_1$的偏导数,
\begin{align*}
    &\pare{\+D{c_1}D{E}}\pare{c_1^2 S_{11} + 2c_1c_2 S_{12} + c_2^2 S_{22}} + E\pare{2c_1 S_{11} + 2c_2 S_{12}} = 2c_1H_{11} = 2c_2 H_{12}, \\
    & \xLongrightarrow{\partial E/\partial c_1 = 0} \pare{H_{11} - ES_{11}}c_1 + \pare{H_{12} - E S_{12}}c_2 = 0.
\end{align*}
类似取对$c_2$的偏导数,
\[ \pare{H_{21} - ES_{21}}c_1 = \pare{H_{22} - ES_{22}}c_2 = 0. \]
可得联立方程
\[ \begin{cases}
    \pare{H_{11} - ES_{11}}c_1 + \pare{H_{12} - ES_{12}}c_2 = 0, \\
    \pare{H_{21} - ES_{21}}c_1 + \pare{H_{22} - ES_{22}}c_2 = 0.
\end{cases} \]
这是久期方程,
\[ \brac{\+vH - E\+vS}\+vc = 0 \Leftrightarrow \+vH\+vc = E\+vS\+vc. \]
完整的特征系统具有如下的形式:
\begin{align*}
    & \+vH \+vC = \+vS\+vC\+v\epsilon.\\
    & \+vH = \begin{pmatrix}
        H_{11} & H_{12} \\
        H_{21} & H_{22}
    \end{pmatrix},\quad
    \+vC = \begin{pmatrix}
        c_{11} & c_{12} \\
        c_{12} & c_{22}
    \end{pmatrix},\\
    & \+vS = \begin{pmatrix}
        S_{11} & S_{12} \\
        S_{21} & S_{22}
    \end{pmatrix},\quad
    \+v\epsilon = \begin{pmatrix}
        \epsilon_1 & 0 \\
        0 & \epsilon_2
    \end{pmatrix}.
\end{align*}
其中$\+vH$谓Fock矩阵(\gloss{Fock matrix}). $\+vC$谓系数矩阵(\gloss{\mbox{coefficients matrix}}), 各列为一组系数. $\+vS$谓重叠矩阵(\gloss{\mbox{overlap matrix}}). $\epsilon_i$是相应的轨道能级.
\par
对于$S_{ij} = \delta_{ij}$之情形,
\[ \+vH\+vC = \+vC\+v\epsilon \Rightarrow \+vH = \+vC\epsilon \+vC^{-1}. \]
定义Coulomb积分(\gloss[-\baselineskip]{\mbox{Coulomb integral}})$\alpha$和共振积分(\gloss{\mbox{resonance integral}})如下:
\[ \alpha = H_{ii} = \bra{\phi_i}H\ket{\phi_i},\quad \beta = H_{ij} = \bra{\phi_i}H\ket{\phi_j}. \]
前者对于同一个原子上的基函数定义, 后者对于相邻原子上的两个基函数定义. 对于并非同一分子或相邻分子上的基函数, 定义
\[ \int \phi_i \hat H \phi_j \,\rd{v} = H_{ij} = \int \phi_j \hat H \phi_i \,\rd{v} = H_{ji} = 0. \]
$\alpha$相应的能量零点应当解释为电子和基函数(轨道)相距无穷远的情形. $\beta$是重叠区域内的电子能量, 能量零点应当解释为电子和该二中心轨道相聚无穷远的情形. 设$\beta = -1$, $\alpha = 0$, 则乙烯相应的矩阵为
\[ \+vH = \begin{pmatrix}
    0 & -1 \\
    -1 & 0
\end{pmatrix}. \]
对角化可得
\[ \+vc_1 = \begin{pmatrix}
    1/\sqrt{2} \\
    1/\sqrt{2}
\end{pmatrix},\quad \epsilon_1 = -1,\quad \+vc_2 = \begin{pmatrix}
    1/\sqrt{2} \\
    -1/\sqrt{2}
\end{pmatrix},\quad \epsilon_2 = 1. \]
分别是成键轨道与反键轨道.
\par
丙烯\marginnote{\parbox{3cm}{\centerline{\chemfig{C=[::30]C-[::-60]C}}\centerline{丙烯}}}相应的矩阵为
\[ \+vH = \begin{pmatrix}
    0 & -1 & 0 \\
    -1 & 0 & -1 \\
    0 & -1 & 0
\end{pmatrix} \]
对角化可得
\[ \+vc_1 = \begin{pmatrix}
    1/2 \\
    1/\sqrt{2} \\
    1/2
\end{pmatrix},\quad \epsilon_1 = -\sqrt{2},\quad \+vc_2 = \begin{pmatrix}
    1/\sqrt{2} \\
    -1/\sqrt{2}
\end{pmatrix},\quad \epsilon_2 = 0,\quad \+vc_3 = \begin{pmatrix}
    1/2 \\
    -1/\sqrt{2} \\
    1/2
\end{pmatrix},\quad \epsilon_3 = \sqrt{2}. \]
分别是成键轨道, 非键轨道与反键轨道.
\par
环丁二烯\marginnote{\parbox{3cm}{\centerline{\chemfig{C*4(-C=[90]C-[180]C=)}}\centerline{环丁二烯}}}相应的矩阵为
\[ \+vH = \begin{pmatrix}
    0 & -1 & 0 & -1 \\
    -1 & 0 & -1 & 0 \\
    0 & -1 & 0 & -1 \\
    -1 & 0 & -1 & 0
\end{pmatrix}. \]
对角化可得
\[ \+vc_1 = \begin{pmatrix}
    1/2 \\
    \pm 1/2 \\
    1/2 \\
    \pm 1/2
\end{pmatrix},\quad \epsilon_1 = \mp 2,\quad \+vc_2 = \begin{pmatrix}
    1/2 \\
    \mp 1/2 \\
    -1/2 \\
    \pm 1/2
\end{pmatrix},\quad \epsilon_2 = 0. \]
分别是成键轨道, 反键轨道和两个非键轨道.

% subsubsection 简单Huckel理论 (end)

\mathsubsubsection{SHMApplication}{应用}{简单H\"uckel理论的应用}{简单Huckel理论的应用} % (fold)
\label{ssub:简单Huckel理论的应用}

两个原子之间的键级谓
\[ B_{i,j} = 1 + \sum nc_i c_j. \]
其中$1$由$\sigma$键贡献, 剩下的项由$\pi$键贡献. 求和遍历被占据的轨道.
\begin{ex}
    乙烯有$c_1 = c_2 = 1/\sqrt{2}$,
    \[ B_{i,j} = 1 + 2\rec{\sqrt{2}}\rec{\sqrt{2}} = 2. \]
\end{ex}
\begin{ex}
    乙烯基阴离子的成键轨道有$c_1 = c_2 = 1/\sqrt{2}$, 反键轨道有$c_1 = -c_2 = 1/\sqrt{2}$
    \[ B_{i,j} = 1 + 2\rec{\sqrt{2}}\rec{\sqrt{2}} - \rec{\sqrt{2}}\rec{\sqrt{2}} = 1.5. \]
\end{ex}
\begin{pitfall}
    H\"uckel理论仅适用于处理$\mathrm{p}$轨道.
\end{pitfall}
一般的SHM需要求解如下问题:
\[ \begin{vmatrix}
    H_{11} - ES_{11} & H_{12} - ES_{12} & \cdots & H_{1n} - ES_{1n} \\
    H_{21} - ES_{21} & H_{22} - ES_{22} & \cdots & H_{2n} - ES_{2n} \\
    \vdots & \vdots & \ddots & \vdots \\
    H_{n1} - ES_{n1} & H_{n2} - ES_{n2} & \cdots & H_{nn} - ES_{nn}
\end{vmatrix} = 0. \]
在$S_{ij} = \delta_{ij}$之近似下变为
\[ \begin{vmatrix}
    x & 1 & \cdots & 0 \\
    1 & x & \cdots & 0 \\
    \vdots & \vdots & \ddots & \vdots \\
    0 & 0 & \cdots & x
\end{vmatrix} = 0,\quad x = \frac{\alpha - E}{\beta}. \]

% subsubsection 简单Huckel理论的应用 (end)

% subsection huckel的相关工作 (end)

\mathsubsection{HuckelE}{H\"uckel方法的拓展}{H\"uckel方法的拓展}{Huckel方法的拓展} % (fold)
\label{sub:huckel方法的拓展}

\subsubsection{理论} % (fold)
\label{ssub:理论}

拓展的H\"uckel方法(\gloss{EHM})不仅考虑了$\mathrm{p}$轨道, 而且也将$\mathrm{s}$轨道纳入基函数, 且各个矩阵元都被数值算出. 通常取
\[ \bra{\phi_i}\hat H\ket{\phi_i} = -I_i,\quad \bra{\phi_i}\hat H\ket{\phi_j} = -\half KS_{ij}\pare{I_i + I_j}. \]
其中$I_i$是相应原子的电离能. $K\approx 2$. 使用Slater基函数
\[ \phi\pare{\mathrm{1s}} = \sqrt{\frac{\zeta_1^3}{\pi}} \exp\pare{-\zeta_1\abs{\+vr - \+vR\+_1s_}},\quad \phi\pare{\mathrm{2s}} = \sqrt{\frac{\zeta_2^5}{96\pi}} \abs{\+vr - \+vR\+_2s_} \exp\pare{-\frac{\zeta_2 \abs{\+vr - \+vR\+_2s_}}{2}}. \]
使用正交基函数简化本征值问题, 即可有
\[ \+vH \+vC = \+vS\+vC\+v\epsilon \rightarrow \+vH'\+vC' = \+vC'\+v\epsilon. \]
设$\+vC' = \+vS^{1/2}\+vC$, $\+vH' = \+vS^{-1/2}\+vH\+vS^{-1/2}$即可. 此时即有
\[ \+vH' = \+vC' \+v\epsilon \+vC'^{-1}. \]
\begin{resume}
    EHM之一般步骤如下:
    \begin{cenum}
        \item 分子的几何构型作为输入.
        \item 计算重叠矩阵$\+vS$.
        \item 计算Fock矩阵$\bra{\phi_i}\hat H\ket{\phi_j}$. 可以参照电离能与参量$K$.
        \item 将$\+vS$对角化, 并算出$\+vS^{1/2}$.
        \item 计算相应的$\+vH'$.
        \item 将$\+vH'$对角化得到$\+vC'$和$\+v\epsilon$.
        \item 将$\+vC'$变换到原先的基下的表示$\+vC$.
    \end{cenum}
\end{resume}

% subsubsection 理论 (end)

\subsubsection{氦氢分子离子} % (fold)
\label{ssub:氦氢分子离子}

考虑 \chemfig{He-H^+} 分子.

\paragraph{结构} % (fold)
\label{par:结构}

取键长$\SI{0.8}{\angstrom}$, 即$\mathrm{H}_1\pare{0,0,0}$, $\mathrm{He}_2\pare{0,0,0.800}$.

% paragraph 结构 (end)

\paragraph{重叠积分} % (fold)
\label{par:重叠积分}

取Slater基
\[ \phi_1\pare{\mathrm{H}\+_1s_} = \sqrt{\frac{\zeta\+_H_^3}{\pi}} \exp \pare{-\zeta\+_H_\abs{\+vr - \+vR\+_H_}} \]
以及
\[ \phi_2\pare{\mathrm{He}\+_1s_} = \sqrt{\frac{\zeta\+_He_^3}{\pi}} \exp \pare{-\zeta\+_H_\abs{\+vr - \+vR\+_He_}}. \]
取$\zeta\+_H_ = \SI{1.24}{\per\bohr}$, $\zeta\+_He_ = \SI{2.0925}{\per\bohr}$, 则
\[ \+vS = \begin{pmatrix}
    1 & 0.435 \\
    0.435 & 1
\end{pmatrix}. \]

% paragraph 重叠积分 (end)

\paragraph{Fock矩阵} % (fold)
\label{par:fock矩阵}

取$K = 1.75$, $I\+_H_ = I_1 = \SI{13.6}{\eV}$, $I\+_He_ = I_2 = \SI{24.6}{\eV}$, 即有
\[ H = \begin{pmatrix}
    -13.6 & -14.5 \\
    -14.5 & -24.6
\end{pmatrix}. \]


% paragraph fock矩阵 (end)

\paragraph{正交化} % (fold)
\label{par:正交化}

$\displaystyle \+vS^{1/2} = \begin{pmatrix}
    1.083 & -0.248 \\
    -0.248 & 1.083
\end{pmatrix}$.

% paragraph 正交化 (end)

\paragraph{Fock矩阵变换} % (fold)
\label{par:fock矩阵变换}

$\displaystyle \+vH' = \+vS^{-1/2}\+vH\+vS^{-1/2} = \begin{pmatrix}
    -9.67 & -7.65 \\
    -7.68 & -21.74
\end{pmatrix}$.

% paragraph fock矩阵变换 (end)

\paragraph{Fock矩阵对角化} % (fold)
\label{par:fock矩阵对角化}

$\displaystyle \+vH' = \+vC' \+v\epsilon \+vC'^{-1} = \begin{pmatrix}
    0.436 & 0.899 \\
    0.900 & -0.437
\end{pmatrix}\begin{pmatrix}
    -25.5 \\
    & -5.95
\end{pmatrix}\begin{pmatrix}
    0.436 & 0.900 \\
    0.899 & -0.437
\end{pmatrix}$.

% paragraph fock矩阵对角化 (end)

\paragraph{系数矩阵变换} % (fold)
\label{par:系数矩阵变换}

$\displaystyle \+vC = \+vS^{-1/2} \+vC' = \begin{pmatrix}
    0.249 & 1.082 \\
    0.867 & -0.696
\end{pmatrix} = \begin{pmatrix}
    c_{11} & c_{12} \\
    c_{21} & c_{22}
\end{pmatrix}$. 相应的轨道为
\[ \psi_1 = c_{11}\phi_1 + c_{21}\phi_2,\quad \psi_2 = c_{12}\phi_1 + c_{22}\phi_2. \]

\begin{pitfall}
    SHM和EHM皆未考虑电子-电子排斥作用和电子自旋的相互作用.
\end{pitfall}

% paragraph 系数矩阵变换 (end)

% subsubsection 氦氢分子离子 (end)

% subsection huckel方法的拓展 (end)

% section 基础 (end)

\end{document}
