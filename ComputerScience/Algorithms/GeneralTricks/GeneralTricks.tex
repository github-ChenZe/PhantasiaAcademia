\documentclass{ctexart}

\usepackage[singleton, margintoc, nova]{van-de-la-sehen}

\newtcbox{\codephrase}{enhanced,nobeforeafter,tcbox raise base,boxrule=0.4pt,top=0mm,bottom=0mm,
  right=0mm,left=0.15mm,arc=1pt,boxsep=2pt,fontupper=\ttfamily,
  colframe=gray,coltext=gray!25!black,colback=gray!10!white
}

\robustify{\codephrase}

\begin{document}

\showtitle{算法}

\section{一般算法} % (fold)
\label{sec:一般算法}

\subsection{位运算} % (fold)
\label{sub:位运算}

\subsubsection{二进制序列中1的个数} % (fold)
\label{ssub:二进制序列中1的个数}

\paragraph{Brian Kernighan算法} % (fold)
\label{par:brian_kernighan算法}

参考\href{https://www.geeksforgeeks.org/count-set-bits-in-an-integer/}{统计二进制序列中1的个数}, 其正确性考虑到如下事实: \emph{\codephrase{n-1}将\codephrase{n}的最低位1置零后在其后填充1, 因此\codephrase{n\&=n-1}恰好减少\codephrase{n}的二进制表示中的一个1.} 这一算法之时间复杂度为$O\pare{n_1}$, 其中$n_1$为1的个数.
\begin{cpplst}
int count_ones(unsigned int n)
{
    return n ? count_ones(n & (n - 1)) : 0;
}
\end{cpplst}

% paragraph brian_kernighan算法 (end)

\paragraph{1计数的分治算法} % (fold)
\label{par:1计数的分治算法}

注意如下事实: \emph{\codephrase{n}中1的个数, 等于其前半部分1的个数加上后半部分1的个数.} 惟以递归方法实现时, 复杂度$\displaystyle T\pare{n} = 2T\pare{\frac{n}{2}} + C$为线性复杂度, 仍有改进之余地.
\par
如果\codephrase{count}的前半部分表示和后半部分分别表示\codephrase{n}的前半部分和后半部分的1的个数, 而\codephrase{mask=0...01...1}与\codephrase{n}等长且长度为\codephrase{2l}, 则\\\centerline{\codephrase{((i >> l) \& mask) + (i \& mask)}}\\可得\codephrase{n}中1的个数. 自底向上实现即可. 其时间复杂度为$O\pare{\log\log n}$.
\begin{cpplst}
#define MASK(space) ((unsigned int) -1 / ((1 << (space)) + 1))
// MASK(1) = 0b01010101010101010101010101010101
// MASK(2) = 0b00110011001100110011001100110011

#define MERGE(n, space) \
    ((n & MASK(space)) + ((n >> space) & MASK(space)))

int count_ones(unsigned int n)
{
    n = MERGE(n, 1);
    n = MERGE(n, 2);
    n = MERGE(n, 4);
    n = MERGE(n, 8);
    n = MERGE(n, 16);
    return n;
}
\end{cpplst}

% paragraph 1计数的分治算法 (end)

% subsubsection 二进制序列中1的个数 (end)

\subsubsection{唯一未重复} % (fold)
\label{ssub:唯一未重复}

设序列中除一个仅出现一次的元素外其他元素都出现两次, 求出这个仅出现一次的元素. 考虑到\codephrase{x\^{}y}是交换且结合且自反的, 知序列全体元素之\codephrase{xor}的结果为所求.

\begin{cpplst}
int find1_non_dup(int* arr, int n)
{
    return n ? *arr ^ find1_non_dup(arr + 1, n - 1) : 0;
}
\end{cpplst}

% subsubsection 唯一未重复 (end)

\subsubsection{仅有两个未重复} % (fold)
\label{ssub:仅有两个未重复}

设序列中除两个出现一次的元素外其他元素都出现两次, 求出这两个仅出现一次的元素. 设这两个仅出现一次元素为\codephrase{x}和\codephrase{y}, 则\codephrase{find1\_non\_dup}之最终结果为\codephrase{z=x\^{}y}. \codephrase{z}非零, 故存在最低位1,\\
%\centerline{\codephrase{u = ((z\^{}(z-1)) >> 1) + 1}}
\centerline{\codephrase{u = z \& (z \^{} (z - 1))}}
将这一位提取出来. 对序列中与\codephrase{u}有共同非零位之全体取\codephrase{xor}可得其中一者, 对无共同非零位者得另一者.

\begin{cpplst}
void find2_non_dup(int* arr, int n, int* non_dup_pair)
{
    non_dup_pair[0] = non_dup_pair[1] = 0;
    int pair_xor = find1_non_dup(arr, n);
    int xor_lowest = pair_xor & (pair_xor ^ (pair_xor - 1));
    for (int i = 0; i < n; i++)
        non_dup_pair[!(arr[i] & xor_lowest)] ^= arr[i];
}
\end{cpplst}

% subsubsection 仅有两个未重复 (end)

% subsection 位运算 (end)

\subsection{序列} % (fold)
\label{sub:序列}

\subsubsection{最长上升子序列} % (fold)
\label{ssub:最长上升子序列}

\paragraph{单纯求出长度} % (fold)
\label{par:单纯求出长度}

设有序列$a_0,\cdots, a_{N-1}$, 则如下命题成立: \emph{如果$a_{n_0}, \cdots, a_{n_i}$是以$a_{n_i}$结尾的最长上升子序列, 则$a_{n_0}, \cdots, a_{n_{i-1}}$是以$a_{n_{i-1}}$结尾的最长上升子序列.} 如果数组$l_n$标记以$a_n$结尾的最长上升子序列, 则$l_n$中的最大值就最长上升子序列的长度.

\begin{cpplst}
int length_of_longest_ascending_subsequence(int* arr, int n)
{
    int len_partial[n];
    for (int i = 0; i < n; i++)
        len_partial[i] = 1;
    for (int i = 0; i < n; i++)
    {
        for (int j = 0; j < i; j++)
        {
            if (arr[j] > arr[i]) continue;
            len_partial[i] = max(len_partial[i],
                len_partial[j] + 1);
        }
    }
    return max_in(len_partial, n);
}
\end{cpplst}

% paragraph 单纯求出长度 (end)

% subsubsection 最长上升子序列 (end)

\subsubsection{三目等差数列计数} % (fold)
\label{ssub:三目等差数列计数}

给定$\curb{1,\cdots,n}$的一个子集$S$, 求其长度为$3$的等差数列的个数, \href{https://www.quora.com/How-can-you-count-the-number-of-3-term-arithmetic-progressions-in-a-subset-of-1-n-using-an-algorithm-which-runs-in-O-n-log-n-time/answer/Alon-Amit?ch=99&share=eebd96a1&srid=hX9A9}{可以用FFT解决}.
\par
这一问题转化为对$\curb{1,\cdots,n}$中的所有$a$求$x+y=2a$在$S$中的解的数目. 设$f\pare{k}\in \curb{0,1}$标记$k$是否在$S$中, 则问题转化为求卷积$\displaystyle \sum_k f\pare{k}f\pare{m-k}$, 其中$m$应取遍$\curb{1,\cdots,2n}$.
\par
构造$\displaystyle \pare{f\pare{1},\cdots,f\pare{n}, \underbrace{0,\cdots,0}_{n\text{个}}}$, 取其DFT$\displaystyle \pare{\hat f\pare{1},\cdots, \hat f\pare{2n}}$, 逐项平方, 得到$\displaystyle \pare{\hat f\pare{1}^2,\cdots, \hat f\pare{2n}^2}$, 取其逆DFT, 得到$\displaystyle \pare{g\pare{1},\cdots,g\pare{2n}}$, $g\pare{k}$即$x+y=k$在$S$中的解数. $\displaystyle \sum_{k\in S}\pare{g\pare{k}-1}$即为等差数列之数目.

% subsubsection 三目等差数列计数 (end)

\subsubsection{马鞍查找} % (fold)
\label{ssub:马鞍查找}

若数组\codephrase{A[n][n]}各行各列皆为升序, 则可在$O\pare{r+s}$的时间内完成对所有$x$的查找, 其中$r$和$s$分别为\codephrase{A[0][:]}和\codephrase{A[:][0]}中小于$x$的元素的数目.
\par
从\codephrase{A[i][j] = A[0][r-1]}开始搜索, 搜索范围为\codephrase{A[:s][:r]}, 每次可以移除一行或一列, 从而\codephrase{i}或\codephrase{j}视\codephrase{A[i][j] <=> x}每次可步进\codephrase{1}.

% subsubsection 马鞍查找 (end)

% subsection 序列 (end)

\subsection{集合} % (fold)
\label{sub:集合}

\href{https://en.wikipedia.org/wiki/Disjoint-set_data_structure}{并查集算法.}

% subsection 集合 (end)

% section 一般算法 (end)

\end{document}
