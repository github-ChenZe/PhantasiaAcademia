\documentclass[hidelinks]{ctexart}

\usepackage{van-de-la-illinoise}

\newtcbox{\codephrase}{enhanced,nobeforeafter,tcbox raise base,boxrule=0.4pt,top=0mm,bottom=0mm,
  right=0mm,left=0.15mm,arc=1pt,boxsep=2pt,fontupper=\ttfamily,
  colframe=gray,coltext=gray!25!black,colback=gray!10!white
}

\robustify{\codephrase}

\begin{document}

\paragraph{栈混洗} % (fold)
\label{par:栈混洗}

用尖括号和方括号分别表示栈顶和栈底, 设有栈\codephrase{A},\\ 
\centerline{\codephrase{A = < a\textsubscript{1}, a\textsubscript{2}, ..., a\textsubscript{n} ]}},
\codephrase{B}和\codephrase{S}, 其中\codephrase{B}和\codephrase{S}初始为空, 每次允许\\
\centerline{\codephrase{S.push(A.pop())}}
或\\
\centerline{\codephrase{B.push(S.pop())},}
最终将\codephrase{A}和\codephrase{S}皆清空, \codephrase{A}中的元素悉数转移至\codephrase{B}, 得到\\
\centerline{\codephrase{B = [ a\textsubscript{k1}, a\textsubscript{k2}, ..., a\textsubscript{kn} >}},
谓该序列为原序列之栈混洗.

% paragraph 栈混洗 (end)

\end{document}
