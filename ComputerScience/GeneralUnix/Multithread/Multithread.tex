\documentclass[hidelinks]{ctexart}

\usepackage[margintoc, singleton, nova]{van-de-la-sehen}

\newtcbox{\mylib}{enhanced,nobeforeafter,tcbox raise base,boxrule=0.4pt,top=0mm,bottom=0mm,
  right=0mm,left=0.15mm,arc=1pt,boxsep=2pt,fontupper=\ttfamily,
  colframe=gray,coltext=gray!25!black,colback=gray!10!white
}

\robustify{\mylib}
\newcommand{\mcode}[1]{\ \mylib{#1} \ }

\begin{document}

\showtitle{多线程编程}

\section{线程API} % (fold)
\label{sec:线程api}

\subsection{线程结构} % (fold)
\label{sub:线程结构}

\subsubsection{线程标识} % (fold)
\label{ssub:线程标识}

\begin{clst}
/@\cinclude{pthread.h}@/
int pthread_equal(/@\+c{pthread\_t}@/ tid1, /@\+c{pthread\_t}@/ tid2);
/* 线程ID相等时返回非零值, 否则返回零 */
\end{clst}

\begin{clst}
/@\cinclude{pthread.h}@/
/* 返回当前线程ID */
/@\+c{pthread\_t}@/ pthread_self(void);
/@\lhend @/
printf("thread id is %lx.", (unsigned long) pthread_self());
\end{clst}

\mcode{pthread\_t}的具体类型是平台相关的.

% subsubsection 线程标识 (end)

\subsubsection{分离状态} % (fold)
\label{ssub:分离状态}

线程若处于分离状态, 则线程的底层资源会在线程终止时立即被收回. 默认情形下, 线程的终止状态会一直被保存, 直到被分离.

\begin{clst}
/@\cinclude{pthread.h}@/
/* 分离线程 */
int pthread_detach(/@\+c{pthread\_t}@/ tid);
/* 成功时返回0, 否则返回错误代码 */
\end{clst}

% subsubsection 分离状态 (end)

\subsection{线程操作} % (fold)
\label{sub:线程操作}

\subsubsection{线程创建} % (fold)
\label{ssub:线程创建}

\begin{clst}
/@\cinclude{pthread.h}@/
int pthread_create(/@\+c{pthread\_t}@/ *thread, const /@\+c{pthread\_attr\_t}@/ *attr,
    void *(*/@\+c{start\_routine}@/) (void *), void *arg);
/* 成功时返回0, 否则返回错误代码 */
/@\lhend @/
/@\+c{pthread\_t}@/ ntid;
void* thr_fn(void* arg);
err = pthread_create(&ntid, NULL, thr_fn, NULL);
\end{clst}
\mcode{attr}可以为\mcode{NULL}.

% subsubsection 线程创建 (end)

\subsubsection{线程退出} % (fold)
\label{ssub:线程退出}

\begin{clst}
/@\cinclude{pthread.h}@/
void pthread_exit(void* rval_ptr);
/@\lhend @/
void* thr_fn(void* arg)
{
    pthread_exit((void*) 1);
}
\end{clst}
\begin{clst}
/@\cinclude{pthread.h}@/
int pthread_join(/@\+c{pthread\_t}@/ thread, void** rval_ptr);
/* 成功时返回0, 否则返回错误代码 */
/@\lhend @/
void* thr_fn(void* arg);
/@\+c{pthread\_t}@/ tid;
void* tret;
err = pthread_join(tid, tret);
\end{clst}
\mcode{pthread\_join}使线程进入分离状态.\\对已经分离的线程调用\mcode{pthread\_join}会产生未定义行为.
\begin{clst}
/@\cinclude{pthread.h}@/
int pthread_cancel(/@\+c{pthread\_t}@/ tid);
/* 成功时返回0, 否则返回错误代码 */
\end{clst}
\mcode{pthread\_exit}由线程自行调用以主动退出, \mcode{pthread\_join}由外部线程调用等待其退出.
\par
\mcode{rval\_ptr}可以为\mcode{NULL}.
\par
如下机制都可以退出线程:
\begin{cenum}
    \item \mcode{return (void*) 8};
    \item \mcode{pthread\_cancel(...)};
    \item \mcode{pthread\_exit(...)}.
\end{cenum}

% subsubsection 线程退出 (end)

\subsubsection{退出回调} % (fold)
\label{ssub:退出回调}

\begin{clst}
/@\cinclude{pthread.h}@/
void pthread_cleanup_push(void (*rtn)(void*), void* arg);
/@\lhend @/
void cleanup(void* arg);
pthread_cleanup_push(cleanup, "thread 1 first handler");
pthread_cleanup_pop(0);
\end{clst}

\begin{clst}
/@\cinclude{pthread.h}@/
void pthread_cleanup_pop(int execute);
\end{clst}

回调函数会在如下条件下被触发:
\begin{cenum}
    \item 调用\mcode{pthread\_exit(...)};
    \item 相应取消请求;
    \item 用非零的\mcode{execute}调用\mcode{pthread\_cleanup\_pop};
\end{cenum}

\begin{pitfall}
    回调程序需要注意
    \begin{cenum}
        \item 线程通过\mcode{return}返回时不会触发回调函数;
        \item 即使已经调用\mcode{pthread\_cleanup\_pop(...)},\\再\mcode{pthread\_exit(...)}时仍然会调用回调;
        \item 调用顺序与注册顺序相反;
        \item 在FreeBSD或者OSX上,\\如果在\mcode{pthread\_cleanup\_push(...)}和\\\mcode{pthread\_cleanup\_pop(...)}之间\mcode{return}, 可能出现错误, 因此必须使用\mcode{pthread\_exit(...)};
        \item \mcode{pthread\_cleanup\_push(...)}和\\\mcode{pthread\_cleanup\_pop(...)}必须配对出现.
    \end{cenum}
\end{pitfall}

% subsubsection 退出回调 (end)

% subsection 线程操作 (end)

% subsection 线程结构 (end)

% section 线程api (end)

\section{互斥量API} % (fold)
\label{sec:互斥量api}

\subsection{互斥量操作} % (fold)
\label{sub:互斥量操作}

\subsubsection{创建} % (fold)
\label{ssub:创建}

\begin{clst}
/@\cinclude{pthread.h}@/
int pthread_mutex_init(/@\+c{pthread\_mutex\_t}@/* restrict mutex,
    const /@\+c{pthread\_mutexattr\_t}@/* restrict attr);
/* 成功时返回0, 否则返回错误代码 */
/@\lhend @/
struct /@\+c{foo}@/
{
    int f_count;
    /@\+c{pthread\_mutex\_t}@/ f_lock;
};
struct /@\+c{foo}@/* fp;
if (pthread_mutex_init(&fp->f_lock, NULL) != 0)
    /* error */;
\end{clst}
\begin{clst}
/@\cinclude{pthread.h}@/
int pthread_mutex_destroy(/@\+c{pthread\_mutex\_t}@/* mutex);
/* 成功时返回0, 否则返回错误代码 */
/@\lhend @/
pthread_mutex_destroy(&fp->f_lock);
\end{clst}
\mcode{attr}可以为\mcode{NULL}.
\par
\mcode{PTHREAD\_MUTEX\_INITIALIZER}可以初始化静态分配的互斥量.

% subsubsection 创建 (end)

\subsubsection{上锁与解锁} % (fold)
\label{ssub:上锁与解锁}

\begin{clst}
/@\cinclude{pthread.h}@/
int pthread_mutex_lock(/@\+c{pthread\_mutex\_t}@/* mutex);
int pthread_mutex_trylock(/@\+c{pthread\_mutex\_t}@/* mutex);
int pthread_mutex_unlock(/@\+c{pthread\_mutex\_t}@/* mutex);
/* 成功时返回0, 否则返回错误代码 */
/@\lhend @/
pthread_mutex_lock(&fp->f_lock);
fp->f_count++;
pthread_mutex_unlock(&fp->f_lock);
\end{clst}
\mcode{pthread\_mutex\_trylock}会尝试锁住互斥量. 如果调用时其未锁住, 则返回零, 否则返回\mcode{EBUSY}.

\begin{pitfall}
    考虑通过维持固定的上锁顺序以防止死锁. 注意上锁顺序可能影响代码复杂度.
\end{pitfall}

% subsubsection 上锁与解锁 (end)

\subsubsection{限时加锁} % (fold)
\label{ssub:限时加锁}

\begin{clst}
/@\cinclude{pthread.h}@/
/@\cinclude{time.h}@/
int pthread_mutex_timedlock(/@\+c{pthread\_mutex\_t}@/* restrict mutex,
    const struct /@\+c{timespec}@/* restrict tsptr);
/* 成功时返回0, 否则返回错误代码 */
/@\lhend @/
struct /@\+c{timespec}@/ tout;
/@\+c{pthread\_mutex\_t}@/ lock = PTHREAD_MUTEX_INITIALIZER;
clock_gettime(CLOCK_REALTIME, &tout);
tout.tv_sec += 10;
err = pthread_mutex_timedclock(&lock, &tout);
\end{clst}
\mcode{tsptr}是愿意等待至的时刻. 超时将返回\mcode{ETIMEDOUT}.
\begin{pitfall}
    OSX尚不支持\mcode{pthread\_mutex\_t}.
\end{pitfall}

% subsubsection 限时加锁 (end)

% subsection 互斥量操作 (end)

\subsection{读写锁操作} % (fold)
\label{sub:读写锁操作}

\subsubsection{读写锁创建} % (fold)
\label{ssub:读写锁创建}

\begin{clst}
/@\cinclude{pthread.h}@/
int pthread_rwlock_init(/@\+c{pthread\_rwlock\_t}@/* restrict rwlock,
    const /@\+c{pthread\_rwlockattr\_t}@/* restrict attr);
/* 成功时返回0, 否则返回错误代码 */
\end{clst}
\begin{clst}
/@\cinclude{pthread.h}@/
int pthread_rwlock_destroy(/@\+c{pthread\_rwlock\_t}@/* rwlock);
/* 成功时返回0, 否则返回错误代码 */
\end{clst}
\mcode{attr}可以为\mcode{NULL}.
\par
\mcode{PTHREAD\_RWLOCK\_INITIALIZER}可以初始化静态分配的读写锁.

% subsubsection 读写锁创建 (end)

\subsubsection{读写上锁与解锁} % (fold)
\label{ssub:读写上锁与解锁}

\begin{clst}
/@\cinclude{pthread.h}@/
int pthread_rwlock_rdlock(/@\+c{pthread\_rwlock\_t}@/* rwlock);
int pthread_rwlock_tryrdlock(/@\+c{pthread\_rwlock\_t}@/* rwlock);
int pthread_rwlock_wrlock(/@\+c{pthread\_rwlock\_t}@/* rwlock);
int pthread_rwlock_trywrlock(/@\+c{pthread\_rwlock\_t}@/* rwlock);
int pthread_rwlock_unlock(/@\+c{pthread\_rwlock\_t}@/* rwlock);
/* 成功时返回0, 否则返回错误代码 */
\end{clst}
\mcode{pthread\_rwlock\_trylock}会尝试锁住互斥量. 如果调用时其未锁住, 则返回零, 否则返回\mcode{EBUSY}.
\par
读写锁可能处于三种状态:
\begin{cenum}
    \item 读加锁: 可以进行读加锁, 但写加锁阻塞;
    \item 写加锁: 阻塞任何加锁;
    \item 不加锁.
\end{cenum}

% subsubsection 读写上锁与解锁 (end)

\subsubsection{读写限时加锁} % (fold)
\label{ssub:读写限时加锁}

\begin{clst}
/@\cinclude{pthread.h}@/
/@\cinclude{time.h}@/
int pthread_rwlock_timedrdlock(/@\+c{pthread\_rwlock\_t}@/* restrict rwlock,
    const struct /@\+c{timespec}@/* restrict tsptr);
int pthread_rwlock_timedwrlock(/@\+c{pthread\_rwlock\_t}@/* restrict rwlock,
    const struct /@\+c{timespec}@/* restrict tsptr);
/* 成功时返回0, 否则返回错误代码 */
\end{clst}

% subsubsection 读写限时加锁 (end)

\subsection{自旋锁操作} % (fold)
\label{sub:自旋锁操作}

\subsubsection{自旋锁创建} % (fold)
\label{ssub:自旋锁创建}

自旋锁不会令等待的线程休眠, 而是使之自旋忙等.
\begin{clst}
/@\cinclude{pthread.h}@/
int pthread_spin_init(/@\+c{pthread\_spinlock\_t}@/* restrict lock,
    int pshared);
/* 成功时返回0, 否则返回错误代码 */
\end{clst}
\begin{clst}
/@\cinclude{pthread.h}@/
int pthread_spin_destroy(/@\+c{pthread\_spinlock\_t}@/* lock);
/* 成功时返回0, 否则返回错误代码 */
\end{clst}
\mcode{pshared}可以设定为\mcode{PTHREAD\_PROCESS\_SHARED}, 表示自旋锁能被访问底层锁内存的线程获取, 即使它们属于不同进程.\\或者\mcode{PTHREAD\_PROCESS\_PRIVATE}, 表示只能被进程内部的线程访问.

% subsubsection 自旋锁创建 (end)

\subsubsection{自旋上锁与解锁} % (fold)
\label{ssub:自旋上锁与解锁}

\begin{clst}
/@\cinclude{pthread.h}@/
int pthread_spin_lock(/@\+c{pthread\_spinlock\_t}@/* lock);
int pthread_spin_trylock(/@\+c{pthread\_spinlock\_t}@/* lock);
int pthread_spin_unlock(/@\+c{pthread\_spinlock\_t}@/* lock);
/* 成功时返回0, 否则返回错误代码 */
\end{clst}

% subsubsection 自旋上锁与解锁 (end)

% subsection 自旋锁操作 (end)

% subsection 读写锁操作 (end)

\subsection{条件变量} % (fold)
\label{sub:条件变量}

\subsubsection{变量创建} % (fold)
\label{ssub:变量创建}

所有条件需要由一互斥量保护.
\begin{clst}
/@\cinclude{pthread.h}@/
int pthread_cond_init(/@\+c{pthread\_cond\_t}@/* restrict cond,
    const /@\+c{pthread\_condattr\_t}@/* restrict attr);
/* 成功时返回0, 否则返回错误代码 */
\end{clst}
\begin{clst}
/@\cinclude{pthread.h}@/
int pthread_cond_destroy(/@\+c{pthread\_cond\_t}@/* cond);
/* 成功时返回0, 否则返回错误代码 */
\end{clst}
\mcode{attr}可以为\mcode{NULL}.
\par
\mcode{PTHREAD\_COND\_INITIALIZER}可以初始化静态分配的条件.

% subsubsection 变量创建 (end)

\subsubsection{等待条件} % (fold)
\label{ssub:等待条件}

\begin{clst}
/@\cinclude{pthread.h}@/
int pthread_cond_wait(/@\+c{pthread\_cond\_t}@/* restrict cond,
    /@\+c{pthread\_mutex\_t}@/* restrict mutex);
int pthread_cond_timedwait(/@\+c{pthread\_cond\_t}@/* restrict cond,
    /@\+c{pthread\_mutex\_t}@/* restrict mutex,
    const struct /@\+c{timespec}@/* restrict tsptr);
/@\lhend @/
/@\+c{pthread\_cond\_t}@/ qready;
/@\+c{pthread\_mutex\_t}@/ qlock;
pthread_mutex_lock(&qlock);
pthread_cond_wait(&qready, &qlock);
/* deque message */
pthread_mutex_unlock(&qlock);
/* process message */
\end{clst}
调用上述的等待函数后, 会发生
\begin{cenum}
    \item 线程休眠;
    \item 互斥量解锁;
\end{cenum}
直到条件被激活, 互斥量再次被锁住, 线程被激活.

% subsubsection 等待条件 (end)

\subsubsection{唤醒线程} % (fold)
\label{ssub:唤醒线程}

\begin{clst}
/@\cinclude{pthread.h}@/
int pthread_cond_signal(/@\+c{pthread\_cond\_t}@/* cond);
int pthread_cond_broadcast(/@\+c{pthread\_cond\_t}@/* cond);
/@\lhend @/
pthread_mutex_lock(&qlock);
/* enque message */
pthread_mutex_unlock(&qlock);
pthread_cond_signal(&qready);
\end{clst}
上述唤醒函数将激活因调用等待函数被挂起的线程.
\par
\mcode{pthread\_cond\_signal}唤醒至少一个线程,\\而\mcode{pthread\_cond\_broadcast}唤醒所有线程.

% subsubsection 唤醒线程 (end)

% subsection 条件变量 (end)

\subsection{屏障} % (fold)
\label{sub:屏障}

\subsubsection{屏障创建} % (fold)
\label{ssub:屏障创建}

\begin{clst}
/@\cinclude{pthread.h}@/
int pthread_barrier_init(/@\+c{pthread\_barrier\_t}@/* restrict barrier,
    const /@\+c{pthread\_barrierattr\_t}@/* restrict attr,
    unsigned int count);
/* 成功时返回0, 否则返回错误代码 */
/@\lhend @/
/@\+c{pthread\_barrier\_t}@/ b;
/* 主线程也参与屏障 */
pthread_barrier_init(&b, NULL, NTHR + 1);
/* 分多线程对数组分部排序 */
pthread_barrier_wait(&b);
/* 合并排序 */
\end{clst}
\begin{clst}
/@\cinclude{pthread.h}@/
int pthread_barrier_destroy(/@\+c{pthread\_barrier\_t}@/* barrier);
/* 成功时返回0, 否则返回错误代码 */
\end{clst}
\mcode{attr}可以为\mcode{NULL}.\\
\mcode{count}指定在允许所有线程继续运行之前, 必须到达屏障的线程数目.

% subsubsection 屏障创建 (end)

\subsubsection{屏障等待} % (fold)
\label{ssub:屏障等待}

\begin{clst}
/@\cinclude{pthread.h}@/
int pthread_wait_destroy(/@\+c{pthread\_barrier\_t}@/* barrier);
/* 成功时返回0或PTHREAD_BARRIER_SERIAL_THREAD, 否则返回错误代码 */
/@\lhend @/
/* 排序线程: 排序自己的部分 */
heapsort(...);
pthread_barrier_wait(&b);
\end{clst}
对其中一个线程会返回\mcode{PTHREAD\_BARRIER\_SERIAL\_THREAD}, 剩下的返回零.

% subsubsection 屏障等待 (end)

% subsection 屏障 (end)

% section 互斥量api (end)

\end{document}
