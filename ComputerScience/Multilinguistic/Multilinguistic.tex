\documentclass[hidelinks]{ctexart}

\usepackage[margintoc, singleton, epu, nova]{van-de-la-sehen}

\begin{document}

\showtitle{不同语言共有特性的对比}

\section{字符串处理} % (fold)
\label{sec:字符串处理}

\subsection{正则表达式} % (fold)
\label{sub:正则表达式}

\subsubsection{模式提取} % (fold)
\label{ssub:模式提取}

\begin{pylst}
import re

data = "id: 8, id: 42, id: 312"
regex = re.compile("id: ([0-9]+)")

it = re.finditer(regex, data)
for match in it:
    print match.group(1)
\end{pylst}
\begin{javalst}
import java.util.regex.*;

public class RegexMatch {
    public static void main(String[] args) {
        String s = "id: 8, id: 42, id: 312";
        Pattern pattern = Pattern.compile("id: ([0-9]+)");
        Matcher match = pattern.matcher(s);
        while (match.find()) {
            System.out.println(match.group(1));
        }
    }
}
\end{javalst}
\begin{jslst}
var re = @</id: ([0-9]+)/g>@;
var s = "id: 8, id: 42, id: 312";
var m;

do {
    m = re.exec(s);
    if (m) {
        console.log(m[1]);
    }
} while (m);
\end{jslst}
\begin{cshlst}
using System;
using System.Text.RegularExpressions;

namespace RegexTest
{
    class MainClass
    {
        public static void Main(string[] args)
        {
            string text = "id: 8, id: 42, id: 312";
            string search = "id: ([0-9]+)";
            MatchCollection matches = Regex.Matches(text, search);
            foreach (Match m in matches)
            {
                Console.WriteLine(m.Groups[1]);
            }
        }
    }
}
\end{cshlst}

% subsubsection 模式提取 (end)

% subsection 正则表达式 (end)

% section 字符串处理 (end)

\end{document}
