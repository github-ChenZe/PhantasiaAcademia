% !TEX options=--shell-escape
\documentclass[hidelinks]{ctexart}

\usepackage[margintoc, singleton, epu, nova]{van-de-la-sehen}
\usepackage{ifplatform}

\newtcbox{\mylib}{enhanced,nobeforeafter,tcbox raise base,boxrule=0.4pt,top=0mm,bottom=0mm,
  right=0mm,left=0.15mm,arc=1pt,boxsep=2pt,fontupper=\ttfamily,
  colframe=gray,coltext=gray!25!black,colback=gray!10!white
}

\robustify{\mylib}

\begin{document}

\showtitle{不同语言共有特性的对比}

\section{语法} % (fold)
\label{sec:语法}

\subsection{函数} % (fold)
\label{sub:函数}

\subsubsection{函数的定义} % (fold)
\label{ssub:函数的定义}

\begin{pylst}
def add(a: /@\+c{int}@/, b: /@\+c{int}@/) -> /@\+c{int}@/:
    return a + b
\end{pylst}
\begin{javalst}
int add(int a, int b) {
    return a + b;
}
\end{javalst}
\begin{jslst}
function add(a, b) {
    return a + b;
}
\end{jslst}
\begin{cshlst}
int add(int a, int b) {
    return a + b;
}
\end{cshlst}
\begin{shlst}
add()
{
    echo $((/@\lstparam{1}@/+/@\lstparam{2}@/))
}
\end{shlst}

% subsubsection 函数的定义 (end)

% subsection 函数 (end)

% section 语法 (end)

\section{字符串处理} % (fold)
\label{sec:字符串处理}

\subsection{字符串IO} % (fold)
\label{sub:字符串io}

\subsubsection{命令行读取} % (fold)
\label{ssub:命令行读取}

\begin{pylst}
line = input()
\end{pylst}
\begin{javalst}
/@\javaimport{java.util.*;}@/
/@\lhend @/
try (/@\+c{Scanner}@/ scanner = new /@\+c{Scanner}@/(System.in)) {
    /@\+c{String}@/ line = scanner.nextLine();
}
\end{javalst}
\begin{nodejslst}
/@\nodeimport{readline}{readline}@/
/@\lhend @/
const rl = readline.createInterface({
    input: process.stdin,
    output: process.stdout
});
rl.question('', (line) => {
    rl.close();
});

\end{nodejslst}
\begin{cshlst}
string line = /@\+c{Console}@/.ReadLine();
\end{cshlst}
\begin{shlst}
read LINE
\end{shlst}

% subsubsection 命令行读取 (end)

\subsubsection{格式化输出} % (fold)
\label{ssub:格式化输出}

\begin{pylst}
num1 = 6
num2 = 8
print("Numbers are %d and %d."%(num1, num2))
\end{pylst}
\begin{javalst}
int num1 = 6;
int num2 = 8;
System.out.format("Numbers are %d and %d.\n", num1, num2);
\end{javalst}
\begin{jslst}
var num1 = 6;
var num2 = 8;
console.log("Numbers are ${num1} and ${num2}.");
\end{jslst}
\begin{cshlst}
int num1 = 6;
int num2 = 8;
/@\+c{Console}@/.WriteLine($"Numbers are {num1} and {num2}.");
\end{cshlst}
\begin{shlst}
NUM1=6
NUM2=8
echo "Numbers are $NUM1 and $NUM2."
\end{shlst}
{\shname} 使用双引号在字符串内插入变量, 单引号无法达到此效果.

% subsubsection 格式化输出 (end)

% subsection 字符串io (end)

\subsection{简单处理} % (fold)
\label{sub:简单处理}

\subsubsection{子串} % (fold)
\label{ssub:子串}

\begin{longtable}{|c|c|c|c|c|c|}
    \hline
    语言 & \pyheader & \javaheader & \+:c2{c|}{\jsheader} & \cshheader \\
    \hline
    函数名 & \+`[]' & \+:c2{c|}{\+`substring'} & \+`substr' & \+`Substring' \\
    \hline
    单参数 & \+`[begin:]' & \+:c2{c|}{\+`(begin)'} & \+:c2{c|}{\+`(begin)'} \\
    \hline
    双参数 & \+`[begin: end]' & \+:c2{c|}{\+`(begin, end)'} & \+:c2{c|}{\+`(begin, length)'} \\
    \hline
\end{longtable}
{\pyname} 使用切片访问子串, 例如\mylib{\lstinline[language=python]!"apple"[2:4]!}给出\mylib{\lstinline[language=python]!"pl"!}. 此外, {\jsname} 的子串函数有两种(长度和终止).

% subsubsection 子串 (end)

\subsubsection{字符串开头} % (fold)
\label{ssub:字符串开头}

\begin{longtable}{|c|c|c|c|c|}
    \hline
    语言 & \pyheader & \javaheader & \jsheader & \cshheader \\
    \hline
    函数名 & \+`startswith' & \+:c2{c|}{\+`startsWith'} & \+`StartsWith' \\
    \hline
\end{longtable}

% subsubsection 字符串开头 (end)

\subsubsection{字符串包含} % (fold)
\label{ssub:字符串包含}

\begin{longtable}{|c|c|c|c|c|}
    \hline
    语言 & \pyheader & \javaheader & \jsheader & \cshheader \\
    \hline
    函数名 & \+`in' & \+`contains' & \+`includes' & \+`Contains' \\
    \hline
\end{longtable}

% subsubsection 字符串包含 (end)

\subsubsection{字符串查找} % (fold)
\label{ssub:字符串查找}

\begin{longtable}{|c|c|c|c|c|c|}
    \hline
    语言 & \pyheader & \javaheader & \jsheader & \cshheader & \cheader \\
    \hline
    函数名 & \+`index' & \+:c2{c|}{\+`indexOf'} & \+`IndexOf' & \+`strcspn' \\
    \hline
    不存在时 & 异常 & \+:c3{c|}{\+`return -1;'} & \+`strlen(\lstparam{1})' \\
    \hline
\end{longtable}

% subsubsection 字符串查找 (end)

% subsection 简单处理 (end)

\subsection{正则表达式} % (fold)
\label{sub:正则表达式}

\subsubsection{定义正则表达式} % (fold)
\label{ssub:定义正则表达式}

\begin{longtable}{|c|c|c|c|c|}
    \hline
    语言 & \pyheader & \javaheader & \jsheader \\
    \hline
    \+`regex' & \+`re.compile("re")' & \+`Pattern.compile("re")' & \+`/re/' \\
    \hline
    \+`matches/' & \+`re.finditer' & \+`pattern.matcher' & \\
    \+`matcher' & \+`(regex, text)' & \+`(text)' & \\
    \hline
\end{longtable}

% subsubsection 定义正则表达式 (end)

\subsubsection{模式判定} % (fold)
\label{ssub:模式判定}

\begin{longtable}{|c|c|c|c|c|}
    \hline
    语言 & \pyheader & \javaheader & \jsheader & \cshheader \\
    \hline
    函数名 & \+`match' & \+`matches' & \+`test' & \+`Match' \\
    \hline
    前缀 & \+`\+c{Pattern}' & \+`\dstr{string}' & \+`\jsrstr{/\^{}re\$/}' & \+`\+c{Regex}' \\
    \hline
    参数 & \+`\dstr{string}' & \+`\dstr{re}' & \+`\dstr{string}' & \+`\dstr{string}, \cshrstr{\^{}re\$}' \\
    \hline
    正返回 & \+`\+c{Match}' & \+:c{3}{c|}{\+`\lsttrue'} \\
    \hline
    零返回 & \+`\pynone' & \+:c{3}{c|}{\+`\lstfalse'} \\
    \hline
\end{longtable}

% subsubsection 模式判定 (end)

\subsubsection{模式提取} % (fold)
\label{ssub:模式提取}

\begin{pylst}
/@\pyimport{re}@/
/@\lhend @/
text = "id: 8, id: 42, id: 312"
regex = re.compile("id: ([0-9]+)")
matches = pattern.finditer(text)
for match in matches:
    print match.group(1)
\end{pylst}
\begin{javalst}
/@\javaimport{java.util.regex.*;}@/
/@\lhend @/
/@\+c{String}@/ text = "id: 8, id: 42, id: 312";
/@\+c{Pattern}@/ regex = /@\+c{Pattern}@/.compile("id: ([0-9]+)");
/@\+c{Matcher}@/ matcher = pattern.matcher(text);
while (matcher.find()) {
    System.out.println(matcher.group(1));
}
\end{javalst}
\begin{jslst}
var text = "id: 8, id: 42, id: 312";
var regex = @</id: ([0-9]+)/g>@;
var match;
do {
    match = regex.exec(text);
    if (match) {
        console.log(match[1]);
    }
} while (match);
\end{jslst}
\begin{cshlst}
/@\cshusing{System;}@/
/@\cshusing{System.Text.RegularExpressions;}@/
/@\lhend @/
string text = "id: 8, id: 42, id: 312";
string regex = "id: ([0-9]+)";
/@\+c{MatchCollection}@/ matches = /@\+c{Regex}@/.Matches(text, regex);
foreach (/@\+c{Match}@/ match in matches)
{
    /@\+c{Console}@/.WriteLine(match.Groups[1]);
}
\end{cshlst}

% subsubsection 模式提取 (end)

% subsection 正则表达式 (end)

% section 字符串处理 (end)

\section{文件处理} % (fold)
\label{sec:文件处理}

\subsection{文件结构} % (fold)
\label{sub:文件结构}

\subsubsection{遍历文件夹} % (fold)
\label{ssub:遍历文件夹}

\begin{longtable}{|c|c|c|c|c|}
    \hline
    语言 & \pyheader & \javaheader & \nodejsheader & \cshheader \\
    \hline
    函数名 & \+`listdir' & \+`listFiles' & \+`readdirSync' & \+`GetFiles'  \\
    \hline
    前缀 & \+`os' & \+`\+c{File}' & \+`require(\sstr{fs})' & \+`\+c{DirectoryInfo}' \\
    \hline
    参数 & \+`\dstr{/usr}' & & \+`\dstr{/usr}' & \\
    \hline
    返回值 & \+`[\+c{str}]' & \+`\+c{File}[]' & \+`[\+c{str}]' & \+`\+c{FileInfo}[]' \\
    \hline
\end{longtable}

% subsubsection 遍历文件夹 (end)

% subsection 文件结构 (end)

\subsection{文件读取} % (fold)
\label{sub:文件读取}

\subsubsection{逐行读取} % (fold)
\label{ssub:逐行读取}

\begin{pylst}
/@file@/ = open('sampleInput')
for line in /@file@/.readlines():
    print(line)
\end{pylst}
\begin{javalst}
/@\javaimport{java.io.*;}@/
/@\lhend @/
try (/@\lstvar @/ br = new /@\+c{BufferedReader}@/(new /@\+c{FileReader}@/(filename))) {
    /@\+c{String}@/ line;
    while ((line = br.readLine()) != null) {
       System.out.println(line);
    }
}
\end{javalst}
\begin{nodejslst}
var lineReader = require('readline').createInterface({
    input: require('fs').createReadStream('file.in')
});
lineReader.on('line', function (line) {
    console.log('Line from file:', line);
});
\end{nodejslst}
\begin{cshlst}
/@\cshusing{System.IO;}@/
/@\lhend @/
/@\lstvar @/ lines = /@\+c{File}@/.ReadLines(filename);
foreach (var line in lines)
    /@\+c{Console}@/.WriteLine(line);
\end{cshlst}

% subsubsection 逐行读取 (end)

% subsection 文件读取 (end)

\subsection{文件写入} % (fold)
\label{sub:文件写入}

\subsubsection{逐行写入} % (fold)
\label{ssub:逐行写入}

\begin{pylst}
/@\pywith@/ open(filename, 'w') /@\pyas@/ myfile:
    myfile.write("Stuff")
\end{pylst}
\begin{javalst}
/@\javaimport {java.io.*;}@/
/@\lhend @/
try (/@\+c{PrintWriter}@/ pw = new /@\+c{BufferedReader}@/(filename)) {
    br.println("Stuff");
}
\end{javalst}
\begin{cshlst}
/@\cshusing{System.IO;}@/
/@\lhend@/
using (/@\+c{StreamWriter}@/ file = new /@\+c{StreamWriter}@/(filename))
{
    file.WriteLine("Stuff");
}
\end{cshlst}

% subsubsection 逐行写入 (end)

% subsection 文件写入 (end)

% section 文件处理 (end)

\section{模版元编程} % (fold)
\label{sec:模版元编程}

\subsection{定义模版} % (fold)
\label{sub:定义模版}

\subsubsection{模版函数} % (fold)
\label{ssub:模版函数}

\begin{cpplst}
template<typename T>
int less_than(const T& v1, const T& v2)
{
    return v1 < v2;
}
\end{cpplst}

% subsubsection 模版函数 (end)

% subsection 定义模版 (end)

% section 模版元编程 (end)

\section{函数式编程} % (fold)
\label{sec:函数式编程}

\subsection{Lambda表达式} % (fold)
\label{sub:lambda表达式}

\subsubsection{Lambda表达式的声明} % (fold)
\label{ssub:lambda表达式的声明}
\begin{cpplst}
std::/@\+c{function}@/<double (double)> f = 
    [] (double x) { return x * x; };
\end{cpplst}
\begin{pylst}
square = lambda x: x * x
\end{pylst}
\begin{javalst}
/@\javaimport{java.util.function.Function;}@/
/@\lhend @/
/@\+c{Function}@/</@\+c{Double}@/, /@\+c{Double}@/> square = x -> x * x;
\end{javalst}
\begin{jslst}
var square = x => x * x;
\end{jslst}
\begin{cshlst}
/@\cshusing{System;}@/
/@\lhend @/
/@\+c{Func}@/<double, double> square = x => x * x;
\end{cshlst}

% subsubsection lambda表达式的声明 (end)

% subsection lambda表达式 (end)

% section 函数式编程 (end)

\section{数学} % (fold)
\label{sec:数学}

\subsection{随机数} % (fold)
\label{sub:随机数}

\subsubsection{生成随机数} % (fold)
\label{ssub:生成随机数}

\begin{clst}
/@\cinclude{stdlib.h}@/
/@\lhend @/
int randInt = rand() % (upper - lower) + lower;
\end{clst}
\begin{pylst}
/@\pyimport{random}@/
/@\lhend @/
random.seed(seed);
randInt = random.randrange(lower, upper);
\end{pylst}
\begin{javalst}
/@\javaimport{java.util.Random;}@/
/@\lhend @/
/@\+c{Random}@/ random = new /@\+c{Random}@/(seed);
int randInt = random.nextInt(upper - lower) + lower;
\end{javalst}
\begin{jslst}
var randInt = Math.floor(Math.random() * (upper - lower)) + lower;
\end{jslst}
\begin{cshlst}
/@\+c{Random}@/ random = new /@\+c{Random}@/(seed);
int randInt = random.Next(lower, upper);
\end{cshlst}
\begin{phplst}
$randInt = rand(min, max);
\end{phplst}
\begin{shlst}
$RANOM=$$
$RAND_INT=$(($RANDOM%($UPPER-$LOWER)+$LOWER))
\end{shlst}

% subsubsection 生成随机数 (end)

% subsection 随机数 (end)

% section 数学 (end)

\section{安全} % (fold)
\label{sec:安全}

\subsection{资源管理} % (fold)
\label{sub:资源管理}

\subsubsection{安全释放} % (fold)
\label{ssub:安全释放}

\begin{pylst}
/@\pywith@/ open(filename, 'w') /@\pyas@/ myfile:
    myfile.write("Stuff")
\end{pylst}
\begin{javalst}
/@\javaimport {java.io.*;}@/
/@\lhend @/
try (/@\+c{PrintWriter}@/ pw = new /@\+c{BufferedReader}@/(filename)) {
    br.println("Stuff");
}
\end{javalst}
\begin{cshlst}
/@\cshusing{System.IO;}@/
/@\lhend@/
using (/@\+c{StreamWriter}@/ file = new /@\+c{StreamWriter}@/(filename))
{
    file.WriteLine("Stuff");
}
\end{cshlst}

% subsubsection 安全释放 (end)

% subsection 资源管理 (end)

% section 安全 (end)

\end{document}
