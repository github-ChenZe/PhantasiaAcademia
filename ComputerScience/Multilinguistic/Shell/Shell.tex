\documentclass[hidelinks]{ctexart}

\usepackage[margintoc, singleton, epu, nova]{van-de-la-sehen}

\begin{document}

\showtitle{Shell技巧}

\section{通用} % (fold)
\label{sec:通用}

\href{https://www.redhat.com/sysadmin/stupid-bash-tricks}{Bash tricks}
\href{https://www.redhat.com/sysadmin/more-stupid-bash-tricks}{More Bash Tricks}
\href{https://www.howtogeek.com/439199/15-special-characters-you-need-to-know-for-bash/amp/}{Bash special characters}
\href{https://devhints.io/bash}{Bash cheatsheet}


\subsection{语法} % (fold)
\label{sub:语法}

\subsubsection{不记录} % (fold)
\label{ssub:不记录}

\begin{shlst}
unset HISTFILE && exit
# or
kill -9 $$ # $$ stands for the pid of the current shell
\end{shlst}
The above code should be executed before the shell exits normally.

% subsubsection 不记录 (end)

\subsubsection{历史选择} % (fold)
\label{ssub:历史选择}

\begin{shlst}
history 3
# 1002 ls
# 1003 tail audit.log
# 1004 history 3
!1003
# tail audit.log
\end{shlst}

% subsubsection 历史选择 (end)

\subsubsection{历史替换} % (fold)
\label{ssub:历史替换}

\begin{shlst}
sudo systemctl status sshd
!!:s/status/start # sudo systemctl start sshd
\end{shlst}
In the above example, \texttt{!!} repreats the last command from history, and \texttt{:s/status/start/} substitute \texttt{status} with \texttt{start}.

% subsubsection 历史替换 (end)

\subsubsection{上一参数} % (fold)
\label{ssub:上一参数}

\begin{shlst}
ls /home/username/
# files
cd $_
\end{shlst}
The \texttt{\$\_} in the above code stands for the last argument of the previous command.

% subsubsection 上一参数 (end)

% subsection 语法 (end)

% section 通用 (end)

\section{文件管理} % (fold)
\label{sec:文件管理}

\subsection{文件夹} % (fold)
\label{sub:文件夹}

\subsubsection{批量创建} % (fold)
\label{ssub:批量创建}

\begin{shlst}
mkdir -v dir_{rmp,txt,zip,pdf}
\end{shlst}

% subsubsection 批量创建 (end)

\subsubsection{复制} % (fold)
\label{ssub:复制}

\begin{shlst}
cp /etc/long/path/to/dir/file{,.back}
# cp /etc/long/path/to/dir/file /etc/long/path/to/dir/file.back
\end{shlst}

% subsubsection 复制 (end)

\subsubsection{移动} % (fold)
\label{ssub:移动}

\begin{shlst}
mv -- *.rpm dir_rpm
\end{shlst}
In the above example, the double dash characters \texttt{--} mean ``end of options'', which is to prevent files that begin begin with a dash from being treated as arguments.

% subsubsection 移动 (end)

% subsection 文件夹 (end)

\subsection{文件结构} % (fold)
\label{sub:文件结构}

\subsubsection{清单} % (fold)
\label{ssub:清单}

\begin{shcommand}{ls}
\ttfamily
\begin{tabular}{@{$\bullet\quad$}ll}
   -a      & 显示隐藏文件('.'开头); \\
   -l & 显示完整信息; \\
\end{tabular} 
\end{shcommand}
\begin{shlst}
ls -la
\end{shlst}

\texttt{ls}表示list.

% subsubsection 清单 (end)

\subsubsection{重命名} % (fold)
\label{ssub:重命名}

\begin{shcommand}{rename}
\ttfamily
\begin{tabular}{@{$\bullet\quad$}ll}
   -n/--just-print/--dry-run      & 不实际执行; \\
   -v/--verbose & 显示完整步骤; \\
\end{tabular} 
\end{shcommand}
\begin{shlst}
rename 's/prefix([0-9])/file_$1' prefix*
\end{shlst}

% subsubsection 重命名 (end)

\subsubsection{测试类型} % (fold)
\label{ssub:测试类型}

\begin{shlst}
[[ -L /path/to/file ]] && echo "File is a symlink."
\end{shlst}

% subsubsection 测试类型 (end)

% subsection 文件结构 (end)

\subsection{压缩} % (fold)
\label{sub:压缩}

\subsubsection{压缩} % (fold)
\label{ssub:压缩}

See \href{https://www.howtogeek.com/248780/how-to-compress-and-extract-files-using-the-tar-command-on-linux/}{here}.

\begin{shcommand}{tar}
\ttfamily
\begin{tabular}{@{$\bullet\quad$}ll}
   -x & extract to disk from the archive \\
   -c & create an archive \\
   -z & compress the archive with gzip \\
   -j & compress the archive with bzip \\
   -v & display progress in the terminal \\
   -f & allows you to specify the filename of the archive
\end{tabular}
\end{shcommand}
\begin{shlst}
tar -czvf name-of-archive.tar.gz /path/to/directory-or-file
tar -xzvf archive.tar.gz -C /tmp
\end{shlst}

% subsubsection 压缩 (end)

\subsubsection{批量} % (fold)
\label{ssub:批量}

\begin{shlst}
for f in ./*.gz;
do tar zxvf "$f";
done
\end{shlst}

% subsubsection 批量 (end)

% subsection 压缩 (end)

% section 文件管理 (end)

\section{网络} % (fold)
\label{sec:网络}

\subsection{配置} % (fold)
\label{sub:配置}

\subsubsection{网络参数} % (fold)
\label{ssub:网络参数}

\begin{shcommand}{ifconfig}
\ttfamily
\begin{tabular}{@{$\bullet\quad$}ll}
    \underline{address}     & 显示该接口的参数; \\
\end{tabular} 
\end{shcommand}
\begin{shlst}
ifconfig
ifconfig en0
\end{shlst}
\texttt{ifconfig}表示interface configuration.

% subsubsection 网络参数 (end)

% subsection 配置 (end)

% section 网络 (end)

\section{系统} % (fold)
\label{sec:系统}

\subsection{用户} % (fold)
\label{sub:用户}

\subsubsection{切换用户} % (fold)
\label{ssub:切换用户}

\begin{shcommand}{su}
\ttfamily
\begin{tabular}{@{$\bullet\quad$}ll}
   -m      & 保持当前shell;
\end{tabular} 
\end{shcommand}
\begin{shlst}
su -m
su -m cyan
\end{shlst}
\texttt{su}表示substitute user.

% subsubsection 切换用户 (end)

\subsubsection{超级权限} % (fold)
\label{ssub:超级权限}

\begin{shcommand}{sudo}
\ttfamily
\begin{tabular}{@{$\bullet\quad$}ll}
   -A      & 从\texttt{\$SUDO\_ASK\_PASS}的\texttt{stdout}读密码, 需要\texttt{\textbackslash n};
\end{tabular} 
\end{shcommand}
\begin{shlst}
alias sudo="sudo -A"
sudo ls
\end{shlst}
\texttt{sudo}表示super user do.

% subsubsection 超级权限 (end)

% subsection 用户 (end)

\subsection{配置} % (fold)
\label{sub:配置}

\subsubsection{配置文件} % (fold)
\label{ssub:配置文件}

\begin{shlst}
vim ~/.bashrc
source ~/.bashrc
\end{shlst}
\texttt{rc}表示runtime configuration.

% subsubsection 配置文件 (end)

% subsection 配置 (end)

\subsection{包安装} % (fold)
\label{sub:包安装}

\subsubsection{apt} % (fold)
\label{ssub:apt}

\texttt{apt}表示advanced package control.

% subsubsection apt (end)

% subsection 包安装 (end)

\subsection{加密} % (fold)
\label{sub:加密}

\subsubsection{随机串} % (fold)
\label{ssub:随机串}

\begin{shlst}
openssl rand -base64 12
# zw03I1n6MZ4FHn6g
\end{shlst}

% subsubsection 随机串 (end)

% subsection 加密 (end)

% section 系统 (end)

\end{document}
