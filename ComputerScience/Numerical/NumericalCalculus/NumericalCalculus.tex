\documentclass{ctexart}

\usepackage[margintoc]{van-de-la-sehen}
\DeclareMathOperator{\cond}{cond}
\usepackage{arydshln}

\begin{document}

\section{数值微积分} % (fold)
\label{sec:数值微积分}

\subsection{数值微分和积分} % (fold)
\label{sub:数值微分和积分}

\subsubsection{数值微分} % (fold)
\label{ssub:数值微分}

\begin{theorem}[二点前向差分公式]
    $\displaystyle f'\pare{x} = \frac{f\pare{x+h}-f\pare{x}}{h} - \frac{h}{2}f''\pare{c}$, 其中$c$在$x$和$x+h$之间.
\end{theorem}
\begin{theorem}[推广的中值定理]
    设$f\in C\brac{a,b}$, $x_1,\cdots,x_n \in \brac{a,b}$, $a_1,\cdots,a_n > 0$, 则存在$c\in\brac{a,b}$满足
    \[ \pare{a_1+\cdots+a_n}f\pare{c} = a_1f\pare{x_1} + \cdots + a_nf\pare{x_n}. \]
\end{theorem}
由Taylor公式出发,
\begin{align*}
    f\pare{x+h} &= f\pare{x} + hf'\pare{x} + \frac{h^2}{2}f''\pare{x} + \frac{h^3}{6}f'''\pare{c_1}, \\
    f\pare{x-h} &= f\pare{x} - hf'\pare{x} + \frac{h^2}{2}f''\pare{x} - \frac{h^3}{6}f'''\pare{c_2},
\end{align*}
作差可得$f'\pare{x}$的表达式.
\begin{theorem}[三点中心差分公式]
    \label{thm:三点中心差分公式}
    $\displaystyle f'\pare{x} = \frac{f\pare{x+h}-f\pare{x-h}}{2h} - \frac{h^2}{6}f'''\pare{c}$, 其中$x-h<c<x+h$.
\end{theorem}
\begin{theorem}[二阶导数的三点中心差分公式]
    \label{thm:二阶导数的三点中心差分公式}
    存在$c\in\brac{x-c,x+c}$满足
    \[ f''\pare{x} = \frac{f\pare{x-h} - 2f\pare{x} + f\pare{x+h}}{h^2} - \frac{h^2}{12}f^{\pare{4}}\pare{c}. \] 
\end{theorem}
浮点舍入将导致舍入误差, 对于三点中心差分公式, 总误差上界为
\[ E\pare{h} = \frac{h^2}{6}f'''\pare{c} + \frac{\epsilon\+_mach_}{h}, \]
当$h=\pare{3\epsilon\+_mach_/M}^{1/3}$这一上界具有最小值.

\paragraph{外推} % (fold)
\label{par:外推}

设$Q \approx F\pare{h} + Kh^n$, 例如$Q=f'\pare{x}$的步长$h$的差分近似, 有
\[ Q - F\pare{\frac{h}{2}} \approx \rec{2^n}\pare{Q-F\pare{h}}. \]
\begin{theorem}[$n$阶公式的外推]
    若$F_n\pare{h}$具有$O\pare{h^n}$的误差, 则
    \[ F_{n+1}\pare{h}\approx \frac{2^nF_n\pare{h/2} - F_n\pare{h}}{2^n-1} \]
    具有$O\pare{h^{n+1}}$的误差.
\end{theorem}
\begin{ex}
    从三点前向差分公式
    \[ f'\pare{x} = \frac{f\pare{x+h}-f\pare{x-h}}{2h} - \frac{h^2}{6}f'''\pare{c} \]
    出发, 应用一次外推, 则
    \[ F_4\pare{x} = \frac{2^2 F_2\pare{h/2}}{2^2-1} = \frac{f\pare{x-h} - 8\pare{x-h/2} + 8f\pare{x+h/2} - f\pare{x+h}}{6h}, \]
    且至多具有$O\pare{h^3}$的误差, 而实际上误差为$O\pare{h^4}$.
\end{ex}

% paragraph 外推 (end)

% subsubsection 数值微分 (end)

\subsubsection[Newton-Cotes公式]{数值积分的Newton-Cotes公式} % (fold)
\label{ssub:数值积分的newton_cotes公式}

给定$f\pare{x}$在$x_0$和$x_1$处的值$y_0$和$y_1$, 由Lagrange插值公式,
\[ f\pare{x} = y_0 \frac{x-x_1}{x_0 - x_1} + y_1 \frac{x-x_0}{x_1-x_0} + \frac{\pare{x-x_0}\pare{x-x_1}}{2!}f''\pare{c_x} = P\pare{x} + E\pare{x}. \]
主项的积分为
\[ \int_{x_0}^{x_1} P\pare{x}\,\rd{x} = h\cdot \frac{y_0+y_1}{2}. \]
误差项为
\begin{align*}
    \int_{x_0}^{x_1} E\pare{x}\,\rd{x} &= \rec{2!}\int_{x_0}^{x_1}\pare{x-x_0}\pare{x-x_1}f''\pare{c\pare{x}}\,\rd{x}\\
    &= \frac{f''\pare{c}}{2}\int_{x_0}^{x_1} \pare{x-x_0}\pare{x-x_1}\,\rd{x} = -\frac{h^3}{12}f''\pare{c}.
\end{align*}
\begin{theorem}[梯形法则]
    存在$c\in\brac{x_0,x_1}$满足
    \[ \int_{x_0}^{x_1} f\pare{x}\,\rd{x} = \frac{h}{2}\pare{y_0+y_1} - \frac{h^3}{12}f''\pare{c}. \]
\end{theorem}
如果改用抛物插值, 并设$x_1 - x_0 = x_2 - x_1 = h$, 则
\begin{align*}
    f\pare{x} &= y_0 \frac{\pare{x-x_1}\pare{x-x_2}}{\pare{x_0-x_1}\pare{x_0-x_2}} + y_1 \frac{\pare{x-x_0}\pare{x-x_2}}{\pare{x_1-x_0}\pare{x_1-x_2}} \\
    &\phantom{= } + y_2\frac{\pare{x-x_0}\pare{x-x_1}}{\pare{x_2-x_0}\pare{x_2-x_1}} + \frac{\pare{x-x_0}\pare{x-x_1}\pare{x-x_2}}{3!}f'''\pare{c_x} \\
    &= P\pare{x} + E\pare{x}.
\end{align*}
主项积分为
\[ \int_{x_0}^{x_2}f\pare{x}\,\rd{x} = \frac{h}{3}\pare{y_0 + 4y_1 + y_2}, \]
误差项为
\begin{align*}
    \int_{x_0}^{x_2} E\pare{x}\,\rd{x} &= -\frac{h^5}{90}f^{\pare{4}}\pare{c}.
\end{align*}
%这可以通过
%\begin{align*}
%    \int_{-h}^h f\pare{x}\,\rd{x} &= \int_{-h}^h f\pare{0}\,\rd{x} + \half f''\pare{0}\int_{-h}^h x^2\,\rd{x} + \rec{24}\int_{-h}^h f^{\pare{4}}\pare{\xi\pare{x}}x^4\,\rd{x} \\
%    &= 2hf\pare{0} + \frac{1}{3}h^3f''\pare{0} + \frac{h^5}{60}f^{\pare{4}}\pare{\xi_3},\\
%    f''\pare{0} &= \frac{f\pare{-h} - 2f\pare{0} + f\pare{h}}{h^2} - \frac{h^2}{12}f^{\pare{4}}\pare{c}
%\end{align*}
%得到.
\begin{remark}
    The proof via Taylor's expansion was flawed. \href{https://math.stackexchange.com/questions/1768667/combining-error-terms-in-simpsons-rule/1770264#1770264}{cf.}
\end{remark}
\begin{theorem}[Simpson法则]
    存在$c\in\brac{x_0,x_2}$满足
    \[ \int_{x_0}^{x_1} f\pare{x}\,\rd{x} = \frac{h}{3}\pare{y_0 + 4y_1 + y_2} - \frac{h^5}{90}f^{\pare{4}}\pare{c}. \]
\end{theorem}
\begin{definition}[精度]
    数值积分方法的精度谓使得积分公式对$k$阶或以下的多项式积分精确成立之最大整数$k$.
\end{definition}
\begin{remark}
    Simpson公式具有三阶精度之几何解释为等距的抛物插值和三次样条具有相同积分值.
\end{remark}
\begin{ex}[Simpson 3/8公式]
    3阶Newton-Cotes公式为
    \[ \int_{x_0}^{x_3}f\pare{x}\,\rd{x} = \frac{3h}{8}\pare{y_0 + 3y_1 + 3y_2 + y_3}. \]
\end{ex}

% subsubsection 数值积分的newton_cotes公式 (end)

\subsubsection[复合公式]{复合Newton-Cotes公式} % (fold)
\label{ssub:复合newton_cotes公式}

梯形法则的复合须以步长$h$划分区间, 分为$m$个区间.
\begin{theorem}[复合梯形法则]
    存在$c\in\brac{a,b}$满足
    \[ \int_a^b f\pare{x}\,\rd{x} = \frac{h}{2}\pare{y_0 + y_m + 2\sum_{i=1}^{m-1}y_i} - \frac{\pare{b-a}h^2}{12}f''\pare{c}. \]
\end{theorem}

Simpson法则的复合同样以步长$h$划分区间, 分为$2m$个区间, 每隔$2h$应用一次Simpson法则后求和.
\begin{theorem}[复合Simpson法则]
    存在$c\in\brac{a,b}$满足
    \[ \int_a^b f\pare{x}\,\rd{x} = \frac{h}{3}\brac{y_0 + y_{2m} + 4\sum_{i=1}^m y_{2i-1} + 2\sum_{i=1}^{m-1} y_{2i}} - \frac{\pare{b-a}h^4}{180}f^{\pare{4}}\pare{c}. \]
\end{theorem}

% subsubsection 复合newton_cotes公式 (end)

\subsubsection[开区间上积分]{开Newton-Cotes方法} % (fold)
\label{ssub:开newton_cotes方法}

\begin{theorem}[中点法则]
    设$\displaystyle h = x_1 - x_0$, $\displaystyle w = x_0 + \frac{h}{2}$, 则存在$c\in\brac{x_0,x_1}$满足
    \[ \int_{x_0}^{x_1}f\pare{x}\,\rd{x} = hf\pare{w} + \frac{h^3}{24}f''\pare{c}. \]
\end{theorem}
将$\brac{a,b}$划分为$m$个相等的子区间, 可得复合的中点法则.
\begin{theorem}[复合中点法则]
    设$w_i$是各个子区间的中点, 则存在$c\in\brac{a,b}$满足
    \[ \int_a^b f\pare{x}\,\rd{x} = h\sum_{i=1}^m f\pare{w_i} + \frac{\pare{b-a}h^2}{24}f''\pare{c}. \]
\end{theorem}
具有3阶精度的Newton-Cotes法则为
\[ \int_{x_0}^{x_4} f\pare{x}\,\rd{x} = \frac{4h}{3}\brac{2f\pare{x_1} - f\pare{x_2} + 2f\pare{x_3}} + \frac{14h^5}{45}f^{\pare{4}}\pare{c}. \]

% subsubsection 开newton_cotes方法 (end)

\subsubsection{Romberg积分} % (fold)
\label{ssub:romberg积分}

对于梯形法则,
\[ \int_a^b f\pare{x}\,\rd{x} = \frac{h}{2}\pare{y_0 + y_m + 2\sum_{i=1}^{m-1}y_i} + c_2h^2 + c_4h^4 + c_6h^6 + \cdots. \]
注意到误差仅含$h$的偶次项, 故每次外推精度阶数增加$2$. 每次将步长减半, $h_1 = b-a$, $h_2 = \pare{b-a}/2$, $\cdots$, $h_j = \pare{b-a}/a^{j-1}$, 可得
\begin{align*}
    R_{11} &= \frac{h_1}{2}\brac{f\pare{a} + f\pare{b}}, \\
    R_{21} &= \frac{h_2}{2}\brac{f\pare{a} + f\pare{b} + 2f\pare{\frac{a+b}{2}}} = \half R_{11} + h_2 f\pare{\frac{a+b}{2}}, \\
    \vdots & \\
    R_{j1} &= \half R_{j-1,1} + h_j \sum_{i=1}^{2^{j-2}} f\pare{a+\pare{2i-1}h}.
\end{align*}
如下的外推
\[ \begin{matrix}
    R_{11} & & & & \\
    R_{21} & R_{22} \\
    R_{31} & R_{32} & R_{33} \\
    R_{41} & R_{42} & R_{43} & R_{44} \\
    \vdots & & & & \ddots
\end{matrix} \]
需要递推式
\begin{align*}
    R_{22} &= \frac{2^2 R_{21} - R_{11}}{2^2-1}, \\
    R_{32} &= \frac{2^2 R_{31} - R_{21}}{2^2-1}, \\
    R_{42} &= \frac{2^2 R_{41} - R_{31}}{2^2-1}, \\
\end{align*}
以及
\begin{align*}
    R_{33} &= \frac{2^4 R_{21} - R_{11}}{2^4-1}, \\
    R_{43} &= \frac{2^4 R_{31} - R_{21}}{2^4-1}, \\
    R_{53} &= \frac{2^4 R_{41} - R_{31}}{2^4-1}. \\
\end{align*}
一般的递推式为
\[ R_{jk} = \frac{4^{k-1}R_{j,k-1} - R_{j-1,k-1}}{4^{k-1}-1}. \]
\begin{matlablst}
/@$R_{11}$@/ = /@$\displaystyle \pare{b-a}\frac{f\pare{a}+f\pare{b}}{2}$@/
for j = 2, 3, ...
    /@$h_j$@/ = /@$\displaystyle \frac{b-a}{2^{j-1}}$@/
    /@$R_{j1}$@/ = /@$\displaystyle \half R_{j-1,1} + h_j \sum_{i=1}^{2^{j-2}} f\pare{a+\pare{2i-1}h_j}$@/
    for k = 2, ... j
        /@$R_{jk}$@/ = /@$\displaystyle \frac{4^{k-1}R_{j,k-1} - R_{j-1,k-1}}{4^{k-1}-1}$@/
    end
end
\end{matlablst}

% subsubsection romberg积分 (end)

\subsubsection{自适应积分} % (fold)
\label{ssub:自适应积分}

梯形法则满足
\[ \int_a^b f\pare{x}\,\rd{x} = S_{\brac{a,b}} - h^3 \frac{f''\pare{c_0}}{12}, \]
其中$h=b-a$. 设$c$为$a,b$之中点, 则有
\[ \int_a^b f\pare{x}\,\rd{x} = S_{\brac{a,c}} + S_{\brac{c,b}} - \frac{h^3}{4}\frac{f''\pare{c_3}}{12}. \]
故
\[ S_{\brac{a,b}} - \pare{S_{\brac{a,c}} + S_{\brac{c,b}}} \approx \frac{3}{4}h^3 \frac{f''\pare{c_3}}{12}. \]
从而通过确定此式的值是否小于$3\times \mathrm{TOL}$, 即可确定第二次积分的误差是否小于$\mathrm{TOL}$.
\begin{matlablst}
对给定容差TOL近似/@$\displaystyle \int_a^b f\pare{x}\,\rd{x}$@/
/@$c$@/ = /@$\displaystyle \frac{a+b}{2}$@/
/@$S_{\brac{a,b}}$@/ = /@$\displaystyle \pare{b-a}\frac{f\pare{a} + f\pare{b}}{2}$@/
if /@$\displaystyle \abs{S_{\brac{a,b}} - S_{\brac{a,c}} - S_{\brac{c,b}}} < 3\times \mathrm{TOL} \times \frac{b-a}{b\+_orig_ - a\+_orig_}$@/
    接受/@$S_{\brac{a,c}} - S_{\brac{c,b}}$@/作为区间/@$\brac{a,b}$@/上的近似
else
    对于区间/@$\brac{a,c}$@/和/@$\brac{c,b}$@/递归重复上面的步骤
end
\end{matlablst}
如果使用Simpson法则, 则相应的公式替换为
\[ \int_a^b f\pare{x}\,\rd{x} = S_{\brac{a,b}} - \frac{h^5}{90} f^{\pare{4}}\pare{c_0}, \]
将区间二分可得
\[ \int_a^b f\pare{x}\,\rd{x} = S_{\brac{a,c}} + S_{\brac{c,b}} - \frac{h^5}{16}\frac{f^{\pare{4}}\pare{c_3}}{90}, \]
从而
\[ S_{\brac{a,b}} - \pare{S_{\brac{a,c}} + S_{\brac{c,b}}} \approx \frac{15}{16}h^5 \frac{f^{\pare{4}}\pare{c_3}}{90}. \]
相应的标准应修改为
\[ \abs{S_{\brac{a,b}} - \pare{S_{\brac{a,c}} + S_{\brac{c,b}}}} < 15\times \mathrm{TOL}. \]

% subsubsection 自适应积分 (end)

\subsubsection{\texorpdfstring{Gauss积分}{Gau\ss 积分}} % (fold)
\label{ssub:gauss_积分}

\begin{theorem}[Gau\ss 积分]
    设$\curb{x_i}$是$P_n\pare{x}$的根, 并设$f\pare{x}$过点$\curb{x_i}$的Lagrange插值多项式为
    \[ Q\pare{x} = \sum_{i=1}^n L_i\pare{x}f\pare{x_i}, \]
    则
    \[ \int_{-1}^1 f\pare{x}\,\rd{x} \approx \sum_{i=1}^n c_i f\pare{x_i}, \]
    其中
    \[ c_i = \int_{-1}^1 L_i\pare{x}\,\rd{x},\quad i = 1,\cdots,n. \]
    系数$c_i$须查表确定. 使用$n$阶的Legendre多项式时具有$2n-1$阶代数精度.
\end{theorem}
\begin{proof}
    使用多项式长除, 有
    \[ P\pare{x} = S\pare{x}P_n\pare{x} + R\pare{x}, \]
    其中$S$和$R$都是小于$n$阶的多项式. 注意到Gau\ss 积分对$R\pare{x}$精确成立, 且$R\pare{x}$和$P\pare{x}$的Gau\ss 积分相等, 且
    \[ \int_{-1}^1 P\pare{x}\,\rd{x} = \int_{-1}^1 S\pare{x}p_n\pare{x}\,\rd{x} + \int_{-1}^1 R\pare{x}\,\rd{x} = \int_{-1}^1 R\pare{x}\,\rd{x} \]
    即$P\pare{x}$的精确积分和$R\pare{x}$的精确积分相等, 故Gau\ss 积分对$P\pare{x}$精确成立.
\end{proof}

% subsubsection gauss_积分 (end)

% subsection 数值微分和积分 (end)

\subsection{常微分方程} % (fold)
\label{sub:常微分方程}

\subsubsection{初值问题} % (fold)
\label{ssub:初值问题}

对于方程
\[ \begin{cases}
    y' = f\pare{t,y},\\
    y\pare{a} = y_a, \\
    t\in\pare{a,b},
\end{cases} \]
以及$\brac{a,b}$的一个分割$\curb{t_i}$, 有如下的Euler方法.
\begin{theorem}[Euler方法]
    $\displaystyle \begin{cases}
        w_0 = y_0, \\
        w_{i+1} = w_i + hf\pare{t_i,w_i}.
    \end{cases}$
\end{theorem}
\begin{definition}
    当存在常数$L$(即Lipschitz常数)对矩形$S=\brac{a,b}\times\brac{\alpha,\beta}$中的每对$\pare{t,y_1}$, $\pare{t,y_2}$满足
    \[ \abs{f\pare{t,y_1} - f\pare{t,y_2}} \le L\abs{y_1-y_2}, \]
    谓函数$f\pare{t,y}$相对变量$y$在$S$上Lipschitz连续.
\end{definition}
\begin{theorem}
    若$f\pare{t,y}$定义在集合$\brac{a,b}\times\brac{\alpha,\beta}$且$\alpha < y_a < \beta$, 函数对于变量$y$是Lipschitz连续, 则在$a,b$之间存在$c$, 使初值问题
    \[ \begin{cases}
        y' = f\pare{t,y}, \\
        y\pare{a} = y_a, \\
        t \in \brac{a,c}
    \end{cases} \]
    有唯一解$y\pare{t}$. 且若$f$在$\brac{a,b}\times \pare{-\infty,\infty}$内Lipschitz连续, 则其在$\brac{a,b}$上有唯一解.
\end{theorem}
\begin{theorem}
    设$f\pare{t,y}$在集合$S=\brac{a,b}\times \brac{\alpha,\beta}$上关于$y$连续. 若$Y\pare{t}$和$Z\pare{t}$是微分方程
    \[ y' = f\pare{t,y} \]
    在$S$上的解, 分别具有初值条件$Y\pare{a}$和$Z\pare{a}$, 则
    \[ \abs{Y\pare{t} - Z\pare{t}} \le e^{L\pare{t-a}}\abs{Y\pare{a} - Z\pare{a}}. \]
\end{theorem}
\begin{remark}
    这是Gr\"onwall不等式的特殊情形.
\end{remark}
对于初值问题
\[ \begin{cases}
    y' = g\pare{t} y + h\pare{t}, \\
    y\pare{a} = y_a, \\
    t\in\brac{a,b},
\end{cases} \]
设$G'\pare{t} = g\pare{t}$, 则有显式解
\[ y\pare{t} = e^{G\pare{t}} \int e^{-G\pare{t}}h\pare{t}\,\rd{t}. \]

% subsubsection 初值问题 (end)

\subsubsection{IVP求解器的分析} % (fold)
\label{ssub:ivp求解器的分析}

全局截断误差为
\[ g_i = \abs{w_i - y_i}, \]
即近似值和初值问题精确解之间的差异. 局部截断误差为
\[ e_{i+1} = \abs{w_{i+1} - z\pare{t_{i+1}}}, \]
这是区间上求解器的值与单步初值问题
\[ \begin{cases}
    y' = f\pare{t,y}, \\
    y\pare{t_i} = w_i, \\
    t\in \brac{t_i,t_{i+1}}
\end{cases} \]
的正确解之间的差异.
\begin{ex}
    假设$y''$连续, 有
    \[ y\pare{t_i + h} = y\pare{t_i} + hf\pare{t_i,w_i} + \frac{h^2}{2}y''\pare{c}, \]
    而Euler方法为
    \[ w_{i+1} = w_i + hf\pare{t_i,w_i}, \]
    可得局部误差
    \[ e_{i+1} = \abs{w_{i+1} - y\pare{t_{i+1}}} = \frac{h^2}{2}\abs{y''\pare{c}} \le \frac{Mh^2}{2}, \]
    其中$M = \sup \abs{y''}$.
\end{ex}
为了得到累积的全局误差, 注意到$w_1$对应的全局误差和局部误差一致. 从$w_1$开始, 求解
\[ \begin{cases}
    y' = f\pare{t,y}, \\
    y\pare{t_1} = w_1,\\
    t\in\brac{t_1,t_2}
\end{cases} \]
可得初始条件$\pare{t_1,w_1}$下的精确解$z\pare{t_2}$, 则$w_2$和精确解的差异由两部分给出,
\begin{align*}
    g_2 &= \abs{w_2 - y_2} \le \abs{w_2 - z\pare{t_2}} + \abs{z\pare{t_2} - y_2} \\
    &= e_2 + e^{Lh} g_1 = e_2 + e^{Lh} e_1.
\end{align*}
对于$i=3$类似,
\[ g_3 = \abs{w_3 - y_3} \le e_3 + e^{Lh}g_2 \le e_3 + e^{Lh}e_2 + e^{2Lh}e_1. \]
类似地, 在第$i$步的全局截断误差满足
\[ g_i = \abs{w_i - y_i} \le e_i + e^{Lh}e_{i-1} + e^{2Lh}e_{i-2} + \cdots + e^{\pare{i-1}Lh}e_1. \]
如果有$e_i \le Ch^{k+1}$, 则
\[ g_i \le Ch^{k+1}\frac{e^{iLh} - 1}{e^{Lh}-1} \le Ch^{k+1}\frac{e^{L\pare{t_i - a}} - 1}{Lh} = \frac{Ch^k}{L}\pare{e^{L\pare{t_i-a}} - 1}. \]
\begin{theorem}
    设$f\pare{t,y}$对$y$有Lipschitz常数$L$, 设局部截断误差满足$e_i \le C h^{k+1}$, 则
    \[ g_i = \abs{w_i - y_i} \le \frac{Ch^k}{L}\pare{e^{L\pare{t_i - a}} - 1}. \]
\end{theorem}
此时谓该方法为$k$阶方法, 例如Euler方法的局部截断误差界为$Mh^2/2$, 故为一阶方法.
\begin{corollary}[Euler方法收敛]
    设$f\pare{t,y}$对$y$有Lipschitz常数$L$, 则Euler方法下
    \[ \abs{w_i - y_i} \le \frac{Mh}{2L}\pare{e^{L\pare{t_i-a}} - 1}, \]
    其中$M = \sup \abs{y''}$.
\end{corollary}
\begin{theorem}[显式梯形方法, 改进的Euler方法, Heun方法]
    \[ \begin{cases}
        \displaystyle w_0 = y_0,\\
        \displaystyle w_{i+1} = w_i + \frac{h}{2}\brac{f\pare{t_i,w_i} + f\pare{t_i+h,w_i+hf\pare{t_i+w_i}}}.
    \end{cases} \]
\end{theorem}
由链式法则,
\[ y''\pare{t} = \+DtDf\pare{t,y} + \+DyDf\pare{t,y}y'\pare{t} = \+DtDf\pare{t,y} + \+DyDf\pare{t,y}f\pare{t,y}. \]
故
\[ y_{i+1} = y_i + hf\pare{t_i,y_i} + \frac{h^2}{2}\pare{\+DtDf\pare{t_i,y_i} + \+DyDf\pare{t_i,y_i}f\pare{t_i,y_i}} + \frac{h^3}{6}y'''\pare{c}. \]
显式梯形方法为
\begin{align*}
    w_{i+1} &= y_i + \frac{h}{2}f\pare{t_i,y_i} + \frac{h}{2}\curb{f\pare{t_i,y_i} + h\brac{\+DtDf\pare{t_i,y_i} + f\pare{t_i,y_i}\+DyDf\pare{t_i,y_i}}+O\pare{h^2}} \\
    &= y_i + hf\pare{t_i,y_i} + \frac{h^2}{2}\brac{\+DtDf\pare{t_i,y_i} + f\pare{t_i,y_i}\+DyDf\pare{t_i,y_i}} + O\pare{h^3}.
\end{align*}
可得局部截断误差
\[ y_{i+1} - w_{i+1} = O\pare{h_3}. \]
从而全局误差$\propto h^3$, 是二阶方法.
\begin{theorem}[$k$阶Taylor方法]
    \[ \begin{cases}
        \displaystyle w_0 = y_0, \\
        \displaystyle w_{i+1} = w_i + hf\pare{t_i,w_i} + \frac{h^2}{2}f'\pare{t_i,w_i} + \cdots + \frac{h^k}{k!}f^{\pare{k-1}}\pare{t_i,w_i}.
    \end{cases} \]
    其中$f'$表示全导数, 例如
    \[ f'\pare{t,y} = f_t\pare{t,y} + f_y\pare{t,y}y'\pare{t} = f_t\pare{t,y} + f_y\pare{t,y} f\pare{t,y}. \]
\end{theorem}
可以发现相应的
\[ y_{i+1} - w_{i+1} = \frac{h^{k+1}}{\pare{k+1}!}y^{\pare{k+1}}\pare{c}, \]
从而该方法为$k$阶方法.

% subsubsection ivp求解器的分析 (end)

\subsubsection{常微分方程组} % (fold)
\label{ssub:常微分方程组}

一阶方程组的形式如
\[ \begin{cases}
    y'_1 = f_1\pare{t,y_1,\cdots,y_n}, \\
    y'_2 = f_2\pare{t,y_1,\cdots,y_n}, \\
    \vdots \\
    y'_n = f_n\pare{t,y_1,\cdots,y_n}.
\end{cases} \]
\begin{ex}
    对两个一阶方程系统使用Euler方法,
    \[ \begin{cases}
        y'_1 = y_2^2 - 2y_1, \\
        y'_2 = y_1 - y_2 - ty_2^2, \\
        y_1\pare{0} = 0,\\
        y_2\pare{0} = 1,
    \end{cases} \Rightarrow \begin{cases}
        w_{i+1,1} = w_{i,1} + h\pare{w_{i,2}^2 - 2w_{i,1}}, \\
        w_{i+1,2} = w_{i,2} + h\pare{w_{i,1} - w_{i,2} + t_i w_{i,2}^2}.
    \end{cases} \]
\end{ex}
高阶方程可以转化为常微分方程组. 对于
\[ y^{\pare{n}} = f\pare{t,y',y'',\cdots,y^{\pare{n-1}}}, \]
设
\[ y_1 = y,\quad y_2 = y',\quad y_3 = y'',\quad \cdots,\quad y_n = y^{\pare{n-1}}, \]
并将原始的常微分方程写成
\[ y'_n = f\pare{t,y_1,y_2,\cdots,y_n} \]
即可.
\begin{ex}
    三阶微分方程
    \[ y''' + a\pare{y''}^2 - y' + yy'' + \sin t \]
    可转化为
    \[ \begin{cases}
        y'_1 = y_2, \\
        y'_2 = y_3, \\
        y'_3 = ay_3^2 - y_2 + y_1y_3 + \sin t.
    \end{cases} \]
\end{ex}

% subsubsection 常微分方程组 (end)

\subsubsection{Runge-Kutta方法} % (fold)
\label{ssub:runge_kutta方法}

\begin{theorem}[中点方法]
    $\displaystyle \begin{cases}
        w_0 = y_0, \\
        \displaystyle w_{i+1} = w_i + hf\pare{t_i + \frac{h}{2}, w_i + \frac{h}{2}f\pare{t_i,w_i}}.
    \end{cases}$
\end{theorem}
为了计算局部截断误差,
\begin{align*}
    y_{i+1} &= y_i + hf\pare{t_i,y_i} + \frac{h^2}{2}\brac{\+DtDf\pare{t_i,y_i} + \+DyDf\pare{t_i,y_i}f\pare{t_i,y_i}} + \frac{h^3}{6}y'''\pare{c}, \\
    w_{i+1} &= y_i + h\brac{f\pare{t_i,y_i} + \frac{h}{2}\+DtDf\pare{t_i,y_i} + \frac{h}{2}\+DyDf\pare{t_i,y_i} + O\pare{h^2}}.
\end{align*}
从而
\[ y_{i+1} - w_{i+1} = O\pare{h^3}, \]
从而中点方法为二阶方法.
\par
右侧每个函数的求值谓该方法的阶段. 梯形方法和中点方法都是二阶Runge-Kutta方法家族中的成员, 形式如
\[ w_{i+1} = w_i + h\pare{1-\rec{2\alpha}}f\pare{t_i,w_i} + \frac{h}{2\alpha}f\pare{t_i + \alpha h, w_i + \alpha hf\pare{t_i,w_i}}. \]
其中$\alpha = 1$对应显式梯形方法, $\alpha = 1/2$对应中点方法.
\begin{theorem}[$4$阶Runge-Kutta方法]
    \[ w_{i+1} = w_i + \frac{h}{6}\pare{s_1 + 2s_2 + 2s_3 + s_4}. \]
    其中
    \begin{align*}
        s_1 &= f\pare{t_i,w_i}, \\
        s_2 &= f\pare{t_i + \frac{h}{2}, w_i + \frac{h}{2}s_1}, \\
        s_3 &= f\pare{t_i + \frac{h}{2}, w_i + \frac{h}{2}s_2}, \\
        s_4 &= f\pare{t_i + h, w_i + hs_3}.
    \end{align*}
\end{theorem}

% subsubsection runge_kutta方法 (end)

\subsubsection{可变步长方法} % (fold)
\label{ssub:可变步长方法}

若某一步的相对误差目标$e_i/\abs{w_i} < T$满足, 假设$e_i = ch_i^{p+1}$, 则满足容差$e_i/\abs{w_i} < T$的最优步长为
\[ T\abs{w_i} = ch^{p+1}. \]
对于$h$和$c$, 可得
\[ h_* = 0.8 \pare{\frac{T\abs{w_i}}{e_i}}^{1/\pare{p+1}}h_i, \]
其中$0.8$为安全因子. 每一步中若相对误差目标$e_i/\abs{w_i} < T$未满足, 则简单将步长减半直到满足目标即可.
\begin{ex}[Runge-Kutta $2$阶/$3$阶嵌入对]
    \[ \begin{cases}
        \displaystyle w_{i+1} = w_i + h \frac{s_1 + s_2}{2}, \\
        \displaystyle z_{i+1} = w_i + h \frac{s_1 + 4s_3 + s_2}{6},
    \end{cases} \]
    其中
    \begin{align*}
        s_1 &= f\pare{t_i,w_i}, \\
        s_2 &= f\pare{t_i + h, w_i + hs_1}, \\
        s_3 &= f\pare{t_i + \half h, w_i + \half h \frac{s_1+s_2}{2}}.
    \end{align*}
    可得局部误差的估计
    \[ e_{i+1} \approx \abs{w_{i+1} - z_{i+1}} = \abs{h\frac{s_1 - 2s_3+s_2}{3}}. \]
\end{ex}
\begin{ex}[Bogacki-Shampine $2$阶/$3$阶嵌入对]
    \[ \begin{cases}
        \displaystyle z_{i+1} = w_i + \frac{h}{9}\pare{2s_1 + 3s_2 + 4s_3}, \\
        \displaystyle w_{i+1} = w_i + \frac{h}{24}\pare{7s_1 + 6s_2 + 8s_3 + 3s_4},
    \end{cases} \]
    其中
    \begin{align*}
        s_1 &= f\pare{t_i,w_i}, \\
        s_2 &= f\pare{t_i + \half h, w_i + \half hs_1}, \\
        s_3 &= f\pare{t_i + \frac{3}{4}h, w_i + \frac{3}{4}hs_2}, \\
        s_4 &= f\pare{t+h, z_{i+1}}.
    \end{align*}
    $z_{i+1}$是$3$阶估计, $w_{i+1}$是$2$阶估计, 步长控制误差估计为
    \[ e_{i+1} \approx \abs{z_{i+1} - w_{i+1}} = \frac{h}{72}\pare{7s_1 + 6s_2 + 8s_3 + 3s_4}. \]
    如果$s_4$被接受, 下一步中可直接作为$s_1$.
\end{ex}
\begin{ex}[Runge-Kutta $4$阶/$5$阶嵌入对]
    \[ \begin{cases}
        \displaystyle w_{i+1} = w_i + h\pare{\frac{25}{216}s_1 + \frac{1408}{2565}s_3 + \frac{2197}{4104}s_4 - \rec{5}s_5}, \\
        \displaystyle z_{i+1} = w_i + h\pare{\frac{16}{135}s_1 + \frac{6656}{12825}s_3 + \frac{28561}{56430}s_4 - \frac{9}{50}s_5 + \frac{2}{55}s_6},
    \end{cases} \]
    其中
    \begin{align*}
        s_1 &= f\pare{t_i,w_i}, \\
        s_2 &= f\pare{t_i + \rec{4}h, w_i + \rec{4}hs_1}, \\
        s_3 &= f\pare{t_i + \frac{3}{8}h, w_i + \frac{3}{32}hs_1 + \frac{9}{32}hs_2}, \\
        s_4 &= f\pare{t_i + \frac{12}{13}, w_i + \frac{1932}{2197}hs_1 - \frac{7200}{2197}hs_2 + \frac{7296}{2197}hs_3}, \\
        s_5 &= f\pare{t_i + h, w_i + \frac{439}{216}hs_1 - 8hs_2 + \frac{3680}{513}hs_3 - \frac{845}{4104}hs_4}, \\
        s_6 &= f\pare{t_i + \half h, w_i - \frac{8}{27} hs_1 + 2hs_2 - \frac{3544}{2565}hs_3 + \frac{1859}{4104}hs_4 - \frac{11}{40}hs_5}.
    \end{align*}
    可以发现$z_{i+1}$是$5$阶近似, $w_{i+1}$是$4$阶近似. 误差估计为
    \[ e_{i+1} \approx \abs{w_{i+1} - z_{i+1}} = h\abs{\rec{360}s_1 - \frac{128}{4275}s_3 - \frac{2197}{75240}s_4 + \rec{50}s_5 + \frac{2}{55}s_6}. \]
\end{ex}
若$i=1$处相对误差测试$e_i/\abs{w_i} < T$成功, 则新的$w_1$被替换为局部外推的$z_1$, 程序移动到下一步. 否则按
\[ h_* = 0.8 \pare{\frac{T\abs{w_i}}{e_i}}^{1/\pare{p+1}}h_i \]
得到新的步长, 其中$p=4$. 若失败则将步长不断减半直到成功通过测试.
\begin{ex}[Dormand-Prince $4$阶/$5$阶嵌入对]
    \[ \begin{cases}
        \displaystyle z_{i+1} = w_i + h\pare{\frac{35}{384}s_1 + \frac{500}{1113}s_3 + \frac{125}{192}s_4 - \frac{2187}{6784}s_5 + \frac{11}{84}s_6}, \\
        \displaystyle w_{i+1} = w_i + h\pare{\frac{5179}{57600}s_1 + \frac{7571}{16695}s_3 + \frac{393}{640}s_4 - \frac{92097}{339220}s_5 + \frac{187}{2100}s_6 + \rec{40}s_7},
    \end{cases} \]
    其中
    \begin{align*}
        s_1 &= f\pare{t_i,w_i}, \\
        s_2 &= f\pare{t_i + \rec{5}h, w_i + \rec{5}hs_1}, \\
        s_3 &= f\pare{t_i + \frac{3}{10}h, w_i + \frac{3}{40}hs_1 + \frac{9}{40}hs_2}, \\
        s_4 &= f\pare{t_i + \frac{4}{5}h, w_i + \frac{44}{45}hs_1 - \frac{56}{15}hs_2 + \frac{32}{9}hs_3}, \\
        s_5 &= f\pare{t_i + \frac{8}{9}h, w_i + h\pare{\frac{19372}{6561}s_1 - \frac{25360}{2187}s_2 + \frac{64448}{6561}s_3 - \frac{212}{729}s_4}}, \\
        s_6 &= f\pare{t_i + h, w_i + h\pare{\frac{9017}{3168}s_1 - \frac{355}{33}s_2 + \frac{46732}{5247}s_3 + \frac{49}{176}s_4 - \frac{5103}{18656}s_5}}, \\
        s_7 &= f\pare{t_i + h, z_{i+1}}.
    \end{align*}
    误差估计为
    \[ e_{i+1} = \abs{z_{i+1} - w_{i+1}} = h\abs{\frac{71}{57600}s_1 - \frac{71}{16695}s_3 + \frac{71}{1920}s_4 - \frac{17253}{339200}s_5 + \frac{22}{525}s_6 - \rec{40}s_7}. \]
\end{ex}

% subsubsection 可变步长方法 (end)

\subsubsection{隐式方法与刚性方程} % (fold)
\label{ssub:隐式方法与刚性方程}

$f\pare{t,y} = 10\pare{1-y}$之类型的方程对于某些$h$难以达到平衡, 即吸引解被附近变化更快的解包围, 谓刚性方程. 通过平衡点处的$\displaystyle \+DyDf = -10$可得其为刚性之结论.
\begin{theorem}[后向Euler方法]%
    $\displaystyle \begin{cases}
        w_0 = y_0, \\
        w_{i+1} = w_i + h\pare{t_{i+1}, w_{i+1}}.
    \end{cases}$
\end{theorem}
\begin{ex}
    对初值问题使用后向Euler方法
    \[ \begin{cases}
        y' = y + 8y^2 - 9y^3, \\
        y\pare{0} = 1/2, \\
        t \in \brac{0,3},
    \end{cases} \]
    即
    \[ w_{i+1} = w_i + hf\pare{t_{i+1},w_{i+1}} = w_i + h\pare{w_{i+1} + 8w_{i+1}^2 - 9w{i+1}^3}. \]
\end{ex}

% subsubsection 隐式方法与刚性方程 (end)

\subsubsection{多步方法} % (fold)
\label{ssub:多步方法}

\begin{theorem}[Adams-Bashforth两步方法]
    \[ w_{i+1} = w_i + h\brac{\frac{3}{2}f\pare{t_i,w_i} - \rec{2}f\pare{t_{i-1}, w_{i-1}}}. \]
\end{theorem}
在使用多步方法时, 必须先用单步方法获得$w_1,w_2,\cdots,w_{s-1}$才能应用多步方法的递推式.
\par
一般的$s$步方法的形式为
\[ w_{i+1} = a_1w_i + a_2w_{i-1} + \cdots + a_s w_{i-s+1} + h\pare{b_0 + f_{i+1} + b_1 f_{i} + b_2f_{i-1} + \cdots + b_s f_{i-s+1}}, \]
其中$f_i = f\pare{t_i,w_i}$. 如果$b_0 = 0$则谓之显式方法, 否则谓隐式方法.
\par
一般的两步方法的形式为
\[ w_{i+1} = a_1 w_i + a_2 w_{i-1} + h\brac{b_0 f_{i+1} + b_1f_i + b_2 f_{i-1}}. \]
Taylor展开后有
\begin{align*}
    w_{i+1} &= a_1 w_i + a_2 w_{i-1} + h\brac{b_0f_{i+1} + b_1 f_i + b_2 f_{i-1}} = a_1\brac{y_i} \\
    &= a_2\brac{y_i - hy'_i + \frac{h^2}{2}y''_i - \frac{h^3}{6}y'''_i + \frac{h^4}{24}y''''_i - \cdots} \\
    &\phantom{=\ }+ b_0\brac{hy'_i + h^2 y'' + \frac{h^3}{2}y'''_i + \frac{h^4}{6}y''''_i + \cdots} \\
    &\phantom{=\ }+ b_1\brac{h y'_i} \\
    &\phantom{=\ }+ b_2\brac{hy'_i - h^2y''_i + \frac{h^3}{2}y'''_i - \frac{h^4}{6}y''''_i + \cdots} \\
    &= \pare{a_1 + a_2}y_i + \pare{b_0 + b_1 + b_2 - a_2}hy'_i + \pare{a_2 - 2b_2 + 2b_0} \frac{h^2}{2} y''_i \\
    &\phantom{=\ } + \pare{-a_2 + 3b_0 + 3b_2}\frac{h^3}{6}y'''_i + \pare{a_2 + 4b_0 - 4b_2}\frac{h^4}{24}y''''_i + \cdots.
\end{align*}
适当选取$a_i$, $b_i$即可使之满足
\[ y_{i+1} = y_i + hy'_i + \frac{h^2}{2}y''_i + \frac{h^3}{6}y'''_i + \cdots. \]

\paragraph{显式多步方法} % (fold)
\label{par:显式多步方法}

设$b_0 = 0$, 可得
\[ \begin{cases}
    a_1 + a_2 = 1,\\
    -a_2 + b_1 + b_2 = 1, \\
    a_2 - 2b_2 = 1
\end{cases} \Rightarrow \begin{cases}
    a_2 = 1-a_1, \\
    \displaystyle b_1 = 2-\half a_1, \\
    \displaystyle b_2 = -\half a_1.
\end{cases} \]
局部误差为
\[ y_{i+1} - w_{i+1} = \frac{4+a_1}{12}h^3 y'''_i + O\pare{h^4}. \]
\begin{cenum}
    \item 设置$a_1 = 1$可得二阶Adams-Bashforth方法. 其局部截断误差为$\displaystyle \frac{5}{12}h^3 y'''_i + O\pare{h^4}$.
    \item 设置$a_1 = 1/2$可得二阶两步方法
    \[ w_{i+1} = \half w_i + \half w_{i-1} + h\pare{\frac{7}{4}f_i - \rec{4}f_{i-1}},\quad y_{i+1} - w_{i+1} = \frac{3}{8}h^3 y'''\pare{t_i} + O\pare{h^4}. \]
    \item 设置$a_1 = -a$可得二阶两步方法
    \[ w_{i+1} = -w_i + 2w_{i-1} + h\pare{\frac{5}{2}f_i + \half f_{i-1}}. \]
\end{cenum}
第三种选择给出了不稳定的方法. 例如对于
\[ \begin{cases}
    y' = 0,\\
    y\pare{0} = 0, \\
    t\in\brac{0,1},
\end{cases} \]
可得
\[ w_{i+1} = -w_i + 2w_{i-1} + h\brac{0}, \]
从而$w_i \equiv 0$式一个解. 但这一递推式的特征多项式有根$\curb{1,-2}$, 故形如$\pare{-2}^i c$的数列也是一解.

% paragraph 显式多步方法 (end)

\begin{definition}
    若多项式$P\pare{x} = x^s - a_1 x^{s-1} - \cdots - a_s$的根的绝对值的界为$1$, 任何绝对值为$1$的根是单根, 则多步方法稳定. 若$1$是唯一一个绝对值为$1$的单根, 则谓之强稳定, 否则谓之弱稳定.
\end{definition}
Adams-Bashforth方法的根为$\curb{0,1}$, 故为强稳定. 可以验证
\[ w_{i+1} = w_{i-1} + 2hf_i \]
是弱稳定的, 特征多项式的根为$\curb{-1,1}$.

\begin{definition}
    若一多步方法至少为$1$阶, 则谓之一致. 若$h\rightarrow 0$时对于每个$t$, 近似解都收敛到精确解, 则谓之收敛.
\end{definition}
\begin{theorem}[Dahlquist]
    若初值正确, 则多步方法收敛当且仅当它稳定并一致.
\end{theorem}
\begin{theorem}[Adams-Bashforth三步方法(三阶)]
    \[ w_{i+1} = w_i + \frac{h}{12}\brac{23 f_i - 16f_{i-1} + 5f_{i-2}}. \]
\end{theorem}
\begin{theorem}[Adams-Bashforth四步方法(四阶)]
    \[ w_{i+1} = w_i + \frac{h}{24}\brac{55f_i - 59f_{i-1} + 37f_{i-2} - 9f_{i-3}}. \]
\end{theorem}

\paragraph{隐式多步方法} % (fold)
\label{par:隐式多步方法}

此时$b_0 \neq 0$.
\begin{theorem}[隐式梯形方法(二阶)]
    \[ w_{i+1} = w_i + \frac{h}{2}\brac{f_{i+1} + f_i}. \]
\end{theorem}
\begin{theorem}[Adams-Moulton两步方法(三阶)]
    \[ w_{i+1} = w_i + \frac{h}{12}\brac{5f_{i+1} +8f_i - f_{i-1}} \]    
\end{theorem}
为了避免对函数球逆, 通常使用预测-矫正方法: 用显式方法预测$w_{i+1}$后代入隐式方法. 在Taylor展开中匹配相应的系数可得
\[ \begin{cases}
    a_1 + a_2 = 1,\\
    -a_2 + b_0 + b_1 + b_2 = 1, \\
    a_2 + 2b_0 - 2b_2 = 1, \\
    -a_2 + 3b_0 + 3b_2,
\end{cases} \Rightarrow \begin{cases}
    \displaystyle a_2 = 1 - a_1, \\
    \displaystyle b_0 = \rec{3} + \rec{12}a_1, \\[.5em]
    \displaystyle b_1 = \frac{4}{3} - \frac{2}{3}a_1, \\[.5em]
    \displaystyle b_2 = \rec{3} - \frac{5}{12}a_1.
\end{cases} \]
局部截断误差为
\[ y_{i+1} - w_{i+1} = -\frac{a_1}{24}h^4y''''_i = O\pare{h^5}. \]
若$a_1 = 1$则可得Adams-Moulton方法. 若选择$a_1 = 0$则可得Milne-Simpson方法.
\begin{theorem}[Milne-Simpson方法]
    \[ w_{i+1} = w_{i-1} + \frac{h}{3}\brac{f_{i+1} + 4f_i + f_{i-1}}. \]
\end{theorem}
大多数多步方法都可以通过积分插值近似得到, 即将
\[ y\pare{t_{i+1}} - y\pare{t_i} = \int_{t_i}^{t_{i+1}} f\pare{t,y}\,\rd{t} \]
以梯形公式近似即得到隐式梯形方法, 以Simpson公式近似即得到Milne-Simpson方法.
\begin{theorem}[Adams-Moulton $3$步方法($4$阶)]
    \[ w_{i+1} = w_i + \frac{h}{24}\pare{9f_{i+1} + 19f_i - 5f_{i-1} + f_{i-2}}. \]
\end{theorem}
\begin{theorem}[Adams-Moulton $4$步方法($5$阶)]
    \[ w_{i+1} = w_i + \frac{h}{720}\pare{251f_{i+1} + 646f_i - 264f_{i-1} + 106f_{i-2} - 19f_{i-3}}. \]
\end{theorem}

% paragraph 隐式多步方法 (end)

% subsubsection 多步方法 (end)

% subsection 常微分方程 (end)

\subsection{边值问题} % (fold)
\label{sub:边值问题}

\subsubsection{打靶方法} % (fold)
\label{ssub:打靶方法}

对于边值问题
\[ \begin{cases}
    y'' = f\pare{t,y,y'}, \\
    y\pare{a} = y_a, \\
    y\pare{b} = y_b,
\end{cases} \]
定义下面的函数:
\[ F\pare{s} = \begin{cases}
    y_b - y\pare{b},\\
    \text{其中$y\pare{t}$为初值问题$y\pare{a} = y_a, y'\pare{a} = s$的解.}
\end{cases} \]
通过方程求解器可以得到$F\pare{s}$的根, 即得到边值问题的解.

% subsubsection 打靶方法 (end)

\subsubsection{有限差分方法} % (fold)
\label{ssub:有限差分方法}

将$y'\pare{t}$和$y''\pare{t}$分别用\cref{thm:三点中心差分公式}和\cref{thm:二阶导数的三点中心差分公式}代替即可将微分方程转化为代数方程.
\par
若微分方程关于$y, y', y''$线性, 则可将方程转化为三对角线性方程. 否则, 方程组为非线形, 须使用Newton方法求解, 迭代式
\[ DF\pare{w^k}\Delta w = -F\pare{w^k}. \]

% subsubsection 有限差分方法 (end)

\subsubsection{排列与有限元方法} % (fold)
\label{ssub:排列与有限元方法}

\paragraph{排列方法} % (fold)
\label{par:排列方法}

选取一组基函数使得
\[ y\pare{t} = \sum_{j=1}^n c_j\phi_j\pare{t} \]
后, 方程即变为代数方程.

% paragraph 排列方法 (end)

\paragraph{有限元以及Galerkin方法} % (fold)
\label{par:有限元以及galerkin方法}

选取基函数后, 设法令
\[ \int_a^b y'' \phi_i\pare{t}\,\rd{t} = \int_a^b f\pare{t,y,y'}\phi_i\pare{t}\,\rd{t}, \]
可得边值问题的弱形式. 分布积分后转化为
\[ \int_a^b f\pare{t,y,y'}\phi_i\pare{t}\,\rd{t} = \phi_i\pare{b}y'\pare{b} - \phi_i\pare{a}y'\pare{a} - \int_a^b y'\pare{t}\phi'_i\pare{t}\,\rd{t}, \]
即可以如下形式求解$c_i$,
\[ y\pare{t} = \sum_{i=0}^{n+1} c_i\phi\pare{t}. \]
设$t$轴上数据格点为$t_0 < t_1 < \cdots < t_n < t_{n+1}$. 对于$i = 1,\cdots,n$, 定义
\[ \phi_i\pare{t} = \begin{cases}
    \pare{t - t_{i-1}} / \pare{t_i - t_{i-1}}, & t_{i-1} < t \le t_i, \\
    \pare{t_{i+1} - t} / \pare{t_{i+1} - t_i}, & t_i < t < t_{i+1}, \\
    0, & \mathrm{otherwise}.
\end{cases} \]
同时定义
\[ \phi_0\pare{t} = \begin{cases}
    \pare{t_1 - t} / \pare{t_1 - t_{0}}, & t_{0} \le t < t_1, \\
    0, & \mathrm{otherwise},
\end{cases}\quad \phi_{n+1}\pare{t} = \begin{cases}
    \pare{t - t_n} / \pare{t_{n+1} - t_{n}}, & t_{n} \le t < t_{n+1}, \\
    0, & \mathrm{otherwise}.
\end{cases} \]
这些函数有性质$\phi_i\pare{t_j} = \delta_{ij}$. 定义分段线性B样条
\[ S\pare{t} = \sum_{i=0}^{n+1} c_i\phi_i\pare{t}, \]
可知系数恰好为$y$坐标. 代入弱形式可得
\[ \int_a^b \phi_i\pare{t} f\pare{t,\sum c_j \phi_j\pare{t}, \sum c_j\phi'_j\pare{t}}\,\rd{t} + \int_a^b \phi'_i\pare{t} \sum c_j\phi'_j\pare{t}\,\rd{t} = 0. \]
对于均匀分布, 步长为$h$的格点, 有
\begin{align*}
    \int_a^b \phi_i\pare{t}\phi_{i+1}\pare{t}\,\rd{t} &= \frac{h}{6},\quad \int \phi_i^2\pare{t}\,\rd{t} = \frac{2}{3}h, \\
    \int_a^b \phi'_i\pare{t}\phi'_{i+1}\pare{t}\,\rd{t} &= -\rec{h},\quad \int \phi_i'^2\pare{t}\,\rd{t} = \frac{2}{h}.
\end{align*}

% paragraph 有限元以及galerkin方法 (end)

% subsubsection 排列与有限元方法 (end)

% subsection 边值问题 (end)

% section 求解方程 (end)

\end{document}
