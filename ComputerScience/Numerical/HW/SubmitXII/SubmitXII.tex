\documentclass[hidelinks]{ctexart}

\usepackage{van-de-la-illinoise}
\usepackage[paper=b5paper,top=.3in,left=.9in,right=.9in,bottom=.3in]{geometry}
\usepackage{calc}
\pagenumbering{gobble}
\setlength{\parindent}{0pt}
\sisetup{inter-unit-product=\ensuremath{{}\cdot{}}}
\usepackage{van-le-trompe-loeil}

\usepackage{stackengine}
\stackMath
\usepackage{scalerel}
\usepackage[outline]{contour}

\newdimen\indexlen
\def\newprobheader#1{%
\def\probindex{#1}
\setlength\indexlen{\widthof{\textbf{\probindex}}}
\hskip\dimexpr-\indexlen-1em\relax
\textbf{\probindex}\hskip1em\relax
}
\def\newprob#1{%
\newprobheader{#1}%
\def\newprob##1{%
\probsep%
\newprobheader{##1}%
}%
}
\def\probsep{\vskip1em\relax{\color{gray}\dotfill}\vskip1em\relax}

\newlength\thisletterwidth
\newlength\gletterwidth
\newcommand{\leftrightharpoonup}[1]{%
{\ooalign{$\scriptstyle\leftharpoonup$\cr%\kern\dimexpr\thisletterwidth-\gletterwidth\relax
$\scriptstyle\rightharpoonup$\cr}}\relax%
}
\def\tensor#1{\settowidth\thisletterwidth{$\mathbf{#1}$}\settowidth\gletterwidth{$\mathbf{g}$}\stackon[-0.1ex]{\mathbf{#1}}{\boldsymbol{\leftrightharpoonup{#1}}}  }
\def\onedot{$\mathsurround0pt\ldotp$}
\def\cddot{% two dots stacked vertically
  \mathbin{\vcenter{\baselineskip.67ex
    \hbox{\onedot}\hbox{\onedot}}%
}}%
\newcommand{\tmpresumath}[1]{\tcboxmath[colback=emphgreen, boxrule=.3pt, sharp corners=all,left=.1mm,right=.1mm,top=.1mm,bottom=.1mm]{#1}}

\begin{document}

\newprob{补充题}%
设$f\pare{x}$在$x_0$处有$r$重根, 则$f\pare{x} = \pare{x-x_0}^r h\pare{x}$,
\begin{align*}
    x_{n+1} &= \varphi\pare{x_n} = x_n - r\frac{f\pare{x_n}}{f'\pare{x_n}},\quad \varphi\pare{x} = x - r\frac{f\pare{x}}{f'\pare{x}}, \\
    \varphi\pare{x} &= x - \frac{\pare{x-x_0}^n h\pare{x}}{n\pare{x-x_0}^{n-1}h\pare{x} + \pare{x-x_0}^n h'\pare{x}} = x_0 + \frac{\pare{x-x_0}^2h\pare{x}}{\pare{x-x_0}h'\pare{x} + rh\pare{x}}.
\end{align*}
因此
\[ \frac{\abs{e_{k+1}}}{\abs{e_k}^2} = \abs{\frac{\varphi\pare{x_k} - {x_0}}{\pare{x_k - x_0}^2}} = \abs{\frac{\varphi\pare{x_k} - \varphi\pare{x_0}}{\pare{x_k - x_0}^2}} = \abs{\frac{O\pare{x_k-x_0}^2}{\pare{x_k-x_0}^2}} = O\pare{1}, \]
故为二阶收敛.

\newprob{3.2}%
$\bullet$等效为$2x^3 - 5x^2 - 18x + 42 = x$,
\[ \left. \+dxd{}\pare{2x^3 - 5x^2 - 18x + 42}\right\vert_{x=3.0} = 6.0 \notin \pare{-1,1}, \]
故在$x=3.0$附近迭代不收敛.
\par
$\bullet$等效为$\displaystyle \rec{5}\pare{2x^2 - 19 + \frac{42}{x}} = x$,
\[ \left. \+dxd{}\brac{\rec{5}\pare{2x^2 - 19 + \frac{42}{x}}}\right\vert_{x=3.0} = 1.467 \notin \pare{-1,1}, \]
故在$x=3.0$附近迭代不收敛.
\par
$\bullet$等效为$\displaystyle \sqrt[3]{\frac{5x^2 + 19x - 42}{2}} = x$,
\[ \left. \+dxd{}\pare{\sqrt[3]{\frac{5x^2 + 19x - 42}{2}}}\right\vert_{x=3.0} = 0.8459 \in \pare{-1,1}, \]
且导数在$x>3$时$\in \pare{-1,1}$, 且$x>3$时$\displaystyle \sqrt[3]{\frac{5x^2 + 19x - 42}{2}} > 3$, 故$x=3.0$附近迭代收敛.

\newprob{3.3}%
$f\pare{0} = -1$, $f\pare{1} = 1$, 故$\brac{0,1}$上有根.
\[ \begin{array}{llllll}
    a & f\pare{a} & b & f\pare{b} & \displaystyle \frac{a+b}{2} & \displaystyle f\pare{\frac{a+b}{2}} \\
    0 & -1 & 1 & 1 & 0.5 & -0.625 \\
0.5 & -0.625 & 1 & 1 & 0.75 & -0.015625 \\
0.75 & -0.015625 & 1 & 1 & 0.875 & 0.43555 \\
0.75 & -0.015625 & 0.875 & 0.43555 & 0.8125 & 0.19653 \\
0.75 & -0.015625 & 0.8125 & 0.19653 & 0.78125 & 0.087189 \\
0.75 & -0.015625 & 0.78125 & 0.087189 & 0.76562 & 0.034977 \\
0.75 & -0.015625 & 0.76562 & 0.034977 & 0.75781 & 0.0094762 \\
0.75 & -0.015625 & 0.75781 & 0.0094762 & 0.75391 & -0.0031242 \\
0.75391 & -0.0031242 & 0.75781 & 0.0094762 & 0.75586 & 0.0031635 \\
0.75391 & -0.0031242 & 0.75586 & 0.0031635 & 0.75488 & 1.6567\times 10^{-5}
\end{array} \]
最终$\displaystyle x^* = \frac{0.75391 + 0.75488}{2} = \boxed{0.75439.}$

\newprob{3.5}%
$x^n - a = 0 \Rightarrow \displaystyle x_{k+1} = x_k - \frac{x_k^n-a}{nx_k^{n-1}}.$
\[ 2 \xrightarrow{x-\frac{x^5-9}{5x^4}} 1.7125 \xrightarrow{x-\frac{x^5-9}{5x^4}} 1.57929 \xrightarrow{x-\frac{x^5-9}{5x^4}} 1.55278 \xrightarrow{x-\frac{x^5-9}{5x^4}} \boxed{1.55185.}  \]

\newprob{3.7}%
记$f\pare{x} = x^3 - 3x - 2$, 则弦截法的迭代式为
\[ x_{n+1} = \frac{-f\pare{x_n} x_{n-1} + f\pare{x_{n-1}}x_n}{f\pare{x_{n-1}} - f\pare{x_n}}. \]
从而
\[ 1 \rightarrow 3 \rightarrow 1.4 \rightarrow 1.68421 \rightarrow 2.23188 \rightarrow 1.94949 \rightarrow 1.99286 \rightarrow 2.00025 \rightarrow \boxed{1.9999988.} \]
\newprob{3.8}%
$\displaystyle J = \begin{pmatrix}
    2x & 2y \\
    3x^2 & -1
\end{pmatrix} \Rightarrow J^{-1} = \rec{2x + 6x^2y}\begin{pmatrix}
    1 & 2y \\
    3x^2 & -2x
\end{pmatrix}.$
\[ \begin{pmatrix}
    0.8 \\
    0.6
\end{pmatrix} \xrightarrow{X=X - J^{-1}F} \begin{pmatrix}
    0.827049 \\
    0.563934
\end{pmatrix} \xrightarrow{X=X - J^{-1}F} \begin{pmatrix}
    0.826032 \\
    0.563624
\end{pmatrix} \xrightarrow{X=X - J^{-1}F} \boxed{\begin{pmatrix}
    x=0.826031 \\
    y=0.563624
\end{pmatrix}.} \]

\end{document}
