\documentclass[hidelinks]{ctexart}

\usepackage{van-de-la-illinoise}
\usepackage[paper=b5paper,top=.3in,left=.9in,right=.9in,bottom=.3in]{geometry}
\usepackage{calc}
\pagenumbering{gobble}
\setlength{\parindent}{0pt}
\sisetup{inter-unit-product=\ensuremath{{}\cdot{}}}
\usepackage{van-le-trompe-loeil}

\usepackage{stackengine}
\stackMath
\usepackage{scalerel}
\usepackage[outline]{contour}

\newdimen\indexlen
\def\newprobheader#1{%
\def\probindex{#1}
\setlength\indexlen{\widthof{\textbf{\probindex}}}
\hskip\dimexpr-\indexlen-1em\relax
\textbf{\probindex}\hskip1em\relax
}
\def\newprob#1{%
\newprobheader{#1}%
\def\newprob##1{%
\probsep%
\newprobheader{##1}%
}%
}
\def\probsep{\vskip1em\relax{\color{gray}\dotfill}\vskip1em\relax}

\newlength\thisletterwidth
\newlength\gletterwidth
\newcommand{\leftrightharpoonup}[1]{%
{\ooalign{$\scriptstyle\leftharpoonup$\cr%\kern\dimexpr\thisletterwidth-\gletterwidth\relax
$\scriptstyle\rightharpoonup$\cr}}\relax%
}
\def\tensor#1{\settowidth\thisletterwidth{$\mathbf{#1}$}\settowidth\gletterwidth{$\mathbf{g}$}\stackon[-0.1ex]{\mathbf{#1}}{\boldsymbol{\leftrightharpoonup{#1}}}  }
\def\onedot{$\mathsurround0pt\ldotp$}
\def\cddot{% two dots stacked vertically
  \mathbin{\vcenter{\baselineskip.67ex
    \hbox{\onedot}\hbox{\onedot}}%
}}%
\newcommand{\tmpresumath}[1]{\tcboxmath[colback=emphgreen, boxrule=.3pt, sharp corners=all,left=.1mm,right=.1mm,top=.1mm,bottom=.1mm]{#1}}
\DeclareMathOperator{\cond}{cond}

\begin{document}

\newprob{8.1 (1)}%
使用规范运算列表如下.
\begin{longtable}{>{$}c<{$}>{$}c<{$}>{$}c<{$}}
k & Y^{\pare{k}} & X^{\pare{k+1}} \\
1 & \pare{1.,-0.655359} & \pare{6.966077365538062,-4.68928} \\
2 & \pare{1.,-0.67316} & \pare{7.019478758378284,-4.65368} \\
3 & \pare{1.,-0.662967} & \pare{6.988900168204065,-4.67407} \\
4 & \pare{1.,-0.668784} & \pare{7.006352834654262,-4.66243} \\
5 & \pare{1.,-0.665458} & \pare{6.996373100353816,-4.66908} \\
6 & \pare{1.,-0.667358} & \pare{7.00207358846886,-4.66528} \\
7 & \pare{1.,-0.666272} & \pare{6.998815443201124,-4.66746} \\
8 & \pare{1.,-0.666892} & \pare{7.0006770041636255,-4.66622} \\
9 & \pare{1.,-0.666538} & \pare{6.9996131778893815,-4.66692} \\
10 & \pare{1.,-0.66674} & \pare{7.000221053421546,-4.66652} \\
11 & \pare{1.,-0.666625} & \pare{6.999873687747939,-4.66675} \\
12 & \pare{1.,-0.666691} & \pare{7.000072179732204,-4.66662} \\
13 & \pare{1.,-0.666653} & \pare{6.999958754864036,-4.66669} \\
14 & \pare{1.,-0.666675} & \pare{7.000023568787992,-4.66665} \\
15 & \pare{1.,-0.666662} & \pare{6.999986532166492,-4.66668} \\
16 & \pare{1.,-0.666669} & \pare{7.000007695919667,-4.66666} \\
17 & \pare{1.,-0.666665} & \pare{6.999995602336452,-4.66667} \\
18 & \pare{1.,-0.666668} & \pare{7.000002512952177,-4.66666} \\
19 & \pare{1.,-0.666666} & \pare{6.999998564027843,-4.66667} \\
20 & \pare{1.,-0.666667} & \pare{7.000000820555687,-4.66667}
\end{longtable}
得到按模最大特征值及其特征向量
\[ \boxed{\lambda_1 = 7.00000,\quad v_1 = \begin{pmatrix}
    1 \\
    0.66667
\end{pmatrix}.} \]
\newprob{8.2 (1)}%
对$\displaystyle A^{-1} = \begin{pmatrix}
    -0.5 & 1.5 \\
    1.5 & -3.5
\end{pmatrix}$使用规范运算列表如下.
\begin{longtable}{>{$}c<{$}>{$}c<{$}>{$}c<{$}}
k & Y^{\pare{k}} & X^{\pare{k+1}} \\
1 & \pare{1.00000000,1.00000000} & \pare{1.00000000,-2.00000000} \\
2 & \pare{0.50000000,-1.00000000} & \pare{-1.75000000,4.25000000} \\
3 & \pare{-0.41176471,1.00000000} & \pare{1.70588235,-4.11764706} \\
4 & \pare{0.41428571,-1.00000000} & \pare{-1.70714286,4.12142857} \\
5 & \pare{-0.41421144,1.00000000} & \pare{1.70710572,-4.12131716} \\
6 & \pare{0.41421362,-1.00000000} & \pare{-1.70710681,4.12132044} \\
7 & \pare{-0.41421356,1.00000000} & \pare{1.70710678,-4.12132034} \\
8 & \pare{0.41421356,-1.00000000} & \pare{-1.70710678,4.12132034} \\
9 & \pare{-0.41421356,1.00000000} & \pare{1.70710678,-4.12132034} \\
10 & \pare{0.41421356,-1.00000000} & \pare{-1.70710678,4.12132034}
\end{longtable}
得到$A^{-1}$按模最大特征值及其特征向量
\[ \lambda_1 = -4.12132034,\quad v_1 = \begin{pmatrix}
    0.41421356 \\
    -1
\end{pmatrix}. \]
故$A$按模最小特征值和特征向量为
\[ \boxed{\lambda_2 = \rec{\lambda_1} = -0.24264069,\quad v_1 = \begin{pmatrix}
    0.41421356 \\
    -1
\end{pmatrix}.} \]
\newprob{8.3 (1)}%
记$Q = \begin{pmatrix}
    \cos\theta & \sin\theta \\
    -\sin\theta & \cos\theta
\end{pmatrix}$, 记$t = \tan\theta$, 则$t$为
\[ t^2 + \frac{a_{22} - a_{11}}{a_{21}}t - 1 = 0 \]
的根中模较小的一者, 即
\[ t = \sqrt{2} - 1,\quad \sin\theta = \frac{t}{\sqrt{1+t^2}} = \frac{\sqrt{2-\sqrt{2}}}{2} = 0.382683,\quad \cos\theta = \frac{1}{\sqrt{1+t^2}} = 0.923880. \]
从而
\[ B = Q^T AQ = \begin{pmatrix}
    2.58579 & 0 \\
    0 & 5.41421
\end{pmatrix} \Rightarrow \boxed{\lambda_1,\lambda_2 = 2.58579, 5.41421.} \]

\end{document}
