\documentclass[hidelinks]{ctexart}

\usepackage{van-de-la-illinoise}
\usepackage[paper=b5paper,top=.3in,left=.9in,right=.9in,bottom=.3in]{geometry}
\usepackage{calc}
\pagenumbering{gobble}
\setlength{\parindent}{0pt}
\sisetup{inter-unit-product=\ensuremath{{}\cdot{}}}
\usepackage{van-le-trompe-loeil}

\usepackage{stackengine}
\stackMath
\usepackage{scalerel}
\usepackage[outline]{contour}

\newdimen\indexlen
\def\newprobheader#1{%
\def\probindex{#1}
\setlength\indexlen{\widthof{\textbf{\probindex}}}
\hskip\dimexpr-\indexlen-1em\relax
\textbf{\probindex}\hskip1em\relax
}
\def\newprob#1{%
\newprobheader{#1}%
\def\newprob##1{%
\probsep%
\newprobheader{##1}%
}%
}
\def\probsep{\vskip1em\relax{\color{gray}\dotfill}\vskip1em\relax}

\newlength\thisletterwidth
\newlength\gletterwidth
\newcommand{\leftrightharpoonup}[1]{%
{\ooalign{$\scriptstyle\leftharpoonup$\cr%\kern\dimexpr\thisletterwidth-\gletterwidth\relax
$\scriptstyle\rightharpoonup$\cr}}\relax%
}
\def\tensor#1{\settowidth\thisletterwidth{$\mathbf{#1}$}\settowidth\gletterwidth{$\mathbf{g}$}\stackon[-0.1ex]{\mathbf{#1}}{\boldsymbol{\leftrightharpoonup{#1}}}  }
\def\onedot{$\mathsurround0pt\ldotp$}
\def\cddot{% two dots stacked vertically
  \mathbin{\vcenter{\baselineskip.67ex
    \hbox{\onedot}\hbox{\onedot}}%
}}%
\newcommand{\tmpresumath}[1]{\tcboxmath[colback=emphgreen, boxrule=.3pt, sharp corners=all,left=.1mm,right=.1mm,top=.1mm,bottom=.1mm]{#1}}
\DeclareMathOperator{\cond}{cond}

\begin{document}

\newprob{补充题}%
$\displaystyle \begin{pmatrix}
    0.2161 & 0.1441 \\
    1.2968 & 0.8648
\end{pmatrix}\begin{pmatrix}
    x_1 \\ x_2
\end{pmatrix} = \begin{pmatrix}
    0.1440 \\
    0.8640
\end{pmatrix}$
\begin{align*}
    & A = \begin{pmatrix}
    0.2161 & 0.1441 \\
    1.2968 & 0.8648
\end{pmatrix},\quad\norm{A}_\infty = 2.1616. \\
    & A^{-1} = \begin{pmatrix}
    60055.6 & -10006.9 \\
    -90055.6 & 15006.9
\end{pmatrix},\quad\norm{A^{-1}}_\infty = 105063. \\
& \cond_\infty\pare{A} = \norm{A}_\infty\norm{A^{-1}}_\infty = 227104.
\end{align*}
变化有上界$\displaystyle \norm{\delta \+vx}\+_\infty_ \le \cond_\infty \pare{A} \frac{\norm{\delta \+vb}_\infty}{\norm{\+vb}_\infty} \cdot \norm{\+vx}_\infty = 0.00525704.$
实际上变化后解为
\[ \begin{pmatrix}
    x'_1 \\ x'_2
\end{pmatrix} = \begin{pmatrix}
    2.0007 \\
    -2.00105
\end{pmatrix},\quad \norm{\delta \+vx}\+_\infty_ = 0.00105 < 0.00525704. \]

\newprob{5.1 (2)}%
$\displaystyle \norm{A}_1 = \max_j {\sum_{i} \abs{a_{ij}}} = \boxed{7.}$\\
$\displaystyle \norm{A}_2 = \sqrt{\max \abs{\lambda \mathrm{\ of\ }AA^T}} = \sqrt{\max \abs{\lambda \mathrm{\ of\ }\begin{pmatrix}
    27 & 3 & 2 \\
    3 & 9 & 3 \\
    2 & 3 & 38
\end{pmatrix} }} = \boxed{\SI{6.22614}{}.}$\\
$\displaystyle \norm{A}_\infty = \max_i {\sum_{j} \abs{a_{ij}}} = \boxed{8.}$\\
$\curb{\lambda\mathrm{\ of\ }A} = \curb{3, 5.5\pm \SI{0.866025}{}i} \Rightarrow \max{\abs{\lambda\mathrm{\ of\ }A}} = \boxed{\SI{5.56776}{}.}$
\newprob{5.3 (1)}%
\\[-2\baselineskip]
\[ \begin{cases}
    \displaystyle x_1^{\pare{k+1}} = \frac{1+x_2^{\pare{k}}}{10}, \\[.5em]
    \displaystyle x_2^{\pare{k+1}} = \frac{x_1^{\pare{k}}+x_3^{\pare{k}}}{10}, \\[.5em]
    \displaystyle x_3^{\pare{k+1}} = \frac{1+x_2^{\pare{k}}+x_4^{\pare{k}}}{10}, \\[.5em]
    \displaystyle x_4^{\pare{k+1}} = \frac{2+x_3^{\pare{k}}}{10}.\end{cases}
    \quad \begin{pmatrix}
        0 \\
        0 \\
        0 \\
        0
    \end{pmatrix} \rightarrow \boxed{\begin{pmatrix}
        0.1 \\
        0 \\
        0.1 \\
        0.2
    \end{pmatrix} \rightarrow \begin{pmatrix}
        0.1 \\
        0.02 \\
        0.12 \\
        0.21
    \end{pmatrix} \rightarrow \begin{pmatrix}
        0.102 \\
        0.022 \\
        0.123 \\
        0.212
    \end{pmatrix}. }
\]
\par
\newprobheader{(2)}%
\\[-2\baselineskip]
\[ \begin{cases}
    \displaystyle x_1^{\pare{k+1}} = \frac{1+x_2^{\pare{k}}}{10}, \\[.5em]
    \displaystyle x_2^{\pare{k+1}} = \frac{x_1^{\pare{k+1}}+x_3^{\pare{k}}}{10}, \\[.5em]
    \displaystyle x_3^{\pare{k+1}} = \frac{1+x_2^{\pare{k+1}}+x_4^{\pare{k}}}{10}, \\[.5em]
    \displaystyle x_4^{\pare{k+1}} = \frac{2+x_3^{\pare{k+1}}}{10}.\end{cases}
    \quad \begin{pmatrix}
        0 \\
        0 \\
        0 \\
        0
    \end{pmatrix} \rightarrow \boxed{\begin{pmatrix}
        0.1 \\
        0.01 \\
        0.101 \\
        0.2101
    \end{pmatrix} \rightarrow \begin{pmatrix}
        0.101 \\
        0.0202 \\
        0.12303 \\
        0.212303
    \end{pmatrix} \rightarrow \begin{pmatrix}
        0.10202 \\
        0.022505 \\
        0.123481 \\
        0.212348
    \end{pmatrix}. }
 \]
\newprobheader{(3)}%
\begin{align*}
& M\+_Jacobi_ = \begin{pmatrix}
    & 0.1 & & \\
    0.1 & & 0.1 & \\
    & 0.1 & & 0.1 \\
    & & 0.1
\end{pmatrix} \Rightarrow \norm{M\+_Jacobi_}_1 = 0.2,\\ & M\+_Gau\text{\ss}_ = \begin{pmatrix}
    0 & 0.1 & & \\
    0 & 0.01 & 0.1 & \\
    0 & 0.001 & 0.01 & 0.1 \\
    0 & 0.0001 & 0.001 & 0.01 
\end{pmatrix} \Rightarrow \norm{M\+_Gau\text{\ss}_}_1 = 0.1111.     
\end{align*}
两者的$p=1$范数皆小于$1$, 故迭代收敛.
\newprob{5.6}%
$\displaystyle L = \begin{pmatrix}
    0 & \\
    t & 0
\end{pmatrix},\quad D = \begin{pmatrix}
    1 & \\
    & 2
\end{pmatrix},\quad U = \begin{pmatrix}
    0 & t \\
    & 0
\end{pmatrix}.$
\par
\newprobheader{(1)}%
$\displaystyle M\+_Jacobi_ = D^{-1}\pare{D-A} = \begin{pmatrix}
    & -t \\
    -t/2 &
\end{pmatrix},\quad M\+_Jacobi_^2 = \begin{pmatrix}
    t^2/2 & \\
    & t^2/2
\end{pmatrix}$. 故$\abs{t} < \sqrt{2}$时Jacobi迭代收敛.
\par
\newprobheader{(2)}%
$\displaystyle M\+_Gau\text{\ss}_ = \pare{D+L}^{-1}\pare{D+L-A} = \begin{pmatrix}
    0 & -t \\
    0 & 0.5t^2
\end{pmatrix},\quad M\+_Gau\text{\ss}_^n = \begin{pmatrix}
    0 & -t^{2n-1}/2^{n-1} \\
    0 & t^{2n}/2^n
\end{pmatrix}$. 故$\abs{t} < \sqrt{2}$时Gau\ss 迭代收敛.
\newprob{5.7 (1)}%
$\displaystyle L = \begin{pmatrix}
    0 & & \\
    1 & 0 & \\
    2 & 2 & 0
\end{pmatrix},\quad D = \begin{pmatrix}
    1 & & \\
    & 1 & \\
    & & 1
\end{pmatrix},\quad U = \begin{pmatrix}
    0 & 2 & -2 \\
    & 0 & 1 \\
    & & 0
\end{pmatrix}.$
\begin{align*}
    & M\+_Jacobi_ = D^{-1}\pare{D-A} = \begin{pmatrix}
        0 & -2 & 2 \\
        -1 & 0 & -1 \\
        -2 & -2 & 0
    \end{pmatrix} \Rightarrow \lambda = \curb{0,0,0}. \\
    & M\+_Gau\text{\ss}_ = \pare{D+L}^{-1}\pare{D+L-A} = \begin{pmatrix}
        0 & -2 & 2 \\
        0 & 2 & -3 \\
        0 & 0 & 2
    \end{pmatrix} \Rightarrow \lambda = \curb{2,2,0}.
\end{align*}
故$M\+_Jacobi_$谱半径为零, $M\+_Gau\text{\ss}_$谱半径大于一, 故前者迭代收敛而后者不收敛.
\par
\newprobheader{(2)}%
$\displaystyle L = \begin{pmatrix}
    0 & & \\
    1 & 0 & \\
    1 & 1 & 0
\end{pmatrix},\quad D = \begin{pmatrix}
    2 & & \\
    & 1 & \\
    & & -1
\end{pmatrix},\quad U = \begin{pmatrix}
    0 & -1 & 1 \\
    & 0 & 1 \\
    & & 0
\end{pmatrix}.$
\begin{align*}
    & M\+_Jacobi_ = D^{-1}\pare{D-A} = \begin{pmatrix}
        0 & 1/2 & -1/2 \\
        -1 & 0 & -1 \\
        1 & 1 & 0
    \end{pmatrix} \Rightarrow \lambda = \curb{\sqrt{2}i,-\sqrt{2}i,0}. \\
    & M\+_Gau\text{\ss}_ = \pare{D+L}^{-1}\pare{D+L-A} = \begin{pmatrix}
        0 & 1/2 & -1/2 \\
        0 & -1/2 & -1/2 \\
        0 & 0 & -1
    \end{pmatrix} \Rightarrow \lambda = \curb{-1,-1/2,0}.
\end{align*}
故$M\+_Jacobi_$谱半径大于一, $M\+_Gau\text{\ss}_$谱半径为一, 故前者迭代不收敛而后者对于大多数初始向量收敛.

\end{document}
