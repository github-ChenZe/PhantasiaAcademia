\documentclass[hidelinks]{ctexart}

\usepackage{van-de-la-illinoise}

\DeclareMathOperator{\cond}{cond}

\begin{document}

\section{迭代解法} % (fold)
\label{sec:迭代解法}

\subsection{向量/矩阵的模/范数} % (fold)
\label{sub:向量_矩阵的模_范数}

\subsubsection{向量的模} % (fold)
\label{ssub:向量的模}

$l_p$模
\[ \norm{\+vx}_p = \pare{\sum_{i=1}^n \abs{x_i}^p}^{1/p},\quad 1\le p \le \infty, \]
对于$p=\infty$,
\[ \norm{\+vx}_\infty = \max_{1\le i\le n}\abs{x_i}. \]
性质:
\begin{cenum}
    \item $\norm{\+vx} \ge 0$. $\norm{\+vx} = 0 \Leftrightarrow \+vx = 0$.
    \item $\forall \alpha \in \+bR, \norm{\alpha \+vx} = \abs{\alpha}\norm{\+vx}$.
    \item $\+vx,\+vy\in \+bR^n, \norm{\+vx+\+vy} \le \norm{\+vx} + \norm{\+vy}$.
    \item $m\norm{\+vx}_1 \le \norm{\+vx}_2 \le M\norm{\+vx}_1.$
\end{cenum}
特别地,
\begin{align*}
    & \norm{\+vx}_\infty \le \norm{\+vx}_1 \le n\cdot \norm{\+vx}_\infty, \\
    & \rec{\sqrt{n}} \norm{\+vx}_1 \le \norm{\+vx}_2 \le \norm{\+vx}_1, \\
    & \rec{\sqrt{n}} \norm{\+vx}_2 \le \norm{\+vx}_\infty \le \norm{\+vx}_2, \\
    & \norm{\+vx}_\infty \le \norm{\+vx}_2 \le \sqrt{n}\norm{\+vx}_\infty.
\end{align*}

% subsubsection 向量的模 (end)

\subsubsection{向量序列的极限} % (fold)
\label{ssub:向量序列的极限}

设有向量序列$\curb{\+vx^{\pare{k}}}$, 若$\forall \epsilon > 0$, 存在$N>a$使得$k\ge N$时
\[ \norm{\+vx^{\pare{k}} - \+vx}<\epsilon, \]
则谓$\curb{\+vx^{\pare{k}}}$是收敛序列, 极限为$\+vx$.
\begin{lemma}
    向量序列收敛当且仅当每一个分量收敛.
\end{lemma}

% subsubsection 向量序列的极限 (end)

\subsubsection{方阵的模} % (fold)
\label{ssub:方阵的模}

算子范数
\[ \norm{A} = \sup \frac{\norm{A\+vx}}{\norm{\+vx}} = \sup_{\norm{\+vx} = 1}\norm{A\+vx}. \]
而
\begin{align*}
    \norm{A}_\infty &= \max_{1\le i\le n} \sum_{j=1}^n \abs{a_{ij}}, \\
    \norm{A}_1 &= \max_{1\le j \le n} \sum_{i=1}^n \abs{a_{ij}}, \\
    \norm{A}_2 &= \sqrt{\max_{1\le i\le n} \abs{\lambda_i \pare{AA^T}}},\quad \text{$\lambda_i$为$AA^T$特征值}.
\end{align*}
上述定义皆满足
\[ \norm{A\+vx}\le \norm{A}\cdot \norm{\+vx}. \]

% subsubsection 方阵的模 (end)

% subsection 向量_矩阵的模_范数 (end)

\subsection{矩阵的极限} % (fold)
\label{sub:矩阵的极限}

\subsubsection{矩阵的收敛性} % (fold)
\label{ssub:矩阵的收敛性}

设$\curb{A^{\pare{k}}}$为矩阵序列, 若$\forall \epsilon > 0$, 存在$N>a$使得$k\ge N$时
\[ \norm{A^{\pare{k}} - A}<\epsilon, \]
则谓$\curb{A^{\pare{k}}}$是收敛序列, 极限为$A$. 矩阵收敛当且仅当各个分量收敛.
\begin{definition}[收敛矩阵]
    $\displaystyle \lim_{k\rightarrow \infty} A^k = 0.$
\end{definition}

% subsubsection 矩阵的收敛性 (end)

\subsubsection{矩阵谱半径} % (fold)
\label{ssub:矩阵谱半径}

\begin{definition}[谱半径]
    $\displaystyle P\pare{A} = \max_{1\le i\le n} \abs{\lambda_i\pare{A}}$.
\end{definition}
设$\norm{A}$为任一范数, 有$P\pare{A} \le \norm{A}$.
\[ \abs{\lambda}\cdot \norm{\+vx} = \norm{\lambda \+vx} = \norm{A\+vx} \le \norm{A}\cdot \norm{\+vx} \Rightarrow \abs{\lambda} \le \norm{A}. \]
$\displaystyle \lim_{k\rightarrow \infty} A^k = 0 \Leftrightarrow P\pare{A} < 1$.
\begin{lemma}
    若在某个范数下$\norm{A} < 1$, 则$\displaystyle \lim_{n\rightarrow \infty} A^k = 0$.
\end{lemma}

% subsubsection 矩阵谱半径 (end)

\subsubsection{条件数} % (fold)
\label{ssub:条件数}

\begin{definition}
    $\cond_p\pare{A} = \norm{A}_p \cdot \norm{A^{-1}}_p$.
\end{definition}
$\cond_p\pare{A}\ge 1$. 若$A$为正交阵, $\cond\pare{A} = 1$.
\par
对于$A\+vx = \+vb$, 考虑微扰$A \rightarrow A+\delta A$, $\+vx \rightarrow \delta \+vx$, 则
\begin{align*}
    & \pare{A+\delta A}\pare{\+vx + \delta \+vx} = \+vb\Rightarrow \norm{\delta \+vx} \le \norm{A^{-1}}\norm{\delta A}\pare{\norm{\+vx} + \norm{\delta \+vx}}. \\
    & \norm{A^{-1}} \cdot \norm{\delta A}\le 1 \Rightarrow \pare{1-\norm{A^{-1}} \cdot \norm{\delta A}}\norm{\delta \+vx} \le \norm{A^{-1}}\cdot \norm{\delta A}\cdot \norm{\+vx}, \\
    & \frac{\norm{\delta \+vx}}{\norm{\+vx}} \le \frac{\norm{A^{-1}}\cdot \norm{\delta A}}{1-\norm{A^{-1}}\norm{\delta A}} = \frac{\norm{A}\norm{A^{-1}} \norm{\delta A}/\norm{A}}{1-\norm{A}\norm{A^{-1}} \norm{\delta A}/\norm{A}} = \frac{\displaystyle \cond \pare{A}\cdot \frac{\norm{\delta A}}{\norm{A}}}{\displaystyle 1-\cond\pare{A}\cdot \frac{\norm{\delta A}}{\norm{A}}}.
\end{align*}
考虑微扰$\+vb \rightarrow \+vb+\delta \+vb$,
\begin{align*}
    & A\pare{\+vx+\delta \+vx} = \+vb + \delta \+vb,\quad \delta \+vx = A^{-1}\delta \+vb,\quad \norm{\delta \+vx} \le \norm{A^{-1}}\cdot \norm{\delta \+vb}, \\
    & A \+vx = \+vb,\quad \norm{A}\cdot \norm{\+vx} \ge \norm{\+vb},\quad \norm{\+vx} \ge \norm{A}^{-1}\cdot \norm{\+vb}, \\
    & \frac{\norm{\delta \+vx}}{\norm{\+vx}} \le \frac{\norm{A^{-1}}\cdot \norm{\delta \+vb}}{\norm{A}^{-1}\cdot \norm{\+vb}} = \norm{A^{-1}} \cdot \norm{A} \cdot \frac{\norm{\delta \+vb}}{\norm{\+vb}} = \cond \pare{A} \cdot \frac{\norm{\delta \+vb}}{\norm{\+vb}}.
\end{align*}
$\cond A$越大, $\norm{\delta x}/\norm{x}$越大.
\begin{ex}
    考虑
    \[ \begin{cases}
        1.0002x_1 + 0.9998x_2 = 2,\\
        0.9998x_1 + 1.0002x_2 = 2,
    \end{cases} \Rightarrow \begin{cases}
        x_1 = 1,\\
        x_2 = 1.
    \end{cases} \]
    若
    \[ \delta \+vb = \begin{pmatrix}
        0.001 \\
        -0.001
    \end{pmatrix} \Rightarrow \begin{pmatrix}
        x_1\\ x_2
    \end{pmatrix} = \begin{pmatrix}
        3.5 \\ 1.5
    \end{pmatrix}. \]
    原因在于
    \[ \cond_\infty A = \norm{A}_\infty \cdot \norm{A^{-1}}_\infty = 5000. \]
\end{ex}

% subsubsection 条件数 (end)

% subsection 矩阵的极限 (end)

\subsection{方程组的迭代法} % (fold)
\label{sub:方程组的迭代法}

将方程组$A\+vx=\+vb$改写为
\[ \+vx = M\+vx + \+vg, \]
即可通过
\[ \+vx^{\pare{k+1}} = M\+vx^{\pare{k}}+\+vg \]
求解之. 设$N$为可逆矩阵, 将方程转化为
\[ \+vx = N^{-1}\pare{N-A}\+vx + N^{-1}\+vb. \]
记$M = N^{-1}\pare{N-A}$, $\+vg = N^{-1}\+vb$, 则迭代格式为
\[ \+vx^{\pare{k+1}} = M\+vx^{\pare{k}}+\+vg. \]
若$P\pare{M} < 1$则上述迭代收敛. 特别地, 若在某种范数下$\norm{M} \le 1$, 迭代收敛.

\subsubsection{Jacobi迭代} % (fold)
\label{ssub:jacobi迭代}

分解为
\[ A = D+L+U,\quad N = D, \]
则
\[ \+vx^{\pare{k+1}} = D^{-1}\pare{D-A}\+vx^{\pare{k}} + D^{-1}\+vb. \]
即
\[ x^{\pare{k+1}}_i = \rec{a_{ii}}\brac{-\sum_{\substack{j=1\\ j\neq i}}^n a_{ij}x_j^{\pare{k}} + v_i}. \]

\begin{theorem}
    左列二情形下Jacobi迭代收敛:
    \begin{cenum}
        \item $A$为严格行对角占优,
        \[ a_{ii} > \sum_{j\neq i}\abs{a_{ij}},\quad i = 1,\cdots,n. \]
        \item $A$为严格列对角占优,
        \[ a_{jj} > \sum_{i\neq j}\abs{a_{ij}},\quad j = 1,\cdots,n. \]
    \end{cenum}
\end{theorem}

% subsubsection jacobi迭代 (end)

\subsubsection{\texorpdfstring{Gau\ss}{Gauss}-Seidel迭代} % (fold)
\label{ssub:gauss_seidel迭代}

$A = D+L+U$, $N=D+L$, 则
\begin{align*}
    & \+vx = -\pare{D+L}^{-1}U\+vx + \pare{D+L}^{-1}\+vb, \\
    & \+vx^{\pare{k+1}} = -\pare{D+L}^{-1}U\+vx^{\pare{k}} + \pare{D+L}^{-1}\+vb, \\
    & M = -\pare{D+L}^{-1}U, \\
    & \pare{D+L}\+vx^{\pare{k+1}} = -U\+vx^{\pare{k}}+\+vb, \\
    & \+vx^{\pare{k+1}} = -\pare{D+L}^{-1}UX^{\pare{k}} + \pare{D+L}^{-1}\+vb, \\
    & S = -\pare{D+L}^{-1}U,\quad f = \pare{D+L}^{-1}b, \\
    & X^{\pare{k+1}} = SX^{\pare{k}} + f. \\
    & x_i^{\pare{k+1}} = \rec{a_{ii}}\brac{-\sum_{j=1}^{i-1}a_{ij}x_k^{\pare{k+1}} - \sum_{j=i+1}^n a_{ij}x_j^{\pare{k}} + b_i}.
\end{align*}
\begin{theorem}
    左列二情形下Gau\ss-Seidel迭代收敛:
    \begin{cenum}
        \item $A$为严格对角占优,
        \[ a_{ii} > \sum_{j\neq i}\abs{a_{ij}},\quad i = 1,\cdots,n. \]
        \item $A$为正定矩阵.
    \end{cenum}
\end{theorem}

% subsubsection gauss_seidel迭代 (end)

\subsubsection{松弛迭代} % (fold)
\label{ssub:松弛迭代}

在Gau\ss-Seidel迭代的基础上加权,
\begin{align*}
    & \+vx^{\pare{k+1}} = \pare{1-\omega}x^{\pare{k}} + \omega \+vx\+_GS_^{\pare{k+1}} \\
    & = \pare{1-\omega} \+vx^{\pare{k}} + \omega\pare{-D^{-1} \pare{L\+vx^{\pare{k+1}}} + D^{-1}\+vb}. \\
    & \Rightarrow \+vx^{\pare{k+1}} = \pare{D+\omega L}^{-1}\brac{\pare{1-\omega}D - \omega U}\+vx^{\pare{k}} + \pare{D+\omega L}^{-1} \omega \+vb.
\end{align*}

\begin{sample}
    \begin{ex}
        方程组
        \[ \begin{cases}
            2x_1 - x_2 - x_3 = 5, \\
            x_1 + 5x_2 - x_3 = 8, \\
            x_1 + x_2 + 10x_3 = 11
        \end{cases} \]
        的等价方程为
        \[ \begin{cases}
            x_1 = 0.5 x_2 + 0.5 x_3 + 2.5, \\
            x_2 = -0.2x_1 + 0.2x_3 + 1.6, \\
            x_3 = -0.1x_1 - 0.1x_2 + 1.1.
        \end{cases} \]
        Jacobi迭代为
        \[ \left\{\begin{array}{lllll}
            x_1^{\pare{k+1}} & = & 0.5x_2^{\pare{k}} & +0.5x_3^{\pare{k}} & +2.5, \\
            x_2^{\pare{k+1}} & =-0.2x_1^{\pare{k}} & & +0.2x_3^{\pare{k}} & +1.6, \\
            x_3^{\pare{k+1}} & =-0.1x_1^{\pare{k}} & -0.1x_2^{\pare{k}} & & +1.1.
        \end{array}\right. \]
        Gau\ss-Seidel迭代为
        \[ \left\{\begin{array}{lllll}
            x_1^{\pare{k+1}} & = & 0.5x_2^{\pare{k}} & +0.5x_3^{\pare{k}} & +2.5, \\
            x_2^{\pare{k+1}} & =-0.2x_1^{\pare{k+1}} & & +0.2x_3^{\pare{k}} & +1.6, \\
            x_3^{\pare{k+1}} & =-0.1x_1^{\pare{k+1}} & -0.1x_2^{\pare{k+1}} & & +1.1.
        \end{array}\right. \]
        松弛迭代为
        \[ \left\{\begin{array}{lllll}
            x_1^{\pare{k+1}} & =\pare{1-\omega}x_1^{\pare{k}} & 0.5\omega x_2^{\pare{k}} & +0.5\omega x_3^{\pare{k}} & +2.5\omega , \\
            x_2^{\pare{k+1}} & =-0.2\omega x_1^{\pare{k+1}} & + \pare{1-\omega}x_2^{\pare{k}} & +0.2\omega x_3^{\pare{k}} & +1.6\omega , \\
            x_3^{\pare{k+1}} & =-0.1\omega x_1^{\pare{k+1}} & -0.1\omega x_2^{\pare{k+1}} & \pare{1-\omega}x_3^{\pare{k}} & +1.1\omega .
        \end{array}\right. \]
        Jacobi迭代的矩阵为
        \[ M = \begin{pmatrix}
            0 & 0.5 & 0.5 \\
            -0.2 & 0 & 0.2 \\
            -0.1 & -0.2 & 0
        \end{pmatrix},\quad \norm{M}_1 = 0.7 < 1. \]
        Gau\ss-Seidel迭代的矩阵为
        \[ M = \begin{pmatrix}
            0 & 0.5 & 0.5 \\
            0 & -0.1 & -0.1 \\
            0 & -0.04 & -0.05
        \end{pmatrix},\quad \norm{M}_1 = 0.66 < 1. \]
    \end{ex}
\end{sample}

% subsubsection 松弛迭代 (end)

% subsection 方程组的迭代法 (end)

% section 迭代解法 (end)

\end{document}
