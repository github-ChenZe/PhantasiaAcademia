\documentclass[hidelinks]{article}

\usepackage[sensei=Academic\ Council,gakka=Styles,section=Quantum,gakkabbr=QM]{styles/kurisuen}
\usepackage{sidenotes}
\usepackage{van-de-la-sehen-en}
\usepackage{van-de-environnement-en}
\usepackage{boite/van-de-boite-en}
\usepackage{van-de-abbreviation}
\usepackage{van-de-neko}
\usepackage{van-le-trompe-loeil}
\usepackage{cyanide/van-de-cyanide}
\setlength{\parindent}{0pt}
\usepackage{enumitem}
\newlist{citemize}{itemize}{3}
\setlist[citemize,1]{noitemsep,topsep=0pt,label={-},leftmargin=1em}

\usepackage{mathtools}
\usepackage{ragged2e}

\DeclarePairedDelimiter\abs{\lvert}{\rvert}%
\DeclarePairedDelimiter\norm{\lVert}{\rVert}%

% Swap the definition of \abs* and \norm*, so that \abs
% and \norm resizes the size of the brackets, and the 
% starred version does not.
\makeatletter
\let\oldabs\abs
\def\abs{\@ifstar{\oldabs}{\oldabs*}}
%
\let\oldnorm\norm
\def\norm{\@ifstar{\oldnorm}{\oldnorm*}}
\makeatother

\newcommand*{\Value}{\frac{1}{2}x^2}%

\usepackage{fancyhdr}
\usepackage{lastpage}

\fancypagestyle{plain}{%
\fancyhf{} % clear all header and footer fields
\fancyhead[R]{\smash{\raisebox{2.75em}{{\hspace{1cm}\color{lightgray}\textsf{\rightmark\quad Page \thepage/\pageref{LastPage}}}}}} %RO=right odd, RE=right even
\renewcommand{\headrulewidth}{0pt}
\renewcommand{\footrulewidth}{0pt}}
\pagestyle{plain}

\newtheorem*{experiment*}{Measurement}
\newtheorem{theorem}{Theorem}
\newtheorem{example}{Example}
\newtheorem{remark}{Remark}

\def\elementcell#1#2#3#4#5#6#7{%
    \draw node[draw, regular polygon, regular polygon sides=4, minimum height=2cm, draw=cyan, line width=0.4mm, fill=cyan!15!white, #1, inner sep=-2mm](#3) {\Large\textbf{\textsf{\color{cyan!50!black}#4}}};
    \draw (#3.corner 1) node[below left] {\footnotesize{\phantom{Hj}#5}};
    \draw (#3.corner 2) node[below right] {\small{\textsf{#6}}};
    \draw (#3.side 3) node[above] {\footnotesize #7};
    \draw (#3.corner 2) ++ (0,-0.4mm) node(nw#3) {};
    \tcbsetmacrotowidthofnode{\elementcellwidth}{#3}
    \node [fill=cyan, line width=0mm, rectangle, rounded corners=1.8mm, rectangle round south east=false, rectangle round south west=false, anchor=south west, minimum width=\elementcellwidth] at (nw#3) {\small\textsf{\color{white}#2}};
}

\DeclareSIUnit\Dq{Dq}
\usepackage{physics}
\usepackage{bbm}
\newtheorem{lemma}{Lemma}
\newtheorem{proposition}{Proposition}

\DeclareMathOperator{\Pfaffian}{Pf}
\DeclareMathOperator{\sign}{sign}
\DeclareMathOperator{\GF}{GF}
\DeclareMathOperator{\GL}{GL}
\DeclareMathOperator{\SU}{SU}
\DeclareMathOperator{\SO}{SO}
\DeclareMathOperator{\SL}{SL}

\makeatletter

\newcommand*\SetSuchThat{\@ifstar\@SetSuchThat@star\@SetSuchThat@nostar}
\newcommand*\@SetSuchThat@star{%
    \mathrel{}%
    % \nobreak % superfluous inside "\left... ... \right..."
    \middle\vert
    \mathrel{}%
}
\newcommand*\@SetSuchThat@nostar[1][]{%
    \mathrel{#1\vert}%
}
\newcommand*\@SetSuchThat{}
\DeclarePairedDelimiterX \SetCond [2] {\lbrace}{\rbrace}
    {\nonscript\,#1\@SetSuchThat@nostar #2\nonscript\,}

\makeatother

\usepackage[all]{xy}

\begin{document}

\section{Style Guide} % (fold)
\label{sec:style_guide}

This style guide applies to lecture notes in English, and is subjected to revision in the future.
\par
Some of the rules here are adopted from the following sources:
\begin{citemize}
    \item \href{https://en.wikipedia.org/wiki/Wikipedia:Manual_of_Style/Mathematics}{Wikipedia: Manual of Style/Mathematics}.
\end{citemize}

\subsection{Grammar} % (fold)
\label{sub:grammar}

The usage of ``denoted'' and ``denoted by'' are \href{https://english.stackexchange.com/questions/25179/denoted-by-or-just-denoted}{both acceptable}.

% subsection grammar (end)

\subsection{Equations} % (fold)
\label{sub:equations}

Equations should be punctuated in the way such that the sentences that contain the equations are grammatical if the equations are substituted by nouns. Equations may be used as apposition.
\par
A sentence which ends with a formula must have a period at the end of the formula. This equally applies to displayed formulae. Similarly, if the conventional punctuation rules would require a question mark, comma, semicolon, or other punctuation at that place, the formula must have that punctuation at the end.
\begin{sample}
    Therefore we have the following relation
    \[ E = mc^2. \]
    Now we begin the next section.
\end{sample}

% subsection equations (end)

\subsection{Lists} % (fold)
\label{sub:lists}

Lists should be punctuated in the way such that the sentences that contain the lists are grammatical if the numbering or bullets in the list are removed.
\begin{sample}
    A number is called a positive real number if
    \begin{cenum}
        \item it is real;
        \item and it is positive.
    \end{cenum}
\end{sample}

% subsection lists (end)

\subsection{Names} % (fold)
\label{sub:names}

Names of people and places should be romanized. If multiple romanization are available, the one occurs the most frequently in authorized sources should be adopted. Accent marks should be retained.

\subsubsection{Japanese Names} % (fold)
\label{ssub:japanese_names}

In the names of people, the following conventions are followed.
\begin{cenum}
    \item The long vowels are indicated by macrons, or circumflex, depending on the most common choices in publications.
\end{cenum}
\begin{sample}
    It\^o's lemma is a key component in the It\^o Calculus.
\end{sample}

% subsubsection japanese_names (end)

% subsection names (end)

% section style_guide (end)

\end{document}
