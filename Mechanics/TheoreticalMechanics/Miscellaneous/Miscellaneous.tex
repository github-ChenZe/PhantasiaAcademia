\documentclass[../TheoreticalMechanics.tex]{subfiles}

\begin{document}

\section{杂项处理} % (fold)
\label{sec:杂项处理}

\begin{ex}
    \label{ex:球坐标系下平方反比问题角向角变量积分}
    对于
    \[ J_\theta = \oint\sqrt{\alpha_\theta^2 - \frac{\alpha_\phi^2}{\sin^2\theta}}\,\rd{\theta}, \]
    其中$\alpha_\theta > \alpha_\phi$, 设$\cos\gamma = \alpha_\phi/\alpha_\theta$, 则
    \[ J_\theta = \alpha_\theta\oint \sqrt{1-\cos^2 i \csc^2\theta}\,\rd{\theta}. \]
    考虑到$\gamma$的物理意义为极轴和角动量的夹角, 可知$\theta$的极限值$\theta_0 = \pi/2 - \gamma$,
    \[ J_\theta = 4\alpha_\theta \int_0^{\theta_0}\csc\theta\sqrt{\sin^2\gamma - \cos^2\theta}\,\rd{\theta}. \]
    换元$\cos\theta = \sin\gamma\sin\psi$, 则
    \[ J_\theta = 4\alpha_\theta\sin^2\gamma\int_0^{\pi/2}\frac{\cos^2\psi\,\rd{\psi}}{1-\sin^2\gamma\sin^2\psi}. \]
    再次换元$u=\tan\psi$,
    \begin{align*}
        J_\theta &= 4\alpha_\theta \sin^2\gamma \int_0^\infty \frac{\rd{u}}{\pare{1+u^2}\pare{1+u^2\cos^2\gamma}} \\
        &= 4\alpha_0 \int_0^\infty\pare{\rec{1+u^2} - \frac{\cos^2\gamma}{1+u^2\cos^2\gamma}}\,\rd{u} \\
        &= 2\pi\alpha_\theta\pare{1-\cos \gamma} \\
        &= 2\pi\pare{\alpha_\theta - \alpha_\phi}.
    \end{align*}
\end{ex}
\begin{figure}[ht]
    \centering
    \incfig{11cm}{ContourIntJ}
    \caption{\cref{ex:球坐标系下平方反比问题径向角变量积分}的围道}
    \label{fig:球坐标系下平方反比问题径向角变量积分的围道}
\end{figure}
\begin{ex}
    \label{ex:球坐标系下平方反比问题径向角变量积分}
    对于
    \[ J_r = \oint \underbrace{\sqrt{2mE + \frac{2mk}{r} - \frac{\pare{J_\theta + J_\phi}^2}{4\pi^2r^2}}}_{\sqrt{A+2B/r-C/r^2}}\,\rd{r}, \]
    考虑到被积分式实际上为$p_r = m\dot{r}$, 在$r_1\rightarrow r_2$过程中具有正值, 在$r_2\rightarrow r_2$过程中具有负值, 于是沿$\pare{r_1,r_2}$在实轴上的部分切开复平面, 可将上述积分替换为环路积分(如\cref{fig:球坐标系下平方反比问题径向角变量积分的围道}), 两处留数分别为
    \[ R_0 = -\sqrt{-C},\quad R_\infty = -\frac{B}{\sqrt{A}},\quad J_r = 2\pi i\pare{\sqrt{-C}+\frac{B}{\sqrt{A}}}. \]
\end{ex}
\begin{ex}
    \label{ex:平方反比问题下第一角变量关联至普通守恒量的积分}
    设$\sin u = \cot\gamma\cot\theta$,
    \begin{align*}
        w_1 &= \frac{\phi}{2\pi} + \frac{J_1}{2\pi} \int \frac{\rd{\theta}}{\sin^2\theta\sqrt{J_2^2 - J_1^2\csc^2\theta}},\\
        2\pi w_1 &= \phi + \cos\gamma \int \frac{\rd{\theta}}{\sin^2\theta\sqrt{1-\cos^2\gamma\csc^2\theta}}\\
        &= \phi + \int \frac{\cot\gamma\csc^2\theta\,\rd{\theta}}{\sqrt{1-\cot^2\gamma\cot^2\theta}}\\
        &\xlongequal{\sin u = \cot\gamma\cot\theta} \phi - u.
    \end{align*}
\end{ex}

% section 杂项处理 (end)

\end{document}
