\documentclass[../LectureNotes.tex]{subfiles}

\begin{document}

\section{刚体的运动} % (fold)
\label{sec:刚体的运动}

\subsection{刚体运动的描述} % (fold)
\label{sub:刚体运动的描述}

\subsubsection{自由度与运动分类} % (fold)
\label{ssub:自由度与运动分类}

\begin{figure}[ht]
    \centering
    \incfig{8cm}{RigidBodyDegreeOfFreedom}
    \caption{刚体的自由度}
\end{figure}

\paragraph{自由度} % (fold)
\label{par:自由度}

刚体的$N$个质点由$3N$个坐标确定, 约束$\abs{\+vr_i - \+vr_j} = r_{ij} = \const$共$\displaystyle \frac{N\pare{N-1}}{2}$个, 惟自由度并非$\displaystyle 3N - \frac{N\pare{N-1}}{2}$. 取不共线三点$\+vr_1$, $\+vr_2$, $\+vr_3$, 共有$9$个坐标, 对于任意$i$, 约束
\[ \abs{\+vr_i - \+vr_1} = r_{i1},\quad \abs{\+vr_i - \+vr_2} = r_{i2},\quad \abs{\+vr_i - \+vr_3} = r_{i3} \]
可唯一确定$\+vr_i$的位置. 再由
\[ \abs{\+vr_1 - \+vr_2} = r_{12},\quad \abs{\+vr_2 - \+vr_3} = r_{23},\quad \abs{\+vr_3 - \+vr_1} = r_{31}, \]
从而$s = 9 - 3 = 6$, 故刚体最大自由度为$6$.

% paragraph 自由度 (end)

\paragraph{确定刚体的位形} % (fold)
\label{par:确定刚体的位形}

基点$A\pare{x,y,z}$, 平动自由度为$3$, 相对于$A$点的取向有轴向$2$个自由度, 绕轴转动$1$个自由度, 故总共$6$个自由度.

% paragraph 确定刚体的位形 (end)

\paragraph{运动分类} % (fold)
\label{par:运动分类}

平动$s=3$, 定轴转动$s=1$, 平面平行运动$s=2+1 = 3$(平面平行运动和转动), 定点运动$s=3$, 一般情形$s=6$.

% paragraph 运动分类 (end)

\begin{remark}
    平面平行运动意味着刚体在运动中有一截面始终在一固定平面上, 故其运动可归于$2$个自由度的平动和绕该平面法线的单自由度转动.
\end{remark}

% subsubsection 自由度与运动分类 (end)

\subsubsection{刚体运动的Euler定理} % (fold)
\label{ssub:刚体运动的euler定理}

\paragraph{惯性坐标系与本体坐标系} % (fold)
\label{par:惯性坐标系与本体坐标系}

惯性坐标系相对于惯性观测者静止, $\dot{\+ve_i} = 0$可直接使用. 本体坐标系相对于刚体固定, 运动定律不可以直接使用, $\dot{\+ve_i}\neq 0$.

% paragraph 惯性坐标系与本体坐标系 (end)

\begin{figure}[ht]
    \centering
    \incfig{8cm}{FixedPointCoor}
    \caption{定点运动的坐标系}
\end{figure}

对于定点运动, 建立本体坐标系$Oe_1e_2e_3$和惯性坐标系$Oe'_1e'_2e'_3$, 刚体上任意一点都可以写作
\begin{align*}
    \+vr\pare{t} &= \sum_i x'_i\pare{t} \+ve'_i = \sum_i x_i \+ve_i\pare{t}. \\
    x_j &= \+vr\cdot\+ve_j = \pare{\sum_i x'_i \+ve'_i}\cdot\+ve_j = \sum_i x'_i\pare{\+ve'_i \cdot\+ve_j} = \sum_i A_{ji}\pare{t}x'_i. \\
    \begin{pmatrix}
        x_1 \\ x_2 \\ x_3
    \end{pmatrix} &= A\pare{t} \begin{pmatrix}
        x'_1\pare{t} \\ x'_2\pare{t} \\ x'_3\pare{t}
    \end{pmatrix},\quad \sum_i x'^2_i = r^2 = \sum_i x_i^2 \Rightarrow A\pare{t} \in SO_3. \\
    \begin{pmatrix}
        x'_1 \\ x'_2 \\ x'_3
    \end{pmatrix} &= A^{-1}\pare{t} \begin{pmatrix}
        x_1 \\ x_2 \\ x_3
    \end{pmatrix} = A^T\pare{t} \begin{pmatrix}
        x_1 \\ x_2 \\ x_3
    \end{pmatrix},\quad A_{ji} = \+ve_j \cdot \+ve'_i.
\end{align*}
实正交矩阵有$6$个独立约束, 从而有$3$个独立参数. 从而本体坐标系和惯性坐标系可相互转换.

\begin{theorem}[Euler定理]
    定点运动的刚体任一位移等效于绕该定点的某一轴的转动
\end{theorem}
\begin{proof}
    任一位移,
    \[ \begin{pmatrix}
        x'\pare{t} \\ y'\pare{t} \\ z'\pare{t}
    \end{pmatrix} = A^{-1}\pare{t} \begin{pmatrix}
        x \\ y \\ z
    \end{pmatrix} = A^{-1}\pare{t} \begin{pmatrix}
        x'\pare{t_0} \\ y'\pare{t_0} \\ z'\pare{t_0}
    \end{pmatrix}. \]
    需证明存在矢量$R$满足$A^{-1}\pare{t}R = R$, 即证明$A\pare{t}$存在本征值$1$.
    \[ \abs{A - \lambda E}  = 0 \]
    构成一$\lambda$的三次方程. 相似对角化后,
    \begin{align*}
        & \abs{C^{-1}AC} = \lambda_1 \lambda_2 \lambda_3 = 1, \\
        & \abs{C^{-1}}\abs{A}\abs{C} = \abs{A} = 1, \\
        & \pare{A-E} A^T = E- A^T = \pare{E-A}^T. \\
        & \abs{A-E}\abs{A^T} = \abs{E-A^T} = \abs{E-A}. \\
        & \Rightarrow\abs{A-E} = \abs{E-A} = -\abs{A-E}. \\
        & \Rightarrow \abs{A-\lambda E} = 0,\quad \exists \, \lambda = 1.
    \end{align*}
    若$\lambda_1 = 1$, 则$\lambda_2\lambda_3 = 1$, 从而$\abs{\lambda_2}=\abs{\lambda_3}=1$, $\lambda_2 = \conj{\lambda_3}$, 故
    \[ \lambda_2 = e^{i\Phi},\quad \lambda_3 = e^{-i\Phi}. \qedhere \]
\end{proof}
\begin{remark}
    如果刚体分别绕$x$和$y$轴先后转动, 则运动最终仍可合称为绕某一第三轴的转动.
\end{remark}
\begin{pitfall}
    角位移并非矢量(不可直接相加). 惟无限小的角位移可以直接相加.
\end{pitfall}

% subsubsection 刚体运动的euler定理 (end)

\subsubsection{无限小转动与角速度} % (fold)
\label{ssub:无限小转动与角速度}

\paragraph{有限转动} % (fold)
\label{par:有限转动}

有限转动角位移并非矢量.
\begin{align*}
    & R\xrightarrow{\+v\varphi_1} R_1 = A_1 R \xrightarrow{\+v\varphi_2} R_{21} = A_2R_1 = A_2A_1 R, \\
     & R\xrightarrow{\+v\varphi_2} R_2 = A_2 R \xrightarrow{\+v\varphi_1} R_{12} = A_1R_2 = A_1A_2 R.
\end{align*}
如果角位移是矢量, 则应有$A_{12} = A_{21}$, 惟一般情形下矩阵相乘不能交换顺序, 故不能成立.

% paragraph 有限转动 (end)

\paragraph{无限小转动} % (fold)
\label{par:无限小转动}

无限小转动满足$A = E\epsilon B$,
\begin{align*}
    & A_1 = E + \epsilon B_1,\quad A_2 = E + \epsilon B_2, \\
    & A_1A_2 = E + \epsilon\pare{B_1+B_2},\quad A_2A_1 = E + \epsilon\pare{B_2+B_1}, \\
    & \Rightarrow A_1 A_2 = A_2 A_1 \Rightarrow \+v\varphi_1 + \+v\varphi_2 = \+v\varphi_2 + \+v\varphi_1.
\end{align*}
现在定义角速度
\[ \+v\omega = \frac{\Delta\+v\varphi}{\Delta t}. \]
为了精确描述角速度的性质, 考虑
\begin{align*}
    &\begin{pmatrix}
        x'\pare{t+\Delta t} \\ y'\pare{t+\Delta t} \\ z'\pare{t+\Delta t}
    \end{pmatrix} = A^{-1}\pare{t+\Delta t}\begin{pmatrix}
        x \\ y \\ z
    \end{pmatrix} \\
    &= A^{-1}\pare{t+\Delta t}A\pare{t} \begin{pmatrix}
        x'\pare{t} \\ y'\pare{t} \\ z'\pare{t}
    \end{pmatrix}
    = A\pare{A+\Delta t, t} \begin{pmatrix}
        x'\pare{t} \\ y'\pare{t} \\ z'\pare{t}
    \end{pmatrix}.
\end{align*}
其中$A\pare{t+\Delta t, t}\in SO_3$. 且对于某$\Omega\pare{t}\in M_{3\times 3}$有
\begin{align*}
    & A\pare{t+\Delta t} = E+\Omega\pare{t}\Delta t, \\
    & A^T A = E \Rightarrow \pare{E+\Omega \Delta t}^T\pare{E + \Omega \Delta t} = E, \\
    & \pare{E + \Omega^T \Delta t} \pare{E + \Omega \Delta t} = E \\
    & \Rightarrow E + \pare{\Omega^T + \Omega} \Delta t = E. \\
    & \Rightarrow \Omega^T + \Omega = 0 \Rightarrow \Omega^T = -\Omega, \\
    & \Omega = \begin{pmatrix}
        0 & \Omega_{12} & \Omega_{13} \\
        -\Omega_{12} & 0 & \Omega_{23} \\
        -\Omega_{13} & -\Omega_{23} & 0
    \end{pmatrix}.
\end{align*}
故$\Omega$构成一实反对称矩阵, 有三个独立参数, 这和$SO_3$矩阵有三个独立参数之事实相符. 位移
\begin{align*}
    & \begin{pmatrix}
        x'\pare{t+\Delta t} \\
        y'\pare{t+\Delta t} \\
        z'\pare{t+\Delta t}
    \end{pmatrix} = \pare{E + \Omega\pare{t}\Delta t} \begin{pmatrix}
        x'\pare{t} \\ y'\pare{t} \\ z'\pare{t}
    \end{pmatrix} 
    = \begin{pmatrix}
        x'\pare{t} \\ y'\pare{t} \\ z'\pare{t}
    \end{pmatrix} + \Omega\pare{t} \begin{pmatrix}
        x'\pare{t} \\ y'\pare{t} \\ z'\pare{t}
    \end{pmatrix}\Delta t. \\
    & \Rightarrow \begin{pmatrix}
        \Delta x' \\ \Delta y' \\ \Delta z'
    \end{pmatrix} = \Omega \begin{pmatrix}
        x' \\ y' \\ z'
    \end{pmatrix}\Delta t 
    \Rightarrow \begin{pmatrix}
        \dot{x}' \\ \dot{y}' \\ \dot{z}'
    \end{pmatrix} = \Omega\pare{t}\begin{pmatrix}
        x' \\ y' \\ z'.
    \end{pmatrix} \\
    & \Rightarrow  \begin{pmatrix}
        \dot{x}' \\ \dot{y}' \\ \dot{z}'
    \end{pmatrix} = \begin{pmatrix}
        \Omega_{12} y' + \Omega_{13} z' \\
        -\Omega_{12}x' + \Omega_{23} z' \\
        -\Omega{13} x' - \Omega_{23} y'
    \end{pmatrix} = \begin{pmatrix}
        \omega_y z' - \omega_z y' \\
        \omega_z x' - \omega_x z' \\
        \omega_x y' - \omega_y x'
    \end{pmatrix} = \+v\omega\times\+vr.
\end{align*}
其中$\omega_x = -\Omega_{23}$, $\omega_y = \Omega_{13}$, $\omega_z = -\Omega_{12}$, 且
\[ \+v\omega = \omega_x \+ve'_x + \omega_y \+ve'_y + \omega_z \+ve'_z. \]
\begin{remark}
    角速度矢量的起源与其它矢量的来源不同, 其本质为一二阶反对称张量, 惟作用在矢量上时可类比叉乘.
\end{remark}
若为一坐标变换, 亦可证明
\[ \omega'_i = \sum_j c_{ij}\omega_j. \]
惟在镜像作用下, 旋转物体的$\+v\omega$不变, 谓轴矢量. 而一般矢量(极矢量)在镜像下反向. $\+J = \+vr\times\+vv$, $\+vM = \+vr\times\+vF$皆为轴矢量.

% paragraph 无限小转动 (end)

\paragraph{作业} % (fold)
\label{par:作业}

4.1, 4.2, 4.3

% paragraph 作业 (end)

% subsubsection 无限小转动与角速度 (end)

\subsubsection{刚体上任一点的速度与加速度} % (fold)
\label{ssub:刚体上任一点的速度与加速度}

纯转动下, 
\[ \Delta \+vr = \Delta\+v\varphi \times\+vr,\quad \+vv = \+v\omega\times\+vr,\quad \+va = \dot{\+v\omega}\times\+vv + \dot{\+v\omega}\times\pare{\+v\omega\times\+vr}. \]
在一般情形下, 取任意点$C$为基点, 则对刚体上任一点$P$都有
\begin{align*}
    \+vr_P &= \+vr_C + \pare{\+vr_P - \+vr_C},\quad \+vr_P = \+vr_C + \+v\omega\times\pare{\+vr_P - \+vr_C}. \\
    \+va_P &= \+va_C + \+v\omega\pare{\+vr_P - \+vr_C} + \+v\omega\times\pare{\+v\omega\times\pare{\+vr_P - \+vr_C}}.
\end{align*}
对于任意另一基点, 都有
\[ \+vv_P = \+vv_{C'} + \+v\omega\pare{\+vr_P - \+vr_{C'}}. \]
从而基点可以任意选取, $\+v\omega$都是相同的.

\begin{figure}[htb]
    \centering
    \begin{subfigure}{5cm}
        \centering
        \incfig{5cm}{InstantaneousCenter}
    \end{subfigure}
    \begin{subfigure}{5cm}
        \centering
        \incfig{5cm}{InstantaneousCenterPara}
    \end{subfigure}
    \caption{瞬心的确定}
\end{figure}

\paragraph{瞬时转轴} % (fold)
\label{par:瞬时转轴}

由$\+vv_C = \+vv_{C'} + \+v\omega\pare{\+vr_P - \+vr_{C'}}$, 故某时刻总能找到点$C$使$\+vv_C = 0$, 此时谓$C$为转动顺心. 若以瞬心为基点,
\[ \+vv_P = \+vv_C + \+v\omega \times\pare{\+vr_P - \+vr_C} = \+v\omega\times\pare{\+vr_P - \+vr_C}. \]
注意该式仅瞬时成立. 满足$\+vr_Q - \+vr_C \parallelsum \+v\omega$的$Q$构成一直线, 且$\+vv_Q = \+v\omega\pare{\+vr_Q - \+vr_C} = 0$谓转动瞬轴.
\begin{pitfall}
    $\+vv_C = 0$仅瞬时成立且可能$\+va_C \neq 0$, 此时仍有
    \[ \Delta \+vr = \Delta\+v\varphi \times\+vr,\quad \+vv = \+v\omega\times\+vr,\quad \+va = \dot{\+v\omega}\times\+vv + \dot{\+v\omega}\times\pare{\+v\omega\times\+vr}. \]
\end{pitfall}
对于平面平行运动的刚体, 任意取$A$, $B$, 若$\+vA$和$\+vB$不平行, 则其垂线的焦点为瞬心. 若$\+vA\parallelsum \+vB$, 则两者矢量端点的交点为瞬心.

% paragraph 瞬时转轴 (end)

% subsubsection 刚体上任一点的速度与加速度 (end)

% subsection 刚体运动的描述 (end)

\subsection{Euler运动学方程} % (fold)
\label{sub:euler运动学方程}

\subsubsection{Euler角} % (fold)
\label{ssub:euler角}

\begin{figure}[ht]
    \centering
    \incfig{8cm}{EulerAngles}
\end{figure}

静态定义要求$\+ve'_i$不变, $\+ve_i\pare{t}$随刚体运动. 设
\[ \begin{pmatrix}
    x \\ y \\ z
\end{pmatrix} = A \begin{pmatrix}
    x' \\ y' \\ z'
\end{pmatrix}, \]
则有$A_{ij} = \+ve_i\cdot\+ve'_j$, 可分解为
\begin{cenum}
    \item 进动$\displaystyle A_\varphi = \begin{pmatrix}
        \cos\varphi & \sin\varphi & 0 \\
        -\sin\varphi & \cos\varphi & 0 \\
        0 & 0 & 1
    \end{pmatrix}$.
    \item 章动$A_\theta = \displaystyle \begin{pmatrix}
        1 & 0 & 0 \\
        0 & \cos\theta & \sin\theta \\
        0 & -\sin\theta & \cos\theta
    \end{pmatrix}$.
    \item 自转$\displaystyle A_\psi = \begin{pmatrix}
        \cos\psi & \sin\psi & 0 \\
        -\sin\psi & \cos\psi & 0 \\
        0 & 0 & 1
    \end{pmatrix}$
\end{cenum}
其中$\varphi \in \lbr{0,2\pi}$, $\theta\in\lbr{0,\pi}$, $\psi\lbr{0,2\pi}$, 总的矩阵为
\[ A = A_\psi A_\theta A_\varphi, \]
于是$\pare{x'} = A^{-1}\pare{x}$. 其中
\[ A^{-1} = A\varphi^{-1}A_\theta^{-1}A_\psi^{-1} = A\varphi^{T}A_\theta^{T}A_\psi^{T} = A^T. \]

% subsubsection euler角 (end)

\subsubsection{Euler的刚体运动学方程} % (fold)
\label{ssub:euler的刚体运动学方程}

设节线为$ON$, 单位矢量$\+un$, 则角速度具有可加性, 因为其对应转动为$\rd{t}$内的无穷小转动, 从而
\[ \+v\omega = \dot\varphi\+ve'_z + \dot\theta\+un + \dot\psi\+ve_z. \]
其中
\begin{align*}
    \+ve'_z &= \sin\theta \sin\psi \+ve_x + \sin\theta\cos\psi\+ve_y + \cos\theta\+ve_z, \\
    \+vn &= \cos\psi \+ve_x - \sin\psi\+ve_y, \\
    \+v\omega &= \dot\varphi\pare{\sin\theta \sin\psi \+ve_x + \sin\theta\cos\psi\+ve_y + \cos\theta\+ve_z} \\
    & + \dot\theta\pare{\cos\psi\+ve_x - \sin\psi\+ve_y} + \dot\psi\+ve_z \\
    &= \pare{\dot\varphi \sin\theta\sin\psi + \dot\theta\cos\psi} \+ve_x \\
    &+ \pare{\dot\varphi \sin\theta\cos\psi - \dot\theta\sin\psi} \+ve_y \\
    &+ \pare{\dot\varphi \cos\theta + \dot\psi}\+ve_z \\
    &\Rightarrow \begin{cases}
        \omega_x = \dot\varphi \sin\theta\sin\psi + \dot\theta\cos\psi, \\
        \omega_y = \dot\varphi \sin\theta\cos\psi - \dot\theta\sin\psi, \\
        \omega_z = \dot\varphi \cos\theta + \dot\varphi.
    \end{cases}
\end{align*}
即得到Euler刚体运动学方程. 这是在本体坐标系中的分解. 若欲得到惯性坐标系中的分解, 由
\begin{align*}
    \+ve_z &= \sin\theta\cos\pare{\frac{\pi}{2} - \varphi}\+ve'_x + \sin\theta \cos\pare{\pi - \varphi} \+ve'_y + \cos\theta\+ve'_z, \\
    &= \sin\theta\sin\varphi \+ve'_x - \sin\theta\cos\varphi\+ve'_y + \cos\theta\+ve'_z. \\
    \+un &= \cos\varphi\+ve'_x + \sin\varphi\+ve'_y. \\
    \+v\omega &= \dot{\varphi}\+ve'_z + \dot\theta\pare{\cos\varphi\+ve'_x + \sin\varphi\+ve'_y}\dot\psi\+ve_z \\
    &\Rightarrow \begin{cases}
        \omega_{x'} = \dot\theta\cos\varphi + \dot\psi\sin\theta\sin\varphi, \\
        \omega_{y'} = \dot\theta\sin\varphi - \dot\psi\sin\theta\cos\varphi, \\
        \omega_{z'} = \dot\varphi + \dot\psi\cos\theta.
    \end{cases}
\end{align*}
这是惯性坐标系下的Euler运动学方程.
\begin{remark}
    也可以通过$-\psi$, $-\theta$, $-\varphi$的转动从本体坐标系得到惯性坐标系得到. 即
    \[ \varphi\rightarrow -\psi,\quad \theta \rightarrow -\theta, \quad \psi\rightarrow -\varphi, \quad \omega \rightarrow -\omega. \]
\end{remark}

% subsubsection euler的刚体运动学方程 (end)

% subsection euler运动学方程 (end)

\subsection{转动惯量张量与惯量主轴} % (fold)
\label{sub:转动惯量张量与惯量主轴}

\subsubsection{转动惯量张量} % (fold)
\label{ssub:转动惯量张量}

对于定点运动,
\begin{align*}
    \+vv_\alpha &= \+v\omega\times\+vr_\alpha,\quad \alpha = 1,\cdots,N, \\
    \+vJ &= \sum_\alpha \+vr_\alpha\times\pare{m_\alpha\+vv_\alpha} = \sum_\alpha m_\alpha\+vr_\alpha\times\pare{\omega\times\+vr_\alpha} \\
    &= \sum_\alpha m_\alpha\brac{r_\alpha^2\+v\omega - \pare{\omega\cdot\+vr_\alpha}\+vr_\alpha}, \\
    \+v\omega\cdot\+vr_\alpha &= \sum_{j=1}^3 \omega_j r_{\alpha j}, \\
    J_i &= \sum_\alpha m_\alpha\brac{r_\alpha^2 \omega_i - \pare{\+v\omega \cdot\+vr_\alpha}r_{\alpha i}} \\
    &= \sum_\alpha m_\alpha\brac{r_\alpha\omega_i - \sum_{j=1}^3 r_{\alpha,i}r_{\alpha,j}\omega_j}, \\
    J_i &= \underbrace{\sum_{j=1}^3 \brac{\sum_\alpha m_\alpha \pare{r_\alpha^2 \delta_{ij} - r_{\alpha i}r_{\alpha j}}}}_{I_{ij}} \omega_j.
\end{align*}
转动惯量张量为
\[ I_{ij} = \sum_\alpha m_\alpha\pare{r_\alpha^2\delta_{ij} - r_{\alpha,i}r_{\alpha,j}}, \]
满足$I_{ij} = I_{ji}$以及$\+vI^T = \+vI$. 特别地, 对角元为
\begin{align*}
    I_{11} &= \sum_\alpha m_\alpha\pare{r_\alpha^2 - x_\alpha^2} \\
    &= \sum_\alpha m_\alpha\pare{y_\alpha^2 + z_\alpha^2}, \\
    I_{22} &= \sum_\alpha m_\alpha\pare{z_\alpha^2 + x_\alpha^2}, \\
    I_{33} &= \sum_\alpha m_\alpha\pare{x_\alpha^2 + y_\alpha^2}.
\end{align*}
三个对角元分别是相对于三个轴的转动惯量.
\begin{align*}
    I_{12} = -\sum_\alpha m_\alpha x_\alpha y_\alpha = I_{21}, \\
    I_{13} = -\sum_\alpha m_\alpha x_\alpha z_\alpha = I_{31}, \\
    I_{23} = -\sum_\alpha m_\alpha y_\alpha z_\alpha = I_{32}.
\end{align*}
这是三个惯量积.
\par
对于连续介质,
\begin{align*}
    I_{ij} &= \int_V \rho\pare{r^2 \delta_{ij} - x_ix_j}\,\rd{V},\quad r^2 = \sum_{i=1}^3 x_i x_i.
\end{align*}
在一些情形下, 例如刚体是一个面/线, 则积分换为面积分/线积分, 密度换为面密度/线密度. 在这一定义下, 对于定点转动,
\[ \begin{pmatrix}
    J_1 \\ J_2 \\ J_3
\end{pmatrix} = \+vI \begin{pmatrix}
    \omega_1 \\ \omega_2 \\ \omega_3
\end{pmatrix}\Rightarrow \+vJ = \+vI \+v\omega. \]
定点转动情形下, 动能
\begin{align*}
    T &= \half \sum_\alpha m_\alpha v_\alpha^2 \\
    &= \half \sum_\alpha m_\alpha \+vv_\alpha\cdot\pare{\+v\omega\times\+vr_\alpha} \\
    &= \half\sum_\alpha m_\alpha \+v\omega \cdot\pare{\+vr_\alpha\times\+vv_\alpha} \\
    &= \half \+v\omega\cdot\+vJ = \half \begin{pmatrix}
        \omega_1 & \omega_2 & \omega_3
    \end{pmatrix} \begin{pmatrix}
        J_1 \\ J_2 \\ J_3
    \end{pmatrix} = \half \+v\omega \cdot \+vI\cdot \+v\omega.
\end{align*}
一般运动下, 取质心为基点, 无论质心是否做惯性运动, 都有
\begin{align*}
    \+vJ &= \+vr_0 \times M\+vv_0 + \+vJ' = \+vJ_0 + \+vI\cdot\+v\Omega. \\
    T &= T_0 + T' = T_0 + \half \+v\omega\cdot\+vI\cdot\+v\omega.
\end{align*}

\paragraph{作业} % (fold)
\label{par:作业}

4.4, 4.5, 4.6

% paragraph 作业 (end)

\begin{pitfall}
    转动惯量张量皆为相对于某点而言.
\end{pitfall}

\paragraph{转动惯量张量的性质} % (fold)
\label{par:转动惯量张量的性质}

这是一个对称张量, $I_{ij} = I_{ji}$, 且具有广延性(即可分为各部分后相加), 并且满足平行轴定理
\[ I_{ij}^Q = \underbrace{M\pare{a^2 \delta_{ij} - a_ia_j}}_{\text{质心相对$Q$}} + \underbrace{I_{ij}^C}_{\text{刚体相对质心}}. \]
\begin{figure}[ht]
    \centering
    \incfig{8cm}{ParallelAxisP}
    \caption{平行轴定理示意}
\end{figure}
这是因为
\begin{align*}
    I_{ij}^Q &= \sum_\alpha m_\alpha\pare{R_\alpha^2 \delta_{ij} - R_{\alpha,i}R_{\alpha,j}}, \\
    I_{ij}^C &= \sum_\alpha m_\alpha\pare{r_\alpha^2 \delta_{ij} - r_{\alpha,i}r_{\alpha,j}}, \\
    I_{ij}^Q - I_{ij}^C &= \sum_\alpha m_\alpha\brac{\pare{R_\alpha^2 - r_\alpha^2} \delta_{ij} + r_{\alpha,i}r_{\alpha,j} - R_{\alpha,i}R_{\alpha,j}} \\
    &= \sum_\alpha m_\alpha \brac{\pare{a^2 + 2\+va\cdot\+vr_\alpha}\delta_{ij} - r_{\alpha,i}a_j - a_ir_{\alpha,j} - a_ia_j} \\
    &= \sum_\alpha m_\alpha\pare{a^2\delta_{ij} - a_i a_j} + \cancelto{0}{\sum_\alpha m_\alpha \pare{2\+va \cdot \+vr_\alpha\delta_{ij} - r_{\alpha,i}a_j - r_{\alpha,j}a_i}} \\
    &= M\pare{a^2\delta_{ij} - a_ia_j}.
\end{align*}
式中的消去由
\begin{align*}
    \+va &= \frac{\sum_\alpha m_\alpha \+vR_\alpha}{M},\quad M = \sum_\alpha m_\alpha, \\
    \+vr_\alpha &= \+vR_\alpha - \+va \Rightarrow \sum_\alpha m_\alpha \+vr_\alpha = 0,\quad \sum_\alpha m_\alpha r_{\alpha,i} = 0
\end{align*}
保证.
\begin{cenum}
    \item $i=j$时, $I_{ii}^Q = I_{ii}^C + M\pare{a^2  -a_i^2}$;
    \item $i\neq j$时, $I_{jj}^Q = I_{ij}^C - Ma_i a_j$.
\end{cenum}

% paragraph 转动惯量张量的性质 (end)

\paragraph{空间转动下的变换} % (fold)
\label{par:空间转动下的变换}

$\+vJ = \+vI\+v\omega$, 在空间转动下
\[ \+vx = \+vA \+vx',\quad \+vJ = \+vA\+vJ',\quad \+v\omega = \+vA\+v\omega'. \]
其中$\+vA$满足$\+vA^T = \+vA^{-1}$. 从而
\[ \left. \begin{array}{c}
    \+vJ' = \+vI'\+v\omega' = \+vI'\+vA^{-1}\+v\omega, \\
    \+vJ' = \+vA^{-1}\+vJ = \+vA^{-1}\+vI\+v\omega,
\end{array} \right\} \Rightarrow \+vI'\+vA^{-1} = \+vA^{-1}\+vI. \]
这对于任何$\+v\omega$都成立, 故
\begin{align*}
    \+vI' &= \+vA^{-1} \+vI \+vA = \+vA^T \+vI \+vA. \\
    x_i &= \sum_j A_{ij}x'_j, \\
    I'_{ij} &= \pare{\+vA^T \+vI \+vA}_{ij} = \sum_{kl}A^T_{ik} I_{kl} A_{lj} \\
    &= \sum_{kl} A_{ki}A_{lj}I_{kl}.
\end{align*}

% paragraph 空间转动下的变换 (end)

\begin{sample}
    \begin{ex}
        设一质点系由位于$\pare{a,-a,a}$, $\pare{-a,a,a}$和$\pare{a,a,a}$的质量分别为$4m$, $3m$和$2m$的质点组成, 求关于坐标原点的惯量张量, 并求过原点, 方向$\pare{1,1,0}$的轴的转动惯量.
    \end{ex}
    \begin{solution}
        $r_1^2 = r_2^2 = r_3^2 = 3a^2$, $m_1 = 4m$, $m_2 = 3m$, $m_3 = 2m$,
        \begin{align*}
            r_{1,1} &= a,\quad r_{1,2} = -a,\quad r_{1,3} = a, \\
            r_{2,1} &= -a,\quad r_{2,2} = a,\quad r_{2,3} = a, \\
            r_{3,1} &= r_{3,2} = r_{3,3} = a. \\
            I_{ij} &= \sum_{\alpha=1}^3 m_\alpha \pare{r_\alpha^2 \delta_{ij} - r_{\alpha,i}r_{\alpha,j}} = 3a^2 \cdot 9m \delta_{ij} - \sum_{\alpha=1}^3 m_\alpha r_{\alpha,i} r_{\alpha,j} \\
            &= 27 ma^2 \delta_{ij} - 4mr_{1,i}r_{1,j} - 3mr_{2,i}r_{2,j} - 2mr_{3,i}r_{3,j}. \\
            I_{11} &= 27ma^2 - 4ma^2 - 3ma^2 - 2ma^2 = 18ma^2 = I_{22} = I_{33}, \\
            I_{12} &= 5ma^2,\quad I_{13} = -3ma^2,\quad I_{23} = -ma^2. \\
            \+vI &= \begin{pmatrix}
                18 & 5 & -3 \\
                5 & 18 & -1 \\
                -3 & -1 & 18
            \end{pmatrix}ma^2. \qedhere
        \end{align*}
    \end{solution}
    \begin{proof}[求轴上转动惯量]
        作坐标变换
        \begin{align*}
            \begin{pmatrix}
            x' \\ y' \\ z'
        \end{pmatrix} &= \underbrace{\begin{pmatrix}
            \cos \pi/4 & \sin \pi/4 & 0 \\
            -\sin\pi/4 & \cos\pi/4 & 0 \\
            0 & 0 & 1
        \end{pmatrix}}_{\+vA^{-1} = \+vA^T}\begin{pmatrix}
            x \\ y \\ z
        \end{pmatrix}. \\
        \+vA &= \begin{pmatrix}
            \sqrt{2}/2 & -\sqrt{2}/2 & 0 \\
            \sqrt{2}/2 & \sqrt{2}/2 & 0 \\
            0 & 0 & 1
        \end{pmatrix}.
        \end{align*}
        所求转动惯量即$I'_{11}$. 从而
        \begin{align*}
            I'_{11} &= \sum_{kl} A_{k1}A_{l1} I_{kl} \\
            &= \sum_{k=1}^3 A_{k1}\pare{A_{11}I_{k1} + I_{21}I_{k2} + A_{31}I_{k3}} \\
            &= A_{11}\pare{A_{11} I_{11} + A_{21}I_{12}} + A_{21}\pare{A_{11}I_{21} + A_{21}I_{22}} \\
            &= A_{11}^2 I_{11} + 2A_{11}A_{21}I_{12} + A_{21}^2I_{22} \\
            &= \half I_{11} + 2\cdot \half I_{12} + \half I_{22} \\
            &= \half \pare{I_{11} + I_{12} + I_{21} + I_{22}} \\
            &= 23ma^2. \qedhere
        \end{align*}
    \end{proof}
\end{sample}
\begin{sample}
    \begin{ex}
        质量$M$, 半径$R$的匀质圆盘, 中心在$O$处且圆盘在$xOy$平面内,
        \begin{align*}
            I_{ij} &= \int \rho \pare{r^2\delta_{ij} - x_ix_j}\,\rd{V} \\
            &= \frac{M}{\pi R^2} \int_0^R \int_0^{2\pi} \pare{r^2 \delta_{ij} - x_ix_j}r\,\rd{\varphi}\,\rd{r}. \\
            x_1 &= r\cos\varphi,\quad x_2 = r\sin\varphi,\quad x_3 = 0. \\
            I_{11} &= \frac{M}{\pi R^2} \int_0^R \int_0^{2\pi} \pare{r^2 - r^2\cos^2\varphi} r\,\rd\varphi \,\rd r \\
            &= \frac{M}{\pi R^2} \int_0^R r^3\,\rd{r}\,\int_0^{2\pi} \sin^2 \varphi\,\rd\varphi \\
            &= \frac{M}{\pi R^2} \cdot \rec{4} R^4 \pi = \rec{4}MR^2. \\
            I_{22} &= I_{11} = \rec{4}MR^2. \\
            I_{33} &= \frac{M}{\pi R^2} \int_0^R \int_{0}^{2\pi} r^3\,\rd\varphi \,\rd r = \half MR^2. \\
            I_{12} &= -\frac{M}{\pi R^2} \int_0^R \int_0^{2\pi} r^2\cos\varphi\sin\varphi\,\rd{\varphi}\,\rd{r} = 0. \\
            I_{13} &= 0,\quad I_{23} = 0. \\
            \+vI &= \begin{pmatrix}
                MR^2/4 & 0 & 0 \\
                0 & MR^2/4 & 0 \\
                0 & 0 & MR^2/2
            \end{pmatrix}.
        \end{align*}
    \end{ex}
\end{sample}

% subsubsection 转动惯量张量 (end)

\subsubsection{惯量主轴} % (fold)
\label{ssub:惯量主轴}

在惯性系中, $\dot{I}_{ij} \neq 0$, 但本体系中$\dot{I}_{ij} = 0$, 这是引入本体系的好处. 惟本体系并非唯一, 虽在一般本体系下$I_{ij}$的非对角元非零, 仍可重新选择本体系令$\+vI$对角化. 设有坐标变换
\[ \tilde{\+vx} = \+vA^{-1}\+vx \]
令$\tilde{\+vI} = \+vA^T \+vI \+vA$为对角矩阵, 则$\tilde{\+vx}$谓主轴系. 相应的坐标轴$\tilde{\+ve}_i$谓惯量主轴. 此时$I_1$, $I_2$, $I_3$谓主转动惯量.
\par
对$\abs{\+vI - \lambda\+vE} = 0$求根, 则三个根$\lambda_1$, $\lambda_2$, $\lambda_3$为本征值. 对本征矢量$\+vI X_i = \lambda_i X_i$正交归一化, 则
\[ \+vA = \begin{pmatrix}
    X_1 & X_2 & X_3
\end{pmatrix} \]
为过渡矩阵. 在新的坐标系下$\displaystyle \tilde{\+vI} = \begin{pmatrix}
    I_1 & 0 & 0 \\
    0 & I_2 & 0 \\
    0 & 0 & I_3
\end{pmatrix}$. 坐标变换
\begin{align*}
    \+vx &= \+vA\tilde{\+vx} \Rightarrow x_i = \sum_j A_{ij}\tilde{x}_j, \\
    \+vr &= \sum_i x_i \+ve_i = \sum_j \tilde{x}_j \tilde{\+ve}_j, \\
    x_i &= \+vr\cdot\+ve_i = \pare{\sum_j \tilde{x}_j\tilde{\+ve}_j}\cdot\+ve_i \\
    &= \sum_j \pare{\+ve_i \cdot \tilde{\+ve}_j} \tilde{x}_j. \\
    A_{ij} &= \+ve_i \cdot \tilde{\+ve}_j = \begin{pmatrix}
        X_1 & X_2 & X_3
    \end{pmatrix}_{ij}. \\
    X_i &= \begin{pmatrix}
        \+ve_1 \cdot \tilde{\+ve}_i \\
        \+ve_2 \cdot \tilde{\+ve}_i \\
        \+ve_3 \cdot \tilde{\+ve}_i
    \end{pmatrix}.
\end{align*}
即$\+vA$的第$i$列为第$i$个主轴在原坐标系的坐标.
\begin{sample}
    \begin{ex}
        立方体在一个顶点处的转动惯量为
        \[ \+vI = \begin{pmatrix}
            2/3 & -1/4 & -1/4 \\
            -1/4 & 2/3 & -1/4 \\
            -1/4 & -1/4 & 2/3
        \end{pmatrix}ma^2. \]
        为了求出对角化形式,
        \[ \abs{\+vI - \lambda\+vE} = 0 \Rightarrow \lambda_1 = \lambda_2 = \frac{11}{12}ma^3,\quad \lambda_3 = \rec{6}ma^2. \]
        \begin{cenum}
            \item 单根: $\lambda_3 = I_3 = ma^2/6$,
            \[ \begin{pmatrix}
                1/4 & -1/4 & -1/4 \\
                -1/4 & 1/2 & -1/4 \\
                -1/4 & -1/4 & 1/2
            \end{pmatrix}X_3 = 0 \Rightarrow X_3 = \pm\rec{\sqrt{3}} \begin{pmatrix}
                1 \\ 1 \\ 1
            \end{pmatrix}. \]
            \item 重根: $\lambda_{1,2} = 11ma^2/12$,
            \[ \begin{pmatrix}
                1 & 1 & 1 \\
                1 & 1 & 1 \\
                1 & 1 & 1
            \end{pmatrix}X_{1,2} = 0\Rightarrow x_1 + x_2 + x_3 = 0. \]
            用Gram-Schmidt正交化方法, 取
            \begin{align*}
                X_1 &= \mu_1 \begin{pmatrix}
                    1 \\ 1 \\ -2
                \end{pmatrix} \Rightarrow X_1 = \pm\rec{\sqrt{6}} \begin{pmatrix}
                    1 \\ 1 \\ -2
                \end{pmatrix}. \\
                a_2 &= \mu_2 \begin{pmatrix}
                    2 \\ -1 \\ -1
                \end{pmatrix} \Rightarrow X_2 = \mu_2 \brac{a - \pare{X_1^T a}X_1} = \pm \frac{\sqrt{2}}{3} \begin{pmatrix}
                    3/2 \\ -3/2 \\ 0
                \end{pmatrix}.
            \end{align*}
            为了不改变坐标的手性, 须$\abs{\+vA} = 1$. 取
            \[ X_3 = \rec{\sqrt{3}}\begin{pmatrix}
                1 \\ 1 \\ 1
            \end{pmatrix},\quad X_1 = \rec{\sqrt{6}}\begin{pmatrix}
                1 \\ 1 \\ -2
            \end{pmatrix}  \Rightarrow  X_2 = -\frac{\sqrt{2}}{3}\begin{pmatrix}
                3/2 \\ -3/2 \\ 0
            \end{pmatrix}. \]
        \end{cenum}
        从而\let\oldrec\rec\def\rec{1/}
            \[ \+vA = \begin{pmatrix}
                X_1 & X_2 & X_3
            \end{pmatrix} = \begin{pmatrix}
                \rec{\sqrt{6}} & -\rec{\sqrt{2}} & \rec{\sqrt{3}} \\
                \rec{\sqrt{6}} & \rec{\sqrt{2}} & \rec{\sqrt{3}} \\
                -2/{\sqrt{6}}   & 0 & \rec{\sqrt{3}}
            \end{pmatrix}. \]
        \let\rec\oldrec
        其中$X_3$就是立方体的对角线轴.
    \end{ex}
\end{sample}
\begin{remark}
    主轴坐标系也并非唯一.
\end{remark}
主轴与主转动惯量之性质有
\begin{cenum}
    \item 主轴彼此垂直, 即$X_1$, $X_2$, $X_3$是正交归一的.
    \item 非唯一性, 即在有多个特征值的情形下, 主轴有多种选择.
    \item 匀质刚体的对称轴, 旋转对称轴一定是惯量主轴, 且刚体对称面的法线一定是惯量主轴.
    \item $I_1 + I_2 \ge I_3$, 对于$xy$平面刚体必定$I_3 = I_1 + I_2$.
\end{cenum}

\paragraph{作业} % (fold)
\label{par:作业}

4.9-4.13.

% paragraph 作业 (end)

% subsubsection 惯量主轴 (end)

\subsubsection{惯量椭球} % (fold)
\label{ssub:惯量椭球}

为求刚体对任一轴$OX=\pare{\alpha,\beta,\gamma}$的转动惯量, 作空间转动变换
\[ X = A\tilde{X},\quad A = \begin{pmatrix}
    \alpha & * & * \\
    \beta & * & * \\
    \gamma & * & *
\end{pmatrix}. \]
则$OX$为$\tilde{\+ve}_1$轴. 问题转化为求$\tilde{I}_{11} = I$, $\tilde{\+vI} = A^T \+vI A$.
\begin{align*}
    I &= I_{11} = \sum_{ij} A_{i1}A_{j1}I_{ij} = \sum_{i} A_{i1} \pare{A_{11}I_{i1} + A_{21}I_{i2} + A_{31}I_{i3}} \\
    &= A_{11}\pare{A_{11}I_{11} + A_{21}I_{12} + A_{31}I_{13}} + A_{21}\pare{A_{11} I_{21} + A_{21}I_{22} + A_{31}I_{23}} \\
    &+ A_{31}\pare{A_{11} I_{31} + A_{21}I_{32} + A_{31}I_{33}} \\
    &= \alpha^2 I_{11} + \beta^2 I_{22} + \gamma^2 I_{33} + 2\alpha\beta I_{12} + 2\beta\gamma I_{23} + 2\gamma\alpha I_{31}.
\end{align*}
若原坐标系恰好为主轴坐标系, 则$I = \alpha^2 I_1 + \beta^2 I_2 + \gamma^2 I_3$.
\par
在$OX$上选取线段$OQ$, 使$OQ = 1/\sqrt{I}$. 则
\[ Q\pare{\frac{\alpha}{\sqrt{I}}, \frac{\beta}{\sqrt{I}}, \frac{\gamma}{\sqrt{I}}} = Q\pare{x,y,z}. \]
则$\pare{x,y,z}$满足一曲面方程
\[ x^2 I_{11} + y^2 I_{22} + z^2 I_{33} + 2xy I_{12} + 2yz I_{23} + 2zx I_{31} = 1. \]
在所有方向如此, 故所有的$Q$构成一椭球面, 谓惯量椭球. 惯量椭球面上总有
\[ I_Q = \rec{OQ^2}, \]
其中$I_Q$是$OQ$轴上的转动惯量. 若转动轴$\+v\omega$和惯量椭球有交点$Q\pare{x,y,z}$, $\+v\omega\parallelsum \+vr_Q$, 则$r_Q = OQ = 1/\sqrt{I}$.
\begin{align*}
    \+v\omega &= c \begin{pmatrix}
        x & y & z
    \end{pmatrix}^T, \\
    \+vJ &= \+vI\+v\omega = c\+vI \begin{pmatrix}
        x & y & z
    \end{pmatrix}.
\end{align*}
椭球面方程
\begin{align*}
    F\pare{x,y,z} &= I_{11}x^2 + I_{22}y^2 + I_{33}z^2 + 2I_{12}xy + 2I_{23}yz + 2I_{31}zx - 1 = 0. \\
    \grad F &= \begin{pmatrix}
        \+DxDF \\
        \+DyDF \\
        \+DzDF
    \end{pmatrix} = 2 \begin{pmatrix}
        I_{11} x + I_{12}y + I_{13}z \\
        I_{21} x + I_{22}y + I_{23}z \\
        I_{31} x + I_{32}y + I_{33}z
    \end{pmatrix} = 2\+vI \begin{pmatrix}
        x \\ y \\ z
    \end{pmatrix}. \\
    \Rightarrow & J \parallelsum \grad F.
\end{align*}
$\+u\omega$和$\+ur_Q$是一致的, $\displaystyle \frac{\omega_i}{\omega} = \frac{x_i}{r_Q} = \sqrt{I}x_i$.
\[ \omega_i = \sqrt{I}\omega x_i \Rightarrow \+u\omega = \sqrt{I}\omega \+ux. \]
惯量椭球的椭球面方程为$\+vx^T \+vI\+vx = 1$. 加上角动量与动能和角速度的关系,
\begin{align*}
    \+vJ &= \+vI\+v\omega = \sqrt{I}\omega \+vI\+vx, \\
    2T &= \+v\omega^T \+vI\+v\omega = \sqrt{I} \omega \+vx^T \+vI \pare{\sqrt{I}\omega\+vx} = I\omega^2 \+vx^T\+vI\+vx = I\omega^2. \\
    \Rightarrow \+vJ &= \sqrt{2T}\+vI\+vx. \\
    \grad F &= 2\+vI\+vx \Rightarrow \grad F = \sqrt{\frac{2}{T}}\+vJ.
\end{align*}
主轴系中椭球为$I_1x^2 + I_2 y^2 + I_3z^2 = 1$, 故三个坐标轴即椭球的三个对称轴.
\par
\begin{figure}[htbp]
    \centering
    \incfig{6cm}{EllipInertia}
    \caption{惯量椭球和角速度与角动量的关系}
\end{figure}
若$\+v\omega$沿着某一主轴, 如$\+ve_z$, 则$\omega_z = \omega$, $\omega_x = \omega_y = 0$, 则在主轴坐标系中,
\[ \+vJ = \+vI\+v\omega = \+vI\+v\omega = \begin{pmatrix}
    I_1 & & \\
    & I_2 & \\
    & & I_3
\end{pmatrix}\begin{pmatrix}
    \omega_x \\ \omega_y \\ \omega_z
\end{pmatrix} = \begin{pmatrix}
    0 \\ 0 \\ I_3\omega
\end{pmatrix} = I_3 \+v\omega. \]
又如若$I_1 = I_2 = I_3 = I$, 则$\+vJ = \+vI\+v\omega = I\+v\omega$, 从而在任何情形下, $\+vJ\parallelsum \+v\omega$.

% subsubsection 惯量椭球 (end)

% subsection 转动惯量张量与惯量主轴 (end)

\subsection{Euler动力学方程及其应用} % (fold)
\label{sub:euler动力学方程及其应用}

\subsubsection{Euler动力学方程的建立} % (fold)
\label{ssub:euler动力学方程的建立}

对于完整系的刚体, 定点运动的动能为$\displaystyle T = \half \+v\omega^T \+vI \+v\omega$. 在主轴系中为
\[ T = \half \pare{I_1 \omega_x^2 + I_2\omega_y^2 + I_3\omega_z^2}. \]
代入
\begin{align*}
    \omega_x &= \dot{\varphi}\sin\theta\sin\psi + \dot\theta\cos\psi, \\
    \omega_y &= \dot\varphi\sin\theta\cos\psi - \dot\theta\sin\psi, \\
    \omega_z &= \dot\varphi\cos\theta + \dot\psi. \\
    T &= \half\bigg[\pare{I_1 - I_2}\pare{\dot\theta\cos\psi + \dot\varphi\sin\theta\sin\psi}^2 \\ &+ I_2\pare{\dot\theta^2 + \dot\varphi^2\sin^2\theta} + I_3\pare{\dot\psi + \dot\varphi\cos\theta}^2\bigg].
\end{align*}
其中$\pare{\varphi,\theta,\psi}$为广义坐标. Lagrange方程为
\begin{align*}
    & \+dtd{} \+D{\dot\psi}D{T} - \+D\psi DT = Q_\psi = N_z. \\
    & \+D{\dot\psi}DT = I_3\omega_z \+D{\dot\psi}D{\omega_z} = I_3\omega_z. \\
    & \+D\psi DT = I_1\omega_x \+D\psi D{\omega_x} + I_2\omega_y \+D\psi D{\omega_y} = I_1\omega_x\pare{\dot\varphi\sin\theta\cos\psi - \dot\theta\sin\psi} \\
    &+ I_2\omega_y\pare{-\dot\varphi\sin\theta\sin\psi - \dot\theta\cos\psi} \\
    &= I_1\omega_x \omega_y - I_2\omega_y\omega_x = \pare{I_1 - I_2}\omega_x \omega_y. \\
    &\Rightarrow I_3 \dot{\omega}_z - \pare{I_1 - I_2}\omega_x\omega_y = N_z.
\end{align*}
可以直接作循环置换得到另外两个动力学方程.
\begin{finale}
    \begin{theorem}[Euler动力学方程]
        \begin{align*}
    I_1\dot{\omega}_x - \pare{I_2 - I_3}\omega_y\omega_z &= N_x, \\
    I_2\dot{\omega}_y - \pare{I_3 - I_1}\omega_z\omega_x &= N_y, \\
    I_3\dot{\omega}_z - \pare{I_1 - I_2}\omega_x\omega_y &= N_z.
\end{align*}
    \end{theorem}
\end{finale}
也可以使用Newton力学直接推导, 在主轴坐标系中
\[ \+vJ = I_1\omega_x \+ve_x + I_2\omega_y\+ve_y + I_3\omega_z\+ve_z = \sum_i J_i\+ve_i = \sum_i I_i\omega_i\+ve_i. \]
角动量定理表明,
\begin{align*}
    \+dtd{\+vJ} &= \+vN,\quad \+vN = \sum_i N_i \+ve_i,\quad \dot{\+ve_i}\neq 0. \\
    \+dtd{\+vJ} &= \sum_i \pare{\dot{J}_i \+ve_i + J_i \dot{\+ve}_i} = \sum_i \pare{I_i \dot{\omega}_i \+ve_i + J_i \+v\omega\times\+ve_i}\\ &= \sum_i I_i\dot\omega_i\+ve_i + \+v\omega\times\+vJ = \sum_i \brac{I_i \dot{\omega}_i \pare{\+v\omega\times\+vJ}_i}\+ve_i \\
    &= \sum_i \brac{I_i \dot\omega_i + \sum_{kj}\epsilon_{ijk}\omega_j J_k}\+ve_i \\
    &= \sum_i \brac{I_i\dot\omega_i + \sum_{kj}\epsilon_{ijk}I_k\omega_j\omega_k}\+ve_i \\
    &= \sum_i N_i \+ve_i. \\
    & \Rightarrow I_i \dot\omega_i + \sum_{kj}\epsilon_{ijk}I_k\omega_j\omega_k = N_i. \\
    i=1 & \Rightarrow I_1\dot\omega_x + I_3\omega_y\omega_z - I_2\omega_z\omega_y = N_x, \\
    &\Rightarrow I_1\dot{\omega}_x - \pare{I_2 - I_3}\omega_y\omega_z = N_x.
\end{align*}
类似对于$i=2$和$i=3$得到上述Euler方程.
\begin{figure}[ht]
    \centering
    \incfig{6cm}{ObtainTorque}
    \caption{}
    \label{fig:求力矩}
\end{figure}
\begin{sample}
    \begin{ex}
        如\cref{fig:求力矩}, 在主轴系中, $\+ve_x$, $\+ve_z$, $\+v\omega$共面,
        \begin{align*}
            & I_3 = \half MR^2,\quad I_1 = I_2 = \rec{4}MR^2, \\
            & \omega_z = \+v\omega\cdot\+ve_z = \omega\cos\theta,\quad \omega_y = 0, \\
            & \omega_x = -\omega\sin\theta. \\
            & \dot\omega_x = \dot\omega_y = \dot\omega_z = 0. \\
            & N_x = \pare{I_3 - I_2}\omega_y\omega_z = 0, \\
            & N_z = \pare{I_2 - I_1}\omega_x\omega_y = 0, \\
            & N_y = \pare{I_1 - I_3}\omega_x\omega_z = \rec{4}MR^2\omega^2 \sin\theta\cos\theta, \\
            &\+vN = N_y\+ve_y = \rec{4}MR^2\omega^2\sin\theta\cos\theta\+ve_y. \\
            & F = \frac{\abs{\+vN}}{d} = \rec{4d}MR^2 \omega^2\sin\theta\cos\theta\+ve_y.
        \end{align*}
        其中$\+vN$的方向将随时间变化.
    \end{ex}
\end{sample}

% subsubsection euler动力学方程的建立 (end)

\subsubsection{Euler陀螺的一般解} % (fold)
\label{ssub:euler陀螺的一般解}

Euler陀螺谓$\+vN=0$的情形, 此时
\begin{align*}
    I_1\dot\omega_x &= \pare{I_2 - I_3}\omega_y \omega_z, \\
    I_2\dot\omega_y &= \pare{I_3 - I_1}\omega_z \omega_x, \\
    I_3\dot\omega_z &= \pare{I_1 - I_2}\omega_x \omega_y.
\end{align*}
由外力不做功, $\+vJ$与$T$皆为运动积分,
\begin{align*}
    J^2 &= I_1\omega_x^2 + I_2\omega_y^2 + I_3\omega_z^2 = \const, \\
    T &= \half \pare{I_1\omega_x^2 + I_2\omega_y^2 + I_3\omega_z^2} = \const.
\end{align*}
从而
\begin{align*}
    \omega_x^2 &= \frac{I_3\pare{I_3 - I_2}\omega_z^2 + 2I_2T - J^2}{I_1\pare{I_2 - I_1}}, \\
    \omega_y^2 &= \frac{I_3\pare{I_1 - I_3}\omega_z^2 + J^2 - 2I_1T}{I_2\pare{I_2 - I_1}}.
\end{align*}

\paragraph{作业} % (fold)
\label{par:作业}

4.16 -- 4.20

% paragraph 作业 (end)

\par
对于自由刚体, $V = 0$,
\begin{align*}
    L &= T = \half\bigg[\pare{I_1 - I_2}\pare{\dot\theta\cos\psi + \dot\varphi\sin\theta\sin\psi}^2\\ & + I_2\pare{\dot\theta^2 + \dot\varphi^2\sin^2\theta} + I_3\pare{\dot\psi + \dot\varphi\cos\theta}^2\bigg]. \\
    H &= T = \const,\quad p_\varphi = \+D{\dot\varphi}DL = J_{z'} = \const,\quad J_{y'} = \const,\quad J_{x'} = \const.
\end{align*}
在主轴系中, $J^2 = \const$. 但$J_x$, $J_y$, $J_z$不能保证.

\begin{figure}[htb]
    \centering
    \incfig{6cm}{Poinsot}
    \caption{Poinsot几何法示意}
\end{figure}

\paragraph{Poinsot几何法} % (fold)
\label{par:poinsot几何法}

在惯性系中取平面使$\+vJ$与之正交, $\+v\omega \parallelsum \+vr_Q$,
\begin{align*}
    \+vr_Q &= \rec{\sqrt{I}}\frac{\+v\omega}{\omega} = \frac{\+v\omega}{\sqrt{2T}},\quad T = \half \+v\omega\cdot J = \half I\omega^2. \\
    R &= \abs{\+vr_Q \cdot \frac{\+vJ}{J}} = \abs{\frac{\+v\omega}{\sqrt{2T}}\cdot\frac{\+vJ}{J}} \\
    &= \abs{\frac{2T}{\sqrt{2T}J}} = \frac{\sqrt{2T}}{J} = \const.
\end{align*}
$OQ$为刚体的转动瞬轴, $\+vv_Q = \+v\omega\times\+vr_Q = 0$, 故$Q$处瞬时速度为零, 从而惯量椭球作无滑动滚动.

% paragraph poinsot几何法 (end)

\paragraph{转动的稳定性} % (fold)
\label{par:转动的稳定性}

设$I_3 > I_2 > I_1$, 若$\+v\omega$沿着某一主轴, 例如$\+ve_x$, 则
\begin{align*}
    \+vJ &= I_1\omega_x = I_1\+v\omega, \\
    \+vN &= 0,\quad \dot{\+vJ} = 0,\quad \+v\omega = 0, \\
    \dot{\omega}_x &= 0,\quad \dot{\omega}_y = 0,\quad \dot{\+v\omega}_z = 0.
\end{align*}
其中
\begin{align*}
    \dot{\+v\omega} &= \dot{\omega}_i \+ve_i + \omega_i \dot{\+ve}_i \\
    &= \dot{\omega}_i \+ve_i + \omega_i\pare{\+v\omega\times\+ve_i} \\
    &= \dot{\omega}_i \+ve_i.
\end{align*}
在$\+v\omega$中引入微扰,
\[ \+v\omega = \omega_x\+ve_x + \omega_y\+ve_y + \omega_z\+ve_z. \]
\begin{cenum}
    \item 当$\abs{\omega_y} , \abs{\omega_z} \ll \abs{\omega_x}$, 有
\begin{align*}
    I_1\dot{\omega}_x &= \pare{I_2 - I_3}\omega_y\omega_z \approx 0\Rightarrow \omega_x \approx \const. \\
    I_2\dot{\omega}_y &= \pare{I_3 - I_1}\omega_z\omega_x \Rightarrow \dot{\omega}_y = \pare{\frac{I_3-I_1}{I_2}\omega_x}\omega_z, \\
    I_3\dot{\omega}_z &= \pare{I_1 - I_2}\omega_x\omega_y \Rightarrow \dot{\omega}_z = \pare{\frac{I_1 - I_2}{I_3}\omega_x}\omega_y. \\
    \ddot{\omega}_y &= \pare{\frac{I_3-I_1}{I_2}\omega_x}\pare{\frac{I_1-I_2}{I_3}\omega_x}\omega_y, \\
    & \ddot{\omega}_y + \brac{\frac{\pare{I_3 - I_1}\pare{I_2-I_1}}{I_2I_3}\omega_x^2}\omega_y = 0, \\
    & \ddot{\omega}_z + \brac{\frac{\pare{I_3 - I_1}\pare{I_2-I_1}}{I_2I_3}\omega_x^2}\omega_z = 0. \\
    \Omega_1^2 &= \frac{\pare{I_3-I_1}\pare{I_2-I_1}}{I_2I_3}\omega_x^2 > 0. \\
    \omega_{y,z} &= A_1\cos\Omega_1 t + B_1\sin\Omega_1 t.
\end{align*}
    \item 当$\abs{\omega_x} , \abs{\omega_y} \ll \abs{\omega_z}$, 有
    \begin{align*}
        & \omega_z \approx \const, \\
        & \ddot{\omega}_{x,y} + \brac{\frac{\pare{I_3 - I_1}\pare{I_3 - I_2}}{I_1I_2}\omega_z^2}\omega_{x,y} = 0, \\
        & \Omega^2_3 > 0 \Rightarrow \text{稳定}, \\
        & \omega_{x,y} = A_3\cos\Omega_3 t + B_3\sin\Omega_3 t.
    \end{align*}
    \item 当$\abs{\omega_x} , \abs{\omega_z} \ll \abs{\omega_y}$, 有
    \begin{align*}
        & \omega_y \approx \const, \\
        & \ddot{\omega}_{x,z} - \brac{\frac{\pare{I_2 - I_1}\pare{I_3 - I_2}}{I_1I_3}\omega_y^2}\omega_{x,z} = 0, \\
        & \Omega_2^2 > 0, \\
        & \omega_{x,z} = A_2e^{\Omega_2 t} + B_2 e^{-\Omega_2 t}.
    \end{align*}
\end{cenum}
因此只有绕着主转动惯量最大或最小的主轴转动时, 平衡才是稳定的.

% paragraph 转动的稳定性 (end)

% subsubsection euler陀螺的一般解 (end)

\subsubsection{对称Euler陀螺} % (fold)
\label{ssub:对称euler陀螺}

$\+vN = 0$时, $I_1 = I_2 \neq I_3$, 从而$\+ve_z$为对称轴,
\begin{align*}
    & \left\{ \begin{aligned}
        &I_1\dot\omega_x = \pare{I_1 - I_3} \omega_y\omega_z, \\
        &I_1\dot\omega_y = \pare{I_3 - I_1} \omega_z\omega_x, \\
        &I_3\dot\omega_z = \pare{I_1 - I_2} \omega_x\omega_y = 0.
    \end{aligned} \right. \Rightarrow \left\{ \begin{aligned}
        &\dot\omega_x = \pare{\frac{I_1-I_3}{I_1}\omega_z} \omega_y, \\
        &\dot\omega_y = \pare{\frac{I_3-I_1}{I_1}\omega_z} \omega_x, \\
        &\omega_z = \const.
    \end{aligned} \right. \\
    & \ddot{\omega}_{x,y} + \pare{\frac{I_3-I_1}{I_1}\omega_z}^2 \omega_{x,y} = 0, \quad \Omega = \abs{\frac{I_3 - I_1}{I_1}\omega_z}, \\
    & \omega_x = A\cos\pare{\Omega t + \phi_0}, \\
    & \dot{\omega_x} = -\omega A\sin\pare{\Omega t + \phi_0}, \\
    & \omega_y = \frac{\dot{\omega}_x}{\frac{I_1-I_3}{I_1}\omega_z} = \frac{\abs{\pare{I_3-I_1}\omega_z}}{\pare{I_3-I_1}\omega_z}A\sin\pare{\Omega t + \phi_0} \\
    &= \begin{cases}
        A\sin\pare{\Omega t + \phi_0},\quad \pare{I_3 -I_1}\omega_z > 0, \\
        -A\sin\pare{\Omega t + \phi_0},\quad \pare{I_3 - I_1}\omega_z < 0.
    \end{cases}
\end{align*}
解的物理意义为
\begin{align*}
    \omega &= \sqrt{\omega_x^2 + \omega_y^2 + \omega_z^2} = \sqrt{A^2 + \omega_z^2} = \const. \\
    J &= \sqrt{J_x^2 + J_y^2 + J_z^2} = \sqrt{I_1^2A^2 + I_3^2\omega_z^2} = \const.
\end{align*}
\begin{figure}[ht]
    \centering
    \incfig{5cm}{EulerEulerAngle}
    \caption{}
\end{figure}
在惯性系中, $\+vJ$守恒, 取$\+ve'_z$为其方向, $\+vJ = J\+ve'_z$. 主轴系中
\begin{align*}
    & J_x = J \sin\theta\sin\psi = I_1\omega_x = I_1 A\cos\pare{\Omega t + \phi_0}, \\
    & J_y = J\sin\theta\cos\psi = I_1\omega_y = \pm I_1 A\sin\pare{\Omega t + \phi_0}, \\
    & J_z = J\cos\theta = I_3\omega_z = \const \Rightarrow \cos\theta = \frac{I_3\omega_z}{J} = \const = \cos\theta_0. \\
    & \omega_z = \dot\varphi\cos\theta+\dot\psi = \dot\varphi\cos\theta_0 \mp \Omega, \\
    & \dot\varphi \cos\theta_0 = \omega_z \mp\Omega = \left\{ \begin{aligned}
        &\omega_z + \frac{I_3-I_1}{I_1}\omega_z, && \pare{I_3 - I_1}\omega_z > 0, \\
        &\omega_z - \pare{-\frac{I_3-I_1}{I_1}\omega_z}, && \pare{I_3-I_1}\omega_z < 0
    \end{aligned} \right. = \frac{I_3}{I_1}\omega_z. \\
    & \Rightarrow \left\{ \begin{aligned}
        & \varphi = \frac{I_3\omega_z}{I_1\cos\theta_0} t + \const, \\
        & \theta = \theta_0 = \const, \\
        & \psi = \frac{\pi}{2} \mp \pare{\Omega t + \phi_0} = \frac{I_1-I_3}{I_1}\omega_z t + \const.
    \end{aligned} \right.
\end{align*}
故章动角保持不变, 进动角和自转角匀速上升, 此种运动谓规则进动.

\paragraph{本体极锥与空间极锥} % (fold)
\label{par:本体极锥与空间极锥}

若$I_3 > I_1$, 则$J = \sqrt{I_1^2 A^2 + I_3^2\omega_z^2} < \sqrt{I_3^2\pare{A^2 + \omega_z^2}} = I_3\omega$.
\[ \frac{\+vJ}{J}\cdot \+ve_z = \frac{J_z}{J} = \frac{I_3\omega_z}{J} > \frac{I_3\omega_z}{I_3\omega} = \frac{\omega_z}{\omega} = \frac{\+v\omega}{\omega}\cdot \+vz. \]
\begin{figure}[ht]
    \centering
    \incfig{6cm}{eJw}
    \caption{$I_3 > I_1$时的本体极锥}
\end{figure}
可以证明$\pare{\+vJ\times\+v\omega}\cdot\+ve_z = 0$, 从而$\+vJ$, $\+v\omega$, $\+ve_z$共面且前二者与$\+ve_z$的夹角为常数. 由
\[ \omega_x = A\cos\pare{\Omega t + \phi_0},\quad \omega_y = A\sin\pare{\Omega t + \phi_0}, \]
$\+v\omega$在本体系中会逆时针旋转.
\begin{figure}[ht]
    \centering
    \incfig{6cm}{ewJ}
    \caption{$I_3 < I_1$时的本体极锥}
\end{figure}
\begin{figure}[ht]
    \centering
    \incfig{6cm}{OmegaConeSpatial}
    \caption{$I_3>I_1$时的空间极锥}
\end{figure}
\begin{figure}[ht]
    \centering
    \incfig{6cm}{OmegaConeSpatial2}
    \caption{$I_3<I_1$时的空间极锥}
\end{figure}
当$I_3 < I_1$, 情况也是类似的, 惟
\[ \frac{\+vJ}{J}\cdot \+ve_z < \frac{\+v\omega}{\omega}\+ve_z, \]
且由
\[ \omega_x = A\cos\pare{\Omega t + \phi_0},\quad \omega_y = -A\sin\pare{\Omega t + \phi_0}. \]
知$\+v\omega$顺时针旋转.
\par
在惯性系中,
\begin{align*}
    \omega_{x'} &= \dot\psi\sin\theta\sin\varphi + \dot\theta\cos\varphi = \frac{I_1-I_3}{I_1}\omega_z \sin\theta_0 \sin\pare{\frac{I_3\omega_z}{I_1\cos\theta_0}t + \const}, \\
    \omega_{y'} &= -\dot\psi\sin\theta\cos\varphi + \dot\theta\sin\varphi = -\frac{I_1-I_3}{I_1}\omega_z \sin\theta_0 \cos\pare{\frac{I_3\omega_z}{I_1\cos\theta_0}t + \const}, \\
    \omega_{z'} &= \dot\psi\cos\theta + \dot\varphi = \frac{I_1\cos^2\theta_0 + I_3\sin^2\theta_0}{I_1\cos\theta_0}\omega_z.
\end{align*}

% paragraph 本体极锥与空间极锥 (end)

% subsubsection 对称euler陀螺 (end)

\subsubsection{Lagrange陀螺} % (fold)
\label{ssub:lagrange陀螺}

\begin{figure}[ht]
    \centering
    \incfig{8cm}{LagrangeTop}
    \caption{Lagrange陀螺}
\end{figure}

关于$\+ve_z$对称, 即$I_1 = I_2$, 且在重力矩下运动的陀螺, 谓Lagrange陀螺.
\begin{align*}
    V &= mgl\cos\theta,\\
    L &= T - V = \half\brac{I_1 \pare{\dot\theta^2 + \dot\varphi^2 \sin^2\theta} + I_3 \pare{\dot\psi + \dot\varphi\cos\theta}^2} - mgl\cos\theta. \\
    \+D\varphi DL &= 0 \Rightarrow p_\varphi = \+D{\dot\varphi}DL = I_1\dot\varphi\sin^2\theta + I_3\cos\theta\pare{\dot\psi + \dot\varphi\cos\theta} = J_{z'} = \const. \\
    \+D\psi DL &= 0 \Rightarrow p_\psi = \+D{\dot\psi}DL = I_3\pare{\dot\psi + \dot\varphi\cos\theta} = I_3\omega_z = J_z = \const. \\
    \+DtDL &= 0 \Rightarrow H = p_\varphi\dot\varphi + p_\psi\dot\psi + p_\theta\dot\theta - L = T+V = E \\
    &\Rightarrow \half I_1\pare{\dot\theta^2 + \dot\varphi^2\sin^2\theta} + \frac{J_z^2}{2I_3}+mgl\cos\theta = E = \const. \\
    \dot\varphi &= \frac{J_{z'} - J_z\cos\theta}{I_1\sin^2\theta},\quad \dot\psi = \frac{J_z}{I_3} - \dot\varphi\cos\theta = \frac{J_z}{I_3} - \frac{J_{z'} - J_z\cos\theta}{I_1\sin^2\theta} \cos\theta. \\
    E' &= E - \frac{J_z^2}{2I_3} = \half I_1\dot\theta^2 + \underbrace{\frac{\pare{J_{z'}-J_z\cos\theta}^2}{2I_1\sin^2\theta} + mgl\cos\theta}_{V\+_eff_\pare{\theta}}. \\
    \half I_1\dot\theta^2 &= E' - V\+_eff_\pare{\theta} \ge 0 \Rightarrow \dot\theta = \+dtd\theta = \pm \sqrt{\frac{2}{I_1}\brac{E' - V\+_eff_\pare{\theta}}}. \\
    t\pare{\theta} &= \pm \frac{\rd{\theta}}{\sqrt{\frac{2}{I_1}\brac{E' - V\+_eff_\pare{\theta}}}}.
\end{align*}
从而可以反解出$\theta\pare{\theta}$, 以及$\varphi\pare{t}$, $\psi\pare{t}$.
\begin{figure}[ht]
    \centering
    \incfig{6cm}{Thetas}
    \caption{$V\+_eff_$和$\theta$的关系}
\end{figure}
\par
章动角$\theta\in\brac{0,\pi}$, 在两端点处显然$V\+_eff_\rightarrow \infty$, 而极值点要求
\begin{align*}
    & \left.\+d\theta d{V\+_eff_} \right\vert_{\theta=\theta_0} = 0, \\
    & \frac{J_{z'} - J_z\cos\theta_0}{I_1\sin\theta_0}J_z - \frac{\pare{J_{z'}-J_z\cos\theta_0}^2}{I_1\sin^3\theta_0} \cos\theta_0 - mgl\sin\theta_0 = 0, \\
    & mgl\cos\theta_0 = \frac{\pare{J_{z'} - J_z\cos\theta_0}J_z\cos\theta_0}{I_1\sin^2\theta_0} - \frac{\pare{J_{z'} - J_z\cos\theta_0}^2\cos^2\theta_0}{I_1\sin^4\theta}, \\
    & \pare{J_{z'} - J_z\cos\theta}^2 - \frac{J_z\sin^2\theta_0}{\cos\theta_0}\pare{J_{z'} - J_z\cos\theta_0} + \frac{mglI_1\sin^4\theta_0}{\cos\theta_0} = 0.
\end{align*}
极值点的存在性要求
\[ \Delta = \frac{J_z^2\sin^4\theta_0}{\cos^2\theta_0} - \frac{4mglI_1\sin^4\theta_0}{\cos\theta_0} \ge 0,\quad J_z^2 \ge 4mglI_1\cos\theta_0. \]
极值点处
\begin{align*}
    & \left.\frac{\rd{^2V\+_eff_}}{\rd\theta^2}\right\vert_{\theta=\theta_0}\\  &= \frac{J_z^2}{I_1} - \frac{4J_z\cos\theta_0\pare{J_{z'} - J_z\cos\theta_0}}{I_1\sin^2\theta_0} + \frac{\pare{4-3\sin^2\theta_0}\pare{J_{z'}-J_z\cos\theta_0}^2}{I_1\sin^4\theta_0} \\
    &= \frac{J_z^2}{I_1} - 4J_z\cos\theta_0 \dot\varphi_0 + I_1\pare{4-3\sin^2\theta_0}\dot\varphi_0^2 \\
    &= I_1 \brac{\pare{\frac{J_z}{I_1} - 2\dot\varphi_0\cos\theta_0}^2 + \dot\varphi_0^2\sin^2\theta_0} > 0.
\end{align*}
故极值点为极小值点且唯一.
\par
在$\theta_1$和$\theta_2$处, $\dot\theta = 0$, 且
\[ \theta_1 \le \theta \le \theta_2,\quad \cos\theta_1 \ge \cos\theta \ge \cos\theta_2. \]
从而进动角速度
\[ \dot\varphi = \frac{J_{z'} - J_z\cos\theta}{I_1\sin^2\theta}. \]
\begin{figure}[ht]
    \centering
    \begin{subfigure}{3.9cm}
        \centering
        \incfig{3.5cm}{LagrangeA}
        \caption{}
        \label{fig:Lagrange陀螺A}
    \end{subfigure}
    \begin{subfigure}{3.9cm}
        \centering
        \incfig{3.5cm}{LagrangeB}
        \caption{}
        \label{fig:Lagrange陀螺B}
    \end{subfigure}
    \begin{subfigure}{3.9cm}
        \centering
        \incfig{3.5cm}{LagrangeC}
        \caption{}
        \label{fig:Lagrange陀螺C}
    \end{subfigure}
    \begin{subfigure}{3.9cm}
        \centering
        \incfig{3.5cm}{LagrangeD}
        \caption{}
        \label{fig:Lagrange陀螺D}
    \end{subfigure}
    \begin{subfigure}{3.9cm}
        \centering
        \incfig{3.5cm}{LagrangeE}
        \caption{}
        \label{fig:Lagrange陀螺E}
    \end{subfigure}
    \caption{}
\end{figure}
\begin{cenum}
    \item $\displaystyle \arccos \frac{J_{z'}}{J_z} < \theta_1 < \theta_2$, $\displaystyle J_{z'}/J_z > \cos\theta_1 > \cos\theta_2$, 则
    \[ J_{z'} > J_z\cos\theta,\quad \dot\varphi > 0. \]
    此时运动如\cref{fig:Lagrange陀螺A}.
    \item $\displaystyle \theta_1 < \theta_2 < \arccos \frac{J_{z'}}{J_z}$, $\displaystyle \cos\theta_1 > \cos\theta_2 > J_{z'}/J_z$, 则
    \[ J_{z'} < J_z\cos\theta,\quad \dot\varphi < 0. \]
    此时运动如\cref{fig:Lagrange陀螺B}.
    \item $\displaystyle \arccos \frac{J_{z'}}{J_z} = \theta_1 < \theta_2$, $J_{z'} = J_z\cos\theta_1$, 则
    \[ J_{z'} - J_z\cos\theta > 0,\quad \dot\varphi > 0,\quad \theta \neq \theta_1,\quad \dot\varphi = 0,\quad \theta = \theta_1. \]
    此时运动如\cref{fig:Lagrange陀螺C}.
    \item $\displaystyle \theta_1 < \theta_2 = \arccos \frac{J_{z'}}{J_z}$, $J_{z'} = J_z\cos\theta_2$, 则
    \[ J_{z'} - J_z\cos\theta < 0,\quad \dot\varphi < 0,\quad \theta \neq \theta_2,\quad \dot\varphi = 0,\quad \theta = \theta_2. \]
    此时运动如\cref{fig:Lagrange陀螺D}.
    \item $\displaystyle \theta_1 <\arccos \frac{J_{z'}}{J_z}< \theta_2$, $\cos\theta_1 > J_{z'}/J_z > \cos\theta_2$, 则
    \[ J_z \cos\theta_1 > J_{z'} > J_z\cos\theta_2, \dot\varphi < 0,\quad \theta=\theta_1, \quad \dot\varphi > 0, \quad \theta<\theta_2. \]
    此时运动如\cref{fig:Lagrange陀螺E}.
    \item $E' = V\+_eff_\pare{\theta}_0$, $\theta=\theta_0$, $\dot\theta = 0$,
    \[ \dot\varphi = \frac{J_{z'} - J_z\cos\theta_0}{I_1\sin^2\theta_0} = \const,\quad \dot\psi = \frac{J_z}{I_3} - \dot\varphi\cos\theta_0 = \const. \]
    此时陀螺发生稳定的规则进动.
\end{cenum}
由$V\+_eff_$的形式, 任何运动都是稳定的, 前提是
\[ J_z^2 \ge 4mglI_1\cos\theta_0. \]

\paragraph{快速陀螺} % (fold)
\label{par:快速陀螺}

初始时刻$\dot\theta_1 = \dot\varphi_1 = 0$, $\dot\psi_1$很大, 此时
\begin{align*}
    J_{z'} &= I_3\cos\theta_1\dot\psi_1,\quad J_z = I_3\dot\psi_1 = I_3\omega_z, \\
    E' &= mgl\cos\theta_1.
\end{align*}
在另一边界
\begin{align*}
    \dot\theta_2 &= 0, \quad J_{z'} = I_1\dot\varphi_2\sin^2\theta_2 + I_3\cos\theta_2\pare{\dot\psi_2 + \dot\varphi_2\cos\theta_2}, \\
    J_z &= I_3\pare{\dot\psi_2 + \dot\varphi_2 \cos\theta_2}, \\
    E' &= \half I_1\dot\varphi^2\sin^2\theta_2 + mgl\cos\theta_2.
\end{align*}
由$J_{z'}$, $J_z$, $E$守恒, 消去$\dot\varphi_2$, $\dot\psi_2$,
\begin{align*}
    \sin^2\theta_2 &= \frac{I_3^2\psi_1^2}{2I_1mgl}\pare{\cos\theta_1 - \cos\theta_2} \\
    &= p\pare{\cos\theta_1 - \cos\theta_2}. \\
    p &\gg 1,\quad \epsilon = \cos\theta_1 - \cos\theta_2, \\
    p\epsilon &= \sin^2\theta_2 = 1-\cos^2\theta_2 = 1-\pare{\cos\theta_1 - \epsilon}^2. \\
    & \epsilon^2 + \pare{p - 2\cos\theta_1} \epsilon -\sin^2\theta_1 = 0.
\end{align*}
从而$\epsilon \ll 1$, $p \gg \abs{2\cos\theta_1}$,
\begin{align*}
    & \epsilon^2 + p\epsilon - \sin^2\theta_1 \approx 0. \\
    & \epsilon = \frac{\sin^2\theta}{p} = \frac{2I_1mgl\sin^2\theta_1}{I_3^2\psi_1^2} \ll 1.
\end{align*}
从而$\dot\psi$越大, 章动越小.
\begin{align*}
    & E' = \half I_1\dot\theta^2 + \frac{\pare{J_z - J_{z'}\cos\theta}^2}{2I_1\sin^2\theta} + mgl\cos\theta = mgl\cos\theta_1. \\
    & E' - mgl\cos\theta = \half I_1\dot\theta_1^2 + \frac{\pare{J_{z'}-J_z\cos\theta}^2}{2I_1\sin^2\theta},\quad J_{z'} = J_z\cos\theta_1. \\
    & mgl\pare{\cos\theta_1 - \cos\theta} = \half I_1\dot\theta^2 + \frac{J_z^2}{2I_1\sin^2\theta}\pare{\cos\theta_1 - \cos\theta}^2. \\
    & mgl\pare{\cos\theta_1 - \cos\theta} \sin^2\theta = \frac{I_1}{2} \pare{\+dtd{\cos\theta}}^2 + \frac{J_z^2}{2I_1}\pare{\cos\theta_1 - \cos\theta}^2.
\end{align*}
令$x = \cos\theta_1 - \cos\theta$, 则$x_1 = 0$, $x_2 = \epsilon$, $0 \le x \le \epsilon$.
\begin{align*}
    & mglx\sin^2\theta = \frac{I_1}{2}\dot x^2 + \frac{J_z^2}{2I_1}x^2, \\
    & \approx mglx\sin^2\theta_1 \\
    & \approx mglp\epsilon x = \frac{I_1}{2}\dot x^2 + \frac{J_z^2}{2I_1}x^2. \\
    & \frac{I_1}{2}\dot x^2 + \frac{J_z^2}{2I_1}\pare{x-\frac{\epsilon}{2}}^2 = \frac{J_z^2}{2I_1} \cdot \frac{\epsilon^2}{4}. \\
    & x = \frac{\epsilon}{2} \brac{1-\cos\pare{\frac{J_z}{I_1}t}}. \\
    & \Rightarrow \dot\varphi = \frac{J_{z'} - J_z\cos\theta}{I_1\sin^2\theta} = \frac{J_z x}{I_1\sin^2\theta} \approx \frac{J_z x}{I_1\sin^2\theta_1} \\ &\approx \frac{I_3\dot\psi }{I_1\sin^2\theta_1} \cdot \frac{\epsilon}{2}\brac{1-\cos\pare{\frac{J_z}{I_1}t}}. \\
    & \expc{\dot\varphi} = \frac{I_3\dot\varphi_1}{I_1\sin^2\theta_1}\cdot \frac{\epsilon}{2} = \frac{mgl}{I_3\dot\psi_1}.
\end{align*}

% paragraph 快速陀螺 (end)

\paragraph{Larmor进动} % (fold)
\label{par:larmor进动}

均匀磁场中带电粒子系统
\[ V = q\pare{\varphi - \+vv\cdot\+vA} = -e\pare{-\+vv\cdot\+vA} = e\+vv\cdot\+vA. \]
其中$\displaystyle \+vA = \half\pare{\+vB\times\+vr}$, $\+vB = \curl \+vA$.
\begin{align*}
    V &= e\+vv \cdot \half\pare{\+vB\times\+vr} = \frac{e}{2}\+vB\cdot\pare{\+vr\times\+vv} = -\+vm\cdot \+vB. \\
    \+vm &= -\frac{e}{2}\pare{\+vr\times\+vv} = -\frac{e}{2}\+vr\times\pare{\+v\omega\times\+vr} \\
    &= \frac{e}{2}\pare{\+v\omega\cdot\+vr}\+vr - \frac{e}{2}r^2\+v\omega = -\frac{e}{2m}\+vJ.
\end{align*}
特别地, 若$\+v\omega\cdot\+vr = 0$, $\+vm = \displaystyle -\frac{e}{2}r^2\+v\omega$.
\begin{align*}
    V &= -\+vm\cdot\+vB = \half e\omega r^2 B\cos\theta \rightarrow mgl\cos\theta.
\end{align*}
因此磁矩的进动和陀螺在重力场中的进动形式一致.

% paragraph larmor进动 (end)

% subsubsection lagrange陀螺 (end)

% subsection euler动力学方程及其应用 (end)

% section 刚体的运动 (end)

\end{document}
