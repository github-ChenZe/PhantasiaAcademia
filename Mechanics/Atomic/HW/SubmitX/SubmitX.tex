\documentclass[hidelinks]{ctexart}

\usepackage{van-de-la-illinoise}
\usepackage[paper=b5paper,top=.3in,left=.9in,right=.9in,bottom=.3in]{geometry}
\usepackage{calc}
\pagenumbering{gobble}
\setlength{\parindent}{0pt}
\sisetup{inter-unit-product=\ensuremath{{}\cdot{}}}

\newdimen\indexlen
\def\newprobheader#1{%
\def\probindex{#1}
\setlength\indexlen{\widthof{\textbf{\probindex}}}
\hskip\dimexpr-\indexlen-1em\relax
\textbf{\probindex}\hskip1em\relax
}
\def\newprob#1{%
\newprobheader{#1}%
\def\newprob##1{%
\probsep%
\newprobheader{##1}%
}%
}
\def\probsep{\vskip1em\relax{\color{gray}\dotfill}\vskip1em\relax}

\def\writetofile#1#2{\immediate\write18{echo "#1" > #2}}
\def\econfig#1{\immediate\write18{echo "#1" | python /Users/zechen/Documents/PhantasiaAcademia/electronconfig.py > electronsconfigtmp.tex}$\mathrm{3s^{1}\ 3p^{1}}$%
}

\begin{document}

\newprob{4.1}%
$s=1$, 故
\begin{cenum}
    \item 电子不再是Fermion, 而是Boson, Pauli不相容原理不适用之.
    \item 总的波函数不再是交换反对称的, 而是交换对称的.
    \item 每个电子的自旋又三种可能$\curb{\uparrow,-,\downarrow}$. 总自旋可以取$s=0,1,2$. 使用CG系数展开,
    \begin{align*}
        \ket{2,2} &= \ket{\uparrow\uparrow}, \\
        \ket{2,1} &= \rec{\sqrt{2}}\ket{\uparrow -} + \rec{\sqrt{2}}\ket{-\uparrow}, \\
        \ket{2,0} &= \rec{\sqrt{6}}\ket{\uparrow\downarrow} + \sqrt{\frac{2}{3}}\ket{--} + \rec{\sqrt{6}}\ket{\downarrow\uparrow}, \\
        \ket{2,-1} &= \rec{\sqrt{2}}\ket{-\downarrow} + \rec{\sqrt{2}}\ket{\downarrow-}, \\
        \ket{2,-2} &= \ket{\downarrow\downarrow}.
    \end{align*}
    故$s=2$的态是交换对称的.
    \begin{align*}
        \ket{1,1} &= \rec{\sqrt{2}}\ket{\uparrow-} - \rec{\sqrt{2}}\ket{-\uparrow}, \\
        \ket{1,0} &= \rec{\sqrt{2}}\ket{\uparrow\downarrow} - \rec{\sqrt{2}}\ket{\downarrow\uparrow}, \\
        \ket{1,-1} &= \rec{\sqrt{2}}\ket{-\downarrow} - \rec{\sqrt{2}}\ket{\downarrow -}.
    \end{align*}
    故$s=1$的态是交换反对称的.
    \begin{align*}
        \ket{0,0} = \rec{\sqrt{3}}\ket{\uparrow\downarrow} - \rec{\sqrt{3}}\ket{--} + \rec{\sqrt{3}}\ket{\downarrow\uparrow}.
    \end{align*}
    故$s=0$的态是交换对称的.
    \item 基态$\mathrm{1s^2}$, 空间波函数必定为交换对称的, 故自旋波函数也必定为交换对称的, 此时有$S=2$或$S=0$(五重态$+$单重态).
    \item 如果$S=1$, 则自旋波函数为交换反对称的, 空间波函数必定亦为交换反对称的, 最低能态为$\ce{1s}\,\ce{2s}$, 是三重态.
\end{cenum}

\newprob{4.2}%
$\Delta E_2$: 三重态的能级低于单重态的能级. 三重态的自旋交换对称, 故空间波函数交换反对称, 两电子的波函数重合减少, 故Coulomb排斥能降低.
\par
$\Delta E_1$: $\ce{2s}$态由穿透效应在原子核附近其波函数有一峰值, 故$\ce{2s}$态的屏蔽效应较低, 所受吸引更大, 其势能较低.
\par
$\Delta E_2 < \Delta E_1$.%

\newprob{4.3}%
$\ce{Ne}$: \econfig{1s2 2s2 2p6}.\\
$\ce{Mg}$: \econfig{[Ne] 3s2}.\\
$\ce{P}$: \econfig{[Ne] 3s2 3p3}.\\
$\ce{Co}$: \econfig{[Ar] 3d7 4s2}.\\
$\ce{Ge}$: \econfig{[Ar] 3d10 4s2 4p2}.\\
($\ce{Ar}$: \econfig{[Ne]3s2 3p6}.)

\end{document}
