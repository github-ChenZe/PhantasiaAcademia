\documentclass[hidelinks]{ctexart}

\usepackage{van-de-la-illinoise}
\usepackage[paper=b5paper,top=.3in,left=.9in,right=.9in,bottom=.3in]{geometry}
\usepackage{calc}
\pagenumbering{gobble}
\setlength{\parindent}{0pt}
\sisetup{inter-unit-product=\ensuremath{{}\cdot{}}}

\newdimen\indexlen
\def\newprobheader#1{%
\def\probindex{#1}
\setlength\indexlen{\widthof{\textbf{\probindex}}}
\hskip\dimexpr-\indexlen-1em\relax
\textbf{\probindex}\hskip1em\relax
}
\def\newprob#1{%
\newprobheader{#1}%
\def\newprob##1{%
\probsep%
\newprobheader{##1}%
}%
}
\def\probsep{\vskip1em\relax{\color{gray}\dotfill}\vskip1em\relax}

\def\writetofile#1#2{\immediate\write18{echo "#1" > #2}}
\def\econfig#1{\immediate\write18{echo "#1" | python /Users/zechen/Documents/PhantasiaAcademia/electronconfig.py > electronsconfigtmp.tex}$\mathrm{3s^{1}\ 3p^{1}}$%
}

\begin{document}

\newprob{4.4}%
等效电子: $C_{10}^2 = \boxed{45.}$\\
非等效电子: $10\times 10 = \boxed{100.}$
\newprob{4.5 (1)}%
\econfig{1p2d}不能存在, $n=2$没有d壳层.\\
\econfig{2s3f}不能存在, $n=3$没有f壳层.\\
\econfig{2p3d}存在.
\par
\newprobheader{(2)}%
$^1\mathrm P_2$不能存在, $S=0$, $L=1$, $\Rightarrow J = 1$.\\
$^3\mathrm F_2$存在, $S=1$, $L=3$, $J = \curb{2,3,4}$.
\newprob{4.6}%
矩阵元表示相应的$\pare{M_L,M_S}$对应的组合数,
\[ \underbrace{\begin{array}{c|cccc}
    &   & \+:c2{c}{M_S} & \\
    \hline
    &   & 1   & 1    &  \\
    &   & 2   & 2   & \\
M_L & 1 & 3   & 3   & 1 \\
    &   & 2   & 2   & \\
    &   & 1   & 1   &
\end{array}}_{\displaystyle \text{\econfig{2p2}}} = \underbrace{\begin{array}{c|cccc}
    &   & \+:c2{c}{M_S} & \\
    \hline
    &   & 1   & 1   &  \\
    &   & 1   & 1   & \\
M_L &   & 1   & 1   & \\
    &   & 1   & 1   & \\
    &   & 1   & 1   &
\end{array}} _{\displaystyle ^2\mathrm D} + \underbrace{\begin{array}{c|cccc}
    &   & \+:c2{c}{M_S} & \\
    \hline
    &   &     &     &  \\
    &   & 1   & 1   & \\
M_L &   & 1   & 1   & \\
    &   & 1   & 1   & \\
    &   &     &     &
\end{array}} _{\displaystyle ^2\mathrm P} + \underbrace{\begin{array}{c|cccc}
    &   & \+:c2{c}{M_S} & \\
    \hline
    &   &     &     &  \\
    &   &     &     & \\
M_L & 1 & 1   & 1   & 1. \\
    &   &     &     & \\
    &   &     &     &
\end{array}}_{\displaystyle ^4\mathrm S} \]
状态凡$\boxed{20}$种.
\[ ^2\mathrm D_{3/2,5/2}: 4 + 6 = 10\text{种},\quad ^2\mathrm P_{1/2,3/2}: 2 + 4 = 6\text{种},\quad ^4\mathrm S_{3/2}: 4\text{种}. \]
\newprob{4.7}
$\displaystyle J_2 = \frac{3}{2}$, $\displaystyle J_3 = \frac{5}{2}$, $\displaystyle J_1 = \half$, $\displaystyle \Rightarrow L+S = J_3 = \frac{5}{2}$, $\displaystyle \abs{L-S} = J_1 = \half$, $\displaystyle \Rightarrow L=1$, $\displaystyle S=\frac{3}{2}$, $\Rightarrow \boxed{^4 \mathrm{P_{1/2,3/2,5/2}}.}$
\newprob{4.8}%
Al: \econfig{1s2 2s2 2p6 3s2 3p}. Hund III要求$\min J$, $\Rightarrow \boxed{^2\mathrm P_{1/2}.}$\\
Mg: \econfig{1s2 2s2 2p6 3s2}. $\Rightarrow \boxed{^1\mathrm S_{0}.}$\\
Ti: \econfig{1s2 2s2 2p6 3s2 3p6 4s2 3d2}. Hund III要求$\min J$, $\Rightarrow \boxed{^3\mathrm F_{2}.}$
\newprob{4.9 (1)}%
\econfig{4d8} $\cong$ \econfig{4d2} $\Rightarrow 2\divs \pare{L+S}$
\begin{align*}
    \text{\econfig{4d2}} &= \curb{L=4,S=0}\cup \curb{L=3,S=1}\cup \curb{L=2,S=0}\cup \curb{L=1,S=1}\cup \curb{L=0,S=0},
    %&= {^1\mathrm G_{}}\cup{^3\mathrm F_{}}\cup {^1\mathrm D_{}}\cup {^3\mathrm P_{}}\cup {^1\mathrm S_{}}.
\end{align*}
再和 \econfig{5s1} 耦合,
\begin{align*}
    \text{\econfig{5s1 4d2}} &= \curb{L=4,S=\half}\cup \curb{L=3,S=\curb{\half,\frac{3}{2}}}\cup \curb{L=2,S=\half} \\
    &\phantom{=\ } \cup \curb{L=1,S=\curb{\half,\frac{3}{2}}}\cup \curb{L=0,S=\half}\\
    &= \boxed{\begin{cases}
        {^2\mathrm G_{7/2,9/2}} \cup {^2\mathrm F_{5/2,7/2}} \cup {^4\mathrm F_{3/2,5/2,7/2,9/2}} \cup {^2\mathrm D_{3/2,5/2}}\\ \cup {^2\mathrm P_{1/2,3/2}} \cup {^4\mathrm P_{1/2,3/2,5/2}} \cup {^2\mathrm S_{1/2}}.
    \end{cases}}
\end{align*}
\par
\newprobheader{(2)}%
取$\max S$, 后$\max L$, 后$\max J$, 得到$\boxed{^4\mathrm F_{9/2}}$.
\newprob{4.10}%
\vspace{-1.75\baselineskip}
\begin{align*}
    \pare{\half,\half} &= \pare{\half,\half}_0 \cup \pare{\half,\half}_1, \\
    \pare{\half, \frac{3}{2}} &= \pare{\half,\frac{3}{2}}_1 \cup \pare{\half,\frac{3}{2}}_2, \\
    \pare{\frac{3}{2}, \half} &= \pare{\frac{3}{2}, \half}_1 \cup \pare{\frac{3}{2}, \half}_2, \\
    \pare{\frac{3}{2},\frac{3}{2}} &= \pare{\frac{3}{2},\frac{3}{2}}_0 \cup \pare{\frac{3}{2},\frac{3}{2}}_1 \cup \pare{\frac{3}{2},\frac{3}{2}}_2 \cup \pare{\frac{3}{2},\frac{3}{2}}_3.
\end{align*}
\newprob{4.11}%
\econfig{3p5} $\cong$ \econfig{3p1} $= {^2\mathrm P_{3/2,1/2}}$. Hund III要求取$\max J$, 故基态为${^2\mathrm P_{3/2}}$. \\
$m_j = \curb{-3/2,-1/2,1/2,3/2}$凡$\boxed{4}$条谱线.
\[ g\+_Cl_ = \frac{3}{2} + \frac{S\pare{S+1} - L\pare{L+1}}{2J\pare{J+1}} = \frac{4}{3}. \]
而氢原子的基态${^2\mathrm S_{1/2}}$,
\[ g\+_H_ = \frac{3}{2} + \frac{s\pare{s+1} - l\pare{l+1}}{2j\pare{j+1}} = 2, \]
从而
\[ d\+_Cl_ = \frac{g\+_Cl_}{g\+_H_}d\+_H_ = \frac{2}{3}d\+_H_ = \boxed{\SI{0.4}{\centi\meter}.} \]
\newprob{4.12}%
弱磁场适用. 对于${^1\mathrm D_2}$,
\[ g_1 = \frac{3}{2} + \frac{S\pare{S+1} - L\pare{L+1}}{2J\pare{J+1}} = 1, \]
对于${^1\mathrm P_1}$,
\[ g_2 = \frac{3}{2} + \frac{S\pare{S+1} - L\pare{L+1}}{2J\pare{J+1}} = 1, \]
从而$g_1 = g_2 = 1$,
\[ \Delta \nu = \frac{\mu\+_B_ B}{h} = \frac{e}{m}\frac{B}{4\pi} \Rightarrow \frac{e}{m} = 4\pi\frac{\Delta \nu}{B} = \boxed{\SI{1.76e11}{\coulomb\per\kilo\gram}.} \]
\newprob{4.13 (1)}%
$\displaystyle g = \frac{3}{2} + \frac{S\pare{S+1} - L\pare{L+1}}{2J\pare{J+1}} = \boxed{\frac{3}{2}.}$
\par
\newprobheader{(2)}%
$\displaystyle g = \frac{3}{2} + \frac{S\pare{S+1} - L\pare{L+1}}{2J\pare{J+1}} = \frac{3}{2} + \frac{S\pare{S+1}}{2S\pare{S+1}} = 2$.
\par
\newprobheader{(2)}%
$\displaystyle g = \frac{3}{2} + \frac{S\pare{S+1} - L\pare{L+1}}{2J\pare{J+1}} = \frac{3}{2} + \frac{- L\pare{L+1}}{2\pare{L}\pare{L+1}} = 1$.
\newprob{4.14 (1)}%
\underline{允许}.
\par
\newprobheader{(2)}%
\underline{禁戒}. $\Delta S \neq 0$.
\par
\newprobheader{(3)}%
\underline{允许}.% 其中一个电子$\Delta l \neq \pm 1$.
\par
\newprobheader{(4)}%
\underline{禁戒}. $J=0 \not\rightarrow J=0$.
\par
\newprobheader{(5)}%
\underline{禁戒}. $\Delta J \neq 1$.
\newprob{4.15}%
\econfig{3p1 4p1} $\cong$ ${^3\mathrm D_{1,2,3}} \cup {^1\mathrm D_{2}} \cup {^3\mathrm P_{0,1,2}} \cup {^1\mathrm P_{1}} \cup {^3\mathrm S_{1}} \cup {^1\mathrm S_{0}}$.\\
\econfig{3s1 3p1} $\cong$ ${^3\mathrm P_{0,1,2}} \cup {^1\mathrm P_{1}}$.\\
${^3\mathrm P_{0}} \leftarrow {^3\mathrm D_{1}}, {^3\mathrm P_{1}}, {^3\mathrm S_{1}}$.\\
${^3\mathrm P_{1}} \leftarrow {^3\mathrm D_{1,2}}, {^3\mathrm P_{0,1,2}}, {^3\mathrm S_{1}}$.\\
${^3\mathrm P_{2}} \leftarrow {^3\mathrm D_{1,2,3}}, {^3\mathrm P_{1,2}}, {^3\mathrm S_{1}}$.\\
${^1\mathrm P_{1}} \leftarrow {^1\mathrm D_{2}}, {^1\mathrm P_{1}}, {^1\mathrm S_{0}}$.\\
凡$\boxed{18}$种.

\end{document}
