\documentclass[hidelinks]{ctexart}

\usepackage{van-de-la-illinoise}

\begin{document}

\section{静电学} % (fold)
\label{sec:静电学}

\subsection{Laplace方程} % (fold)
\label{sub:laplace方程}

\subsubsection{调和函数} % (fold)
\label{ssub:调和函数}

\begin{theorem}[平均值定理]
    设$\varphi$是一调和函数, 则
    \[ \varphi\pare{\+vr} = \rec{4\pi R^2}\oiint_S \varphi\pare{\+vr'}\rd{S}, \]
    其中$S$是半径为$R$的球面.
\end{theorem}
\begin{proof}
    记$\displaystyle \psi = \rec{r}$, 则由Green第二恒等式,
    \[ \iiint_{V\cap V'} \cancelto{0}{\pare{\psi\laplacian \varphi - \varphi\laplacian \psi}}\,\rd{V} = \oiint_{S\cup S'} \pare{\psi\+D{\+vn}D{\varphi} - \varphi\+D{\+vn}D{\psi}}\,\rd{S}. \]
    右侧积分第一项
    \begin{align*}
        \oiint_{S\cup S'} \psi\+D{\+vn}D{\varphi}\,\rd{S} &= \rec{R} \oiint_{S} \grad\varphi\cdot\rd{\+vS} - \rec{R'} \oiint_{S'}\grad\varphi\cdot\rd{\+vS} \\ &= \rec{R} \iiint_{V} \laplacian \varphi\,\rd{V} - \rec{R'} \iiint_{V'} \laplacian \varphi\,\rd{V} = 0. 
    \end{align*}
    从而
    \[ \oiint_{S\cup S'} \varphi\+D{\+vn}D{\psi}\,\rd{S} = \rec{R'^2} \oiint_{S'}\varphi\,\rd{S} - \rec{R^2}\oiint_{S} \varphi\,\rd{S} = 0. \]
    即
    \[ \rec{R^2}\oiint_{S} \varphi\,\rd{S} \]
    和$R$无关.
\end{proof}
\begin{theorem}[唯一性定理]
    设$\varphi$和$\psi$是调和函数, 且$\restr{\varphi}{S} = \restr{\psi}{S}$, 则
    \[ \restr{\varphi}{V} \equiv \restr{\psi}{V}. \]
\end{theorem}
\begin{lemma}
    设$\varphi$是一调和函数, 且$\restr{\varphi}{S} = 0$, 则
    \[ \restr{\varphi}{V} \equiv 0. \]
\end{lemma}
\begin{proof}
    由Green第一恒等式,
    \[ \iiint_V \pare{\cancelto{0}{\varphi\laplacian \varphi} + \pare{\grad\varphi}^2}\,\rd{V} = \oiint_{S} {\varphi\+D{\+vn}D{\varphi}}\,\rd{S} = 0. \]
    从而$\varphi$为常量, 必定为$0$.
\end{proof}

% subsubsection 调和函数 (end)



% subsection laplace方程 (end)

% section 静电学 (end)

\end{document}