\documentclass[hidelinks]{ctexart}

\usepackage[sensei=潘海俊,gakka=電気力学,gakkabbr=ED,section=Seidenba]{styles/kurisu}
\usepackage{van-de-la-illinoise}
\usepackage{van-le-trompe-loeil}
\usepackage{stackengine}
\stackMath
\usepackage{scalerel}
\usepackage[outline]{contour}

\newlength\thisletterwidth
\newlength\gletterwidth
\newcommand{\leftrightharpoonup}[1]{%
{\ooalign{$\scriptstyle\leftharpoonup$\cr%\kern\dimexpr\thisletterwidth-\gletterwidth\relax
$\scriptstyle\rightharpoonup$\cr}}\relax%
}
\def\tensor#1{\settowidth\thisletterwidth{$\mathbf{#1}$}\settowidth\gletterwidth{$\mathbf{g}$}\stackon[-0.1ex]{\mathbf{#1}}{\boldsymbol{\leftrightharpoonup{#1}}}  }

\begin{document}

\section{静电场} % (fold)
\label{sec:静电场}

Sommerfeld曾将电磁场问题归为求和问题和边界问题, 前者谓空间中每一点处之电荷分布已知时求场, 后者谓已知边界条件时求场.
\par
唯一性定理保证了边界问题的场可求.

\subsection{静电场的基本规律} % (fold)
\label{sub:静电场的基本规律}

\subsubsection{基本方程} % (fold)
\label{ssub:基本方程}

$\begin{cases}
    \div \+vE\pare{\+vr} = \rho\pare{\+vr}/\epsilon_0,\\
    \curl \+vE\pare{\+vr} = 0
\end{cases}$.
\begin{cenum}
    \item 边值关系: $\+un\cdot \pare{\+vE_2 - \+vE_1} = \sigma/\epsilon_0$, $\+un\times \pare{\+vE_2 - \+vE_1} = 0$.
    \item Helmholtz定理: 电场在某点处总能写成无旋场和无源场之和. 即
    \begin{align*}
        \+vE\pare{\+vr} &= -\grad \rec{4\pi} \iiint \rd{V'}\, \frac{\grad'\+v\cdot \+vE\pare{\+vr'}}{\+gr} + \curl \rec{4\pi} \iiint \rd{V'}\, \frac{\grad'\times \+vE\pare{\+vr'}}{\+gr}. \\
        &= \grad \rec{4\pi\epsilon_0} \iiint \rd{V'}\, \frac{\rho\pare{\+vr'}}{\+gr}
    \end{align*}
    \item 无旋场可以定义静电势
    \[ \+vE = -\grad \varphi,\quad \varphi\pare{\+vr} = -\int_P^{\+vr} \rd{\+vl}\cdot \+vE,\quad \pare{\rd{\+vl} = \rd{\+vr}}. \]
    \begin{cenum}
        \item 对于局域的电流分布, 可以将参考点设定为$P=\infty$.
        \item 可以定义两点之间的电势差(电势降)为
        \[ V_{12} = \varphi_1 - \varphi_2 = \int_1^2 \rd{\+vl}\cdot \+vE. \]
        \item 可将$\varphi$视为$\+vE$的不定积分.
    \end{cenum}
    \item 势方程: $\laplacian \varphi = -\rho/\epsilon_0$. 可得面电荷边界处
    \[ \begin{cases}
        \displaystyle \+DnD{\varphi_1} - \+DnD{\varphi_2} = \frac{\sigma}{\epsilon_0},\\
        \displaystyle \varphi_1 = \varphi_2 \Rightarrow \+vE_{\parallel}^{\pare{1}} = \+vE_{\parallel}^{\pare{2}}.
    \end{cases} \]
\end{cenum}
\begin{proof}[可取$\infty$为参考点的证明]
    \begin{align*}
        \varphi\pare{\+vr} &= -\int_\infty^{\+vr} \rd{\+vl}\cdot \+vE = -\int_\infty^{\+vr} \rd{\+vl}\cdot \rec{4\pi\epsilon_0} \iiint \rd{V'}\, \frac{\rho\pare{\+vr'}\+v{\+gr}}{\+gr^3} \\
        &= -\rec{4\pi\epsilon_0} \iiint \rd{V'}\, \rho\pare{\+vr} \int_\infty^{\+vr} \frac{\+v{\+gr}\cdot\rd{\pare{\+vr - \+vr'}}}{\+gr^2} \\
        &= + \rec{4\pi\epsilon_0} \iiint \rd{V'}\, \frac{\rho\pare{\+vr'}}{\+gr}. \qedhere
    \end{align*}
\end{proof}
\begin{figure}[ht]
    \centering
    \incfig{6cm}{GreenRec}
\end{figure}
\newpoint{Green互易关系} $\displaystyle \iiint \rd{V'}\,\rho'\pare{\+vr'} \varphi\pare{\+vr'}$\\
$\displaystyle = \iiint \rd{V'}\, \rho'\pare{\+vr'} \cdot \rec{4\pi\epsilon_0} \iiint \rd{V}\, \frac{\rho\pare{\+vr'}}{\abs{\+vr - \+vr'}} = \iiint \rd{V}\,\rho\pare{\+vr} \varphi'\pare{\+vr}$.
\begin{sample}
    \begin{ex}
        一个$\pare{Q,a}$均匀带电球面$S'$, 在外部一电荷密度$\rho\pare{\+vr}$的体积$V$内激发的电势为
        \[ \iiint \rd{V'}\, \rho'\pare{\+vr'} \varphi \pare{\+vr'} = \frac{Q}{4\pi a^2} \oiint \rd{\sigma'}\, \varphi\pare{\+vr'} = Q\expc{\varphi}. \]
        由Green互易关系, 这等于
        \[ \iiint \rd{V}\, \rho\pare{\+vr} \varphi' \pare{\+vr} = \iiint \rd{V}\, \rho\pare{\+vr} \frac{Q}{4\pi\epsilon_0 r} = Q\varphi_0. \]
        其中$\varphi_0$是$\rho$在球心处产生的电势.
    \end{ex}
\end{sample}

% subsubsection 基本方程 (end)

\subsubsection{物质中的基本方程} % (fold)
\label{ssub:物质中的基本方程}

$\begin{cases}
    \div \+vD\pare{\+vr} = \rho_0 \pare{\+vr},\quad \+vD = \epsilon_0 \+vE + \+vP,\\
    \curl \+vE\pare{\+vr} = 0,
\end{cases}$
\begin{cenum}
    \item 边值关系: $\+un\cdot\pare{\+vD_2 - \+vD_1} = \sigma_0$, $\+un\times\pare{\+vE_2 - \+vE_1} = 0$.
    \item 势方程(简单介质中$\epsilon = \epsilon\pare{\+vr}, \+vD = \epsilon \+vE = -\epsilon \grad \varphi$)为
    \[ \div\pare{\epsilon \grad \varphi} = -\rho_0 \Rightarrow \begin{cases}
        \displaystyle \text{均匀介质内} & \laplacian \pare{\varphi} = \rho_0 / \epsilon,\\
        \displaystyle \text{边值关系}   & \displaystyle \varphi_1 = \varphi_2,\quad \epsilon_1 \+DnD{\varphi_1} - \epsilon_2 \+DnD{\varphi_2} = \sigma_0.
    \end{cases} \]
\end{cenum}

% subsubsection 物质中的基本方程 (end)

\subsubsection{静电能} % (fold)
\label{ssub:静电能}

\begin{figure}[ht]
    \centering
    \incfig{4cm}{Assemble}
\end{figure}
将电荷自无穷远处移入, $\begin{cases}
    \div \+vD = \rho_0, \\
    \delta \rho_0 = \div \pare{\delta \+vD}.
\end{cases}$
\begin{align*}
    \delta A &= \iiint \rd{V} \,\varphi \rho_0 \\
    &= \iiint \rd{V}\, \varphi \div \delta \+vD \\
    &= \iiint \rd{V}\, \div \pare{\varphi \delta \+vD} - \iiint \rd{V}\, \delta \+vD\cdot \grad \varphi \\
    &= \iiint \rd{V}\, \+vE\cdot \delta \+vD.
\end{align*}
对于简单介质, $\epsilon = \epsilon\pare{\+vr}$,
\begin{align*}
    \+vE\cdot \delta \+vD &= \+vE\pare{\+vr}\cdot \delta\brac{\epsilon\pare{\+vr}\+vE\pare{\+vr}} \\
    &= \+vE\pare{\+vr} \cdot \epsilon\pare{\+vr} \delta \+vE\pare{\+vr} \\
    &= \+vD\cdot \delta \+vE \\
    &= \delta \pare{\half \+vD\cdot \+vE}, \\
    \delta A &= \iiint \rd{V}\+vE\cdot \delta \+vD = \delta \iiint \rd{V} \, \half \+vD\cdot \+vE.
\end{align*}
\begin{cenum}
    \item 能量局域于场: $\displaystyle W = \iiint \rd{V}\,\half \+vD\cdot \+vE$,
    \[ w = \half \+vD\cdot \+vE = \underbrace{\half \epsilon_0 E^2}_{\text{宏观}} + \underbrace{\half \+vP\cdot \+vE}_{\text{极化}}. \]
    增加的极化能部分源于将分子极化(正负电荷拉开)时所做功.
    \item 能量局域于电荷:
    \begin{align*}
        W &= \half \iiint \rd{V}\, \+vD\cdot \+vE = -\half \iiint \rd{V}\, \+vD\cdot \grad \varphi \\
        &= -\half \iiint \rd{V}\,\div \pare{\varphi \+vD} + \half \iiint \rd{V}\,\varphi \div \+vD \\
        &= -\half \oiint \rd{\+v\sigma}\cdot \varphi \+vD + \half \iiint \rd{V}\,\rho_0 \varphi.
    \end{align*}
    从而$\displaystyle W = \half \iiint \rd{V}\rho_0 \varphi$.
    \item 导体系统: 每一个导体是等势的, 从而
    \[ W = \half \sum_i Q_i\varphi_i. \]
    导体系统电荷和电势之间有关系
    \[ Q_i = \sum_j C_{ij}\varphi_j,\quad \varphi_i = \sum_j p_{ij}Q_j. \]
    从而$\displaystyle W = \half \sum_{ij} C_{ij}\varphi_i\varphi_j$.
\end{cenum}
\begin{sample}
    \begin{ex}
        考虑带电$Q$, 内外厚度$a,b$的球壳, 自无穷远处移动电荷$e$至球心, 求做功.
    \end{ex}
    \begin{solution}
        $e$在无穷远处时, $\displaystyle W_1 = \half Q\varphi_b + \half e\varphi_e = \half Q\frac{Q}{4\pi\epsilon_0 b} + \half e\frac{e}{4\pi\epsilon_0 r_0}$.\\
        终态$\displaystyle W_2 = \half Q\frac{Q+e}{4\pi\epsilon_0 b} + \half e\brac{\frac{e}{4\pi\epsilon_0 r_0} + \frac{Q+e}{4\pi\epsilon_0 b} - \frac{e}{4\pi\epsilon_0 a}}$.\\
        $\displaystyle W = W_1 - W_1 = \frac{e}{8\pi\epsilon_0} \brac{\frac{2Q+e}{b} - \frac{e}{a}}.$
    \end{solution}
\end{sample}
\begin{remark}
    移动到导体内非球心位置时能量如何?
\end{remark}

% subsubsection 静电能 (end)

\subsubsection{唯一性定理} % (fold)
\label{ssub:唯一性定理}

设区域$V$内部电荷分布$\rho_0\pare{\+vr}$已知, $\epsilon\pare{\+vr}$已知, 则势方程为
\[ \div \pare{\epsilon \grad \varphi} = -\rho_0. \]
设有二解$\varphi_1$, $\varphi_2$, 令$\Phi = \varphi_1 - \varphi_2$, 则
\[ \div\pare{\epsilon \grad \Phi} = 0. \]
且边界条件为齐次边界条件. 注意到
\begin{align*}
    0 &= \iiint_V \rd{V}\, \brac{\epsilon\pare{\grad\Phi}^2 + \cancelto{0}{\Phi \div\pare{\epsilon \grad \Phi}}} = \iiint \rd{V}\,\div\pare{\Phi\epsilon \grad \Phi} \\
    &= \oiint_{\partial V} \rd{\+v\sigma}\cdot \Phi \epsilon \grad \Phi = \oiint_{\partial V}\rd{\sigma}\, \Phi\epsilon \+D{n}D{\Phi}.
\end{align*}
因此当$\displaystyle \Phi\vert_{\partial V} = 0$或$\displaystyle \left.\+DnD{\Phi}\right\vert_{\partial V} = 0$时$\Phi \equiv 0$.
\begin{cenum}
    \item Dirichlet边界条件: $\varphi\pare{\+vr_s} = f\pare{\+vr_s}$, $\+vr_s \in \partial V$, 可确定唯一电场和电势.
    \begin{cenum}
        \item 导体条件: 对应于$\+vf = \const$.
        \item $\+vE_\tau \vert_{\partial V}$得到$\Delta \varphi\vert_{\partial V}$. 此时须有
        \[ \lim_{\Sigma \rightarrow 0} \rec{\Sigma} \oint_{\partial \Sigma}\rd{\+vl}\cdot \+vE_\tau = 0. \]
    \end{cenum}
    \item Neumann边界条件: $\displaystyle \left.\+DnD{\varphi}\right\vert_{\partial V} = g\pare{\+vr_s}$, 可唯一确定电场和电势.
    \begin{cenum}
        \item 可给定表面的$\+vE_n$.
        \item 由$\displaystyle \oiint_{\partial V} \rd{\+v\sigma}\cdot \+vD = D_0 = -\oiint_{\partial V} \rd{\+v\sigma}\cdot \epsilon \grad \varphi = 0$.
    \end{cenum}
    \item 若$\partial V$均为导体表面, 只需知道每个内导体边界的总电量即可保证$V$内$\+vE$的唯一性. 下面用$\Phi$表示二解之差,
    \begin{align*}
        -\oiint_{S_k} \rd{\+v\sigma}\cdot \+vD = Q_k = \oiint_{S_k}\rd{\sigma}\epsilon \+DnD\varphi \Rightarrow  \oiint_{S_k} \rd{\sigma} \epsilon \+DnD\Phi = 0.
    \end{align*}
    设有解$\begin{cases}
        \varphi_k' = C_k', \varphi_k'' = C_k'', \\
        \varphi_0' = C_0',\varphi_0'' = C_0''
    \end{cases} \Rightarrow \begin{cases}
        \Phi_k = C'_k - C''_k = C_k,\\
        \Phi_0 = C_0' - C''_0 = C_0.
    \end{cases}$
    \begin{align*}
        \oiint_{\partial V}\rd{\sigma}\, \Phi \epsilon \+DnD\Phi &= C_0 \oiint_{S_0} \rd{\sigma}\,\epsilon \+DnD\Phi + \sum_k C_k \cancelto{0}{\oiint_{S_k} \rd{\sigma}\,\epsilon \+DnD\Phi}, \\
        -\oiint_{S_0}\rd{\sigma}\epsilon \+DnD\varphi &= Q_0 + \sum_k Q_k \Rightarrow \oiint_{S_0} \rd{\sigma}\,\epsilon \+DnD\Phi = 0.
    \end{align*}
\end{cenum}

% subsubsection 唯一性定理 (end)

\subsubsection{其它定解条件} % (fold)
\label{ssub:其它定解条件}

\newpoint{}由对称性选择合适的坐标系.
\newpoint{}由单值性, 连续性或有限性也可以得到一些边界条件.
\newpoint{}渐进条件(叠加原理), 例如在外有均匀电场$\+vE_0$的情形, 存在无限线电荷分布以及有限电荷分布时,
\[ \varphi \sim -\+vE_0 \cdot \+vr - \frac{\lambda}{2\pi\epsilon_0}\ln s + o\pare{1}. \]

% subsubsection 其它定解条件 (end)

% subsection 静电场的基本规律 (end)

\subsection{分离变量法} % (fold)
\label{sub:分离变量法}

\subsubsection{概述} % (fold)
\label{ssub:概述}

问题为$\laplacian \varphi = 0$, 附带边界条件. 若问题为Poisson方程, 则可以设
\[ \varphi = \rec{4\pi\epsilon_0} \iiint \rd{V}\,\rho\pare{\+vr'}\cdot \rec{\+gr} + \varphi_0 \pare{\+vr}, \]
即可转化为$\laplacian \varphi_0 = 0$.

% subsubsection 概述 (end)

\subsubsection{对称性} % (fold)
\label{ssub:对称性}

由问题(边界条件)的对称性可选择合适的坐标系, 使得$\varphi = \varphi\pare{u_1,u_2,u_3}$.

% subsubsection 对称性 (end)

\subsubsection{分离变量要点} % (fold)
\label{ssub:分离变量要点}

令$\varphi\pare{u_1,u_2,u_3} = X\pare{u_1}Y\pare{u_2}Z\pare{u_3}$, 其满足$\laplacian\varphi = 0$, 则该类型的解可叠加得到满足边界条件的调和函数.

% subsubsection 分离变量要点 (end)

\subsubsection{正交完备函数系} % (fold)
\label{ssub:正交完备函数系}

正交归一化谓
\[ \int_a^b \rd{x}\, \psi^*_n\pare{x} \psi_m\pare{x} = \delta_{nm}, \]
完备性谓
\[ f\pare{x} = \sum_{n=1}^\infty C_n\psi_n\pare{x},\quad C_n = \int_a^b\rd{x'} \psi_n^*\pare{x'} f\pare{x'}. \]
从而
\[ f\pare{x} = \int_a^b \rd{x'} \brac{\sum_{n=1}^\infty \psi_n^*\pare{x'}\psi_n\pare{x}}f\pare{x'}. \]
可以发现
\[ \sum_{n=1}^\infty \psi_n^*\pare{x'}\psi_n\pare{x} = \delta\pare{x-x'}. \]
即$\displaystyle \begin{cases}
    \braket{\psi_n}{\psi_m} = \delta_{mn}, \\
    \sum_n \ket{\psi_n}\bra{\psi_n} = 1.
\end{cases}$
\begin{ex}
    可以验证$\displaystyle \curb{\sqrt{\frac{2}{a}}\sin \frac{n\pi x}{a}}$对于$f\pare{0} = f\pare{a} = 0$的空间满足条件.
\end{ex}
\begin{ex}
    设平面上有区域$\curb{\phi\in\pare{0,\beta},s\in\pare{s_1,s_2}}$. $s=s_1$, $s=s_2$, $\phi = 0$处$\varphi = 0$, $\varphi\pare{s,\phi = \beta} = f\pare{s}$, 求$\varphi$.
\end{ex}

% subsubsection 正交完备函数系 (end)

\subsubsection{直角坐标系} % (fold)
\label{ssub:直角坐标系}

Laplace方程有形式
\[ \grad \varphi = \+D{x^2}D{^2X} + \+D{y^2}D{^2\varphi} + \+D{z^2}D{^2\varphi} = 0. \]
令$\displaystyle \varphi\pare{x,y,z} = X\pare{x}Y\pare{y}Z\pare{z}$, 有
\[ \frac{\laplacian\varphi}{\varphi} = \underbrace{\rec{X}\+d{x^2}d{^2X}}_{\alpha^2} + \underbrace{\rec{Y}\+d{y^2}d{^2Y}}_{\beta^2} + \underbrace{\rec{Z}\+d{z^2}d{^2Z}}_{\gamma^2}. \]
其中$\alpha,\beta,\gamma$谓分离参数, 
\[ \alpha^2 + \beta^2 + \gamma^2 = 0. \]
对不同的符号有
\begin{align*}
    \alpha^2 = 0 &\Rightarrow X = \left\{\begin{aligned}
    1 \\ x
    \end{aligned} \right\},\\
    \alpha^2 > 0 &\Rightarrow X = \left\{\begin{aligned}
    \cosh \alpha x \\ \sinh \alpha h
    \end{aligned} \right\},\\
    \alpha^2 = -a^2 < 0 &\Rightarrow X = \left\{\begin{aligned}
    \cos ax \\ \sin ax
    \end{aligned} \right\}.
\end{align*}
组合后
\[ \varphi\pare{x,y,z} = \sum_{\alpha,\beta,\gamma} X_\alpha\pare{x} Y_\beta\pare{y} Z_\gamma\pare{z} \delta\pare{\alpha^2 + \beta^2 + \gamma^2}. \]
\begin{figure}[ht]
    \centering
    \incfig{5cm}{CubicLaplacian}
    \caption{}
    \label{fig:方块laplacian}
\end{figure}
\begin{sample}
    \begin{ex}
        \cref{fig:方块laplacian}中, 假设只有在灰色面上有非零$\varphi_0\pare{x,y}$, 则需选取$X$和$Y$为SL问题,
        \[ \frac{\laplacian\varphi}{\varphi} = \underbrace{\rec{X}\+d{x^2}d{^2X}}_{-\alpha^2} + \underbrace{\rec{Y}\+d{y^2}d{^2Y}}_{-\beta^2} + \underbrace{\rec{Z}\+d{z^2}d{^2Z}}_{\gamma^2 = \alpha^2 + \beta^2} = 0. \]
        因此
        \begin{align*}
            X &= \left\{\begin{aligned}
                \cancel{\cos \alpha x} \\ \sin \alpha x
                \end{aligned} \right\},
            Y &= \left\{\begin{aligned}
                \cancel{\cos \beta y} \\ \sin \beta y
                \end{aligned} \right\},
            Z &= \left\{\begin{aligned}
                \cancel{\cosh \gamma y} \\ \sinh \gamma y
                \end{aligned} \right\}.
        \end{align*}
        边界条件还要求$\alpha a = m\pi$, $\beta b = n\pi$, 从而一般解为
        \[ \varphi = \sum_{m,n=1}^\infty A_{mn}\sin\frac{m\pi x}{a} \sin \frac{n\pi y}{b} \sinh \pare{\gamma_{mn}z}. \]
        其中$\displaystyle \gamma_{mn} = \pi \sqrt{\pare{\frac{m}{a}}^2 + \pare{\frac{n}{b}}^2}$. 而灰色面的边界条件要求
        \[ \sum_{m,n=1}^\infty A_{mn}\sin\frac{m\pi x}{a}\sin\frac{n\pi y}{b}. \]
        故
        \[ A_{mn} = \frac{4}{ab} \int_0^a \rd{x} \int_0^b\rd{y} \,\varphi_0\pare{x,y} \sin \frac{m\pi x}{a} \sin \frac{n\pi y}{b}. \]
        若$\varphi_0 = V_0 = \const$, 则
        \[ A_{mn}  = \frac{4V_0}{ab}\rec{m\pi}\rec{n\pi}\brac{1-\pare{-1}^m}\brac{1-\pare{-1}^n} = \begin{cases}
            \displaystyle \frac{16V_0}{mn\pi^2}, & m=n,\\
            0, & m\neq n.
        \end{cases} \]
        最终解为
        \begin{align*}
            \varphi\pare{x,y,z} &= \frac{16V_0}{\pi^2} \sum_{m,n=1,3,5,\cdots} \rec{mn}\sin \frac{m\pi x}{a}\sin\frac{n\pi y}{b} \frac{\sinh \gamma_{mn}z}{\sinh \gamma_{mn}c} \\
            &= \varphi_{1,1} + \\
            &\phantom{=\ } \varphi_{3,1} + \varphi_{1,3} \\
            &\phantom{=\ } \varphi_{5,1} + \varphi_{3,3} + \varphi_{1,5} \\
            &\phantom{=\ } \varphi_{7,1} + \varphi_{5,3} + \varphi_{3,5} + \varphi_{1,7}+\cdots.
        \end{align*}
    \end{ex}
\end{sample}
\begin{figure}[ht]
    \centering
    \incfig{4cm}{SquareTubeLaplacian}
    \caption{}
    \label{fig:方管laplacian}
\end{figure}
\begin{sample}
    \begin{ex}
        \cref{fig:方管laplacian}中, $\displaystyle \frac{\laplacian\varphi}{\varphi} = \underbrace{\rec{X}\+d{x^2}d{^2X}}_{-\alpha^2} + \underbrace{\rec{Y}\+d{y^2}d{^2Y}}_{\alpha^2} = 0.$从而
        \begin{align*}
            \alpha = 0 & \Rightarrow X = \left\{\begin{aligned}
                \cancel{1} \\ x
                \end{aligned} \right\},\quad Y = \curb{1}, \\
            \alpha \neq 0 & \Rightarrow \Rightarrow X = \left\{\begin{aligned}
                \cancel{\cos \alpha x} \\ \sin \alpha x
                \end{aligned} \right\},\quad Y = \curb{\cosh \alpha y}.
        \end{align*}
        可得
        \[ \varphi\pare{x,y} = A_0 x \sum_{\alpha \neq 0} A_\alpha \sin \alpha x \frac{\cosh\alpha y}{\cosh \alpha b}. \]
        边界条件要求
        \[ V_0 = A_0 a \sum_{\alpha \neq 0} A_\alpha \sin \alpha a \frac{\cosh\alpha y}{\cosh \alpha b}. \]
        即可令$\displaystyle A_0 = V_0/a$, 则$\alpha a = n\pi$自动消去求和. 故
        \[ \varphi\pare{x,y} = \frac{V_0}{a}x + \sum_{n=1}^\infty A_n \sin \frac{n\pi x}{a} \frac{\displaystyle \cosh \frac{n\pi y}{a}}{\displaystyle \cosh \frac{n\pi b}{a}}. \]
        $y=\pm b$两个表面的条件要求
        \[ -\frac{V_0}{x} = \sum_{n=1}^\infty A_n \sin\frac{n\pi x}{a} \Rightarrow A_n = - \frac{V_0}{a} \frac{2}{a}\int_0^1\rd{x}\,\pare{x\sin \frac{n\pi x}{a}}. \]
        可得
        \[ A_n = -\frac{2V_0}{n^2\pi^2}\pare{-n\pi \cos n\pi + 0} = \frac{2V_0}{n\pi}\pare{-1}^n. \]
    \end{ex}
    \begin{remark}
        选择另一SL问题仍可得到正确的解.
    \end{remark}
\end{sample}

% subsubsection 直角坐标系 (end)

\subsubsection{柱坐标系} % (fold)
\label{ssub:柱坐标系}

设问题具有沿$z$轴的平易对称性, $\varphi = \varphi\pare{s,\phi}$, 令
\[ \laplacian \varphi = \rec{s}\+DsD{}\pare{s\+DsD{\varphi}} + \rec{s^2}\+D{\phi^2}D{^2\varphi} = 0. \]
令$\varphi = R\pare{s}\Phi\pare{\phi}$, 有
\[ s^2 \frac{\laplacian \varphi}{\varphi} = \underbrace{\frac{s}{R}\+dsd{}\pare{s\+dsdR}}_{m^2} + \underbrace{\rec{\Phi}\+d{\phi^2}d{^2\Phi}}_{-m^2} = 0. \]
即$\displaystyle s\+dsd{}\pare{s\+dsdR} = m^2 R$, $\+d{\phi^2}d{^2\Phi} = -m^2\Phi$.
\begin{align*}
    m=0 &\Rightarrow R = \left\{\begin{aligned}
        1 \\ \ln s
    \end{aligned} \right\}, \quad \Phi = \left\{\begin{aligned}
        1 \\ \phi
    \end{aligned} \right\}, \\
    m\neq 0  &\Rightarrow R = \left\{\begin{aligned}
        s^m \\ \ln -m
    \end{aligned} \right\}, \quad \Phi = \left\{\begin{aligned}
        1 \\ \phi
    \end{aligned} \right\}.
\end{align*}
可得
\[ \varphi = \pare{A_0 + B_0 s}\pare{C_0 + D_0\phi} + \sum_{m\neq 0} \pare{A_m s^m + \frac{B_m}{s^m}} \pare{C_m \cos m\phi + D_m \sin m\phi}. \]
\begin{cenum}
    \item 若$\phi = \brac{0,2\pi}$, 则周期性要求
    \[ \varphi = A_0 B_0\ln s + \sum_{m=1}^\infty\pare{A_ms^m + \frac{B_m}{s^m}}\pare{C_m \cos m\phi + D_m \sin m\phi}. \]
    \item 若无周期性, 则不存在上述约束, 则$m$未必为实数.
\end{cenum}
\begin{figure}[ht]
    \centering
    \incfig{4cm}{UniformOuterLaplacian}
    \caption{}
    \label{fig:圆柱laplacian}
\end{figure}
\begin{sample}
    \begin{ex}
        \cref{fig:圆柱laplacian}中, 假设外场均匀, 圆柱单位长度带电量$\lambda$. 当$s\rightarrow\infty$,
        \[ \varphi \sim -E_0 s\cos\phi - \frac{\lambda}{2\pi s_0}\ln s. \]
        边界条件为$\varphi\vert_{s=R} = \const$. 由柱坐标下的通解可设
        \[ \varphi = -E_0 s\cos\phi - \frac{\lambda}{2\pi\epsilon_0}\ln s + \frac{B}{s}\cos\phi, \]
        由边界条件可得
        \[ -E_0 R + \frac{B}{R} = 0,\quad B = E_0 R^2. \]
        因此
        \begin{align*}
            \varphi &= -\frac{\lambda}{2\pi\epsilon_0} \ln s - E_0 s\cos\phi + \frac{E_0 R^2}{s}\cos\phi \\
            &= -\frac{\lambda}{2\pi \epsilon_0} \ln s - \+vE_0\cdot \+vr\pare{1-\frac{R^2}{s^2}}.
        \end{align*}
        表面处电场为
        \begin{align*}
            \+vE\vert_{s=R} &= -\pare{\grad \varphi}_{s=R} = \+us \brac{\frac{\lambda}{2\pi\epsilon_0 R} + \pare{-\frac{2R^2}{R^3}}\+vE_0\cdot \+vr} \\
            &= \+us \brac{\frac{\lambda}{2\pi\epsilon_0 R} - 2E_0\cos\phi}.
        \end{align*}
        从而
        \[ \sigma = \epsilon_0 E_s = \frac{\lambda}{2\pi R} - 2\epsilon_0 E_0 \cos\phi. \]
        单位长度电量自动为$\lambda$.
    \end{ex}
\end{sample}
\begin{figure}[ht]
    \centering
    \incfig{4cm}{SectorLaplacian}
    \caption{}
    \label{fig:扇形laplacian}
\end{figure}
\begin{sample}
    \begin{ex}
        \cref{fig:扇形laplacian}中, 分离变量后可设
        \[ R \sim s^{\pm i\alpha} \sim e^{\pm i\alpha \ln s} \sim \left\{\begin{aligned}
            \cos \alpha \ln s \\
            \sin \alpha \ln s
        \end{aligned}\right\} \sim \left\{\begin{aligned}
            \cos \alpha \ln s/s_1 \\
            \sin \alpha \ln s/s_1
        \end{aligned}\right\}. \]
        最终解为
        \[ \varphi = \sum_\alpha \neq 0 A_\alpha \sin \pare{\alpha \ln \frac{s}{s_1}}\frac{\sinh \alpha \phi}{\sinh \alpha\beta}. \]
        $s_2$处接地的条件要求
        \[ \varphi = \sum_{n=1}^\infty A_n \sin \frac{n\pi \ln \frac{s}{s_1}}{\ln \frac{s_2}{s_1}} \frac{\sinh \frac{n\pi \phi}{\ln s_2/s_1}}{\sinh \frac{n\pi\beta}{\ln s_2/s_1}}. \]
    \end{ex}
\end{sample}

% subsubsection 柱坐标系 (end)

\subsubsection{球坐标系} % (fold)
\label{ssub:球坐标系}

Laplace方程为
\[ \laplacian \varphi = \rec{r^2}\+DrD{}\pare{r^2\+DrD\varphi} + \rec{r^2\sin\theta}\+D\theta D{}\pare{\sin\theta \+D\theta D\varphi} + \rec{r^2\sin^2\theta} \+D{\phi^2}D{^2\varphi} = 0. \]
令$\varphi = R\pare{r}\Theta\pare{\theta}\Phi\pare{\phi}$, 有
\begin{align*}
    & r^2 \frac{\laplacian \varphi}{\varphi} = \underbrace{\rec{R}\+drd{}\pare{r^2 \+drdR}}_{l\pare{l+1}} + \rec{\sin^2\theta}\underbrace{\brac{\overbrace{\frac{\sin\theta}{\Theta}\+D\theta D{} \pare{\sin\theta \+d\theta d\Theta}}^{\smash{-l\pare{l+1}\sin^2\theta + m^2}} + \overbrace{\+d{\phi^2}d{^2\Phi}}^{\smash{-m^2}}}}_{-l\pare{l+1}\sin^2\theta} = 0.\\
    & \begin{cases}
        \displaystyle \+drd{}\pare{r^2 \+drdR} = l\pare{l+1} R, \\
        \displaystyle \+d{\phi^2}d{^2\Phi} = -m^2\Phi, \\
        \displaystyle \rec{\sin\theta}\+d\theta d{}\pare{\sin\theta\+d\theta d\Theta} + \brac{l\pare{l+1} - \frac{m^2}{\sin^2\theta}}\Theta = 0\\ 
        \displaystyle \xLongrightarrow[\+d\theta d{} = -\sin\theta \+dxd{}]{x=\cos\theta}
        \+dxd{}\brac{\pare{1-x^2}\+dxd\Theta} + \brac{l\pare{l+1} - \frac{m^2}{1-x^2}}\Theta = 0.
    \end{cases}
\end{align*}

\begin{cenum}
    \item 分离变量解的性质:
    \begin{cenum}
        \item $\displaystyle R \sim \curb{r^l,\rec{r^{l+1}}}$.
        \item $\displaystyle \phi \in \brac{0,2\pi}$的单值性要求$\Phi \sim \curb{\cos m\phi, \sin m\phi}$, $m=0,1,2,\cdots$.
        \item $\theta \in \brac{0,\pi}$内解的有界性要求$l=0,1,2,\cdots$, 且对于每一个特定的$l$, $m = 0,1,2,\cdots, l$. 且对于$l,m$, 解为
        \[ \Theta\pare{\theta} = P_l^m\pare{\cos\theta}. \]
        特别地在转动不变的情况下, $m=0$, $P$退化为Legendre多项式.
    \end{cenum}
    \item Legendre多项式: $P_l\pare{x} = P_l^0\pare{x}$.
    \begin{cenum}
        \item Rodrigues公式:
        \[ P_l\pare{x} = \rec{2^l l!}\+d{x^l}d{^l}\pare{x^2-1}^l. \]
        \item $P_l\pare{-x} = \pare{-1}^l P_l\pare{x}$, $P_l\pare{1} = 1$,\\
        $P_{2k-1}\pare{0} = 0$, $\displaystyle P_{2k}\pare{0} = \pare{-1}^k\frac{\pare{2k-1}!!}{\pare{2k}!!}$.
        \item 正交完备性:
        \begin{cenum}
            \item $\displaystyle \int_{-1}^{1} \rd{x}\, P_l\pare{x}P_{l'}\pare{x} = \frac{2}{2l+1} \delta_{ll'}$.
            \item $\displaystyle \sum_{l=0}^\infty \frac{2l+1}{2}P_l\pare{x'} P_l\pare{x} = \delta\pare{x-x'}$.
        \end{cenum}
    \end{cenum}
    \item 关联Legendre多项式:
    \[ P_l^m\pare{x} = \pare{-1}^m \pare{1-x^2}^{m/2}\+d{x^m}d{^m}P_l\pare{x},\quad m=0,1,\cdots,l. \]
    \item 球谐函数:
    \begin{align*}
        &Y_{lm}\pare{\theta,\phi} = Y_{lm}\pare{\Omega} = Y_{lm}\pare{\+ur} = \sqrt{\frac{2l+1}{4\pi}\frac{\pare{l-m}!}{\pare{l+m}!}}P_l^m\pare{\cos\theta}e^{\pm im\phi},\quad m>0,\\
        &Y_{l,-m} = \pare{-1}^m Y_{l,m}^*.
    \end{align*}
    正交完备性:
    \begin{cenum}
        \item $\displaystyle \iint \rd{\Omega}\, Y_{l'm'}^*\pare{\Omega}Y_{lm}\pare{\Omega} = 
        \delta_{ll'}\delta_{mm'}$.
        \item $\displaystyle \sum_{l=0}^\infty \sum_{m=-l}^{l} Y_{l'm'}^*\pare{\Omega} Y_{lm}\pare{\Omega} = \delta\pare{\cos \theta' - \cos\theta} \delta\pare{\phi' - \phi}.$
    \end{cenum}
    \item 一般解: 单值有限,
    \begin{align*}
        \varphi\pare{r,\theta,\phi} &= \sum_{l=0}^\infty \sum_{m=0}^l \pare{A_{lm}r^l + \frac{B_{lm}}{r^{l+1}}}P_l^m\pare{\cos\theta}\pare{C_{lm}\cos m\phi + D_{lm}\sin m\phi} \\
        &\rightarrow \sum_{l=0}^\infty \sum_{m=-l}^l \pare{A_{lm}r^l + \frac{B_{lm}}{r^{l+1}}}Y_{lm}\pare{\theta,\phi}.
    \end{align*}
    若问题具有方位角对称性, 则
    \[ \varphi\pare{r,\theta} = \sum_{l=0}^\infty \pare{A_l r^l + \frac{B_l}{r^{l+1}}}P_l\pare{\cos\theta}. \]
\end{cenum}
\begin{ex}
    $P_0\pare{x} = 1$, $P_1\pare{x} = x$, $\displaystyle P_2\pare{x} = \frac{3x^2 - 1}{2}$, $\displaystyle P_3\pare{x} = \frac{5x^3 - 3x}{2}$.
\end{ex}
\begin{ex}
    $\displaystyle \begin{aligned}[t]
        &P_0^0 \pare{x} = 1, \\
        &P_1^0 \pare{x} = x, && P_1^1\pare{x} = -\sqrt{1-x^2}, \\
        &P_2^0 \pare{x} = \half\pare{3x^2 - 1}, && P_2^1 = -3x\sqrt{1-x^2}, && P_2^2\pare{x} = 3\pare{1-x^2}.
    \end{aligned}$
\end{ex}
\begin{ex}
    $\displaystyle \begin{aligned}[t]
        &Y_{00} = \rec{\sqrt{4\pi}}, \\
        &Y_{10} = \sqrt{\frac{3}{4\pi}}\cos\theta, && \rightarrow \sqrt{\frac{3}{4\pi}}\frac{z}{r}, \\
        & Y_{1,\pm 1} = \mp \sqrt{\frac{3}{8\pi}}\sin\theta e^{\pm i\phi}, && \rightarrow \mp\sqrt{\frac{3}{8\pi}}\frac{x\pm iy}{r}, \\
        &Y_{20} = \sqrt{\frac{5}{16\pi}}\pare{3\cos^2\theta - 1}, && \rightarrow \sqrt{\frac{5}{16\pi}}\frac{3z^2-r^2}{r^2}, \\
        & Y_{2,\pm 1} = \mp\sqrt{\frac{15}{8\pi}}\sin\theta\cos\theta e^{\pm i\phi}, && \rightarrow \mp\sqrt{\frac{15}{8\pi}}\frac{z\pare{x\pm iy}}{r^2}, \\
        & Y_{2,\pm 2} = \sqrt{\frac{15}{32\pi}}\sin^2\theta e^{\pm 2i\phi}, && \rightarrow \sqrt{\frac{15}{32\pi}}\frac{\pare{x\pm iy}^2}{r^2}.
    \end{aligned}$
\end{ex}
\begin{figure}[ht]
    \centering
    \incfig{4cm}{SphereInUnif}
    \caption{}
    \label{fig:均匀场中的介质球}
\end{figure}
\begin{sample}
    \begin{ex}
        如\cref{fig:均匀场中的介质球}, $\epsilon$为相对介电常数.
        \begin{align*}
            \varphi_1 &= \sum_{l=0}^\infty A_l \frac{r^l}{R^l}P_l\pare{\cos\theta},\\
            \varphi_2 &= -E_0 r P_1\pare{\cos\theta} + \sum_{l=0}^\infty B_l \frac{R^{l+1}}{r^{l+1}}P_l\pare{\cos\theta}.
        \end{align*}
        边值关系要求$r=R$处,
        \begin{align*}
            & \varphi_1 = \varphi_2 \Rightarrow \sum_l A_l P_l\pare{\cos\theta} = \sum_l B_l P_l\pare{\cos\theta} - E_0 RP_1\pare{\cos\theta}, \\
            & \epsilon \+DrD{\varphi_1} = \+DrD{\varphi_2} \Rightarrow \epsilon \sum_l \frac{lA_l}{R}P_l\pare{\cos\theta} = -\sum_l \frac{\pare{l+1}B_l}{R} P_l - E_0 P_1.
        \end{align*}
        各$P_l$系数相等,
        \begin{align*}
            & A_l = B_l,\quad \pare{l\neq 1},\quad A_1 = B_1 - E_0 R, \\
            & \epsilon lA_l = -\pare{l+1}B_l,\quad \pare{l\neq 1},\quad \epsilon A_1 = -2B_1 - E_0R.\\
            &\Rightarrow A_l = 0 = B_l,\quad \pare{l\neq 1},\quad \begin{cases}
                \displaystyle A_1 = -\frac{3}{\epsilon + 2}E_0 R, \\[.5em]
                \displaystyle B_1 = \frac{\epsilon - 1}{\epsilon + 2}E_0 R.
            \end{cases}
        \end{align*}
        从而
        \begin{align*}
            & \varphi_1 = -\frac{3}{\epsilon + 2}E_0 r\cos\theta = -\frac{3}{\epsilon+2}\+vE_0\cdot \+vr, \\
            & \varphi_2 = -E_0 r\cos\theta + \frac{\epsilon - 1}{\epsilon + 2}\frac{R^3}{r^2}E_0 \cos\theta. \\
            & \+vE_1 = -\grad\varphi_1 = \frac{3}{\epsilon + 2}\+vE_0 \Rightarrow \+vP = %\chi\epsilon_0 \+vE_1 = 
            \pare{\epsilon - 1}\epsilon_0 \+vE_1 = \frac{\epsilon - 1}{\epsilon + 2}3\epsilon_0 \+vE_0. \\
            & \begin{cases}
                \displaystyle \varphi_1 = -\pare{\+vE_0 - \frac{\+vP}{3\epsilon_0}}\cdot \+vr, \\[.5em]
                \displaystyle \varphi_2 = -\+vE_0 \cdot \+vr = \frac{\+vp\cdot \+vr}{4\pi\epsilon_0 r^2}.
            \end{cases} \\
            & \+vp = \frac{4\pi R^3}{3}\+vP.
        \end{align*}
    \end{ex}
\end{sample}
\begin{sample}
    \begin{ex}
        若将一小介质球由$\infty$处移至外场$\+vE_0\pare{\+vr}$中, 外界做功如何? 一种方法为采用静电力$\+vF = \pare{\+vp\+v\cdot \grad}\+vE_0$, 另一种方法为$\+vF = -\grad U = \grad\pare{\+vp\cdot \+vE_0}$, $U = -\+vp\cdot \+vE_0$. 此处应采用前者. 近似$\+vp = \alpha \epsilon_0 \+vE_0$,
        \begin{align*}
            & A = -\int_\infty^{\+vr}\+vF\cdot \rd{\+vl} \approx -\int_\infty^{\+vr} \pare{\+vp\+v\cdot\grad} \+vE_0 \cdot \rd{\+vl} \\
            & = -\alpha \epsilon_0\int_\infty^{\+vr} \pare{\+vE_0 \+v\cdot \grad \+vE_0}\cdot \rd{\+vl} \\
            & = -\alpha \epsilon_0\int_\infty^{\+vr} \brac{\cancelto{0}{\pare{\curl \+vE_0}}\times \+vE_0 + \half \grad E_0^2}\cdot \rd{\+vl} \\
            &= -\half \alpha \epsilon_0 \int_{\infty}^{\+vr}\pare{\grad E_0^2}\cdot \rd{\+vl} \\
            &= -\half \alpha \epsilon E_0^2\pare{\+vr}.
        \end{align*}
        也可以
        \begin{align*}
            A &= W_2 - W_1 = \Delta W = \half \iiint\rd{V}\,\rho_0\varphi - \half \iiint\rd{V}\,\rho_0 \varphi_0 \\
            &= \half \iiint \rd{V}\,\rho_0 \varphi' = \half \iiint \rd{V}\, \rho'\varphi_0\,\rd{V} = -\half \iiint \rd{V}\,\varphi_0 \div \+vP \\
            &= -\half \iiint \rd{V}\,\div\pare{\varphi \+vP} + \half \iiint\rd{V}\,\+vP\cdot \grad\varphi_0\\
            &= -\half \iiint \rd{V}\,\+vP\cdot \+vE_0 \approx -\half \pare{\iiint \rd{V}\, \+vP}\cdot \+vE_0 \\
            &= -\half \+vp\cdot \+vE_0 = -\half \alpha \epsilon_0 E_0^2.
        \end{align*}
    \end{ex}
\end{sample}
\begin{sample}
    \begin{ex}
        球面上由面电荷分布, 且已知球面上的电势$\varphi_0 = V_0 \cos 3\theta$, 求球面上的电荷分布.
        \[ \cos 3\theta = 4\cos^3\theta - 3\cos\theta = -\frac{3}{5}\cdot P_1\pare{\cos\theta} + \frac{8}{5}\cdot P_3\pare{\cos\theta}. \]
        从而球内
        \[ \varphi_1 = A_1 \frac{r}{R}P_1\pare{\cos\theta} + A_3 \frac{r^3}{R^3}P_3\pare{\cos\theta}, \]
        球外
        \[ \varphi_2 = B_1 \frac{R^2}{r^2}P_1\pare{\cos\theta} + B_3 \frac{R^4}{r^4}P_3\pare{\cos\theta}. \]
        立刻有
        \[ A_1 = B_1 = -\frac{3V_0}{5},\quad A_3 = B_3 = \frac{8V_0}{5}. \]
        由
        \[ \sigma = \epsilon_0 \pare{\+DrD{\varphi_1} - \+DrD{\varphi_2}}_{r=R} \]
        可得表面电荷密度.
    \end{ex}
\end{sample}

% subsubsection 球坐标系 (end)

% subsection 分离变量法 (end)

\subsection{Green函数方法} % (fold)
\label{sub:green函数方法}

\subsubsection{概述} % (fold)
\label{ssub:概述}

对于$\displaystyle \begin{cases}
    \laplacian \varphi\pare{\+vr} = -\rho\pare{\+vr}/\epsilon_0,\quad \+vr\in V, \\
    \text{Dirichlet或Neumann边界条件}.
\end{cases}$ 考虑$\rho\pare{\+vr} = \delta\pare{\+vr-\+vr'}$, $\varphi\pare{r} = G\pare{\+vr;\+vr'}$. 即$G\pare{\+vr;\+vr'}$是$\+vr'$处的$\delta$输入在$\+vr$处的响应. 问题转化为
\[ \begin{cases}
    \laplacian G\pare{\+vr;\+vr'} = -\delta\pare{\+vr - \+vr'}/\epsilon_0,\quad \+vr,\+vr'\in V,\\
    \text{Dirichlet或Neumann边界条件}.
\end{cases} \]
\begin{cenum}
    \item $\varphi$与$G$的联系: 由Green恒等式,
    \begin{align*}
        & \iiint_V \rd{V'}\, \brac{\varphi\nabla'^2\psi - \psi \nabla'^2\varphi} = \oiint_{\partial V} \rd{\sigma'}\,\brac{\varphi \+D{n'}D{\psi} - \psi \+D{n'}D{\varphi}}. \\
        & \varphi = \varphi\pare{\+vr'},\quad \psi = G\pare{\+vr',\+vr},\\
        & \varphi\pare{\+vr} = \iiint_V \rd{V'}\,\rho\pare{\+vr'}G\pare{\+vr';\+vr} - \epsilon_0\oiint_{\partial V}\rd{\sigma'}\, \varphi\pare{\+vr'}\+D{n'}D{G\pare{\+vr';\+vr}}\\
        &\phantom{\varphi\pare{\+vr} = \,} + \epsilon_0\oiint_{\partial V}\rd{\sigma'}\, \+D{n'}D{\varphi\pare{\+vr'}}G\pare{\+vr';\+vr}.
    \end{align*}
    \item Dirichlet边界条件: $\laplacian G\+_D_\pare{\+vr;\+vr'} = -\delta\pare{\+vr-\+vr'}.\epsilon_0$, $G_D\pare{\+vr_S;\+vr'} = 0$, $\+vr_S\in \partial_V$.
    \begin{align*}
        & \varphi\pare{\+vr} = \iiint_V \rd{V'}\,\rho\pare{\+vr'}G_D\pare{\+vr';\+vr} - \epsilon_0\oiint_{\partial V}\rd{\sigma'}\, \varphi\pare{\+vr'}\+D{n'}D{G_D\pare{\+vr';\+vr}}
    \end{align*}
    \item Neumann边界条件: $\laplacian G\+_N_\pare{\+vr;\+vr'} = -\delta\pare{\+vr-\+vr'}.\epsilon_0$, $\displaystyle \+D{n}D{G_N\pare{\+vr_S;\+vr'}} = -\rec{\epsilon_0 S}$, $\+vr_S\in \partial_V$.
    \begin{align*}
        \varphi\pare{\+vr} = \iiint_V \rd{V'}\,\rho\pare{\+vr'}G_N\pare{\+vr';\+vr} + \epsilon_0\oiint_{\partial V}\rd{\sigma'}\, \+D{n'}D{\varphi\pare{\+vr'}}G_N\pare{\+vr';\+vr} + \expc{\varphi}_S.
    \end{align*}
\end{cenum}
\begin{remark}
    Neumann边界条件不能设边界法向导数为零, 否则相应的$E_n = 0$从而电通量为零.
\end{remark}

% subsubsection 概述 (end)

\subsubsection{Dirichlet Green函数} % (fold)
\label{ssub:dirichlet_green函数}

\begin{cenum}
    \item $G_D\pare{\+vr;\+vr'}$是接地导体的导体壳内$\+vr'$处放置$Q=1$时$\+vr$处的电势.
    \item $G_D\pare{\+vr;\+vr'} = G_D\pare{\+vr';\+vr}$.
    \begin{align*}
        & \iiint_V \rd{V}\,\brac{\varphi \laplacian \psi - \psi\laplacian \varphi} = \oiint_{\partial V}\rd{\sigma}\,\brac{\varphi \+DnD\psi - \psi \+DnD\phi},\\
        & \varphi = G_D\pare{\+vr;\+vr_1},\quad \psi = G_D\pare{\+vr,\+vr_2}. \\
        & \Rightarrow G_D\pare{\+vr_1;\+vr_2} = G_D\pare{\+vr_2;\+vr_2}.
    \end{align*}
    也可以直接由Green互易定理得到.
    \item 形式解: $\displaystyle \begin{cases}
        \laplacian \varphi\pare{\+vr} = -\rho\pare{\+vr}/\epsilon_0,\\
        \varphi\pare{\+vr_S}\text{已知},\quad \+vr_S \in \partial V.
    \end{cases}$
    \[ \varphi\pare{\+vr} = \iiint_V \rd{V'}\,\rho\pare{\+vr'} G_D\pare{\+vr;\+vr'} - \epsilon_0 \oiint_{\partial V}\rd{\sigma'}\, \varphi\pare{\+vr'} \+D{n'}D{G_D\pare{\+vr;\+vr'}}. \]
    第一项由内部自由电荷贡献, 第二项由边界感应电荷贡献.
\end{cenum}
\begin{ex}
    对于全空间的情形, 显然$\displaystyle G_D\pare{\+vr;\+vr'} = \rec{4\pi\epsilon_0 \+gr}$. 故对于局域电荷分布,
    \[ \varphi\pare{\+vr} = \iiint_{V'}\rd{V'}\, \frac{\rho\pare{\+vr'}}{4\pi\epsilon_0 \+gr}. \]
\end{ex}
\begin{figure}[ht]
    \centering
    \incfig{6cm}{UpperGreen}
    \caption{}
    \label{fig:上半Green}
\end{figure}
\begin{sample}
    \begin{ex}
        如\cref{fig:上半Green}, 为了求解上半平面的Green函数, 构造电像后即有
        \[ G_D\pare{\+vr;\+vr'} = \rec{4\pi\epsilon_0}\pare{\rec{\+gr} - \rec{\+gr'}}. \]
        边界
        \begin{align*}
            & \left.-\epsilon_0 \+D{n'}D{G_D}\right\vert_{z'=0} = \left.\epsilon_0 \+D{z'}D{G_D}\right\vert_{z'=0} \\
            & = \rec{4\pi}\brac{\frac{z - z'}{\+gr^3} + \rec{z + z'}{\+gr_1^3}}_{z'=0} \\
            & = \frac{z}{2\pi\+gr^3}.
        \end{align*}
        故总的解为
        \[ \varphi\pare{\+vr} = \rec{4\pi\epsilon_0} \iiint_{z'>0}\rd{V'}\rho\pare{\+vr'}\pare{\rec{\+gr} - \rec{\+gr_1}} + \frac{z}{2\pi}\iint_{z'=0} \frac{\varphi}{\+gr^3}. \]
    \end{ex}
\end{sample}
\begin{figure}[ht]
    \centering
    \incfig{6cm}{UpperRingGreen}
    \caption{}
    \label{fig:上半圆Green}
\end{figure}
\begin{sample}
    \begin{ex}
        如\cref{fig:上半圆Green}, 有
        \[ \+gr = \sqrt{\pare{s-s'\cos\phi}^2 + \pare{-s'\sin\phi'}^2 + z^2} = \sqrt{r^2 - s'^2 - 2ss'\cos\phi}. \]
        而
        \[ \varphi = \frac{V_0 z}{2\pi}\int_0^{2\pi}\rd{\phi'}\int_0^a \frac{s'\,\rd{s}}{\+gr^3}. \]
        当$r\gg a$, 有
        \[ \frac{s'}{r} = \epsilon \ll 1 \Rightarrow \+gr = r\sqrt{1+\epsilon^2 - \epsilon\cdot 2\lambda \cos\phi'},\quad \lambda = \frac{s}{r}. \]
        令$\Delta = \epsilon^2 - \epsilon\cdot 2\lambda \cos\phi'$, 则
        \begin{align*}
            \frac{s'}{\+gr^3} &= \frac{s'}{r^3}\pare{1+\Delta}^{-3/2} = \frac{s'}{r^3}\brac{1-\frac{3}{2}\Delta + \rec{2!}\pare{-\frac{3}{2}}\cdot \pare{-\frac{5}{2}}\Delta^2 + \cdots} \\
            &= \frac{s'}{r^3}\brac{1-\frac{3}{2}\pare{\epsilon^2 - \epsilon\cdot 2\lambda \cos\phi'} + \rec{2!}\pare{-\frac{3}{2}}\cdot \pare{-\frac{5}{2}}\pare{\epsilon^2 - \epsilon\cdot 2\lambda \cos\phi'}^2 + \cdots} \\
            &\approx \frac{s'}{r^3}\brac{1+\epsilon\cdot 3\lambda \cos\phi' + \epsilon^2 \cdot \frac{3}{2}\pare{5\lambda^2 \cos^2\phi' - 1}}.
        \end{align*}
        从而
        \[ \int_0^{2\pi}\rd{\phi'}\frac{s'}{R^3} = \frac{s'}{r^3}\brac{2\pi + 0 + \epsilon^2\cdot \frac{3}{4}\pare{5\lambda^2 - 2}\cdot 2\pi}. \]
        故
        \begin{align*}
            \varphi &= \frac{V_0 z}{r^3} \int_0^a \rd{s'}\brac{s' + \frac{s'^3}{r^2}\cdot \frac{3}{4}\pare{5\lambda^2 - 2}} \\
            &= \frac{V_0 a^2 z}{2r^3}\pare{1+\frac{3}{8}\pare{5\lambda^2 - 2}\frac{a^2}{r^2}}.
        \end{align*}
    \end{ex}
\end{sample}
\begin{figure}[ht]
    \centering
    \incfig{6cm}{SphereGreen}
    \caption{}
    \label{fig:球Green}
\end{figure}
\begin{sample}
    \begin{ex}
        如\cref{fig:球Green}, 有
        \[ \frac{Q'}{Q} = \frac{a}{r'},\quad \frac{r''}{r'} = \pare{\frac{a}{r'}}^2. \]
        定义
        \begin{align*}
            & \+gr = \+vr - \+vr' = r\+ur - r'\+ur',\\
            & \+gr_1 = \+vr - \+vr'' = r\+ur - \frac{a^2}{r'}\+ur',\\
            & \+gr_2 = \frac{r'}{a}\+gr_1 = \frac{rr'}{a}\+ur - a\+ur'.
        \end{align*}
        从而
        \[ G_D\pare{\+vr;\+vr'} = \rec{4\pi\epsilon_0}\pare{\rec{\+gr} - \frac{a/r'}{\+gr_1}} = \rec{4\pi\epsilon_0}\pare{\rec{\+gr} - \rec{\+gr_2}}. \]
        其中
        \begin{align*}
            &\+gr = \sqrt{r^2 + r'^2 - 2rr'\cos \alpha},\\
            &\+gr_2 = \sqrt{\pare{\frac{rr'}{a}}^2 + a^2 - 2rr'\cos\alpha},\\
            &\cos\alpha = \+ur\cdot \+ur'.
        \end{align*}
        边界
        \begin{align*}
            \left.-\epsilon_0 \+D{n'}D{G_D}\right\vert_{r'=a} &= \left.-\epsilon_0 \+D{r'}D{G_D}\right\vert_{r'=a} = \rec{4\pi}\brac{-\frac{r' - r\cos\alpha}{\+gr^3} + \frac{\frac{r^2r'}{a} - r\cos\alpha}{\+gr_2^3}}_{r' = a}\\
            &= \frac{r^2 - a^2}{4\pi a\+gr^3}.
        \end{align*}
        其中$\+gr = \sqrt{a^2 + r^2 - 2ar\cos\alpha}$. 从而
        \[ \varphi\pare{\+gr} = \rec{4\pi\epsilon_0}\int_{r'>a}\rd{V'}\,\rho\pare{\+vr'}\pare{\rec{\+gr} - \rec{\+gr_2}} + \frac{r^2-a^2}{4\pi a}\oiint_{r'=a}\rd{\sigma'}\,\frac{\varphi}{R}. \]
        如果要得到球内的电势, 只需讲积分替换为对球内空间的积分, 且$r^2 - a^2\mapsto a^2 - r^2$.
    \end{ex}
\end{sample}
\begin{sample}
    \begin{ex}
        在无电荷区域划一球面, 则球心电势为
        \[ \varphi_0 = \frac{a^2}{4\pi a}\oiint_{r'=a}\rd{\sigma'}\frac{\varphi}{a^3} = \rec{4\pi a^2}\oiint_{r'=a}\rd{\sigma'}\varphi. \]
        故无电荷球形区域内$\varphi\pare{0} = \expc{\varphi}_{r=a}$.
    \end{ex}
\end{sample}
\begin{sample}
    \begin{ex}
        设原点处有一球面半径$a$, 上半球面电势$V_0$, 下面球面电势$-V_0$, 则
        \begin{align*}
            & \varphi\pare{r,\theta} = \frac{V_0\pare{r^2 - a^2}}{4\pi a} \cdot a^2\int_0^{2\pi}\rd{\phi'} \brac{\int_0^{\pi/2}\rd{\theta'} - \int_{\pi/2}^{\pi}\rd{\theta'}}\frac{\sin\theta'}{\+gr^3}.
        \end{align*}
        其中
        \[ \begin{cases}
            \+gr = \sqrt{r^2 + a^2 - 2ar\cos\alpha}, \\
            \cos\alpha = \+ur\cdot \+ur',
        \end{cases}\quad \begin{cases}
            \+ur = \pare{\sin\theta,0,\cos\theta},\\
            \+ur' = \pare{\sin\theta'\cos\phi',\sin\theta'\sin\phi',\cos\theta'}.
        \end{cases} \]
        从而
        \[ \cos\alpha = \cos\theta\cos\theta' + \sin\theta\sin\theta'\cos\phi'. \]
        将$\theta'$替换为$\pi - \theta'$, 则$\displaystyle \int_{\pi/2}^\pi \rd{\theta'} = \int_0^{\pi/2}\rd{\theta'}$. 再替换$\phi'\mapsto \pi + \phi'$, 并设$\epsilon = a/r$.
        \begin{align*}
            \varphi\pare{r,\theta} &= \frac{V_0 a\pare{r^2 - a^2}}{4\pi}\int_0^{2\pi}\rd{\phi'}\int_0^{\pi/2}\rd{\theta'}\brac{\frac{\sin\theta'}{\pare{r^2 + a^2 - 2ar\cos\alpha}^{3/2}} - \frac{\sin\theta'}{\pare{r^2 + a^2 + 2ar\cos\alpha}^{3/2}}} \\
            &= \frac{V_0a\pare{r^2 - a^2}}{4\pi r^3}\int_0^{2\pi}\rd{\phi'}\int_0^{\pi/2}\rd{\theta'}\sin\theta' \brac{\pare{1+\epsilon^2 - \epsilon\cdot 2\cos\alpha}^{-3/2} - \pare{1+\epsilon^2 + \epsilon\cdot 2\cos\alpha}^{-3/2}}
            &\approx \frac{V_0a\pare{r^2 - a^2}}{4\pi r^3}\int_0^{2\pi}\rd{\phi'}\int_0^{\pi/2}\rd{\theta'}\sin\theta'\cdot 6\epsilon \cos \alpha \\
            &= \frac{V_0a\pare{r^2 - a^2}}{4\pi r^3}\int_0^{2\pi}\rd{\phi'}\int_0^{\pi/2}\rd{\theta'}\sin\theta'\cdot 6\epsilon \cos\theta\cos\theta' \\
            &= \frac{3}{2}V_0\cdot \frac{a}{r}\cdot \frac{a}{r}\cos\theta = \frac{\+vp\cdot \+vr}{4\pi\epsilon_0 r^3}.
        \end{align*}
    \end{ex}
\end{sample}
\begin{figure}[ht]
    \centering
    \incfig{12cm}{GreenEx3}
    \caption{}
    \label{fig:球面green}
\end{figure}
\begin{sample}
    \begin{ex}
        \cref{fig:球面green}, 若空间中有一半径$a$的球面, 球面上电势为$\varphi_0 = V_0\cos 3\theta$. 球外
        \begin{align*}
            & \varphi = \frac{\pare{r^2 - a^2}V_0}{4\pi a}\oiint_{r'=a}\rd{\sigma'}\, \frac{\cos 3\theta'}{\+gr^3}.
        \end{align*}
        其中
        \[ \cos\theta' = \+un\cdot \+ur' = \cos\theta\cos\alpha + \sin\theta\sin\alpha\cos\beta. \]
    \end{ex}
\end{sample}

% subsubsection dirichlet_green函数 (end)

\subsubsection{本征函数展开} % (fold)
\label{ssub:本征函数展开}

对于一区域$\pare{V,\partial V}$, 考虑如下问题
\[ \begin{cases}
    \laplacian \psi_n\pare{\+vr} = -\lambda_n \psi_n\pare{\+vr}, & \+vr\in V,\\
    \psi_n\pare{\+vr_S} = 0, & \+vr_S \in \partial_V.
\end{cases} \]
则$\lambda_n > 0$, 且归一化后
\[ \sum_n \psi_n^*\pare{\+vr'}\psi_n\pare{\+vr} = \delta\pare{\+vr - \+vr'}. \]
则$V$内的Green函数
\[ G_D\pare{\+vr;\+vr'} = \rec{\epsilon_0}\sum_n \frac{\psi^*_n\pare{\+vr'}\psi_n\pare{\+vr}}{\lambda_n}. \]
显然$\displaystyle \laplacian G_D = -\rec{\epsilon_0} \sum_n \psi_n^*\pare{\+vr'}\psi_n\pare{\+vr} = -\rec{\epsilon_0}\delta\pare{\+vr - \+vr'}$.
\begin{sample}
    \begin{ex}
        在立方体表面上设计一面电荷分布, 使其激发的电场与位于中心的点电荷相同. 假设虚拟中心电荷为$Q$, 则要求面电荷与中心处$-Q$电荷激发之电场为零. 这等价于求中心有电荷$-Q$的接地立方体的感应电荷. 设立方体内的Green函数为$G_D$, 则
        \[ \varphi\pare{x,y,z} = -Q G_D\pare{x,y,z;-\frac{a}{2},-\frac{a}{2},-\frac{a}{2}}. \]
        若Green函数已知, 则
        \[ \sigma = \epsilon_0 \+DnD{\varphi}. \]
        本征值问题为
        \[ \begin{cases}
            \laplacian \psi = -\lambda \psi, \\
            \psi\vert_{\partial V} = 0.
        \end{cases} \]
        设$\psi = X\pare{x}Y\pare{y}Z\pare{z}$, 则
        \begin{align*}
            & \frac{\laplacian \psi}{\psi} = \underbrace{\rec{X}\+d{x^2}d{^2 X}}_{-\alpha^2} + \underbrace{\rec{Y}\+d{y^2}d{^2Y}}_{-\beta^2} + \underbrace{\rec{Z}\+d{z^2}d{^2z}}_{-\gamma^2} = -\lambda.\\
            & \Rightarrow \lambda = \alpha^2 + \beta^2 + \gamma^2. \\
            & X \sim \curb{\cos \alpha x, \sin \alpha x},\quad \alpha a = l\pi, \\
            & Y \sim \curb{\cos \beta y, \sin \beta y},\quad \beta a= m\pi, \\
            & Z \sim \curb{\cos \gamma z, \sin \gamma z}\quad \gamma a = n\pi. \\
            & \lambda_{lmn} = \frac{\pi^2}{a^2}\pare{l^2+m^2+n^2}. \\
            & \psi_{lmn} = \sqrt{\frac{8}{a^3}}\sin \frac{l\pi x}{a}\sin\frac{m\pi y}{a} \sin \frac{n\pi z}{a}.
        \end{align*}
        最终Green函数为
        \[ G_D\pare{\+vr;\+vr'} = \rec{\epsilon_0}\sum_{l,m,n=1}^\infty \frac{8}{a^3}\cdot \frac{a^2}{\pi^2}\cdot \rec{l^2+m^2+n^2} \sin \frac{l\pi x}{a}\sin\frac{m\pi y}{a} \sin \frac{n\pi z}{a}\sin \frac{l\pi x'}{a}\sin\frac{m\pi y'}{a} \sin \frac{n\pi z'}{a}. \]
    \end{ex}
\end{sample}

% subsubsection 本征函数展开 (end)

% subsection green函数方法 (end)

\subsection{电多极展开} % (fold)
\label{sub:电多极展开}

设电荷分布在半径$R$的球面内, 则
\[ \begin{cases}
    \displaystyle \varphi\pare{\+vr} = \rec{4\pi\epsilon_0} \iiint \frac{\rd{q}}{\+gr}, \\[.5em]
    \displaystyle \+vE\pare{\+vr} = -\grad \varphi\pare{\+vr}.
\end{cases} \]

\subsubsection{电势的电多极展开} % (fold)
\label{ssub:电势的电多极展开}

在$r\neq 0$处, $\displaystyle \tensor{I}:\grad\grad \rec{r} = \laplacian \rec{r} = 0$.
\begin{align*}
    \rec{\+gr} &= \rec{\abs{\+vr - \+vr'}} = e^{-\+vr\cdot \grad} \rec{r} \\
    &= \brac{1 - \+vr'\cdot \grad + \rec{2!}\+vr'\+vr':\grad\grad - \cdots} \rec{r} \\
    &= \brac{1 - \+vr'\cdot \grad + \rec{6}\pare{3\+vr'\+vr' - r'^2\tensor{I}}:\grad\grad - \cdots}\rec{r}. \\
    \grad \rec{r} &= -\frac{\+ur}{r^2} = -\frac{\+vr}{r^3}, \\
    \grad\grad\rec{r}  &= -\grad \frac{\+vr}{r^3} = -\pare{\grad \rec{r^3}}\+vr - \rec{r^3}\grad \+vr = \frac{3\+ur\+ur - \tensor{I}}{r^3}.
\end{align*}
定义
\begin{flalign*}
    &\text{总电量} && Q = \iiint \rd{q}, &&\\
    &\text{电偶极矩} && \+vp = \iiint \+vr'\,\rd{q}, &&\\
    &\text{电四极矩} && \+vD = \iiint \pare{3\+vr'\+vr' - r'^2\tensor{I}}\,\rd{q}, \begin{cases}
        \text{对称:} & D_{ij} = D_{ji},\\
        \text{迹零:} & \trace \tensor{D} = \tensor{D}:\tensor{I} = 0.
    \end{cases} &&
\end{flalign*}
有
\begin{align*}
    \rec{\+gr} &= \rec{4\pi\epsilon_0}\brac{\frac{Q}{r} - \+vp\cdot \grad \rec{r} + \rec{6}\tensor{D}:\grad\grad\rec{r} - \cdots} \\
    &= \frac{Q}{4\pi\epsilon_0 r} + \frac{\+vp\cdot \+ur}{4\pi\epsilon_0 r^2} + \half \frac{\+ur\cdot \tensor{D}\cdot \+ur}{4\pi\epsilon_0 r^3} + \cdots.
\end{align*}

% subsubsection 电势的电多极展开 (end)

\subsubsection{电偶极子} % (fold)
\label{ssub:电偶极子}

当$Q = 0$, $\displaystyle \varphi\pare{\+vr} = \frac{\+vp\cdot \+vr}{4\pi\epsilon_0 r^3}$, $\+vE\pare{\+vr} = -\grad \varphi\pare{\+vr}$.
\begin{align*}
    & \+vE = -\grad \varphi = -\rec{4\pi\epsilon_0} \grad \frac{\+vp\cdot \+vr}{r^3} = -\rec{4\pi\epsilon_0}\pare{\grad \frac{\+vr}{r^3}}\cdot \+vp, \\
    & \+vE = -\grad \varphi = -\grad\pare{-\rec{4\pi\epsilon_0}\+vp\cdot \grad \rec{r}}.
\end{align*}
\begin{cenum}
    \item 一般而言, 电偶极矩的选择与原点有关. 除非$Q=0$.
    \[ \+vp_1 = \iiint \+v{r}'_1\,\rd{q} = \iiint \+vr'_2\,\rd{q} + \iiint \+vd\,\rd{q} \Rightarrow \+vp_1 = \+vp_2 + Q\+vd. \]
    \item 电荷分布的对称性与$\+vp$的关系:
    \begin{cenum}
        \item 若电荷分布关于原点对称, $\rho\pare{\+vr} = \rho\pare{-\+vr}$, 则$\+vp = 0$. 在变换
        \[ \pare{x_1,x_2,x_3}\mapsto \pare{x'_1,x'_2,x'_3} = \pare{-x_1,-x_2,-x_3} \]
        下, 电偶极矩
        \[ \pare{p_1,p_2,p_3}\mapsto \pare{p'_1,p'_2,p'_3} = \pare{-p_1,-p_2,-p_3}, \]
        由电荷分布的对称性知$\pare{p_1,p_2,p_3} = \pare{-p_1,-p_2,-p_3}\Rightarrow \+vp = 0$.
        \item 若$z=0$是电荷分布的对称平面, 即$\rho\pare{x,y,z} = \rho\pare{x,y,-z}$, 则$\+vp\perp \+uz$. 在变换
        \[ \pare{x_1,x_2,x_3}\mapsto \pare{x'_1,x'_2,x'_3} = \pare{x_1,x_2,-x_3} \]
        下, 电偶极矩
        \[ \pare{p_1,p_2,p_3}\mapsto \pare{p'_1,p'_2,p'_3} = \pare{p_1,p_2,-p_3}, \]
        由电荷分布的对称性知$\pare{p_1,p_2,p_3} = \pare{p_1,p_2,-p_3}\Rightarrow p_3 = 0$.
        \item 若$z$轴是电荷分布的$n$次对称轴,
        \[ \rho\pare{\+vr} = \rho\pare{R\+vr},\quad R = \begin{pmatrix}
            \cos\theta & -\sin\theta & 0 \\
            \sin\theta & \cos\theta & 0 \\
            0 & 0 & 1
        \end{pmatrix},\quad \theta = \frac{2\pi}{n}. \]
        则$\+vp = p\+uz$. 考虑坐标变换
        \begin{align*}
            & x'_i = R_{ij}x_j \Rightarrow p'_i = R_{ij}p_j = p_i. \\
            & p_1 = \lambda_{11}p_1 + \lambda{12}p_2, \\
            & p_2 = \lambda_{21}p_1 + \lambda{22}p_2. \\
            &\Rightarrow \begin{pmatrix}
                1-\cos\theta & \sin\theta \\
                -\sin\theta & 1-\cos\theta
            \end{pmatrix}\begin{pmatrix}
                p_1 \\ p_2
            \end{pmatrix} = \begin{pmatrix}
                0 \\ 0
            \end{pmatrix}.
        \end{align*}
    \end{cenum}
\end{cenum}
\begin{ex}
    对于$\pm q$构成的系统,
    \[ \+vp = q\+vr_+ + \pare{-q}\+vr_- = q\+vl \]
    与原点选择无关.
\end{ex}
\begin{ex}
    点电荷的电偶极矩为$Q\+vd$, 和原点有关.
\end{ex}
\begin{ex}
    设半径$R$的球$V$将电荷分布包围,
    \begin{align*}
        \iiint_V \rd{V}\, \+vE\pare{\+vr} &= \iiint_V \rd{V}\,\rec{4\pi\epsilon_0} \iiint \rd{V'} \frac{\rho\pare{\+vr'}\+gr}{\+gr^3} \\
        &= -\iiint \rd{V'}\,\rec{4\pi\epsilon_0} \iiint_V\rd{V}\, \frac{\rho\pare{\+vr'}\pare{-\+v{\+gr}}}{\+gr^3} \\
        &= -\iiint \rd{V'}\,\text{在$V$内均匀的电荷$\rho\pare{\+vr'}$在$\+vr'$处激发的电场} \\
        &= -\iiint \rd{V'}\, \frac{\rho\pare{\+vr'}\+vr'}{3\epsilon_0} \\
        &= -\rec{3\epsilon_0}\iiint \rd{V'}\rho\pare{\+vr'} \+vr'.
    \end{align*}
\end{ex}

% subsubsection 电偶极子 (end)

\subsubsection{电四极子} % (fold)
\label{ssub:电四极子}

\begin{resume}
    若$\curl \+vF = 0$, 则$\grad \+vF = \+vF\grad$.
\end{resume}

当$Q = 0 = \+vp$, $\displaystyle \varphi\pare{\+vr} = \frac{\+ur\cdot \tensor{D}\cdot \+ur}{8\pi\epsilon_0 r^3}$, 从而(注意到$\+vr\grad = \grad \+vr$)
\begin{align*}
    & -\grad \varphi\pare{\+vr} = -\rec{8\pi\epsilon_0} \grad \frac{\+vr\cdot \tensor{D}\cdot \+vr}{r^5}, \\
    & \grad \frac{\+vr\cdot \tensor{D}\cdot \+vr}{r^5} = -\pare{\grad \rec{r^5}}\pare{\+vr\cdot \tensor{D}\cdot \+vr} - \rec{r^5}\grad \pare{\+vr\cdot\tensor{D}\cdot \+vr} \\
    & = \frac{5\pare{\+vr\cdot\tensor{D}\cdot\+vr}\+ur}{r^6} - \rec{r^5}\brac{\pare{\grad \+vr}\cdot \tensor{D}\cdot \+vr + \+vr\cdot \tensor{D}\cdot \grad \+vr} \\
    & = \frac{5\pare{\+ur\cdot \tensor{D}\cdot \+ur}\+ur}{r^4} - \frac{\tensor{D}\cdot \+ur + \+ur \cdot \tensor{D}}{r^4} \\
    & \Rightarrow \+vE\pare{\+vr} = \frac{5\pare{\+ur\cdot \tensor{D}\cdot \+ur}\+ur - 2\tensor{D}\cdot \+ur}{8\pi\epsilon_0 r^4},\quad \pare{r\gg R}.
\end{align*}
\begin{cenum}
    \item 一般$\tensor{D}$与原点的选择有关, 除非总电量以及电偶极矩皆为零.
    \begin{align*}
        \tensor{D}_1 &= \iiint \brac{3\+vr'_1 \+vr'_1 - r^2\tensor{I}}\,\rd{q} \\
        &= 3\iiint \pare{\+vr'_2 + \+vd}\pare{\+vr'_2 + \+vd}\,\rd{q} - \iiint \pare{\+vr'_2 + \+vd}^2\,\rd{q}\,\tensor{I} \\
        &= \iiint \pare{3\+vr'_2 \+vr'_2 - r'^2_2\tensor{I}}\,\rd{q} + \iiint \pare{3\+vd\+vd - d^2\tensor{I}}\,\rd{q} \\
        &\phantom{=\ } + 3\iiint \pare{\+vr'_2 \+vd + \+vd\+vr'_2}\,\rd{q} - 2\iiint \pare{\+vr'_2 \cdot \+vd}\,\rd{q}\, \tensor{I} \\
        \Rightarrow \tensor{D}_1 &= \tensor{D}_2 + Q\pare{3\+vd\+vd - d^2\tensor{I}} + 3\pare{\+vp_2 \+vd - \+vd\+vp_2}- 2\pare{\+vd\cdot \+vp_2}\tensor{I}.
    \end{align*}
    \item $\tensor{D}$是对称的迹零张量. 变换为$D'_{ij} = \lambda_{ik}\lambda_{gl}D_{kl}$, 即$D' = \lambda D\lambda^T$. 对于固定原点, 总可以选择坐标系(主轴系)使
    \[ D = \begin{pmatrix}
        D_1 & 0 & 0 \\
        0 & D_2 & 0 \\
        0 & 0 & D_3
    \end{pmatrix} = \diag\curb{D_1,D_2,D_3},\quad D_1+D_2+D_3 = 0. \]
    \item 对称性与$\tensor{D}$:
    \begin{cenum}
        \item 若电荷分布关于原点具有球对称性, 即$\rho\pare{\+vr} = \rho\pare{R \+vr}$, 其中$R\in \mathrm{SO}_3$, 则$\tensor{D} = 0$. 此时
        \[ D' = \lambda D \lambda^T = D \Rightarrow  \lambda D = D\lambda \Rightarrow D \propto I \xLongrightarrow{\trace D = 0} D = 0. \]
        \item 如果$z=0$为对称平面, 则$z$轴为主轴之一. 此时$D$有形式
        \[ D = \begin{pmatrix}
            * & * & 0 \\
            * & * & 0 \\
            0 & 0 & *
        \end{pmatrix}. \]
        \item 若$z$轴为电荷分布的$n\ge 2$次对称轴, 则$z$轴为主轴之一. 如果$n\ge 3$, 则与对称轴垂直的任意轴皆为主轴. 此时$D = \diag \curb{D_1,D_2,D_3}$. 考虑
        \[ R = \begin{pmatrix}
            \cos \theta & -\sin\theta & 0 \\
            \sin \theta & \cos\theta & 0 \\
            0 & 0 & 1
        \end{pmatrix},\quad \theta = \frac{2\pi}{n},\quad \begin{cases}
            n\ge 2,\quad \theta \in \lbr{0,\pi}, \\
            n\ge 3,\quad \theta \in \pare{0,\pi}.
        \end{cases} \]
        则
        \begin{align*}
            & D'_{ij} = D_{ij} = \lambda_{ik}\lambda_{jl}D_{kl}, \\
            & D_{13} = \lambda_{1k}\lambda_{3l}\lambda_{kl} = \lambda_{11}D_{13} + \lambda_{12}D_{23}, \\
            & D_{23} = \lambda_{2k}\lambda_{3l}\lambda{kl} = \lambda_{21}D_{13} + \lambda_{22}D_{23}. \\
            & \Rightarrow \begin{pmatrix}
                1 - \cos\theta & \sin\theta \\
                -\sin\theta & 1-\cos\theta
            \end{pmatrix}\begin{pmatrix}
                D_{13} \\
                D_{23}
            \end{pmatrix} = \begin{pmatrix}
                0\\ 0
            \end{pmatrix} \\
            & \Rightarrow D_{13} = D_{23} = 0. \\
            & D_{11} = \lambda_{11}^2 D_{11} + \lambda_{12}^2 D_{22} + 2\lambda_{11}\lambda_{12} D_{12}, \\
            & D_{12} = \lambda_{11}\lambda_{21}D_{11} + \lambda_{12}\lambda_{22}D_{22} + \lambda_{11}\lambda_{22}D_{12} + \lambda_{12}\lambda_{21}D_{12}, \\
            & \begin{cases}
                \pare{D_{11} - D_{22}}\sin^2\theta + 2D_{12}\sin\theta \cos\theta = 0, \\
                -\pare{D_{11} - D_{22}}\sin\theta \cos\theta + 2D_{12}\sin^2\theta = 0,
            \end{cases}\\ 
            & \Rightarrow \begin{pmatrix}
                1-\cos\theta & \sin\theta \\
                -\sin\theta & 1-\cos\theta
            \end{pmatrix}\begin{pmatrix}
                D_{11} - D_{22} \\
                D_{12}
            \end{pmatrix} = \begin{pmatrix}
                0 \\ 0
            \end{pmatrix}. \\
            & n\ge 3 \Rightarrow D_12 = 0,\quad D_{11} = D_{22} = 0.
        \end{align*}
    \end{cenum}
\end{cenum}
\begin{figure}[ht]
    \centering
    \begin{tikzpicture}[scale=0.7]
        \draw
        (0,0) node[circ] {$O$}
        (2,2) node[circ] {$+q$}
        (-2,2) node[circ] {$-q$}
        (-2,-2) node[circ] {$+q$}
        (2,-2) node[circ] {$-q$}
        (2,2) -- (2,-2) -- (-2,-2) -- (-2,2) -- (2,2)
        (1,0) node[right] {$a$}
        ;
        \draw [->] (0,0) -- (3,3) node {$x$};
        \draw [->] (0,0) -- (-3,3) node {$y$};
    \end{tikzpicture}
    \caption{}
    \label{fig:四极矩例}
\end{figure}
\begin{sample}
    \begin{ex}
        对于\cref{fig:四极矩例}中的系统,
        \begin{align*}
            \tensor{D} &= \sum_k q_k\pare{3\+vr_k \+vr_k - r_k^2 \tensor{I}} \\
            &= 6q\pare{\+vr_1 \+vr_1 - \+vr_2 \+vr_2} \\
            &= 3qa^2 \pare{\+ux\+ux - \+uy\+uy} \\
            &= 3qa^2 \begin{pmatrix}
                1 & & \\
                & -1 & \\
                & & 0
            \end{pmatrix}.
        \end{align*}
        如果直接以横纵轴为坐标, 则
        \begin{align*}
            \tensor{D} &= 6q\cdot \frac{a^2}{4}\brac{\pare{\+ux'\+ux' + \+uy'\+uy' + \+ux'\+uy' + \+uy'\+ux'} - \pare{\+ux'\+ux' + \+uy'\+uy' - \+ux'\+uy' - \+uy'\+ux'}} \\
            &= 3qa^2 \pare{\+ux'\+uy' + \+uy'\+ux'}.
        \end{align*}
    \end{ex}
\end{sample}
\begin{sample}
    \begin{ex}
        对于在椭球$\displaystyle \frac{x_1^2}{a_1^2} + \frac{x_2^2}{a_2^2} + \frac{x_3^2}{a_3^2} \le 1$内均匀分布的电荷, 总电量$Q$, 则由对称性知三个轴皆为主轴, 从而
        \begin{align*}
            Q &= \iiint \rho\,\rd{V'} = \frac{4\pi}{3} \rho a_1a_2a_3 \Rightarrow Q = \frac{4\pi a_1a_2a_3}{3}\rho. \\
            D_{11} &= \iiint \pare{3x'^2_1 - r'^2}\,\rd{q} = \frac{3Q}{4\pi a_1a_2a_3} \iiint \pare{2x'^2_1 - x'^2_2 - x'^2_3}\,\rd{x'_1}\,\rd{x'_2}\,\rd{x'_3} \\
            &= \frac{3Q}{4\pi}\iiint_{r\le 1} \pare{2a_1^2x_1^2 - a_2^2 x_2^2 - a_3^2 x_3^3} \,\rd{x_1}\,\rd{x_2}\,\rd{x_3},\\
            \text{其中} &\iiint x_1^2\,\rd{V} = \iiint x_2^2 \,\rd{V} = \iiint x_3^2\,\rd{V}\\ &= \rec{3} \iiint \pare{x_1^2 + x_2^2 + x_3^2}\,\rd{V} = \rec{3}\cdot 4\pi \int_0^1 r^4\,\rd{r} = \frac{4\pi}{15}, \\
            D_{11} &= \frac{Q}{5} \pare{2a_1^2 - a_2^2 - a_3^2}, \\
            D_{22} &= \frac{Q}{5} \pare{2a_2^2 - a_3^2 - a_1^2}, \\
            D_{33} &= \frac{Q}{5} \pare{2a_3^2 - a_1^2 - a_2^2}.
        \end{align*}
        如果具有关于$z$的旋转对称性, $a_1 = a_2 = a$, $a_3 = b$, 则
        \[ D_{33} = \frac{2Q}{5}\pare{b^2 - a^2} = -2D_{11} = -2D_{22}. \]
        对于雪茄形的原子核, $D_{33} > 0$. 对于南瓜形的原子核, $D_{33} < 0$.
    \end{ex}
\end{sample}

% subsubsection 电四极子 (end)

\subsubsection{外场中的小带电体} % (fold)
\label{ssub:外场中的小带电体}

设外场$\varphi_e\pare{\+vr}$, $\+vE_e\pare{\+vr}$. 设带电体位于$\+vr$,
\begin{cenum}
    \item 电势能:
    \begin{align*}
         U &= \iiint \rd{V'}\,\rho\pare{\+vr'}\varphi_e\pare{\+vr + \+vr'} = \iiint \varphi_e\pare{\+vr+ \+vr'}\,\rd{q}, \\
         \varphi_e\pare{\+vr + \+vr'} &= e^{\+vr'\cdot \grad}\varphi_e\pare{\+vr} = \brac{1+\+vr'\cdot \grad + \rec{2!}\+vr'\+vr':\grad\grad + \cdots} \varphi_e\pare{\+vr} \\
         &= \brac{1+\+vr'\cdot \grad + \rec{r}\pare{3\+vr'\+vr' - r'^2\tensor{I}}::\grad\grad + \cdot}\varphi_e\pare{\+vr}.
    \end{align*}
    若假定外场满足Laplace方程, 即球内无激发外场的电势,
    \[ \tensor{I} : \grad\grad \varphi_e\pare{\+vr} = \laplacian \varphi_e\pare{\+vr} = 0 \]
    可以保证上面的变换有效.
    \begin{align*}
        U &= U^{\pare{0}} + U^{\pare{1}} + U^{\pare{2}} \\
        &= Q\varphi_e\pare{\+vr} + \+vp\cdot \grad \varphi_e\pare{\+vr} + \rec{6}\tensor{D}:\grad\grad\varphi_e\pare{\+vr} + \cdots \\
        &= Q\varphi_e\pare{\+vr} - \+vp\cdot \+vE_e\pare{\+vr} - \rec{6}\tensor{D}:\grad \+vE_e\pare{\+vr}.
    \end{align*}
    \item 力: 由$\laplacian \+vE_e\pare{\+vr} = -\laplacian \grad \varphi_e\pare{\+vr} = -\grad\pare{\laplacian \varphi_e} = 0$,
    \begin{align*}
        \+vF &= \iiint \+vE_e\pare{\+vr + \+vr'}\,\rd{q} \\
        &= Q\+vE_e\pare{\+vr} + \+vp\+v\cdot\grad \+vE_e\pare{\+vr} + \rec{6}\pare{\tensor{D}\+v: \grad\grad} \+vE_e\pare{\+vr} + \cdots + \cdots \\
        &=-\grad U^{\pare{0}} - \grad U^{\pare{1}} - \grad U^{\pare{2}} -\cdots.\\
        -\grad U^{\pare{1}} &= +\grad{\+vp\cdot \+vE_e} = \pare{\grad \+vp}\cdot \+vE_e + \pare{\grad \+vE_e}\cdot \+vp = \+vp\+v\cdot\grad \+vE_e.\\
        -\grad U^{\pare{2}} &= +\rec{6}\grad\pare{\tensor{D}\+v:\grad \+vE_e} = \rec{6}\tensor{D}\+v:\grad\grad \+vE_e.
    \end{align*}
    其中$\grad$和$\+vE$可交换.
    \item 力矩:
    \begin{align*}
        \+v\tau &= \iiint \brac{\+vr'\times \+vE_e\pare{\+vr+\+vr'}}\,\rd{q} \\
        &= \iiint \curb{\+vr'\times \+vE_e\pare{\+vr}}\,\rd{q} + \iiint \curb{\+vr'\times \brac{\pare{\+vr'\+v\cdot\grad}\+vE_e\pare{\+vr}}}\,rd{q} \\
        &= \+vp\times \+vE_e\pare{\+vr} + \brac{\iiint \+vr'\+vr'\,\rd{q}\+v\cdot \grad} \times \+vE_e\pare{\+vr'}.
    \end{align*}
    注意到$\pare{\tensor{I}\+v\cdot\grad}\times \+vE_e = \curl \+vE_e = 0$, 从而
    \begin{align*}
        \+v\tau &= \+vp\times \+vE_e\pare{\+vr}  + \brac{\rec{3}\iiint \pare{3\+vr'\+vr' - r'^2 \tensor{I}}\,\rd{q} \+v\cdot \grad} \+vE \\
        &= \+vp\times \+vE_e\pare{\+vr} + \rec{3}\pare{\tensor{D}\+v\cdot \grad}\times \+vE_e\pare{\+vr}.
    \end{align*}
\end{cenum}

% subsubsection 外场中的小带电体 (end)

\subsubsection{球内和球外展开} % (fold)
\label{ssub:球内和球外展开}

若电荷分布完全处于某球内部, 则可将球外电势展开, 若电荷分布完全位于某球外部, 则可将球内电势展开.
\begin{align*}
    \+gr &= \abs{\+vr - \+vr'} = \sqrt{r^2 - 2rr'\+ur\cdot \+ur' + r'^2}, \\
    \+ur\cdot \+ur' &= \cos\theta \cos\theta' + \sin\theta\sin\theta' \cos\pare{\varphi - \varphi'}.
\end{align*}
如下结论成立:
\begin{align*}
    & \rec{\sqrt{1-2xt+t^2}} = \sum_{l=0}^\infty t^lP_l\pare{x},\quad 0<t<1,\quad \abs{x} < 1, \\
    & \rec{\+gr} = \rec{\sqrt{r^2 - 2rr'\+ur\cdot\+ur' + r'^2}} = \rec{r_> \sqrt{\displaystyle 1 - 2\frac{r_<}{r_>}\pare{\+ur\cdot \+ur'} + \pare{\frac{r_<}{r_>}}^2}} \\
    &= \rec{\+gr} = \rec{r_>}\sum_{l=0}^\infty \sum_{l=0}^\infty \pare{\frac{r_<}{r_>}}^l P_l\pare{\+ur\cdot \+ur'}. \\
    & \resumath{P_l\pare{\+ur\cdot \+ur'} = \frac{4\pi}{2l+1}\sum_{n=-l}^l Y_{lm}\pare{\+ur}\cdot Y_{lm}^*\pare{\+ur'}.}\\
    & \Rightarrow \rec{\+gr} = \rec{r_>}\sum_{l=0}^\infty \frac{4\pi}{2l+1}\pare{\frac{r_<}{r_>}}^{l} \sum_{m=-l}^l Y^*_{lm}\pare{\+ur'}Y_lm\pare{\+ur}.
\end{align*}

\paragraph{球内展开} % (fold)
\label{par:球内展开}

此时$r'=r_>$, $r = r_<$, 故
\[ \varphi\pare{\+vr} = \rec{4\pi\epsilon_0}\sum_{l=0}^\infty \sum_{m=-l}^l A_{lm}r^l Y_{lm}\pare{\+ur}. \]
其中
\[ A_{lm} = \frac{4\pi}{2l+1} \iiint \frac{Y_{lm}^*\pare{\+ur'}}{r'^{l+1}}\,\rd{q}. \]
若恰好有$\rho = \rho\pare{r,\theta}$, 则
\[ \varphi\pare{r,\theta} = \rec{4\pi\epsilon_0} \sum_{l=0}^\infty A_l r^l P_l\pare{\cos\theta}. \]
其中
\[ A_l = \iiint \frac{P_l\pare{\cos\theta'}}{r'^{l+1}}\,\rd{q}. \]

% paragraph 球内展开 (end)

\paragraph{球外展开} % (fold)
\label{par:球外展开}

此时$r' = r_<$, $r = r_>$, 故
\[ \varphi\pare{\+vr} = \rec{4\pi\epsilon_0}\sum_{l=0}^\infty \sum_{m=-l}^l \frac{B_{lm}}{r^{l+1}}Y_{lm}\pare{\+ur}. \]
其中
\[ B_{lm} = \frac{4\pi}{2l+1} \iiint r'^l Y^*_{lm}\pare{\+ur'}\,\rd{q}. \]
若恰好有$\rho = \rho\pare{r,\theta}$, 则
若恰好有$\rho = \rho\pare{r,\theta}$, 则
\[ \varphi\pare{r,\theta} = \rec{4\pi\epsilon_0} \sum_{l=0}^\infty \frac{B_l}{r^{l+1}} P_l\pare{\cos\theta}. \]
其中
\[ B_l = \iiint r'^l P_l\pare{\cos\theta'}\,\rd{q}. \]
注意到这和之前的多极展开的关系,
\[ Q\leftrightarrow Y_{0,0},\quad \+vp\leftrightarrow Y_{1,0}, Y_{1,\pm 1}, \quad \tensor{D} \leftrightarrow Y_{2,0}, Y_{2,\pm 1}, Y_{2,\pm 2}. \]

\begin{sample}
    \begin{ex}
        若球面上有电荷分布, 且电势满足
        \[ \varphi_0 = \frac{Q}{4\pi\epsilon_0 R} \sin\theta\cos\theta\cos\phi. \]
        由$\theta$的项为$\sin \theta\cos \theta$知$l=2$, $\phi$的项为$\cos \phi$知$m=\pm 1$. 由
        \[ Y_{2,\pm 1} = \mp \sqrt{\frac{15}{8\pi}} \sin\theta \cos\theta e^{\pm im\phi} \]
        可得
        \[ \varphi_0 = \frac{Q}{4\pi\epsilon_0 R}\sqrt{\frac{2\pi}{15}}\pare{Y_{2,-1} - Y_{2,1}}. \]
        从而
        \[ \varphi = \begin{cases}
            \displaystyle \frac{Qr^2}{4\pi\epsilon_0 R^3}\sqrt{\frac{2\pi}{15}}\pare{Y_{2,-1} - Y_{2,1}}, & r<R, \\[.5em]
            \displaystyle \frac{QR^2}{4\pi\epsilon_0 r^3}\sqrt{\frac{2\pi}{15}}\pare{Y_{2,-1} - Y_{2,1}}, & r>R.
        \end{cases} \]
    \end{ex}
\end{sample}

\begin{sample}
    \begin{ex}
        若电荷分布具有球对称性, $\rho=\rho\pare{r}$, 则每一点处的电势可分为球内贡献和球外贡献.
        \begin{align*}
            \varphi &= \rec{4\pi\epsilon_0}\sum_{l=0}^\infty \pare{A_l r^l + \frac{B_l}{r^{l+1}}}P_l\pare{\cos\theta}, \\
            A_l &= \iiint \frac{P_l\pare{\cos\theta}}{r'^{l+1}}\,\rd{q} \\
            &= \iiint \frac{P_l\pare{\cos\theta'}}{r'^{l+1}}\,\rd{q} \\
            &= 2\pi \int_r^\infty \frac{\rho\pare{r'}}{r'^{l-1}}\,\rd{r'}\int_{-1}^1 P_l\pare{\cos\theta'}\,\rd{\cos\theta'} \\
            &= 4\pi \delta_{l,0} \int_r^\infty \frac{\rho\pare{r'}}{r'^{l-1}}\,\rd{r'}. \\
            B_l &= \iiint r'^l P_l\pare{\cos\theta'}\,\rd{q} \\
            &= \int_0^r r'^{l+2}\rho\pare{r'}\,\rd{r'} \cdot 4\pi\delta_{l,0} \\
            &= 4\pi \delta_{l,0} \int_0^r \rho\pare{r'}r'^2\,\rd{r}.\\
            \varphi &= \rec{\epsilon_0}\brac{\rec{r} \int_0^r \rho\pare{r'}r'^2\,\rd{r} + \int_r^\infty\rho\pare{r'}r'\,\rd{r'}}.
        \end{align*}
    \end{ex}
\end{sample}
\begin{sample}
    \begin{ex}
        设半径为$R$的球面上有电荷分布$\sigma = \sigma_0 \cos\theta$, 则
        \begin{align*}
            \text{(球内)}\quad A_l &= \iiint \frac{P_l\pare{\cos\theta'}}{r'^{l+1}}\,\rd{q}, \\
            \sigma\,\rd{S'} &= \sigma_0 P_1\pare{\cos\theta'} R^2\,\rd{\cos\theta'}\,\rd{\phi'}, \\
            A_l &= \frac{\sigma_0}{R^{l-1}}\cdot 2\pi \cdot \frac{2}{2l+1}\delta_{l,1}.\\
            \text{(球外)}\quad B_l &= \iiint r'^l P_l\pare{\cos\theta'}\,\rd{q} = \sigma_0 R^{l+2}\cdot \frac{4\pi}{2l+1}\delta_{l,1}. \\
            \varphi &= \begin{cases}
                \displaystyle \frac{\sigma_0}{3\epsilon_0} r\cos\theta, & r<R, \\
                \displaystyle \frac{\sigma_0}{3\epsilon_0} \frac{R^3}{r^2}\cos\theta, & r>R.
            \end{cases}
        \end{align*}
    \end{ex}
\end{sample}
\begin{sample}
    \begin{ex}
        设电荷均匀分布在圆环上, 求空间中任何一点的电势,
        \begin{align*}
            A_l &= \iiint \frac{P_l\pare{\cos\theta'}}{r'^{l+1}}\,\rd{q} = \frac{Q}{R^{l+1}}P_l\pare{0}, \\
            B_l &= \iiint r'^l P_l\pare{\cos\theta'}\,\rd{q} = QR^l P_l\pare{0}, \\
            \varphi &= \begin{cases}
                \displaystyle \frac{Q}{4\pi\epsilon_0 R} \sum_l \pare{\frac{r}{R}}^l P_l\pare{0} P_l\pare{\cos\theta}, & r<R \\[.5em]
                \displaystyle \frac{Q}{4\pi\epsilon_0 r} \sum_l \pare{\frac{R}{r}}^{l}P_l\pare{0} P_l\pare{\cos\theta}, & r>R.
            \end{cases}
        \end{align*}
        或者注意到轴线上的电势为
        \begin{align*}
            & \varphi\pare{r,\theta = 0,\pi} = \frac{Q}{4\pi\epsilon_0 \sqrt{R^2+r^2}} = \begin{cases}
                \displaystyle \frac{Q}{4\pi\epsilon_0 R} \rec{\sqrt{1+\pare{r/R}^2}}, & r<R, \\
                \displaystyle \frac{Q}{4\pi\epsilon_0 r} \rec{\sqrt{1+\pare{R/r}^2}}, & r>R.
            \end{cases}\\
            & \Rightarrow \varphi\pare{r,\theta = 0,\pi} = \begin{cases}
                \displaystyle \frac{Q}{4\pi \epsilon_0 R} \sum_{l=0}^\infty \pare{\frac{r}{R}}^l P_l\pare{0}, & r<R, \\
                \displaystyle \frac{Q}{4\pi \epsilon_0 r} \sum_{l=0}^\infty \pare{\frac{R}{r}}^l P_l\pare{0}, & r>R.
            \end{cases}
        \end{align*}
        注意到
        \[ P_{2n}\pare{0} = \pare{-1}^n \frac{\pare{2n-1}!!}{\pare{2n}!!}, \]
        可得完整的展开式.
    \end{ex}
\end{sample}

% paragraph 球外展开 (end)

% subsubsection 球内和球外展开 (end)

% subsection 电多极展开 (end)

% section 静电场 (end)

\end{document}
