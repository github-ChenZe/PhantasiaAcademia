\documentclass[hidelinks]{ctexart}

\usepackage{van-de-la-illinoise}
\usepackage[paper=b5paper,top=.3in,left=.9in,right=.9in,bottom=.3in]{geometry}
\usepackage{calc}
\pagenumbering{gobble}
\setlength{\parindent}{0pt}
\sisetup{inter-unit-product=\ensuremath{{}\cdot{}}}
\usepackage{van-le-trompe-loeil}
\usepackage{makecell}

\usepackage{stackengine}
\stackMath
\usepackage{scalerel}
\usepackage[outline]{contour}

\newdimen\indexlen
\def\newprobheader#1{%
\def\probindex{#1}
\setlength\indexlen{\widthof{\textbf{\probindex}}}
\hskip\dimexpr-\indexlen-1em\relax
\textbf{\probindex}\hskip1em\relax
}
\def\newprob#1{%
\newprobheader{#1}%
\def\newprob##1{%
\probsep%
\newprobheader{##1}%
}%
}
\def\probsep{\vskip1em\relax{\color{gray}\dotfill}\vskip1em\relax}

\newlength\thisletterwidth
\newlength\gletterwidth
\newcommand{\leftrightharpoonup}[1]{%
{\ooalign{$\scriptstyle\leftharpoonup$\cr%\kern\dimexpr\thisletterwidth-\gletterwidth\relax
$\scriptstyle\rightharpoonup$\cr}}\relax%
}
\def\tensor#1{\settowidth\thisletterwidth{$\mathbf{#1}$}\settowidth\gletterwidth{$\mathbf{g}$}\stackon[-0.1ex]{\mathbf{#1}}{\boldsymbol{\leftrightharpoonup{#1}}}  }
\def\onedot{$\mathsurround0pt\ldotp$}
\def\cddot{% two dots stacked vertically
  \mathbin{\vcenter{\baselineskip.67ex
    \hbox{\onedot}\hbox{\onedot}}%
}}%

\begin{document}

\newprob{2.1}%
$\displaystyle \left.\+DnDE\right\vert_{r=R} = \left.\+DnD{\pare{\sqrt{E_n^2 + E_\parallel^2}}}\right\vert_{r=R} = \left. \rec{\sqrt{E_n^2 + {E_\parallel^2}}}\pare{E_n \+DnD{E_n} + {E_\parallel}\+DnD{E_\parallel}} \right\vert_{r=R} = \left.\+DnD{E_n}\right|_{r=R}$.\\
$\displaystyle \left.\laplacian \varphi\right\vert_{r=R} = \left.\rec{r^2}\+DrD{}\pare{r^2\+DrD{\varphi}} + \rec{r^2\sin\theta}\+D\theta D{}\pare{\sin\theta\+D\theta D\varphi} + \rec{r^2\sin^2\theta} \+D{\phi^2}D{^2\varphi}\right|_{r=R} = 0$.\\
但$r=R$时$\varphi=\const$, 故对$\theta$和$\phi$的偏导数为零, 从而\\
$\displaystyle \left.\rec{r^2}\+DrD{}\pare{r^2\+DrD{\varphi}}\right|_{r=R} = \left.\frac{2}{r}\+DrD{\varphi} + \+D{r^2}D{^2\varphi}\right|_{r=R} = \left.\frac{2E_r}{R} + \+DrD{E_r}\right\vert_{r=R} = 0 \Rightarrow  \rec{E}\+DnD{E} = -\frac{2}{R}$.
\newprob{2.2}%
假设电荷$\pm q$分别位于$\+vr_1$和$\+vr_2$, 则$\rho\pare{\+vr} = q\delta\pare{\+vr-\+vr_1} - q\delta\pare{\+vr - \+vr_2}$.\\
令$\+vr_1 \rightarrow \+vr_0$, $\+vr_2 \rightarrow \+vr_0$, 从而$\+vr_2 - \+vr_1 \rightarrow 0$, 则有($\+v{\+gr}$表示源点到场点的位矢)\\
$\displaystyle \rho\pare{\+vr} = q\pare{\+vr_1 - \+vr_2}\+v\cdot \grad_0\delta\pare{\+vr - \+vr_0} = -q\pare{\+vr_1 - \+vr_2}\+v\cdot \grad\delta\pare{\+vr-\+vr_0} = -\+vp\+v\cdot \grad\delta\pare{\+vr-\+vr_0}$.\\
$\displaystyle \varphi = \rec{4\pi\epsilon_0} \iiint \frac{\rho}{\+gr}\,\rd{V} = \frac{-\+vp}{4\pi\epsilon_0} \cdot \iiint \frac{\grad'\delta\pare{\+vr' - \+vr_0}}{\+gr}\,\rd{V}$\\
$\displaystyle = \frac{-\+vp}{4\pi\epsilon_0} \cdot \iiint \pare{\grad' \frac{\delta\pare{\+vr' - \+vr_0}}{\+gr} - \delta\pare{\+vr' - \+vr_0}\grad'\rec{\+gr}}\,\rd{V}$\\
$\displaystyle = \frac{-\+vp}{4\pi\epsilon_0} \cdot \brac{ \oiint \rd{\+v\sigma}\, \frac{\delta\pare{\+vr' - \+vr_0}}{\+gr} -\iiint \delta\pare{\+vr' - \+vr_0}\grad'\rec{\+gr}\,\rd{V}}$\\
$\displaystyle = \frac{\+vp}{4\pi\epsilon_0}\grad_0 \rec{\abs{\+vr - \+vr_0}} = \boxed{\frac{\+vp}{4\pi\epsilon_0} \cdot \frac{\+u{\+gr}}{\+gr^2}.}$

\end{document}
