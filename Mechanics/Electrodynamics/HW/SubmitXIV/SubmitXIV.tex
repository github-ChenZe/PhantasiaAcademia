\documentclass[hidelinks]{ctexart}

\usepackage{van-de-la-illinoise}
\usepackage[paper=b5paper,top=.3in,left=.9in,right=.9in,bottom=.3in]{geometry}
\usepackage{calc}
\pagenumbering{gobble}
\setlength{\parindent}{0pt}
\sisetup{inter-unit-product=\ensuremath{{}\cdot{}}}
\usepackage{van-le-trompe-loeil}
\usetikzlibrary{quotes,angles}
\usetikzlibrary{arrows.meta}
\usepackage{makecell}

\usepackage{stackengine}
\stackMath
\usepackage{scalerel}
\usepackage[outline]{contour}

\newdimen\indexlen
\def\newprobheader#1{%
\def\probindex{#1}
\setlength\indexlen{\widthof{\textbf{\probindex}}}
\hskip\dimexpr-\indexlen-1em\relax
\textbf{\probindex}\hskip1em\relax
}
\def\newprob#1{%
\newprobheader{#1}%
\def\newprob##1{%
\probsep%
\newprobheader{##1}%
}%
}
\def\probsep{\vskip1em\relax{\color{gray}\dotfill}\vskip1em\relax}

\newlength\thisletterwidth
\newlength\gletterwidth
\newcommand{\leftrightharpoonup}[1]{%
{\ooalign{$\scriptstyle\leftharpoonup$\cr%\kern\dimexpr\thisletterwidth-\gletterwidth\relax
$\scriptstyle\rightharpoonup$\cr}}\relax%
}
\def\tensor#1{\settowidth\thisletterwidth{$\mathbf{#1}$}\settowidth\gletterwidth{$\mathbf{g}$}\stackon[-0.1ex]{\mathbf{#1}}{\boldsymbol{\leftrightharpoonup{#1}}}  }
\def\onedot{$\mathsurround0pt\ldotp$}
\def\cddot{% two dots stacked vertically
  \mathbin{\vcenter{\baselineskip.67ex
    \hbox{\onedot}\hbox{\onedot}}%
}}%

    \tikzset{
    partial ellipse/.style args={#1:#2:#3}{
        insert path={+ (#1:#3) arc (#1:#2:#3)}
    }}

\begin{document}

\newprob{6.1}%
$\displaystyle T = \frac{mc^2}{\displaystyle \sqrt{1-\beta^2}} - mc^2 = \frac{mc^2}{\displaystyle \sqrt{1-\pare{1/n}^2}} - mc^2 = \boxed{\SI{30.6}{\mega\eV}.}$
\newprob{6.2 (1)}%
记(和课堂上记号一致)$\+vn^* = \+u{\+gr} - \+v\beta$, 加速度场为
\[ \+vE = \frac{e}{4\pi\epsilon_0 c^2 \+gr^*} \frac{\+u{\+gr}^* \times \pare{\+vn^*\times \+va^*}}{\pare{\+u{\+gr}^* \cdot \+vn^*}^3},\quad c\+vB = \+u{\+gr}^* \times \+vE. \]
故
\[ c\+vB\times \+u{\+gr}^* = \pare{\+u{\+gr}^*\times \+vE}\times \+u{\+gr}^* = \pare{\+u{\+gr}^*\cdot \+u{\+gr}^*}\+vE - \pare{\+u{\+gr}^*\cdot \+vE}\+u{\+gr}^* = \+vE. \]
\par
\newprobheader{(2)}%
$\displaystyle \beta \ll 1 \Rightarrow \+vn^* \approx \+u{\+gr}$, $\pare{\+u{\+gr}\cdot \+vn^*}^3 \approx 1$.
\begin{align*}
    & \+vE \approx \frac{e}{4\pi\epsilon_0 c^2 \+gr^*}\+u{\+gr}^*\times \pare{\+u{\+gr}^*\times \+va^*} = \frac{e}{4\pi\epsilon_0 c^2 \+gr^*}\brac{\pare{\+u{\+gr}^* \cdot \+va^*}\+u{\+gr}^* - \+va^*}\\
    & \Rightarrow \+vB = \+/\+u{\+gr}^*/c/\times \+vE = \frac{\mu_0 e}{4\pi c\+gr^*}\+va^*\times \+u{\+gr}^*,\quad \+vE = c\+vB\times \+u{\+gr}.
\end{align*}
\newprob{6.3 (1)}%
由$\displaystyle \half mv^2 + V\pare{r} = \half mv_0^2 = V\pare{r\+_min_}$,
\begin{align*}
    W &= \int_{-\infty}^{\infty}\rd{t}\,P = 2\int_{r\+_min_}^\infty\rd{r}\, P = 2\cdot \frac{\mu_0 q^2}{6\pi c m^2} \int_{r\+_min_}^\infty \pare{\+drdV}^2 \cdot \frac{\rd{r}}{v}\\ &= \frac{q^2}{3\pi\epsilon_0 m^2 c^3}\sqrt{\frac{m}{2}}\int_{r\+_min_}^\infty \pare{\+drdV}^2 \frac{\rd{r}}{\sqrt{V\pare{r\+_min_} - V\pare{r}}}.
\end{align*}
\par
\newprobheader{(2)}%
记$u = 1/r$, $u\+_max_ = 1/r\+_min_$, 有
\begin{align*}
    & \int_{r\+_min_}^\infty \rec{r^4}\frac{\rd{r}}{\sqrt{\displaystyle \rec{r\+_min_} - \rec{r}}} = \int_0^{u\+_max_} \frac{u^2\,\rd{u}}{\sqrt{u\+_max_ - u}} = \int_0^1 \frac{u^2\,\rd{u}}{\sqrt{1-u}}\cdot u\+_max_^{5/2} = \frac{16}{15}u\+_max_^{5/2}. \\
    & W = \frac{q^2}{3\pi\epsilon_0 m^2 c^3}\sqrt{\frac{m}{2}}\pare{\frac{qQ}{4\pi\epsilon_0}}^{3/2} \times \frac{16}{15}\pare{\rec{r\+_min_}}^{5/2} \\
    & = \frac{q}{m^2 c^3 Q} \sqrt{\frac{m}{2}} V\pare{r\+_min_}^{5/2}\cdot \frac{16}{15}\cdot \frac{4}{3} = \frac{8qmv_0^5}{45c^3Q}.
\end{align*}
\newprob{6.4}%
记$\displaystyle D = \+dtd{}$, 则$\displaystyle \ddot{\+vr} = -\omega_0^2 \+vr + \frac{q\+vv\times \+vB}{m}$可写为
\begin{align*}
    & \begin{pmatrix}
        D^2 + \omega_0^2 & -\omega_c D \\
        \omega_c D       & D^2 + \omega_0^2
    \end{pmatrix} \begin{pmatrix}
        x \\ y
    \end{pmatrix} = \begin{pmatrix}
        0 \\ 0
    \end{pmatrix} \Rightarrow \begin{vmatrix}
        D^2 + \omega_0^2 & -\omega_c D \\
        \omega_c D       & D^2 + \omega_0^2
    \end{vmatrix} = \pare{D^2 + \omega_0}^2 + \omega_c^2 D^2 = 0 \\
    & \Rightarrow D\approx -i\pare{\omega_0 \pm 0.5 \omega_c}.
\end{align*}
相应的本征矢为
\[ D = -i\pare{\omega_0 + 0.5\omega_c} \Rightarrow \begin{pmatrix}
    x_0 \\ y_0
\end{pmatrix} = C \begin{pmatrix}
    1 \\ -i
\end{pmatrix},\quad D = -i\pare{\omega_0 - 0.5\omega_c} \Rightarrow \begin{pmatrix}
    x_0 \\ y_0
\end{pmatrix} = C \begin{pmatrix}
    1 \\ i
\end{pmatrix}. \]
而$\+uz$分量显然满足
\[ \ddot{z} = -\omega_0^2 z \Rightarrow z = Ce^{-i\omega_0 t}. \]
从而
\[ \boxed{\+vr = C_1 e^{-i\omega_0 t}\+uz + C_2 \pare{\+ux - i\+uy}e^{-i\pare{\omega_0 + 0.5\omega_c}t} + C_3\pare{\+ux + i\+uy}e^{-i\pare{\omega_0  - 0.5\omega_c }t}. } \]
故沿着磁场方向有频率为$\omega_0 + 0.5\omega_c$的左旋圆偏振(潘氏定义)和频率为$\omega_0 - 0.5\omega_c$的右旋圆偏振. 垂直于磁场方向有频率为$\omega_0, \omega_0 \pm 0.5\omega_c$的线偏振.
\newprob{6.5}%
由
\[ \int_{-\infty}^\infty \frac{\rd{x}}{\pare{a^2+x^2}} = \frac{\pi}{2a^3} \]
可得
\begin{align*}
    W &= \frac{\mu_0 q^2}{6\pi c} \int_L \+/\rd{x}/v/\, \pare{\frac{qq_1}{4\pi\epsilon_0 m \pare{a^2 + x^2}}}^2 = \frac{\mu_0 q^2}{6\pi c}\cdot \frac{q^2 q_1^2}{16\pi^2\epsilon_0^2 m^2 v}\cdot \frac{\pi}{2a^3} = \boxed{\frac{q^4 q_1^2}{192 \pi^2 \epsilon_0^3 m^2 c^3 a^3 v}.}
\end{align*}
\newprob{6.6}%
参考6.2的结论, 记$r$为二粒子相对位矢, $v$为二粒子相对速度,
\begin{align*}
    \+vB &= \frac{\mu_0}{4\pi c R} \pare{e_1 \+va_1 + e_2 \+va_2} \times \+u{R} \\
    &= \frac{\mu_0}{4\pi cR}\pare{\frac{e_1}{m_1} - \frac{e_2}{m_2}}\+ur\times \+uR\cdot \frac{e_1 e_2}{4\pi\epsilon_0 r^2},\\
    \+d\Omega dP &= \frac{c}{\mu_0}\abs{B}^2 R^2 = \frac{c}{\mu_0}\pare{\frac{\mu_0}{4\pi c}}^2 \pare{\frac{e_1}{m_1} - \frac{e_2}{m_2}}^2 \pare{\frac{e_1e_2}{4\pi\epsilon_0}}^2 \rec{r^4}\sin^2\theta.
\end{align*}
由$\displaystyle \iint\rd{\Omega}\, \sin^2\theta = \frac{8\pi}{3}$,
\[ P = \frac{8\pi}{3}\frac{c}{\mu_0}\pare{\frac{\mu_0}{4\pi c}}^2 \pare{\frac{e_1}{m_1} - \frac{e_2}{m_2}}^2 \pare{\frac{e_1e_2}{4\pi\epsilon_0}}^2 \rec{r^4}. \]
与6.3类似,
\[ 2\int_{r\+_min_}^\infty \frac{\rd{r}}{vr^4} = 2\times \frac{\sqrt{\displaystyle \+/\overbar{m}/2/}}{\sqrt{\displaystyle \+/e_1e_2/4\pi\epsilon_0/}} \times \frac{16}{15}\times \pare{\rec{r\+_min_}}^{5/2}. \]
从而
\begin{align*}
    W &= \int \rd{t}\, P = \frac{8\pi}{3}\frac{c}{\mu_0}\pare{\frac{\mu_0}{4\pi c}}^2 \pare{\frac{e_1}{m_1} - \frac{e_2}{m_2}}^2 \pare{\frac{e_1e_2}{4\pi\epsilon_0}}^2 \times 2\int_{r\+_min_}^\infty \frac{\rd{r}}{vr^4} \\
    &= \frac{8\pi}{3}\frac{c}{\mu_0}\pare{\frac{\mu_0}{4\pi c}}^2\pare{\frac{e_1}{m_1} - \frac{e_2}{m_2}}^2 \times \frac{32}{15}\sqrt{\frac{\overbar{m}}{2}}\frac{4\pi\epsilon_0}{e_1e_2} \pare{\frac{e_1e_2}{4\pi\epsilon_0 r\+_min_}}^{5/2} \\
    &= \frac{8\pi}{3}\frac{c}{\mu_0}\pare{\frac{\mu_0}{4\pi c}}^2\pare{\frac{e_1}{m_1} - \frac{e_2}{m_2}}^2 \times \frac{32}{15}\sqrt{\frac{\overbar{m}}{2}}\frac{4\pi\epsilon_0}{e_1e_2} \pare{\frac{\overbar{m}}{2}v_0^2}^{5/2} \\
    &= \frac{8\overbar{m}^3 v_0^5}{45c^3e_1e_2}\pare{\frac{e_1}{m_1} - \frac{e_2}{m_2}}^2.
\end{align*}
\newprob{6.7}%
辐射角分布
\[ \+d\Omega dP \propto g\pare{\theta} = \frac{\sin^2\theta}{\pare{1-\beta \cos\theta}^5} = \frac{1-u^2}{\pare{1-\beta u}^5}. \]
最值出现在
\[ \+dudg = -\frac{2u}{\pare{1-\beta u}^5} + \frac{5\beta\pare{1-u^2}}{\pare{1-\beta u}^6} = 0 \]
处, 即(只有一根满足$u\in \brac{-1,1}$)
\[ u = \boxed{\cos\theta = \frac{\sqrt{1+15\beta^2} - 1}{3\beta}.} \]
\newprob{6.8 (1)}%
$\displaystyle P_\perp = \frac{\mu_0 q^2}{6\pi c}a_\perp^2 \gamma^4 = \frac{\mu_0 q^2}{6\pi c}\pare{\frac{vqB}{\gamma m}}^2\gamma^4 = \frac{\mu_0 q^4 B^2}{6\pi cm^2} \cdot c^2 \cdot \frac{v^2/c^2}{1-v^2/c^2} = \boxed{\frac{\gamma^2 q^4 B^2}{6\pi \epsilon_0 m^2 c}\pare{1-\rec{\gamma^2}}.}$
\par
\newprobheader{(2)}%
记$\displaystyle P_0 = \frac{q^4 B^2}{6\pi\epsilon_0 m^2 c}\pare{\gamma^2 - 1}$, 则
\begin{align*}
    & \+dtdE = -P_0\pare{\gamma^1 - 1} \Rightarrow \frac{\rd{\gamma}}{\gamma^2 - 1} = -\frac{P_0}{mc^2}\,\rd{t}, \\
    & \half \ln \frac{\gamma-1}{\gamma+1} = -\frac{P_0}{mc^2}\,\rd{t}, \\
    & \gamma \gg 1 \Rightarrow \rec{\gamma_0} - \rec{\gamma} \approx -\frac{P_0}{mc^2}T \Rightarrow \boxed{T \approx \frac{6\pi\epsilon_0 m^3c^3}{q^4B^2}\pare{\rec{\gamma} - \rec{\gamma_0}}.}
\end{align*}
\par
\newprobheader{(3)}%
非相对论极限下, $\gamma - 1 \propto T$, $\gamma + 1\approx 2$, 从而
\[ \half \ln \pare{\frac{T}{T_0}} = -\frac{P_0}{mc^2}T \Rightarrow \boxed{T = \frac{3\pi\epsilon_0 m^3 c^3}{q^4 B^2} \ln \pare{\frac{T_0}{T}}.} \]
\par
\newprobheader{(4)}%
$P \propto \pare{\gamma^2 - 1}$故速度较大处辐射功率较大. $mv_\perp^2 / \pare{2B}\approx \const$而磁镜点处$B$最大, 故\underline{磁镜点处辐射功率大}.

\end{document}
