\documentclass[hidelinks]{ctexart}

\usepackage{van-de-la-illinoise}
\usepackage[paper=b5paper,top=.3in,left=.9in,right=.9in,bottom=.3in]{geometry}
\usepackage{calc}
\pagenumbering{gobble}
\setlength{\parindent}{0pt}
\sisetup{inter-unit-product=\ensuremath{{}\cdot{}}}
\usepackage{van-le-trompe-loeil}
\usetikzlibrary{quotes,angles}
\usetikzlibrary{arrows.meta}
\usepackage{makecell}

\usepackage{stackengine}
\stackMath
\usepackage{scalerel}
\usepackage[outline]{contour}

\newdimen\indexlen
\def\newprobheader#1{%
\def\probindex{#1}
\setlength\indexlen{\widthof{\textbf{\probindex}}}
\hskip\dimexpr-\indexlen-1em\relax
\textbf{\probindex}\hskip1em\relax
}
\def\newprob#1{%
\newprobheader{#1}%
\def\newprob##1{%
\probsep%
\newprobheader{##1}%
}%
}
\def\probsep{\vskip1em\relax{\color{gray}\dotfill}\vskip1em\relax}

\newlength\thisletterwidth
\newlength\gletterwidth
\newcommand{\leftrightharpoonup}[1]{%
{\ooalign{$\scriptstyle\leftharpoonup$\cr%\kern\dimexpr\thisletterwidth-\gletterwidth\relax
$\scriptstyle\rightharpoonup$\cr}}\relax%
}
\def\tensor#1{\settowidth\thisletterwidth{$\mathbf{#1}$}\settowidth\gletterwidth{$\mathbf{g}$}\stackon[-0.1ex]{\mathbf{#1}}{\boldsymbol{\leftrightharpoonup{#1}}}  }
\def\onedot{$\mathsurround0pt\ldotp$}
\def\cddot{% two dots stacked vertically
  \mathbin{\vcenter{\baselineskip.67ex
    \hbox{\onedot}\hbox{\onedot}}%
}}%

    \tikzset{
    partial ellipse/.style args={#1:#2:#3}{
        insert path={+ (#1:#3) arc (#1:#2:#3)}
    }}

\DeclareMathOperator{\arcsinh}{arcsinh}
\DeclareSIUnit\year{yr}

\begin{document}

\newprob{8.1}%
\vspace{-\baselineskip}
\[ \Delta t = \rec{c}\cdot \gamma\pare{c\Delta t' + \beta \Delta x} = \SI{3.75e-8}{\second}. \]
故$B$领先$\boxed{\SI{3.75e-8}{\second}.}$
\newprob{8.2}%
在车厢系中,
\[ \Delta x' = l,\quad \Delta t' = \frac{l}{u}. \]
故地面系中
\[ \Delta t = \rec{c}\cdot \gamma\pare{c\Delta t' + \beta \Delta x} = \boxed{\frac{l}{c^2u}\frac{c^2+uv}{\sqrt{1-\pare{v/c}^2}}.} \]
\newprob{8.3}%
二尺之间的相对速度为
\[ \beta' = \frac{2\beta}{1+\beta^2} \Rightarrow \rec{\gamma'} = \sqrt{1-\beta'^2} = \frac{1-\beta^2}{1+\beta^2}. \]
故
\[ l = \frac{l_0}{\gamma} = \boxed{\frac{l_0\pare{c^2-v^2}}{c^2+v^2}.} \]
\newprob{8.4 (1)}%
由加速度变换公式, 地面系
\[ a_x = \pare{\sqrt{1-\beta^2}}^3 g \Rightarrow \frac{\beta}{\sqrt{1-\beta^2}} = \frac{gt}{c}. \]
解得(在第一个加速阶段)
\[ \beta = \frac{gt/c}{\sqrt{1+\pare{gt/c}^2}} \Rightarrow \frac{\rd{t}}{\sqrt{1+\pare{gt/c}^2}} = \rd{\tau}. \]
即(在完成一个加速或减速阶段时)
\[ \frac{c \arcsinh \pare{gt/c}}{g} = \tau \Rightarrow t = \SI{11.136}{\year}. \]
考虑到出发时$20$岁, 共经过$4$个加速/减速阶段, 回来后地球上的人\boxed{${64.5}\text{岁}.$}
\par
\newprobheader{(2)}%
积分可得在第一个加速阶段
\[ x = c\brac{\sqrt{\frac{c^2}{g^2} + t^2} - \frac{c}{g}} = \SI{10.2}{\lightyear}. \]
从而在经过第一个加速阶段和第一个减速阶段后$x = \boxed{\SI{20.4}{\lightyear}.}$
\newprob{8.5}%
设宇航员观测到的波长为$\lambda'$, 地面观测者观测到的波长为$\lambda$, 则由Doppler频移公式,
\[ \frac{\lambda'}{\lambda} = \sqrt{\frac{1-\beta}{1+\beta}} \Rightarrow v = \boxed{\SI{4721}{\kilo\meter\per\second}.} \]
\newprob{8.6 (1)}%
保留到$\beta$的一阶项,
\begin{align*}
    \tan \theta' - \tan\theta &\approx \frac{\alpha}{\cos^2\theta}, \\
    \tan \theta' - \tan\theta &= \frac{\sin\theta}{\cos\theta - \beta} - \frac{\sin\theta}{\cos\theta}
    = \frac{\beta \sin \theta}{\cos\theta\pare{\cos\theta - \beta}}
     \approx \frac{\beta\sin\theta}{\cos^2\theta}.
\end{align*}
从而
\[ \alpha \approx \beta \sin\theta. \]
\par
\newprobheader{(2)}%
$\alpha = \beta = v/c = \SI{e-4}{\radian} = \boxed{\SI{20.6}{\arcsecond}.}$
\newprob{8.7}%
对于$v$和光发射方向相同的情形, $S'$系下
\[ \omega' = \sqrt{\frac{1-\beta}{1+\beta}}\omega \Rightarrow \omega' - \omega \approx -\beta\omega. \]
从而$S'$系下光速
\[ c' = \frac{c}{n\pare{\omega'}} \approx \frac{c}{n\pare{\omega}} - \frac{c}{n\pare{\omega}^2} \+d\omega d{n\pare{\omega}}\,\pare{\omega' - \omega} \approx \frac{c}{n} - \frac{c}{n^2}\+d\omega d{n\pare{\omega}}\cdot {-\beta\omega}. \]
由速度合成律,
\begin{align*}
    u &= \frac{v+c'}{1+vc'/c^2} \approx v + c' - v \pare{\frac{c'}{c}}^2 \approx v+\frac{c}{n} + \rec{n^2}\+d\omega d{n\pare{\omega}}\cdot {v \omega} - \frac{v}{n^2} \\
    &= \boxed{\frac{c}{n} + v\pare{1 - \rec{n^2} + \frac{\omega}{n^2} \+d\omega d{n\pare{\omega}}}.}
\end{align*}
对于$v$和光发射方向相反的情形, 替换$v\rightarrow -v$即可.
\newprob{8.8}%
光的4-动量为$\displaystyle p = \frac{\+cE}{c}\pare{1,\+vn}$. 在两个参考系中分别为
\[ p = \frac{\+cE}{c}\pare{1,\+un},\quad p' = \frac{\+cE'}{c}\pare{1,\+un'}. \]
两者由Lorentz变换联系.
\[ \frac{\+cE'}{c} = \frac{\+cE}{c}\cdot \gamma\pare{1-\+v\beta\cdot \+un},\quad \frac{\+cE'}{c}\+un'_\parallel = \frac{\+cE}{c}\cdot\gamma \pare{\+un_\parallel - \frac{\+vv}{c}},\quad \frac{\+cE}{c}\+un_\perp = \frac{\+cE}{c}\+un_\perp. \]
故
\begin{align*}
    \cos\phi' &= \frac{\gamma\pare{\hat u_\parallel - v/c}}{\gamma\pare{1-\+v\beta\cdot \+un}} = \boxed{\frac{\cos\phi - v/c}{1-v\cos\phi/c}.}\\
    \cos \psi' &= \frac{\cos\psi}{\gamma\pare{1 - \+v\beta\cdot \+un}} = \boxed{\frac{\cos\psi}{\gamma\pare{1-v\cos\phi/c}}.} \\
    \cos \chi' &= \frac{\cos\chi}{\gamma\pare{1 - \+v\beta\cdot \+un}} = \boxed{\frac{\cos\chi}{\gamma\pare{1-v\cos\phi/c}}.} 
\end{align*}

\end{document}
