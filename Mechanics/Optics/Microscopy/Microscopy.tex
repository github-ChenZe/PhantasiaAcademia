\documentclass[hidelinks]{ctexart}

\usepackage{van-de-la-illinoise}

\begin{document}

\section{显微术} % (fold)
\label{sec:显微术}

\subsection{光学显微术} % (fold)
\label{sub:光学显微术}

相衬显微镜(Phase-Contrast Microscopy)通过将光线穿过透明样品时产生的相位差转化为强度以成像.

% subsection 光学显微术 (end)

\subsection{电子显微术} % (fold)
\label{sub:电子显微术}

场离子显微镜(Field Ion Microscope), 通过高电位针状样品电离附近的气体分子, 气体离子带正电后被排斥, 通过探测离子的最终位置得到样品的原子排布.

\subsubsection{扫描探针显微镜} % (fold)
\label{ssub:扫描探针显微镜}

扫描探针显微镜(Scanning Probe Microscopy, SPM)是所有机械式地用物理探针在样本上扫描移动以探测样本以探测样本影像的显微镜的统称.
\par
扫描隧道显微镜(Scanning Tunneling Microscope, STM), 探针和样品之间存在一电压. 通过针尖的电流大小得到表面起伏图像.
\par
原子力显微镜(Atomic Force Microscope, AFM), 探针接近样品表面时会弯曲偏移, 观测偏移量(例如压电效应)可对表面成像.

% subsubsection 扫描探针显微镜 (end)

% subsection 电子显微术 (end)

% section 显微术 (end)

\end{document}
