\documentclass{ctexart}

\usepackage{van-de-la-sehen}

\begin{document}

\headerstamp

\begin{ex}[毛细现象]
毛细管中, $A$ and $D$ be on the axis of the tube, $D$ be on the surface of liq, $A$ be on the same altitude as the periph, $B$ be on bottom quo connected to aer;
\[ p_B = p_A, \]
point $A$ be viewed as in a bubble,
\[ p_A - p_D = \frac{2\sigma}{R} = \rho gh,\quad R = \frac{r}{\cos\theta}. \]
\[ h = \frac{2\sigma\cos\theta}{\rho gr}. \]
\end{ex}

\paragraph{相变} % (fold)
\label{par:相变}

\[ \delta S = \frac{\rd{U}}{T} + p\frac{\rd{V}}{T}. \]
\begin{finale}
	\[ \mu = u+pv-Ts. \]
\end{finale}
\begin{align*}
&\brac{S_1 - S_2 - \frac{U_1 - U_2}{T_2} - \frac{p_2\pare{v_1-v_2}}{T_2}}\delta \nu_1 +\\& \nu_1\pare{\rec{T_1} - \rec{T_2}}\delta u_1 + \nu_1\pare{\frac{p_1}{T_1}-\frac{p_2}{T_2}}\delta v_1 = 0.
\end{align*}

\begin{align*}
	T_1 &= T_2,\\
	p_1 &= p_2,\\
	\mu_1 &= \mu_2.
\end{align*}

% paragraph 相变 (end)

\begin{ex}
	等$p$线从左至右三点$A,B,C$, $B$点在气-液平衡点, $A$点过饱和. $A$处液化, 则熵变之计算需要构造至平衡线上的循环.
\end{ex}
\begin{ex}
	水-气-冰三态平衡, 做绝热压缩, 问其最终状态如何. 固体先融化, 沿气-液平衡线移动, 最后有可能完全变为气体/液体.
\end{ex}
\begin{ex}
	低于液体内压力低于外界大气压: 凹液面. 反之凸液面.
\end{ex}
\begin{ex}
	二气泡合并为一大气泡, $p$和$r$和$\sigma$皆已知, 求$p_0$, 则联立$p_0-p=4\sigma/r$和$pV=nRT$即可.
\end{ex}

\end{document}