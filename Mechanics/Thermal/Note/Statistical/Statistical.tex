\documentclass[hidelinks]{ctexart}

\usepackage[margintoc, singleton]{van-de-la-sehen}

\begin{document}

\showtitle{统计物理初步}

\section{分布率} % (fold)
\label{sec:分布率}

\subsection{Maxwell分布} % (fold)
\label{sub:maxwell分布}

\subsubsection{速度/速率分布} % (fold)
\label{ssub:速度_速率分布}

\begin{finale}
    \[ \expc{v} = \sqrt{\+/8kT/\pi m/},\quad v_{\mathrm{rms}}=\sqrt{\+/3kT/m/},\quad v_{\mathrm{p}} = \sqrt{\+/2kT/m/}. \]
\end{finale}
\begin{finale}
    \[ n = n_0 \exp\pare{-mgh/kT}. \]
\end{finale}
\begin{finale}
    \[ \varepsilon = \+/i/2/ pV,\quad C_V = \+/i/2/ R,\quad C_p = \+/i+2/i/ R. \]
\end{finale}
\begin{finale}
    \[ f\pare{v_i} = \pare{\+/m/2\pi kT/}^{1/2} e^{-mv_i^2/2kT}. \]
    \[ f\pare{v} = 4\pi\pare{\+/m/2\pi kT/}^{3/2}e^{-mv_i^2/2kT}v^2. \]
\end{finale}
\begin{ex}
    试证能量分布为
    \[ f\pare{\varepsilon} = \+/2/\sqrt{\pi}/ \pare{kT}^{3/2}e^{-\varepsilon/kT}\sqrt{\varepsilon}. \]
\end{ex}
\begin{finale}
    \[ f\pare{x}\,\rd{x} = \+/4/\sqrt{\pi}/ x^2 e^{-x^2}\,\rd{x},\quad x = \+/v/v_p/. \]
\end{finale}

% subsubsection 速度_速率分布 (end)

\subsubsection{应用到势能} % (fold)
\label{ssub:应用到势能}

\begin{finale}
    \begin{equation*}
        n = n_0 e^{-\varepsilon_p / kT}.
    \end{equation*}
\end{finale}
\begin{ex}
    对于势能$cx^2-gx^3-fx^4$, 证明在小位移近似下, $x$的期望值为
    \[ \frac{3kg}{4c^2}T. \]
    Hint: 对指数上的$gx^3+fx^4$作一阶近似.
\end{ex}
\begin{ex}
    在水中的颗粒应当使用的势能为$m^*gh = \pare{\rho - \rho_0}V_m gh$, 如果已知两个高度处的浓度, 即可以反求得$k$值.
\end{ex}

% subsubsection 应用到势能 (end)

\subsubsection{能量均分定理} % (fold)
\label{ssub:能量均分定理}

\begin{ex}
    双原子气体在常温下热容为$5R/2$.
\end{ex}
\begin{ex}
    $\SI{1}{\mol}$\ce{O2}在绝热容器中以$\SI{10}{\meter\per\second}$的速度撞墙, $80\%$的动能被转化为热能, 求最终$T$和$p$.
\end{ex}

% subsubsection 能量均分定理 (end)

% subsection maxwell分布 (end)

% section 分布率 (end)

\section{输运现象} % (fold)
\label{sec:输运现象}

\subsection{基本物理量} % (fold)
\label{sub:基本物理量}

\subsubsection{平均自由程} % (fold)
\label{ssub:平均自由程}

\begin{finale}
    \[ \lambda = \rec{\sqrt{2}n\pi d^2} = \+/kT/\sqrt{2}\pi d^2 p/. \]
\end{finale}
\begin{finale}
    \begin{theorem}
        [碰撞概率分布] 自由程大于$x$的概率为
        \[ N = N_0 e^{-x/\lambda}. \]
    \end{theorem}
\end{finale}

% subsubsection 平均自由程 (end)

\subsubsection{系数} % (fold)
\label{ssub:系数}

\begin{finale}
    \[ \eta = \rec{3}nm\expc{v}\lambda, \]
    \[ \kappa = \rec{3}nc\expc{v}\lambda, \]
    \[ D = \rec{3} \expc{v}\lambda. \]
\end{finale}

\begin{ex}
    使\ce{Na}自发解离的温度大概要求$kT\sim \varepsilon\+_disass_$.
\end{ex}

% subsubsection 系数 (end)

\subsubsection{输运方程} % (fold)
\label{ssub:输运方程}

\begin{ex}
    两个共轴圆柱, 半径分别为$R_1$和$R_2$, 长$L\gg R_1, R_2$. 内/外圆柱分别维持在温度$T_1, T_2$, 则单位时间内传热
    \begin{equation*}
        \Delta Q = \+/2\pi\kappa L/\ln\pare{R_2/R_2}/ \pare{T_1-T_2}.
    \end{equation*}
    Hint: 将$\Delta Q/r$对$r$积分,
    \[ \Delta Q = \kappa \+drdT \cdot 2\pi r\cdot L. \]
\end{ex}
\begin{ex}
    假定热导率$\kappa \propto \sqrt{T}$, 气体在圆柱管中, 两侧被维持在温度$T_1$和$T_2$,
    \[ Q = \kappa_0 \sqrt{T}\+dxdT\Delta S. \]
    可以反解得到$T$.
\end{ex}
\begin{reflex}
    {稳态传热量问题}{稳态传热量问题}
    稳态下传热量的求解, 借助经过各个截面的$\Delta Q$相等之事实, 对尺度积分.
\end{reflex}
\begin{ex}
    两块平行木板, 上方速率$u_A$, 下方速率$u_B$, $p\ll 1$, 求黏滞力.
\end{ex}
\begin{proof}[稀薄气体的黏滞力]
    在各个方向上运动的粒子数为$n/6$, 从而动量输运为
    \[ \rec{6} n\,\rd{s}\,\expc{v}\,\rd{t}\cdot\pare{mu_A - mu_B}, \]
    \[ f = \rec{6} n\expc{v}m\pare{u_A - u_B}. \]
\end{proof}
\begin{ex}
    容器内有绝热隔板, 两侧有\ce{He}气体, 分别$T_1 = \SI{150}{\kelvin}$, $T_2 = \SI{300}{\kelvin}$, 中间开孔.
    \begin{cenum}
         \item $D\ll \lambda_1,\lambda_2$, 分子泻流, $\Phi_1 = \Phi_2$时平衡;
         \item $D>\lambda_1, \lambda_2$, 二者导通, $p_1=p_2$时达到平衡.
    \end{cenum} 
\end{ex}

% subsubsection 输运方程 (end)

% subsection 基本物理量 (end)

% section 输运现象 (end)


\end{document}
