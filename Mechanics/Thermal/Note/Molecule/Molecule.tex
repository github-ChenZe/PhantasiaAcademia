\documentclass{ctexart}

\usepackage{van-de-la-sehen}

\begin{document}

\subsection{\texorpdfstring{th\'eor\`eme EQP}{theorem EQP}} % (fold)
\label{sub:theoreme_EQP}

\begin{finale}
	高温下, 每一个被激发的平方模贡献内能$1/2RT$.
\end{finale}
\begin{ex}
	双原子分子气体, 平动贡献$3/2$, 转动贡献$2/2$, 震动贡献$2/2$.
	\[ C_V = \frac{7}{2}R. \]
\end{ex}
\begin{finale}
	\[ C_V = \half R\pare{t+r+2s}. \]
\end{finale}
\begin{pitfall}
	谐振子的$2$来自于振动同时存在动能和势能.
\end{pitfall}
\begin{pitfall}
	转动和振动仅在高温下激发.
\end{pitfall}
citing von QM,
\[ E_l = l\pare{l+1}\frac{\hbar^2}{2I},\quad E_v = \pare{n+\half}\hbar\omega. \]
Therefore for one asp to have signi contrib, $T = E/k$. 

% subsection theoreme_EQP (end)

\subsection{Transp} % (fold)
\label{sub:transp}

$T$, $n$, $u$之不同分别引发热量、数量、动量流, 即导致热传导、扩散与黏滞.

\subsection{mean free path} % (fold)
\label{sub:mean_free_path}

colli param: distance perpendi $\vv$ rel center. for effective colli to take place, $b < d$. $\pi b^2$ \emph{section efficace} est. for $d_1 \neq d_2$, il est $\sigma = \pi\pare{d_1^2+d_2^2}/4$.
\par
freq colli:
\[ \expc{z} = \pi d^2\cdot\expc{v_{\mathrm{rel}}}\cdot n \]
quo $\expc{z}$ being  num de colli for uno molecule. cum $p=nkT$,
\[ \Rightarrow \expc{z} = \frac{4\sigma p}{\sqrt{\pi mkT}}. \]
\begin{ex}
	dans etat standard, $\expc{z} = \SI{6.5e9}{\Hz}$.
\end{ex}
\[ \lambda = \frac{\expc{v_{\mathrm{rel}}}}{\expc{z}} = \rec{\sqrt{2}n\sigma}. \]
\begin{ex}
	$\lambda \sim 200d$ dans etat standard est typical molecule.
\end{ex}

% subsection mean_free_path (end)

\subsubsection{thermal coeff} % (fold)
\label{ssub:thermal_coeff}

for molecule uno atom, thermal coeff
\begin{finale}
	\[ \kappa = \rec{3}nC\lambda\expc{v} = \frac{\pi}{d^2}\sqrt{\frac{kT}{m\pi}}. \]
\end{finale}

\begin{remark}
	applies only for $d \ll \lambda \ll L$.
\end{remark}

\begin{finale}
	\[ \frac{\eta C_V}{\kappa} = 1. \]
\end{finale}

% subsubsection thermal_coeff (end)

\subsubsection{Vis coeff} % (fold)
\label{ssub:vis_coeff}

\begin{finale}
	\[ \eta = \rec{3}nm\lambda\expc{v} = \frac{m\expc{v}}{3\sqrt{2}\sigma} = \frac{2}{3\sigma}\sqrt{\frac{kmT}{\pi}}. \]
\end{finale}
\begin{ex}[measurement of $\eta$ via rotation and torq]
	outer linear veloc $\omega\pare{R+\delta}$, quod provides grad de veloc
	\[ \frac{\omega\pare{R+\delta}}{\delta} \approx \frac{\omega R}{\delta}, \]
	torq balanced,
	\[ G = 2\pi RL\cdot\eta\cdot\frac{\omega R}{\delta}\cdot R. \]
\end{ex}

% subsubsection vis_coeff (end)

\subsubsection{diff coeff} % (fold)
\label{ssub:diff_coeff}

\begin{finale}
	\[ D = -\rec{3}\lambda\expc{v} = \frac{2}{3\pi nd^2}\sqrt{\frac{kT}{\pi m}} = \frac{2}{3\pi d^2 p}\cdot\sqrt{\frac{k^3}{\pi m}} T^{3/2}. \]
\end{finale}

\begin{finale}
	\[ \frac{D\rho}{\eta} = 1. \]
\end{finale}

% subsubsection diff_coeff (end)

% subsection transp (end)

\end{document}
