\documentclass{ctexart}

\usepackage{van-de-la-sehen}

\begin{document}

\subsection{\texorpdfstring{th\'eor\`eme EQP}{theorem EQP}} % (fold)
\label{sub:theoreme_EQP}

\begin{finale}
	高温下, 每一个被激发的平方模贡献内能$1/2RT$.
\end{finale}
\begin{ex}
	双原子分子气体, 平动贡献$3/2$, 转动贡献$2/2$, 震动贡献$2/2$.
	\[ C_V = \frac{7}{2}R. \]
\end{ex}
\begin{finale}
	\[ C_V = \half R\pare{t+r+2s}. \]
\end{finale}
\begin{pitfall}
	谐振子的$2$来自于振动同时存在动能和势能.
\end{pitfall}
\begin{pitfall}
	转动和振动仅在高温下激发.
\end{pitfall}
citing von QM,
\[ E_l = l\pare{l+1}\frac{\hbar^2}{2I},\quad E_v = \pare{n+\half}\hbar\omega. \]
Therefore for one asp to have signi contrib, $T = E/k$. 

% subsection theoreme_EQP (end)

\subsection{Transp} % (fold)
\label{sub:transp}

$T$, $n$, $u$之不同分别引发热量、数量、动量流, 即导致热传导、扩散与黏滞.

% subsection transp (end)

\end{document}
