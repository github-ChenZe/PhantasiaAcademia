\documentclass[hidelinks]{ctexart}

\usepackage{van-de-la-illinoise}

\begin{document}

\section{基本知识} % (fold)
\label{sec:基本知识}

\subsection{宇宙漫游} % (fold)
\label{sub:宇宙漫游}

\newpoint{}银河系半径大约$5.3$万光年.
\newpoint{}本星系团直径大约$1000$万光年.
\newpoint{}本超星系团.

% subsection 宇宙漫游 (end)

\subsection{天体观测手段} % (fold)
\label{sub:天体观测手段}

略.

% subsection 天体观测手段 (end)

\subsection{天体温度的测量} % (fold)
\label{sub:天体温度的测量}

\newpoint{}通过光谱/赫罗图等手段测定温度.

% subsection 天体温度的测量 (end)

\subsection{恒星的一生} % (fold)
\label{sub:恒星的一生}

气体尘埃团$\Rightarrow $原恒星$\Rightarrow $主序星$\Rightarrow $红巨星$\Rightarrow $白矮星.

% subsection 恒星的一生 (end)

\subsection{星系} % (fold)
\label{sub:星系}

定义.

% subsection 星系 (end)

\subsection{宇宙学简介} % (fold)
\label{sub:宇宙学简介}

人类宇宙观简史.

% subsection 宇宙学简介 (end)

\subsection{宇宙起源} % (fold)
\label{sub:宇宙起源}

\newpoint{}宇宙起源于$137$亿年前的爆炸.

% subsection 宇宙起源 (end)

\subsection{太阳系} % (fold)
\label{sub:太阳系}

\newpoint{}海王星外侧($30$万光年处)存在Kuiper带.
\newpoint{}火星轨道和木星轨道之间存在主小行星带.
\newpoint{}火星轨道在地球参考系的看来存在逆行. 需要将太阳置于中心以解释.
\newpoint{}Titius-Bode law表明, $r = 0.4 + 0.3\times n$, $n = 0,1,2,4,8,\cdots$, 其中$n=8$对应小行星带. 在海王星之前都和实际符合较好.
\newpoint{}太阳能量源于核反应, 表面温度约$\SI{6000}{\kelvin}$, 主要反应发生于内核.
\newpoint{}水星-地球-太阳共线时在太阳表面形成阴影, 发生水星凌日.
\newpoint{}金星也会发生类似的现象, 发生金星凌日.
\newpoint{}小行星带除了谷神星等是球形外, 大多数为不规则状. 有推测认为小行星带上的碎片是一颗大行星分裂的结果.
\newpoint{}木星共有79颗卫星, 其中有4颗较大, 由Galileo发现.
\newpoint{}木星的大红斑是一个巨大的反气旋风暴.
\newpoint{}Galileo号发现了木星周围的光环.
\newpoint{}Juno号发现了木星大红斑附近存在水分子.
\newpoint{}Cassini号于2004年7月到达土星.
\newpoint{}土星的光环可以稳定存在原因在于引力的几何性, 不同质点的轨道大体相同.
\newpoint{}Cassini号在土卫六自毁, 土卫六有比较厚的大气层, 也是Huygens当年发现的卫星.

% subsection 太阳系 (end)

\subsection{海王星} % (fold)
\label{sub:海王星}

\newpoint{}海王星的发现源于天王星的摄动.
\newpoint{}先驱者10号, 先驱者11号携带地球人的信息前往太阳系以外. 旅行者1号, 旅行者2号同.
\newpoint{}Kepler望远镜可搜寻其他类地行星.
\newpoint{}Kepler-22b表面可能由海洋覆盖, 距离地球600光年.
\newpoint{}TESS望远镜发现一颗恒星周边的宜居带, 存在一颗与地球大小相仿的行星, 谓TOI 700d, 距离地球101.5光年.

% subsection 海王星 (end)

% section 基本知识 (end)

\end{document}
