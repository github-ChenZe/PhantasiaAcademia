\documentclass{ctexart}

\usepackage{van-de-la-sehen}

\begin{document}

\section{电磁感应} % (fold)
\label{sec:电磁感应}

\[ \cE = -\eddon{\Phi}{t} = -\eddon{\vB}{t}\cdot\vS - \vB\cdot\eddon{\vS}{t} = \cE_{\mathrm{induced}} + \cE_{\mathrm{motive}}. \]
\begin{ex}[measurement of $\vM$ to $\vB$]
	with $Q = \Delta \Phi/R$, where $\Phi = BS$, $H = nI_0 = B_0/\mu_0$.
\end{ex}

\subsection{Induced emf} % (fold)
\label{sub:induced_emf}

\subsubsection{motive generated E} % (fold)
\label{ssub:motive_generated_e}

\[ \Phi' - \Phi + \Phi_{\mathrm{side}} = 0. \]
\[ \cE = \eddon{\Phi_{\mathrm{side}}}{t} = \iint_{\mathrm{side}} \vB\cdot\rd{S_{\mathrm{side}}}/\rd{t}, \]
quo via expanding $\rd{S} = \rd{l}\times \vv\,\rd{t}$,
\[ \cE = \oint \vK\cdot\rd{\vl} = \oint\pare{\vv\times\vB}\cdot\rd{\vl}. \]

% subsubsection motive_generated_e (end)

\subsubsection{感生电动势} % (fold)
\label{ssub:感生电动势}

由$\vB$的变化而产生, 涡旋电场.
\begin{finale}
	\[ \curl \vE_{\mathrm{edd}} = -\ddelon{\vB}{t},\quad \vE_{\mathrm{edd}} = -\ddelon{\vA}{t}. \]
\end{finale}
With
\[ \vA = \frac{\mu_0}{4\pi} \iiint \frac{\vj\,\rd{V'}}{\rcurs}, \]
$\vE_{\mathrm{edd}}$ may be obtained.

\begin{multicols}{2}
	\noindent
	\begin{align*}
		\curl \vE &= - \ddelon{\vB}{t},\\
		\oint \vE\cdot\rd{\vl} &= -\eddon{\Phi}{t}.
	\end{align*}
	\begin{align*}
		\curl \vB &= \mu_0 \vj,\\
		\oint \vB\cdot\rd{\vl} &= \mu_0 I_0.
	\end{align*}
\end{multicols}
Cum symmetry, quo evaluate ex.p.8.

% subsubsection 感生电动势 (end)

\subsubsection{自感和互感} % (fold)
\label{ssub:自感和互感}

通有电流的线圈$1$所激发的磁场穿过线圈$2$的磁通量为
\[ \Psi_{21} = M_{21}I_1, \]
则$1$中电流变化时,
\[ \cE_{21} = -\eddon{\Phi_{21}}{t} = -M_{21}\eddon{I_1}{t}. \]
类似定义$M_{12}$.
\begin{finale}
	互感系数具有对称性, 即$M_{12}=M_{21}$.
\end{finale}
\begin{proof}
	\[ \Psi_{21} = \oint_{C_2} \vA_1\cdot\rd{\vl_2}, \]
	with
	\[ \vA_1 = \frac{\mu_0 I}{4\pi} \oint_{C_1}\frac{\rd{\vl_1}}{r_{12}}, \]
	\[ \Phi_{21} = \frac{\mu_0 I_1}{4\pi} \oint_{C_2}\oint_{C_1}\frac{\rd{\vl_1}\cdot\rd{\vl_2}}{r_{12}}. \]
\end{proof}
\begin{pitfall}
	cf ex.p.8., on $\Phi$ of m-dipole through $S$, $\Phi = -\Phi_{\mathrm{out-of-}S}$.
\end{pitfall}
\begin{reflex}{计算互感系数}{计算互感系数}
	在计算互感系数时, 使用更容易得到磁通量的$I$计算磁场.
\end{reflex}
通有电流的线圈对自身会有磁通量, $\Phi = LI$. 通过
\[ L = \frac{\mu_0}{4\pi} \oint_C\oint_C \frac{\rd{\vl_1}\cdot\rd{\vl_2}}{r_{12}} \]
会导致发散. Deferred evaluation of $L$.
\paragraph{自感-互感关系} % (fold)
\label{par:自感_互感关系}

若每一个线圈的磁感线都完全穿过另一个线圈, 则
\[ M = \sqrt{L_1 L_2}, \]
实际上则有
\[ M = k\sqrt{L_1 L_2}, \]
变压器主/副线圈之间$k\approx 0.98$.
\begin{proof}
	\begin{align*}
		M&= \frac{N_2\Phi_{21}}{I_1} = \frac{N_1\Phi-{12}}{I_2},\\
		L_1 &= \frac{N_1\Phi_11}{I_1},\\
		L_2 &= \frac{N_2\Phi_{22}}{I_2}.
	\end{align*}
	plus the condition $\Phi{21}=\Phi{11}$, $\Phi{12}=\Phi{22}$.
\end{proof}
\begin{ex}
	通过列出$\cE_1$, $\cE_2$, $I_1$与$I_2$和$I$之间的关系给出并联/串联线圈时得到的$L$.
\end{ex}

% paragraph 自感_互感关系 (end)

% subsubsection 自感和互感 (end)

% subsection induced_emf (end)

% section 电磁感应 (end)

\end{document}