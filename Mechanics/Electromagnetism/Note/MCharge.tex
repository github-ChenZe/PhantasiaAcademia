\documentclass{ctexart}

\usepackage{van-de-la-sehen}

\begin{document}

\section{M Charge} % (fold)
\label{sec:m_charge}

If $\vJ_f = 0$, $\rot \vH = 0$ thus $\vH$ may be ritten as $\grad$ something.
cf. \& distinguish from
\[ \varphi = -\frac{I\Omega}{4\pi}. \]
sic field by magnet
\[ \begin{cases}
	\rot \vH = 0;\\
	\div \vH = -\div\vM;
\end{cases} \]
cf the case of $\vE$ field. where $\rho = -\div \vP$. quo hints introducing
\[ \frac{\rho^*}{\mu_0} = -\div \vM,\quad \vP^* = \mu_0 \vM. \]
sic the following correspon
\begin{align*}
	\vP &\mapsto \vP^*\\
	\vsigma &\mapsto \vsigma^*\\
	\vE &\mapsto \vH\\
	\varphi &\mapsto \varphi\\
	\epsilon_0 &\mapsto \mu_0.
\end{align*}
\begin{ex}
	magnet of total Area $2S$, with $\vM$, eval the attr force to ferro:
	\[ \sigma_1^* = \mu_0 M = +\sigma^*, \]
	\[ \sigma_2^* = -\mu_0 M = -\sigma^*. \]
\end{ex}
\begin{ex}[correction to fringe]
	$\sigma$ of e charge converted to $\sigma^*$ yields $\Delta p^* = =\mu_0 \Delta m = \mu_0 IS$
	\[ I = \frac{\sigma^* d}{\mu_0}. \]
	cancellation sic to boundary current,
	\[ H = \frac{I}{2\pi r}=\frac{\sigma^* d}{2\pi mu_0 r}\Rightarrow E = \frac{\sigma d}{2\pi\epsilon_0 r}. \]
\end{ex}
\begin{ex}
	Method quo may be applied to eval the field prod via uno uniformly polarized sphere.
	\[ \sigma^* = \mu_0 M\cos\theta. \]
\end{ex}

% section m_charge (end)

\section{Supp: EM field in SR} % (fold)
\label{sec:supp_em_field_in_sr}

\[ \gamma_0 = \rec{\sqrt{1-v_0^2}}, \gamma = \rec{\sqrt{1-u^2}}, \gamma' = \rec{\sqrt{q-u'^2}}. \]
\begin{ex}
	Uno $u$ clock moving rel to ground, in frame de clock running $\rd{\tau}$, on ground
	\[ \rd{t} = \gamma \rd{\tau} = \frac{\rd{\tau}}{\sqrt{1-u^2}}. \]
\end{ex}
\begin{ex}
	Uno $u$ ruler moveing rel to ground, in frame de ruler length $l_0$, on ground l.
\end{ex}
\begin{remark}
	citing def de measurement of $l$ and $t$.
\end{remark}
def 4-veloc as
\[ u = \eddon{\pare{t,x,y,z}}{\tau}. \]
\[ u = \gamma\pare{1, v_x,v_y,v_z}. \]
4-veloc transf via Lorentz.
\\
4-moment def as
\[ P = mu, \]
quo transf via Lorentz. $p_0 = \gamma m$.
\\
suppose in $K$ holding $\rho$ and $\vJ = \rho \vvu$, sic in $K'$ quo moving $u$ rel $K$ quo
\[ \rho' = \rho_0,\quad \vJ' = 0. \]
In frame $K$,
\[ \rho = \gamma \rho_0, \vJ = \gamma \rho_0. \]
sic def $j = \gamma\rho_0 u$.
\\
cum
\[ \vF = \eddon{\vp}{t}. \]
sic, si $\vF\cdot\vv = 0$, then $\vF = \gamma m\vva$, otw si $\vF\parallelsum\vv$, $\vF = \gamma^3 m\vva$.
\\
in $K$ uno wire $j$ with $s$ outer $q$ moving $u$, trans quo yielding
\[ \rho' = -\gamma u j_x \Rightarrow E'_s = -\gamma u \frac{I}{2pi \epsilon_0 s}, \]
yielding force attraction, auch in frame quo wire static, quo force magnetic est.
\\
sic	for $\vB$ being zero, $\vE'$ perpendi mul by $\gamma$ sed $E$ parallel immut.
\\
$B'$ follows. Sed $E$ and $B$ getting mixed,
\[ \vE' = \gamma_0 \pare{\vE + \vv\times\vB}, \]
\[ \vB' = \gamma_0 \pare{\vB - \vv\times\vB}, \]
for perpendi comp only. comp $\parallelsum$ immut.

% section supp_em_field_in_sr (end)

\end{document}