\documentclass[hidelinks]{ctexart}

\usepackage[singleton, margintoc]{van-de-la-sehen}

\begin{document}

\showtitle{交流电}

\section{简谐交流电} % (fold)
\label{sec:简谐交流电}

\subsection{物理量} % (fold)
\label{sub:物理量}

\subsubsection{有效值} % (fold)
\label{ssub:有效值}

\begin{definition}
    [有效值]简谐量的有效值谓在同一纯电阻上产生相同Joule热的等效直流电的相应值.
\end{definition}
\begin{finale}
    \begin{proposition}
        电压和电流的有效值分别为
        \[ U\+_rms_ = \frac{U}{\sqrt{2}},\quad I\+_rms_ = \frac{I}{\sqrt{2}}. \]
    \end{proposition}
\end{finale}

% subsubsection 有效值 (end)

\subsubsection{阻抗与幅角} % (fold)
\label{ssub:阻抗与幅角}

\begin{definition}
    [阻抗与幅角]交流电路中元件的阻抗与幅角分别为
    \[ Z = \frac{U\+_m_}{I\+_m_} = \frac{U\+_rms_}{I\+_rms_}, \]
    \[ \varphi = \varphi\+_u_ - \varphi\+_i_. \]
\end{definition}

\begin{longtable}{|c|c|c|}
    \hline
    元件 & 阻抗 & 幅角 \\
    \hline
    \+:r6{\incfig{2cm}{triplet}} & \+:r2{\+$\omega L$} & \+:r2{\+$+\frac{\pi}{2}$} \\
    & & \\
    \cline{2-3}
    & \+:r2{\+$R$} & \+:r2{\+$0$} \\
    & & \\
    \cline{2-3}
    & \+:r2{\+$\rec{\omega C}$} & \+:r2{\+$-\frac{\pi}{2}$} \\
    & & \\
    \hline 
\end{longtable}

% subsubsection 阻抗与幅角 (end)

\subsubsection{功率} % (fold)
\label{ssub:功率}

\begin{definition}
    [视在功率]
    \[ S = I\+_rms_ U\+_rms_. \]
\end{definition}
\begin{definition}
    [有功电阻]
    \[ \+gR = Z\cos\varphi. \]
\end{definition}
\begin{finale}
    \begin{definition}
        [平均功率, 有功功率]
        \[ P = \expc{p} = I\+_rms_ U\+_rms_ \cos\varphi = I^2 Z\cos\varphi = I^2 \+gR. \]
    \end{definition}
\end{finale}


% subsubsection 功率 (end)

% subsection 物理量 (end)

% section 简谐交流电 (end)

\end{document}
