\documentclass[hidelinks]{ctexart}

\usepackage{van-de-la-illinoise}

\begin{document}

\section{反馈} % (fold)
\label{sec:反馈}

\newpoint{}反馈量以电压直接相加者谓串联反馈.
\newpoint{}反馈量以电流直接相加者谓并联反馈.
\newpoint{电压反馈} 反馈信号$x\+_f_$与输出电压成比例.
\newpoint{电流反馈} 反馈信号$x\+_f_$与输出电流成比例.
\newpoint{} 电压负反馈
\[ R\+_L_\downarrow \Rightarrow v\+_o_\downarrow \Rightarrow x\+_f_\downarrow \Rightarrow x\+_id_\uparrow \Rightarrow v\+_o_\uparrow. \]
可生成稳定的输出.
\newpoint{} 电流负反馈可形成稳定的输出电流.
\newpoint{} 将负载短路, 反馈量为零, 则为电压反馈.
\newpoint{} 将负载短路, 反馈量仍然存在, 则为电流反馈.

\subsection{基本组态} % (fold)
\label{sub:基本组态}

\subsubsection{电压串联负反馈} % (fold)
\label{ssub:电压串联负反馈}

% subsubsection 电压串联负反馈 (end)

\subsubsection{电压并联负反馈} % (fold)
\label{ssub:电压并联负反馈}

% subsubsection 电压并联负反馈 (end)

\subsubsection{电流串联负反馈} % (fold)
\label{ssub:电流串联负反馈}

% subsubsection 电流串联负反馈 (end)

\subsubsection{电流并联负反馈} % (fold)
\label{ssub:电流并联负反馈}

% subsubsection 电流并联负反馈 (end)

% subsection 基本组态 (end)

\subsection{增益的一般表达式} % (fold)
\label{sub:增益的一般表达式}

\newpoint{}$\displaystyle A = \frac{x\+_o_}{x\+_id_}$谓开环增益.
\newpoint{}$\displaystyle F = \frac{x\+_f_}{x\+_o_}$谓反馈系数.
\newpoint{}$\displaystyle A\+_f_ = \frac{x\+_o_}{x\+_i_}$谓闭环增益.
\newpoint{}$x\+_id_ = x\+_i_ - x\+_f_ \Rightarrow x\+_i_ = x\+_id_ + x\+_f_\Rightarrow$ 
\[ A\+_f_ = \frac{x\+_o_}{x\+_id_ + x\+_f_} = \resumath{\frac{A}{1+AF}.} \]
\newpoint{}$F$随反馈类型的不同具有不同的量纲.
\newpoint{}$1+AF$谓\gloss{反馈深度}.
\begin{cenum}
    \item $\abs{1+\tilde{A}\tilde{F}} > 1$谓一般负反馈.
    \item $\abs{1+\tilde{A}\tilde{F}} \gg 1$谓深度负反馈.
    \item $\abs{1+\tilde{A}\tilde{F}} < 1$谓正反馈.
    \item $\abs{1+\tilde{A}\tilde{F}} = 0$发生自激振荡.
\end{cenum}

% subsection 增益的一般表达式 (end)

\subsection{负反馈对放大电路性能的影响} % (fold)
\label{sub:负反馈对放大电路性能的影响}

\newpoint{}提高增益的稳定性:
\[ \+dAd{A\+_f_} = \rec{1+AF}\frac{\rd{A}}{A}. \]
\newpoint{}减少环内非线性失真.
\newpoint{}抑制反馈环内噪声, 信噪比提高$A\+_\mathnormal{v}2_$倍.
\newpoint{}引入串联负反馈后, $R\+_if_ = \pare{1+AF}R\+_i_$.
\newpoint{}引入并联负反馈后, $R\+_if_ = R\+_i_/\pare{1+AF}$.
\newpoint{}引入电压负反馈后, $R\+_of_ = R\+_o_/\pare{1+A\+_o_F}$.
\newpoint{}引入电流负反馈后, $R\+_of_ = \pare{1+A\+_s_F}R\+_o_$.
\newpoint{}闭环上限频率$f\+_Hf_ = \pare{1+\tilde{A}\+_M_F}f\+_H_$.
\newpoint{}闭环下限频率$\displaystyle f\+_Lf_ = \frac{f\+_L_}{1+\tilde{A}\+_M_F}$.

% subsection 负反馈对放大电路性能的影响 (end)

\subsection{深度负反馈} % (fold)
\label{sub:深度负反馈}

此时$\abs{1+\tilde{A}\tilde{F}}\gg 1$, 从而
\[ \tilde{A}\+_f_ = \frac{\tilde{A}}{1+\tilde{A}\tilde{F}} \approx \rec{\tilde{F}}. \]
深度负反馈条件下, 闭环增益仅与反馈网络有关.
\newpoint{}串联负反馈
\[ \begin{cases}
    v\+_id_ = v\+_i_ - v\+_f_ \approx 0, \\
    \displaystyle i\+_id_ = \frac{v\+_id_}{r\+_i_} \approx 0.
\end{cases} \]
\newpoint{}并联负反馈
\[ \begin{cases}
    i\+_id_ = i\+_i_ - i\+_f_ \approx 0,\\
    v\+_id_ = i\+_id_r\+_i_ \approx 0.
\end{cases} \]
\begin{ex}
    电压串联负反馈$\displaystyle A\+_\mathnormal{v}f_ = \frac{v\+_o_}{v\+_i_} = 1+\frac{R\+_f_}{R\+_b2_}$.
\end{ex}
\begin{ex}
    电流并联负反馈$\displaystyle A\+_\mathnormal{v}f_ = \frac{v\+_o_}{v\+_s_} = -\pare{1+\frac{R\+_f_}{R}}\frac{R\+_L_}{R\+_s_}$.
\end{ex}
\begin{ex}
    电压并联负反馈$\displaystyle A\+_\mathrm{r}f_ = -R\+_f_$.
\end{ex}

% subsection 深度负反馈 (end)

% section 反馈 (end)

\end{document}
