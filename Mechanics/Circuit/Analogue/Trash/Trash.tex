\documentclass[hidelinks]{ctexart}

\usepackage{van-de-la-illinoise}

\begin{document}

\subsubsection*{模电概念题} % (fold)
\label{ssub:模电概念题}

\newpoint{Th\'evenin等效电路} 将电路等效为理想电压源和源内阻$R\+_si_$串联的等效形式.
\newpoint{Norton等效电路} 将电路等效为理想电流源和源内阻$R\+_si_$并联的等效形式.
\newpoint{幅度频谱} 信号各频率分量的振幅随角频率变化的分布.
\newpoint{相位频谱} 信号各频率分量的相位随角频率变化的分布.
\newpoint{频谱图} \emph{Not implemented}.
\newpoint{电压放大电路} 主要考虑电压增益的电路.
\newpoint{电流放大电路} 主要考虑输出电流和输入电流的关系的电路.
\newpoint{互阻放大电路} 将电流信号转换为电压信号的电路.
\newpoint{互导放大电路} 将电压信号转化为响应的电流输出的电路.
\newpoint{幅频响应} 电压增益的模和角频率之间的关系.
\newpoint{相频响应} 输入正弦电压信号的相位差与角频率之间的关系.
\newpoint{上限(截止)频率} 频率响应的高端半功率点. 即电压增益下降为中频值的$0.707$倍的点.
\newpoint{下限(截止)频率} 频率响应的低端半功率点. 即电压增益下降为中频值的$0.707$倍的点.
\newpoint{幅度失真} 不同频率信号的幅值放大倍数不同而产生的失真.
\newpoint{相位失真} 不同频率的信号产生的时延不同而产生的失真.
\newpoint{频率失真} 幅度失真和相位失真的总称.
\newpoint{线性失真} 由线性电抗元件引起的失真, 频率失真的别称.
\newpoint{非线形失真} 由元器件的非线形特性造成的失真. 由于晶体管输入特性的非线性和动态时管子工作点进入饱和区和截止区所引起的输出波形失真.
\newpoint{传输特性曲线} 描述放大电路输出量和输入量关系的曲线.
\newpoint{本征激发} 在一定温度以上, 被束缚的价电子获得随机热振动的能量而挣脱共价键的束缚, 产生电子-空穴对.
\newpoint{空间电荷区(耗尽层)} 自动电子从N区向P区扩散, 导致P区失去空穴而N区失去电子, 留下带正电的不能移动的离子, 集中在P区和N区的交界面附近, 形成PN结.
\newpoint{正向偏置} 外加电压正端接半导体P区, 负端接N区.
\newpoint{反向偏置} 外加电压正端接半导体N区, 负端接P区.
\newpoint{PN结导通} 正向的PN结表现为一个阻值很小的电阻.
\newpoint{PN结截止} PN结在反向偏置时呈现出一阻值很大的电阻, 可认为它基本上不导电.
\newpoint{反向击穿} PN结两端的电压增加到一定数值时, 反向电流突然增加.
\newpoint{反向击穿电压} 发生击穿所需的反向电压.
\newpoint{最大整流电流} 二极管长期运行时, 允许通过的最大正向平均电流.
\newpoint{碰撞电离} 产生漂移运动的少数载流子通过空间电荷区时, 在很强的电场作用下获得足够的动能, 与晶体原子发生碰撞, 从而打破共价键的束缚, 形成更多的自由电子-空穴对, 谓碰撞电离.
\newpoint{倍增效应} 新产生的电子和空穴与原有的电子和空穴一样, 在强电场的作用下获得足够的能量, 继续碰撞电离, 再产生电子-空穴对.
\newpoint{雪崩击穿} 反向电压增大到某一数值后, 倍增效应导致载流子快速大量增加, 使反向电流极具增大.
\newpoint{扩散电容} PN结正向偏置时, P区的空穴向N区扩散, 到达N区的空穴在靠近结边缘的浓度高于距离结稍远处的浓度, 可以视为有电荷储存到PN结的邻域. 空穴或电子的扩散运动在结附近产生的电和增量比与外加正向电压的增量之比为扩散电容.
\newpoint{势垒电容}外加电压增加, 多数载流子被拉出而远离PN结, 势垒区域增宽, 其内储存的正负电荷将增加.
\newpoint{二极管的理想模型} $I$-$V$特性在正向偏置时一条与纵轴重合的垂线, 反向偏置时是一条与横轴重合的水平线.
\begin{pitfall}
    理想模型的二极管以黑色实心符号表示.
\end{pitfall}
\newpoint{恒压降模型} 认为二极管导通后, 管压降是恒定的.
\newpoint{折线模型} 认为二极管的正向管压降不适恒定的, 在等效电路中用电源和电阻作近似.
\newpoint{小信号模型} 在静态工作点$Q$附近的小范围内将二极管的$I$-$V$特性近似为以$Q$为切点的一条直线, 其斜率的倒数为小信号模型的微变电阻$r\+_d_$.
\newpoint{预夹断} 当$v\+_DS_$增加, 靠近漏端的电位差最大, 耗尽层最宽. 当两耗尽层相遇时, 谓发生预夹断.
\newpoint{夹断电压} 发生预夹断时, 耗尽层两边的电位差, 即此时的$v\+_GD_$.
\newpoint{输出特性} $v\+_GS_$一定的情形下, 漏极电流$i\+_D_$与漏源电压$v\+_DS_$之间的关系.
\newpoint{转移特性} $v\+_DS_$一定的情形下, 栅源电压$v\+_GS_$对漏极电流$i\+_D_$的控制特性.
\newpoint{BJT结构特点} 基区宽度很薄, 掺杂浓度很低, 发射区和集电区是同类型的杂志半导体, 但前者比后者的掺杂浓度高的多, 且集电结的面积大于发射结的面积, 因此它们不是电气对称的.
\newpoint{共射输入特性曲线} 当$v\+_CE_$固定时, 输入电流$i\+_B_$与输入电压$v\+_BE_$之间的关系.
\newpoint{共射输出特性曲线} 当$i\+_B_$固定时, 集电极电流$i\+_C_$与电压$v\+_CE_$之间的关系.
\newpoint{共基输入特性曲线} 当$v\+_CB_$固定时, 输入电流$i\+_E_$与输入电压$v\+_BE_$之间的关系.
\newpoint{共基输出特性曲线} 当$i\+_E_$固定时, 集电极电流$i\+_C_$与电压$v\+_CB_$之间的关系.
\newpoint{共集放大电路特点} 输入电阻最高, 输出电阻最小, 频带宽. 可作多集放大电路的输入级, 输出级或缓冲级.
\newpoint{共基放大电路特点} 有电压放大作用, 输入输出电压相位相同, 没有电流放大作用, 有电流跟随作用. 可用于高频或宽频带电路.
\newpoint{共射放大电路} 既有电压放大作用又有电流放大作用, 输出电压和输入电压相位相反, 输入电阻在三种组态居中, 输出电阻较大, 适合低频情况下作多级放大电路中间级.
\newpoint{差模信号} 差分式放大电路两输入端信号的差值部分. $v\+_id_ = v\+_i1_ - v\+_i2_$.
\newpoint{共模信号} 两输入端信号的公共部分, $v\+_ic_ = \pare{v\+_i1_ + v\+_i2_}/2$.
\newpoint{零点漂移} 当放大电路的输入端短路时, 输出端还有缓慢变化的电压产生, 即输出电压偏离原来的起始点而上下漂动.
\newpoint{差模输入} 在电路的两个输入端各加一个大小相等, 极性相反的电压信号.
\newpoint{单端输出} 输出电压取自其中一管的漏极/集电极.
\newpoint{单端输入} 放大电路的输入端口有一端接地.
\newpoint{反馈} 电路输出电量(电压或电流)的一部分或全部通过反馈网络, 以一定方式送回到输入回路, 以影响输入和输出电量(电压或电流)的过程.
\newpoint{直流反馈} 存在于放大电路的直流通路中的反馈.
\newpoint{交流反馈} 存在于放大电路的交流通路中的反馈.
\newpoint{开环} 没有反馈网络, 不能形成反馈, 此种状态谓开环.
\newpoint{闭环} 有反馈网络, 能形成反馈, 此种状态谓闭环.
\newpoint{负反馈} 反馈信号送回到放大电路的输入回路与原信号共同作用后, 净输入信号比没有引入反馈时减小了.
\newpoint{正反馈} 反馈信号送回到放大电路的输入回路与原信号共同作用后, 净输入信号比没有引入反馈时增加了.
\newpoint{串联反馈} 在放大电路的输入回路, 反馈网络的输出端口与基本放大电路的输入端口串联连接.
\newpoint{并联反馈} 在放大电路的输入回路, 反馈网络的输出端口与基本放大电路的输入端口并联连接.
\newpoint{电压反馈} 把输出电压的一部分或全部取出来回送到放大电路的输入回路.
\newpoint{电流反馈} 把输入电压的一部分或全部取出来回送到放大电路的输入回路.

% subsubsection 模电概念题 (end)

\end{document}
