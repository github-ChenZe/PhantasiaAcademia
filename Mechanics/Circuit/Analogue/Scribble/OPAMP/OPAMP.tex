\documentclass[hidelinks]{ctexart}

\usepackage{cmbright}
\usepackage{van-de-la-illinoise}
\usepackage{nccmath}
\usepackage[paperheight=297mm,paperwidth=240mm,top=.2in,left=.1in,right=.1in,bottom=.2in, landscape]{geometry}
\usepackage{tensor}
\usepackage{calc}
\usepackage{van-le-trompe-loeil}

\definecolor{graybg}{RGB}{242,241,236}
\definecolor{titlepurple}{RGB}{138,47,57}
\definecolor{shadegray}{RGB}{102,119,136}
\definecolor{itemgray}{RGB}{163,149,128}
\definecolor{mathnormalblack}{RGB}{0,0,0}
\pagecolor{graybg}

\setCJKmainfont{STHeitiSC-Light}
\setmainfont{Arial}
\newcommand*{\mysans}{\fontfamily{phv}\selectfont}

\usepackage{multicol}
\setlength{\columnsep}{.1in}

\newcommand{\raisedrule}[2][0em]{\qquad}
%\leaders\hbox{\rule[#1]{1pt}{#2}}\hfill}
\newcommand{\wdiv}{\,·\,}

\setlength{\parindent}{0pt}

\setCJKfamilyfont{pfsc}{STYuanti-SC-Regular}
\newcommand{\titlefont}{\CJKfamily{ttt}}
\setCJKfamilyfont{ttt}{STFangsong}
\newcommand{\mathtextfont}{\CJKfamily{ttt}}
\newcommand{\emphbox}[1]{\colorbox{lightgray!20}{$\displaystyle #1$}}

\newdimen\indexlen
\def\newheader#1{%
\def\probindex{#1}
\setlength\indexlen{\widthof{\Large\color{titlepurple} #1\qquad}}
\vspace{1em}
{\Large\color{titlepurple} #1\qquad}
\raisebox{.5em}{\tikz \fill[titlepurple,opacity=.2,path fading=east] (0,0.05em) rectangle (\dimexpr\linewidth-\indexlen\relax,0em);}
}
\def\mathitem#1{\text{\color{itemgray}#1}}
\def\mathcomment#1{\text{\color{lightgray}\quad \texttt{\#}\kern-0pt#1}}
\def\mathheadcomment#1{\text{\color{lightgray}\texttt{\#}\kern-0pt#1}}
\def\midbreak{\smash{\raisebox{1.5em}{\smash{\tikz \path[opacity=.2,left color=white,right color=white,middle color=black] (0,0.05em) rectangle (\linewidth,0em);}}}
\vspace{-4em}}
\newtcolorbox{cheatresume}{enhanced, arc=.5pt, left=.5em, frame hidden, boxrule=0pt, colback=white, fuzzy halo=.05pt with lightgray, shadow={.4pt}{-.4pt}{0pt}{fill=shadegray,opacity=0.3}}

\usepackage{stackengine}
\stackMath
\usepackage{scalerel}
\usepackage[outline]{contour}

\newlength\thisletterwidth
\newlength\gletterwidth
\newcommand{\leftrightharpoonup}[1]{%
{\ooalign{$\scriptstyle\leftharpoonup$\cr%\kern\dimexpr\thisletterwidth-\gletterwidth\relax
$\scriptstyle\rightharpoonup$\cr}}\relax%
}
\def\tensor#1{\settowidth\thisletterwidth{$\mathbf{#1}$}\settowidth\gletterwidth{$\mathbf{g}$}\stackon[-0.1ex]{\mathbf{#1}}{\boldsymbol{\leftrightharpoonup{#1}}}  }
\def\mitensor#1{\stackon[-0.1ex]{\+v#1}{\boldsymbol{\leftrightharpoonup{#1}}} }
\def\onedot{$\mathsurround0pt\ldotp$}
\def\cddot{% two dots stacked vertically
:}%
\definecolor{emphgreen}{RGB}{238,255,207}
%\newcommand{\resume}[1]{\par
%\noindent\colorbox{emphgreen}{#1}}
\definecolor{CJKblack}{RGB}{72,72,72}

\usetikzlibrary{calc}

\def\equals{=}
\newcommand{\ib}[2]{to[battery1,l=$ #1 $, invert, #2]}
\newcommand{\ic}[2]{to[capacitor,C=$ #1 $, #2]}
\newcommand{\ivs}[2]{to[european voltage source,v<=$ #1 $, #2]}
\newcommand{\iis}[2]{to[european current source,i=$ #1 $, #2]}
\newcommand{\ir}[2]{to[resistor,R=$ #1 $, #2]}
\newcommand{\iso}[1]{to[short,-o,#1]}
\newcommand{\ios}[1]{to[short,o-,#1]}
\newcommand{\is}[1]{to[short,#1]}
%\newcommand{\iu}[3]{\draw #1  #2}
\newcommand{\iou}[3]{\draw ($#1!0.5!#2$) node {$ #3 $};
\draw ($#1!0.12!#2$) node {$\scriptstyle +$};
\draw ($#1!0.88!#2$) node {$\scriptstyle -$};}
\newcommand{\iblock}[3]{\draw[thick] #1 |- #2 |- cycle;
\draw ($#1!0.5!#2$) node {#3};}

\def\lefthandwidth{3cm}%
\newtcolorbox{cheatresumefuta}{enhanced, arc=.5pt, left=.5em, frame hidden, boxrule=0pt, colback=white, fuzzy halo=.05pt with lightgray, shadow={.4pt}{-.4pt}{0pt}{fill=shadegray,opacity=0.3}, sidebyside, lefthand width=\lefthandwidth, sidebyside align=top, lower separated=false}%

\pgfplotsset{compat=newest}
\usepackage{vwcol}  

\begin{document}
\begin{multicols*}{2}[\centerline{\titlefont オペアンプ}]
\raggedcolumns%
\newheader{差動増幅回路}
\begin{cheatresume}
\begin{tabular}{c>{\centering\arraybackslash}p{5.6cm}>{\centering\arraybackslash}p{5.6cm}}
\+:c1c{\color{titlepurple}バイポーラ} & \+:c1c{\color{titlepurple}2つの出力} & \+:c1c{\color{titlepurple}単一出力} \\
\begin{tikzpicture}[yscale=0.85]
        \draw (0,0) node {}
        (0,4) node{\mathitem{基本回路}};
    \end{tikzpicture} & \begin{tikzpicture}[yscale=0.85,xscale=0.75]
        \draw 
            (0,1) node(EE) {}
            (-2,3) node(C1) {}
            (2,3) node(C2) {}
            (0,5) node(CC) {}
            (C1) node[npn,anchor=D] (T1) {}
            (C2) node[npn,anchor=D,xscale=-1] (T2) {}
            (T1.S) -- (T1.S|-EE) to[short,-*] (EE)
            (T2.S) -- (T2.S|-EE) to[short] (EE)
            (EE) to[european current source,i_=$I\+_O_$,-o]
            ++ (0,-2) node[right] {$-V\+_EE_$}
            (T1.D) to[short,*-o]
            (T1.D-|-1,0) to[resistor,R=$R\+_L_$,v=$v\+_o_$]
            (T2.D-|1,0) to[short,o-*]
            (T2.D)
            (T1.D) to[resistor,R=$R\+_c1_$]
            (T1.D|-CC) to[short,-*]
            (CC) --
            (T2.D|-CC) to[resistor,R=$R\+_c2_$]
            (T2.D)
            (CC) to[short,-o]
            ++(0,1) node[right] {$+V\+_CC_$}
            (T1.G) to[open,o-o]
            ++ (0,-2) to[short] ++ (0,-0.06) node[rground] (g1) {}
            (T2.G) to[open,o-o]
            ++ (0,-2) to[short] ++ (0,-0.06) node[rground] (g2) {}
            (T1.G) node[below] {$+$}
            (g1) node[above] {$-$}
            (T2.G) node[below] {$+$}
            (g2) node[above] {$-$}
            ($(T1.G)!0.5!(g1)$) node {$v\+_i1_$}
            ($(T2.G)!0.5!(g2)$) node {$v\+_i2_$}
            ;
    \end{tikzpicture} & \begin{tikzpicture}[yscale=0.85,xscale=0.75]
        \draw 
            (0,1) node(EE) {}
            (-2,3) node(C1) {}
            (2,3) node(C2) {}
            (0,5) node(CC) {}
            (C1) node[npn,anchor=D] (T1) {}
            (C2) node[npn,anchor=D,xscale=-1] (T2) {}
            (T1.S) -- (T1.S|-EE) to[short,-*] (EE)
            (T2.S) -- (T2.S|-EE) to[short] (EE)
            (EE) to[european current source,i_=$I\+_O_$,-o]
            ++ (0,-2) node[right] {$-V\+_EE_$}
            (T1.D) to[short,*-o]
            (T1.D-|-1,0) to[resistor,R=$R\+_L_$,v=$v\+_o1_$,-o]
            ++ (2,0) -- ++(0,-1) node[rground] {}
            (T1.D) to[resistor,R=$R\+_c1_$]
            (T1.D|-CC) to[short,-*]
            (CC) --
            (T2.D|-CC) to[resistor,R=$R\+_c2_$]
            (T2.D)
            (CC) to[short,-o]
            ++(0,1) node[right] {$+V\+_CC_$}
            (T1.G) to[open,o-o]
            ++ (0,-2) to[short] ++ (0,-0.06) node[rground] (g1) {}
            (T2.G) to[open,o-o]
            ++ (0,-2) to[short] ++ (0,-0.06) node[rground] (g2) {}
            (T1.G) node[below] {$+$}
            (g1) node[above] {$-$}
            (T2.G) node[below] {$+$}
            (g2) node[above] {$-$}
            ($(T1.G)!0.5!(g1)$) node {$v\+_i1_$}
            ($(T2.G)!0.5!(g2)$) node {$v\+_i2_$}
            ;
    \end{tikzpicture}\\
    \mathitem{差動利得} & $\displaystyle A\+_\mathnormal{v}d_ = \frac{v\+_o_}{v\+_id_} = -\frac{\displaystyle \beta\pare{R\+_c_\parallel \frac{R\+_L_}{2}}}{r\+_be_}$ & $\hspace{-.3cm}\displaystyle A\+_\mathnormal{v}d1_ = \frac{v\+_o1_}{v\+_id_} = -\frac{v\+_o2_}{v\+_id_} = -\frac{\displaystyle \beta\pare{R\+_c_\parallel {R\+_L_}}}{2r\+_be_}$ \\
    \mathitem{同相利得} & $\displaystyle A\+_\mathnormal{v}c_ \rightarrow 0$ & $\displaystyle A\+_\mathnormal{v}c1_ \approx -\frac{R\+_c_\parallel R\+_L_}{2r\+_O_}$ \\
    \mathitem{差動抵抗} & \+:c2c{$R\+_id_ = 2r\+_be_$} \\
    \mathitem{同相抵抗} & \+:c2c{$\displaystyle R\+_ic_ = \half r\+_be_ + \pare{1+\beta}r\+_O_$} \\
    \mathitem{出力抵抗} & $R\+_o_ = 2R\+_c_$ & $R\+_o_ = R\+_c_$ \\
\+:c3c{\tikz \path[opacity=.2,left color=white,right color=white,middle color=black] (0,0.05em) rectangle (\linewidth,0em);} \\
\+:c1c{\color{titlepurple}FET} & \begin{tikzpicture}[baseline={([yshift={-\ht\strutbox}]current bounding box.north)},yscale=0.85,xscale=0.75]
        \draw 
            (-2,3) node(C1) {}
            (2,3) node(C2) {}
            (0,5) node(CC) {}
            (C1) node[njfet,anchor=D] (T1) {}
            (C2) node[njfet,anchor=D,xscale=-1] (T2) {}
            (T1.D) to[short,*-o]
            (T1.D-|-1,0) to[resistor,R=$R\+_L_$,v=$v\+_o_$]
            (T2.D-|1,0) to[short,o-*]
            (T2.D)
            (T1.D) to[resistor,R=$R\+_d1_$]
            (T1.D|-CC)
            (T2.D|-CC) to[resistor,R=$R\+_d2_$]
            (T2.D)
            ;
    \end{tikzpicture} & \begin{tikzpicture}[baseline={([yshift={-\ht\strutbox}]current bounding box.north)},yscale=0.85,xscale=0.75]
        \draw 
            (-2,3) node(C1) {}
            (2,3) node(C2) {}
            (0,5) node(CC) {}
            (C1) node[njfet,anchor=D] (T1) {}
            (C2) node[njfet,anchor=D,xscale=-1] (T2) {}
            (T1.D) to[short,*-o]
            (T1.D-|-1,0) to[resistor,R=$R\+_L_$,v=$v\+_o1_$,-o]
            ++ (2,0) -- ++(0,-1) node[rground] {}
            (T1.D) to[resistor,R=$R\+_d1_$]
            (T1.D|-CC)
            (T2.D|-CC) to[resistor,R=$R\+_d2_$]
            (T2.D)
            ;
    \end{tikzpicture} \\
    \mathitem{差動利得} & $\displaystyle A\+_\mathnormal{v}d_ = \frac{v\+_o_}{v\+_id_} = -g\+_m_\pare{R\+_d_\parallel \frac{R\+_L_}{2}}$ & $\hspace{-.5cm}\displaystyle A\+_\mathnormal{v}d1_ = \frac{v\+_o1_}{v\+_id_} = -\frac{v\+_o2_}{v\+_id_} = -\frac{g\+_m_\pare{R\+_d_\parallel R\+_L_}}{2}$ \\
    \mathitem{同相利得} & $\displaystyle A\+_\mathnormal{v}c_ \rightarrow 0$ & $\hspace{-.5cm}\displaystyle A\+_\mathnormal{v}c1_ \approx -\frac{g\+_m_\pare{R\+_d_\parallel R\+_L_}}{1+2g\+_m_r\+_o_} \approx -\frac{R\+_d_\parallel R\+_L_}{r\+_o_}$ \\
    \mathitem{入力抵抗} & \+:c2c{$R\+_id_ \rightarrow \infty,\quad R\+_ic_ \rightarrow \infty$} \\
    \mathitem{出力抵抗} & $R\+_o_ = 2R\+_d_$ & $R\+_o_ = R\+_d_$ \\
    \+:c3c{\tikz \path[opacity=.2,left color=white,right color=white,middle color=black] (0,0.05em) rectangle (\linewidth,0em);} \\
\mathitem{注意} & & \+:c1l{\mathheadcomment{$\mathrm{C_1}$出力と$\mathrm{C_2}$出力の場合には、}} \\
    & & \+:c1l{\mathcomment{$v\+_o_$の符号が反対である}}
\end{tabular}
\end{cheatresume}
\columnbreak
\newheader{負帰還増幅回路}
\begin{cheatresume}
\ctikzset{tripoles/op amp/width=1.25}
\ctikzset{tripoles/op amp/height=1}
    \begin{tabular}{@{}c>{\centering\arraybackslash}p{4.6cm}>{\centering\arraybackslash}p{6.6cm}}
        {\color{lightgray}出\textbackslash 入} & \+:c1c{{\color{titlepurple}直列}\ $\displaystyle R\+_i_ \mapsto R\+_i_\pare{1+AF}$} & \+:c1c{{\color{titlepurple}並列} $\displaystyle R\+_i_\mapsto R\+_i_/\pare{1+AF}$} \\
    \begin{tikzpicture}[yscale=0.7]
        \draw (0,0) node {}
        (0,3.5) node (denatsu) {\color{titlepurple}{電圧}};
        \draw node[below=0em of denatsu.south west,anchor=north west] {\makebox[0pt][l]{$\substack{\displaystyle R\+_o_ \mapsto\\[.5em] \displaystyle \frac{R\+_o_}{1+A\+_o_F} }$}};
    \end{tikzpicture} & \begin{tikzpicture}[scale=0.55]
        \draw
        (0,0) node[op amp,yscale=-1] (opamp) {}
        (opamp.+) to[short,-o] ++(-0.5,0) node[left] {$v\+_i_$}
        (opamp.-) to[short] ++ (0,-1) node(n) {}
        to[resistor,R=$R_1$,v=$v\+_f_$] ++(0,-2)
        node[rground]{}
        (opamp.out) to[short] ++(0.5,0) node(o) {}
        to[resistor,R=$R\+_L_$,*-] ++(0,-2)
        node[rground] (og) {}
        (o) to[short,-o] ++(0.5,0) node[right] {$v\+_o_$}
        (opamp.out) to[short,*-]
        (opamp.out|-n) to[resistor,R=$R_2$,-*]
        (n)
        ;
        \draw (3,-3.2) node {$\begin{array}{@{}l}
            \displaystyle F = \frac{U\+_f_}{U\+_o_} = \frac{R_1}{R_1 + R_2}
        \end{array}$};
    \end{tikzpicture} & \begin{tikzpicture}[scale=0.55]
        \draw
        (0,0) node[op amp] (opamp) {}
        (opamp.-) to[resistor,R=$R\+_s_$,-o] ++(-2,0) node[left] {$v\+_i_$}
        (opamp.out) to[short] ++(0.5,0) node(o) {}
        to[resistor,R=$R\+_L_$,*-] ++(0,-2)
        node[rground] (og) {}
        (o) to[short,-o] ++(0.5,0) node[right] {$v\+_o_$}
        (opamp.+) -- (opamp.+|-og)
        node[rground]{}
        (opamp.-) to[short,*-]
        ++(0,1) node(nabove){} to[resistor,R=$R\+_f_$]
        (nabove-|o) to[short]
        (o)
        ;
        \draw (5,-1) node {$\begin{array}{@{}l}
            \displaystyle F = \rec{R\+_f_}
        \end{array}$};
    \end{tikzpicture} \\
    \begin{tikzpicture}[yscale=0.7]
        \draw (0,0) node {}
        (0,4) node (denryuu) {\color{titlepurple}{電流}};
        \draw node[below=0em of denryuu.south west,anchor=north west] {\makebox[0pt][l]{$\substack{\displaystyle R\+_o_ \mapsto\\[.5em] \displaystyle {R\+_o_}\pare{1+A\+_o_F} }$}};
    \end{tikzpicture} & \begin{tikzpicture}[scale=0.55]
        \draw
        (0,0) node[op amp,yscale=-1] (opamp) {}
        (opamp.+) to[short,-o] ++(-0.5,0) node[left] {$v\+_i_$}
        (opamp.-) to[short] ++ (0,-1.5) node(n) {}
        to[resistor,R=$R_1$,v=$v\+_f_$] ++(0,-2)
        node[rground](ng){}
        (opamp.out) to[short] ++(0.5,0) node(o) {}
        to[resistor,l_=$R\+_L_$,*-*] ++(0,-2)
        node (og) {} to[short,-*]
        (og-|ng)
        (o) to[short,-o] ++(1,0) node[below] {$+$}
        (og) to[short,-o] ++(1,0) node[above] {$-$}
        ($(o)!0.5!(og)$) ++(1,0) node {$v\+_o_$}
        ;
        \draw (3,-3.2) node {$\begin{array}{@{}l}
            \displaystyle F = R_1
        \end{array}$};
    \end{tikzpicture} & \begin{tikzpicture}[scale=0.55]
        \draw
        (0,0) node[op amp,yscale=-1] (opamp) {}
        (opamp.-) to[resistor,R=$R\+_s_$,-o] ++(-2,0) node[left] {$v\+_i_$}
        (opamp.out) to[short] ++(0.5,0) node(o) {}
        to[resistor,l_=$R\+_L_$,*-] ++(0,-2)
        node (vop) {}
        (opamp.+) node[rground]{}
        (opamp.-) to[short,*-]
        (opamp.- |- vop) to[resistor,R=$R_1$,-*]
        (vop) to[resistor,R=$R_2$] ++(0,-2)
        node[rground] (og) {}
        (o) to[short,-o] ++(1,0) node[below] {$+$}
        (vop) to[short,-o] ++(1,0) node[above] {$-$}
        ($(o)!0.5!(vop)$) ++(1,0) node {$v\+_o_$}
        ;
        \draw (6,-1) node {$\begin{array}{@{}l}
            \displaystyle F = \frac{R_2}{R_1 + R_2}
        \end{array}$};
        \draw (-1,-3.5) node (exprA) {$\begin{array}{@{}l}
            \displaystyle A \mapsto \frac{A}{1+AF} \approx \rec{F}
        \end{array}$};
        \draw[lightgray,-latex] (exprA.west) ++ (-0.6,0) -- (exprA.west);
        \draw[lightgray,-latex] (exprA.north west) ++ (0,0.6) -- ($(exprA.north west) + (0.6,0)$);
        \draw[lightgray,-latex] ($(exprA.north) + (-1,0)$) ++ (0,0.6) -- ($(exprA.north) + (-1,0)$);
    \end{tikzpicture}
    \end{tabular}
\end{cheatresume}
\newheader{演算回路}
\begin{cheatresume}
\ctikzset{tripoles/op amp/width=1.25}
\ctikzset{tripoles/op amp/height=1}
    \begin{tabular}{ccc}
        %\begin{tikzpicture}[scale=0.55]
        %\draw
        %(0,0) node[op amp] (opamp) {}
        %(opamp.-) to[resistor,R=$R\+_s_$,-o] ++(-2,0) node[left] {$v\+_i_$}
        %(opamp.out) node(o) {}
        %(o) to[short,-o] ++(0.5,0) node[right] {$v\+_o_$}
        %(opamp.+) node[rground]{}
        %(opamp.-) to[short,*-]
        %++(0,1) node(nabove){} to[resistor,R=$R\+_f_$]
        %(nabove-|o) to[short,-*]
        %(o)
        %;
    %\end{tikzpicture} & 
    \begin{tikzpicture}[scale=0.55]
        \draw
        (0,0) node[op amp] (opamp) {}
        (opamp.-) to[resistor,R=$R$,-o] ++(-2,0) to[short] ++(-0.5,0) node[rground] {}
        (opamp.+) to[resistor,R=$R\+_s_$,-o] ++(-2,0) node[left] {$v\+_i_$}
        (opamp.out) node(o) {}
        (o) to[short,-o] ++(0.5,0) node[right] {$v\+_o_$}
        (opamp.-) to[short,*-]
        ++(0,1) node(nabove){} to[resistor,R=$R\+_f_$]
        (nabove-|o) to[short,-*]
        (o)
        ;
    \end{tikzpicture} & \begin{tikzpicture}[scale=0.55]
        \draw
        (0,0) node[op amp] (opamp) {}
        (opamp.-) to[short] ++(-0.5,0) to[resistor,R=$R\+_1_$,*-o] ++(-2,0) node[left] {$v\+_i1_$}
        (opamp.-) ++(-0.5,0) to[short] ++(0,1) to[resistor,l_=$R\+_2_$,-o] ++(-2,0) node[left] {$v\+_2_$}
        (opamp.out) node(o) {}
        (o) to[short,-o] ++(0.5,0) node[right] {$v\+_o_$}
        (opamp.+) node[rground]{}
        (opamp.-) to[short,*-]
        ++(0,1) node(nabove){} to[resistor,R=$R\+_f_$]
        (nabove-|o) to[short,-*]
        (o)
        ;
    \end{tikzpicture} & %\begin{tikzpicture}[scale=0.55]
        %\draw
        %(0,0) node[op amp] (opamp) {}
        %(opamp.-) to[resistor,R=$R\+_s_$,-o] ++(-2,0) node[left] {$v\+_i_$}
        %(opamp.out) node(o) {}
        %(o) to[short,-o] ++(0.5,0) node[right] {$v\+_o_$}
        %(opamp.+) node[rground]{}
        %(opamp.-) to[short,*-]
        %++(0,1) node(nabove){} to[capacitor,C=$C$]
        %(nabove-|o) to[short,-*]
        %(o)
        %;
    %\end{tikzpicture} \\
    %$\displaystyle v\+_o_ = -\frac{R\+_f_}{R}v\+_i_$ & 
    $\displaystyle \begin{array}[b]{l}
        \displaystyle C \rightarrow \rec{C}\int \rd{t} \\
        \displaystyle \rec{C} \rightarrow C\+dtd{}
    \end{array}$ \\
    $\displaystyle v\+_o_ = \pare{1+ \frac{R\+_f_}{R}}v\+_i_$ & $\displaystyle v\+_o_ = -R\+_f_\pare{\frac{v\+_i1_}{R_1} + \frac{v\+_i2_}{R_2}}$ &% $\displaystyle v\+_o_ = -\rec{RC}\int v\+_i_\,\rd{t}$
    \end{tabular}
\end{cheatresume}
\begin{multicols*}{2}
\raggedcolumns\def\lefthandwidth{1.5cm}%%
\newheader{{\mysans Wildar}電流源}
\begin{cheatresumefuta}
    \begin{center}
        \begin{tikzpicture}[yscale=0.65]
            \draw
                (0,0) node (BB) {}
                (BB) node[npn,anchor=G,xscale=-1,label={[label distance=.35cm]-75:{$\mathrm{T}_1$}}] (T1) {}
                (BB) node[npn,anchor=G,label={[label distance=.35cm]255:{$\mathrm{T}_2$}}] (T2) {}
                (0,0) to[short,*-]
                (0,0|-T1.D) to[short,-*]
                (T1.D) to[resistor,l_=$R$,-o]
                ++ (0,2) node[right] {$V\+_CC_$}
                (T2.S) to[resistor,l_=$R\+_e2_$]
                ++ (0,-2) node(E1) {}
                (T1.S) to[short]
                ++ (0,-2) node(E2) {}
                (E1) to[short,-*]
                ($(E1)!0.5!(E2)$) node[rground] {}
                (E2) to[short]
                ($(E1)!0.5!(E2)$)
                (0,0)
                (T2.D) to[short,i<=$I\+_O_$]
                ++ (0,1)
                ;
        \end{tikzpicture}
    \end{center}
    \tcblower
    \begin{flalign*}
        & \mathitem{交流抵抗} && \\
        & r\+_o_ = r\+_ce_\pare{1+\frac{\beta R\+_e2_}{r\+_be2_ + R\+_e2_}} && \\
        & \mathitem{出力電流} && \\
        & I\+_C_ \approx \frac{V\+_T_}{R\+_e_}\ln \frac{I\+_R_}{I\+_C_} && \\
        & \mathitem{カレントミラー} && \\
        & R\+_e_ = 0, \ I\+_C_ \approx I\+_B_,\ r\+_o_ = r\+_ce_ &&
    \end{flalign*}
\end{cheatresumefuta}
\columnbreak
\newheader{周波数特性}
\begin{cheatresume}
\begin{tikzpicture}[scale=0.6]
    \begin{axis}[
        at={(0,0)},
        axis lines=left,
        ylabel=\raisebox{6em}{$20\log\abs{\dot{A}_v}/\SI{}{\decibel}$},xmin=0,xmax=33,ymin=0,ymax=24,%xtick={0,4,8,12,16,20,24,28},xticklabels={$1^{\phantom{0}}$,$10^{\phantom{0}}$,$10^2$,$10^3$,$10^4$,$10^5$,$10^6$,$10^7$}
        xmajorticks=false,ytick={0,6.7,13.4,20.1},yticklabels={$0$, $20$, $40$, $60$}, %
        xlabel style={
        at={(current axis.right of origin)},
        anchor=north,
        right=6.5em,
        above=-2em
        },%
        ylabel style={
        at={(current axis.above origin)},
        anchor=west,
        rotate=-90,
        above=3em,
        right=0em
        },clip=false,height=5cm,width=8cm]
    \draw (0,0) -- (4,13.4) -- (8,20.1) -- (20,20.1) -- (24,13.4) -- (28,0);
    \draw[dashed] (24,0) -- (24,13.4);
    \draw[dashed] (20,0) -- (20,20.1);
    \draw[dashed] (4,0) -- (4,13.4);
    \draw[dashed] (8,0) -- (8,20.1);
    \draw (30,10) node[right] {\color{CJKblack}$\SI{20}{\decibel\per\text{10倍周波数}\per\text{段}}$};
    \end{axis}
    \begin{axis}[
        at={(0,-3.5cm)},
        axis lines=left,xlabel=\mbox{\hspace{14em}$f/\SI{}{\hertz}$},
        ylabel=\raisebox{6em}{$\varphi$},xmin=0,xmax=33,ymin=0,ymax=24,xtick={0,4,8,12,16,20,24,28},xticklabels={$1^{\phantom{0}}$,$10^{\phantom{0}}$,$10^2$,$10^3$,$10^4$,$10^5$,$10^6$,$10^7$},ytick={0,5,10,15,20},yticklabels={$\SI{-180}{\degree}$, $\SI{-90}{\degree}$, $\SI{0}{\degree}$, $\SI{+90}{\degree}$, $\SI{+180}{\degree}$}, %
        xlabel style={
        at={(current axis.right of origin)},
        anchor=north,
        right=6.5em,
        above=-2em
        },%
        ylabel style={
        at={(current axis.above origin)},
        anchor=west,
        rotate=-90,
        above=3em,
        right=0em
        },clip=false,height=5cm,width=8cm]
    \draw (0,20) -- (4,17.5) -- (8,12.5) -- (12,10) -- (16,10) -- (20,7.5) -- (24,2.5) -- (28,0);
    \draw[dashed] (24,0) -- (24,2.5);
    \draw[dashed] (20,0) -- (20,7.5);
    \draw[dashed] (4,0) -- (4,17.5);
    \draw[dashed] (8,0) -- (8,12.5);
    \draw (30,20) node[right] (cutoff) {\color{CJKblack}遮断周波数に$\SI{90}{\degree\per\text{段}}$};
    \draw (cutoff) node[below=1em] {\parbox{3.5cm}{\linespread{.25}\color{CJKblack}遮断周波数で45度となり, その後90度に向かって滑らかに変化していきます.}};
    \end{axis}
\end{tikzpicture}
\end{cheatresume}
\end{multicols*}

\end{multicols*} 

\end{document}
