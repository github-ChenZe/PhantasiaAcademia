\documentclass[hidelinks]{ctexart}

\usepackage{van-de-la-illinoise}

\begin{document}

\subsubsection*{配置} % (fold)
\label{ssub:配置}

\noindent
Wang Yonggang 63606499 wangyg@ustc.edu.cn\\
Zhao Zelong, Zhang Jianfeng(13076063853)\\
考核: 上课+作业(20\%$\sim$ 30\%)+期末考试(70\%$\sim$ 80\%)

% subsubsection 配置 (end)

\section{基础} % (fold)
\label{sec:基础}

\subsection{信号与系统} % (fold)
\label{sub:信号与系统}

\newpoint{电信号} 信号由电压或电流表示者. 电压和电流是电路中的两个基本变量. 基本属性由幅度, 频率和相位. 由这些基本属性表示信息.
\par
\newpoint{电路} 用来处理电信号的器件及其互联网络. 其目的在于在保证被处理信号所携带的信息不失真的前提下, 对信号进行正确的转换, 传输, 提取, 再现及其它深化处理.

% subsection 信号与系统 (end)

\subsection{模拟信号与数字信号} % (fold)
\label{sub:模拟信号与数字信号}

\newpoint{}模拟电路处理模拟信号, 数字电路处理数字信号.
\newpoint{}\gloss{模拟信号}谓在时间和幅值上都连续的信号.
\newpoint{}\gloss{数字信号}谓在时间和幅值上都是离散的信号.
\begin{ex}
    麦克风信号是模拟信号. 自然界中几乎所有信息都是模拟信号.
\end{ex}

% subsection 模拟信号与数字信号 (end)

\subsection{信号的频谱} % (fold)
\label{sub:信号的频谱}

\newpoint{}线性电路可以通过Th\'evenin等效和Norton等效转化为电压源/电流源等效电路.
\newpoint{}信号的时间域表示是信号作为对时间的函数. 信号还可以在频率域表示, 相应表示谓频谱.
\newpoint{正弦波的三要素} 幅度, 角频率, 初相位.
\begin{ex}
    周期$T$, 幅度$V_s$的方波可展开为
    \[ v\pare{t} = \frac{V_s}{2} + \frac{2V_s}{\pi} \pare{\sin\omega_0 t + \rec{3}\sin 3\omega_0 t + \rec{5}\sin 5\omega_0 t + \Lambda}. \]
    其中$\omega_0 = 2\pi/T$, $V_s/2$为直流分量, $2V_s/\pi$谓基波分量, $2V_s/\pi \cdot 1/3$谓三次谐波分量.
\end{ex}
\newpoint{}频谱谓信号在频率域的展开.
\newpoint{}周期信号FT后可得离散频率函数.
\newpoint{}非周期信号FT后可得连续频率函数.

% subsection 信号的频谱 (end)

\subsection{放大电路模型} % (fold)
\label{sub:放大电路模型}

\newpoint{}信号放大后可处理.
\newpoint{}通过信号源的等效表示, 负载的等效表示, 得到放大电路的符号表示. 放大电路一般为有源器件.
\newpoint{}信号放大可有度量
\begin{cenum}
    \item 电压增益$A_v = v_o / v_i$;
    \item 电流增益$A_i = i_o / i_i$;
    \item 互阻增益$A_r = v_o / i_i\pare{\SI{}{\ohm}}$;
    \item 互导增益$A_g = i_o / i_s\pare{\SI{}{\siemens}}$.
\end{cenum}
\newpoint{}通常将负载等效为电阻.
\paragraph{电压放大} % (fold)
\label{par:电压放大}

设$A_{vo}$为负载开路时的电压增益, $R_i$为输入电阻, $R_o$为输出电阻. 输出回路得
\[ v_o = A_{vo}v_i \frac{R_L}{R_o + R_L},\quad A_v = \frac{v_o}{v_i} = \frac{R_L}{R_o+R_L}. \]
为了减少负载的影响, 需要$R_o \ll R_L$. 减少输入回路的衰减, 需要$R_i \gg R_s$.

% paragraph 电压放大 (end)

\paragraph{电流放大} % (fold)
\label{par:电流放大}

设$A_{is}$为负载短路时的电流增益, 由输出回路,
\[ i_o = A_{is}i_i \frac{R_o}{R_o + R_L},\quad A_i = \frac{i_o}{i_i} = A_{is}\frac{R_o}{R_o+R_L}. \]
为了减少负载的影响, 需要$R_o \gg R_L$. 减少输入回路的衰减, 需要$R_i \ll R_s$.

% paragraph 电流放大 (end)

\paragraph{互阻/互导放大模型} % (fold)
\label{par:互阻_互导放大模型}

等效模型如何, 放大器本身输入阻抗与输出阻抗有何希望, 对信号源和负载有何希望?
\newpoint{}四种放大器模型可以相互转化. 即同一模型可用四种模型刻画. 例如大$R_i$小$R_o$的放大器宜用电压-电压放大模型刻画.
\newpoint{}隔离放大电路模型: 输入输出回路, 没有公共端.
\begin{ex}
    部份情况下输入和输出端口不应共地. 例如医疗仪器中, 为了防止漏电引发危险, 应当分离前后端.
\end{ex}

% paragraph 互阻/互导放大模型 (end)

% subsection 放大电路模型 (end)

\subsection{放大电路的主要性能指标} % (fold)
\label{sub:放大电路的主要性能指标}

\newpoint{}\gloss{输入阻抗}$\displaystyle R_i = \frac{v_t}{i_t}$. \gloss{输出阻抗}为
\[ R_o = \left.\frac{v_t}{i_t}\right\vert_{v_s = 0,R_L=\infty}. \]
计算输出阻抗时, 输入信号源置零, 保留信号源内阻.
\newpoint{}输入和输出阻抗可能是频率的函数(直流阻抗, 交流阻抗).
\newpoint{}增益描述输出电压, 电流或功率与输入电压, 电流或功率之间的比例关系.
\[ A_v = \frac{v_0}{v_o},\quad A_i = \frac{i_o}{i_i},\quad A_r = \frac{v_o}{i_i},\quad A_g = \frac{i_o}{v_i},\quad A_p = \frac{P_o}{P_i}. \]
\newpoint{}增益常用分贝表示,
\[ \text{电压增益} = 20\log\abs{A_v},\quad \text{电流增益} = 20\log\abs{A_i},\quad \text{功率增益} = 10\log A_P. \]
可以将乘法转化为加法. 增益可能有量纲.
\newpoint{}放大电路的增益是频率的函数时, 写作
\[ \tilde{A}_v\pare{\omega} = A_v\pare{\omega}\angle\varphi\pare{\omega}, \]
其中$\displaystyle \abs{\frac{V_o\pare{j\omega}}{V_i\pare{j\omega}}}$谓幅频相应, $\angle \varphi\pare{\omega} = \varphi_o\pare{\omega} - \varphi_i\pare{\omega}$谓相频相应.
\newpoint{}输入正弦信号的情况下, 输出随输入信号频率连续变化的稳态响应, 谓放大电路的频率响应.
\newpoint{}\gloss{放大器带宽}$f_{BW} = f_H - f_L$, $f_H \gg f_L$时$f_{BW}\approx f_H$. 当放大倍数与中频相比$\SI{-3}{\decibel}$时得到$f_L$和$f_H$. $\SI{-3}{\decibel}$正好对应功率下降至一般的点. 通常的带宽都按这种方法定义, 谓$\SI{3}{\decibel}$带宽.
\begin{remark}
    直流放大电路不适用之.
\end{remark}
\begin{remark}
    若输入固定频率的正弦波, 无论频率如何皆不会发生失真, 仅增益变小.
\end{remark}
\newpoint{}\gloss{线性失真}分为\gloss{幅度失真}: 对不同频率的信号增益不同产生的失真. \gloss{相位失真}: 对不同频率的信号相移不同产生的失真.
\newpoint{}\gloss{非线性失真}谓由元件非线性特性引起的失真. 输入设定频率的正弦波, 理想输出应当是同频率, 幅度正比的完美正弦波. 但实际输出是变形的正弦波, 频率上有多余的高次谐波. \gloss{非线性失真系数}定义为
\[ \gamma = \frac{\sqrt{\displaystyle \sum_{k=2}^\infty V_{ok}^2}}{V_{o1}}\times 100\%. \]
其中$v_{o1}$是输入电压基波分量的有效值. $V_{ok}$是高次谐波分量的有效值, $k$是正整数.

% subsection 放大电路的主要性能指标 (end)

% section 基础 (end)

\end{document}
