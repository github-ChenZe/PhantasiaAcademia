\documentclass[hidelinks]{ctexart}

\usepackage{van-de-la-illinoise}
\usepackage[paper=b5paper,top=.3in,left=.9in,right=.9in,bottom=.3in]{geometry}
\usepackage{calc}
\usepackage{van-le-trompe-loeil}
\pagenumbering{gobble}
\setlength{\parindent}{0pt}
\pgfplotsset{compat=newest}

\newdimen\indexlen
\def\newprobheader#1{%
\def\probindex{#1}
\setlength\indexlen{\widthof{\textbf{\probindex}}}
\hskip\dimexpr-\indexlen-1em\relax
\textbf{\probindex}\hskip1em\relax
}
\def\newprob#1{%
\newprobheader{#1}%
\def\newprob##1{%
\probsep%
\newprobheader{##1}%
}%
}
\def\probsep{\vskip1em\relax{\color{gray}\dotfill}\vskip1em\relax}
\def\rD{\mathrm{D}}
\newcommand{\mathmin}[2]{(#1+#2)/2-abs(#1-#2)/2}
\newcommand{\mathmax}[2]{(#1+#2)/2+abs(#1-#2)/2}

\begin{document}

\newprob{4.8.2}%
\\[-\baselineskip]
\begin{tikzpicture}
    \draw
        (0,2) node[below] {$+$} to[capacitor,o-*]
        (2,2) to[resistor]
        (2,0) to[short,-o]
        (0,0) node[above] {$-$}
        (0,1) node {$v\+_i_$}
        (2,2) -- (3,2) node (pjf) [pjfet,anchor=G,yscale=-1] {}
        (pjf.S) to[resistor]
        (pjf.S|-0,0) node[rground] {} to[short,*-*]
        (2,0)
        (pjf.S) to[short,*-]
        (pjf.S-|5,0) to[capacitor]
        (5,0) to[short,*-] (pjf.S|-0,0)
        (pjf.D) to[resistor]
        (pjf.D|-0,5) to[short,-o]
        ++(1,0) node[right] {$-V\+_DD_$}
        (pjf.D) to[capacitor,*-*]
        (pjf.D-|6,0) to[resistor,-*]
        (6,0) --
        (5,0)
        (6,0) to[short,-o]
        (7,0) node[above] {$-$}
        (pjf.D-|6,0) to[short,-o]
        (pjf.D-|7,0) node[below] {$+$}
        ($0.5*(pjf.D-|7,0) + 0.5*(7,0)$) node {$v\+_o_$}
    ;
\end{tikzpicture}
\newprob{4.8.3 (1)}%
$v\+_GS_ < 0 \Rightarrow $\underline{N沟道}.
\par
\newprobheader{(2)}%
$V\+_P_ = \boxed{\SI{-4}{\volt}.}$\\
$I\+_DSS_ = \boxed{\SI{3}{\milli\ampere}.}$
\newprob{4.8.6}
\\[-\baselineskip]
\begin{tikzpicture}
    \begin{axis}[
        axis lines=middle,xlabel=$v\+_GS_/\SI{}{\volt}$,
        ylabel=$i\+_D_/\SI{}{\milli\ampere}$,xmin=-1,xmax=0.3,ymin=0,ymax=0.7,xtick={-0.8,-0.6,-0.4,-0.2,0},ytick={0,0.2,0.4,0.6},%
        xlabel style={
        at={(current axis.right of origin)},
        anchor=north
        },%
        ylabel style={
        at={(current axis.above origin)},
        anchor=west,
        %rotate=-90
        },y tick label style={
        anchor=west,
        right=0.3em
        %rotate=-90
        },clip=false]
    \addplot[mark={},domain=-0.8:0,samples=200] function{(x+0.8)^2*0.53/0.64};
    \addplot[mark={},domain=-0.8:0,samples=200] function{-x/1.5};
    \addplot[mark=*,id=intersection] coordinates{(-0.304716,0.203144)} node[right] {$Q$};
    \end{axis}
    %\draw (8,5) node[right] {$\boxed{\begin{cases}
    %    v\+_CEQ_ \approx \SI{9}{\volt},\\
    %    I\+_CQ_\approx\SI{4}{\milli\ampere}.
    %\end{cases}}$};
\end{tikzpicture}\\
$\displaystyle \begin{cases}
    V\+_GS_ = -R\+_s_I\+_D_, \\
    \displaystyle I\+_D_ = I\+_DSS_ \pare{1-\frac{V\+_GS_}{V\+_P_}}
\end{cases} \xLongrightarrow[V\+_P_ = \SI{-0.8}{\volt}]{I\+_DSS_ = \SI{0.53}{\milli\ampere}} \begin{cases}
    V\+_GS_ = \boxed{\SI{-0.3}{\volt},} \\
    I\+_D_ = \boxed{\SI{0.2}{\milli\ampere},} \\
    V\+_DS_ = V\+_DD_ - I\+_D_\pare{R\+_d_ + R\+_s_} = \boxed{\SI{9.7}{\volt}.}
\end{cases}$
\newprob{4.8.8}%
$\displaystyle A_v = \frac{g\+_m_ R\+_s_}{g\+_m_ R\+_s_+1} = \boxed{0.915.}$\\
$\displaystyle R\+_i_ = R\+_g3_ + R\+_g1_\parallel R\+_g2_ = \boxed{\SI{2.08}{\mega\ohm}.}$\\
$\displaystyle R\+_o_ = R\+_s_ \parallel \rec{g_s} = \boxed{\SI{1.02}{\kilo\ohm}.}$
\newprob{4.8.9}
$\displaystyle \begin{cases}
    i\+_ds_ r\+_ds_ + \brac{i\+_ds_ - g\+_m_ \pare{v\+_AB_ - i\+_ds_ r\+_ds_}} R\+_s_ = v\+_AB_, \\
    i\+_ds_ - g\+_m_ \pare{v\+_AB_ - i\+_ds_ r\+_ds_} = i\+_AB_,
\end{cases} r\+_AB_ = \frac{v\+_AB_}{i\+_AB_} = r\+_ds_ + R\+_s_ + g\+_m_ R\+_s_r\+_ds_.$\\
\begin{tikzpicture}
    \draw
        (0,1) node {$v\+_gs_$}
        (0,0) node[above] {$+$} to[short,o-]
        (2,0) to[resistor,R=$R\+_s_$,*-*]
        (2,2) to[short,-o]
        (0,2) node[below] {$-$}
        (2,2) to[european controlled current source,i<=$g\+_m_ v\+_gs_$]
        (2,4) --
        (4,4) to[short,*-]
        (4,2) to[resistor,R=$r\+_ds_$]
        (2,2)
        (2,0) --
        (5,0) to[resistor,R=$R\+_L_$]
        (5,4) --
        (4,4)
        %(5,0) to[short,-o]
        %(6,0) node[above] {$-$}
        %(5,4) to[short,-o]
        %(6,4) node[below] {$+$}
        %(6,2) node {$v\+_o_$}
    ;
\end{tikzpicture}
\newprob{4.8.10}%
\\[-\baselineskip]
\begin{tikzpicture}
    \draw
        (0,0) node[above] {$-$} to[short,o-]
        (3,0)
        (0,4) node[below] {$+$} to[short,o-o]
        (1,4) node[below] {$+$}
        (0,2) node {$v\+_i_$}
        (3,0) to[resistor,R=$R_2$,*-*]
        (3,2) to[short,-o]
        (1,2) node[above] {$-$}
        (1,3) node{$v\+_gs1_$}
        (5,4) --
        (3,4) to[european controlled current source,i=$g\+_m_ v\+_gs1_$]
        (3,2) --
        (5,2) to[resistor,R=$r\+_ds_$,-*]
        (5,4) --
        (8,4)
        (3,0) to[short,i=$i$]
        (5,0) --
        (8,0) to[european controlled current source,i=$g\+_m_ v\+_gs2_$,*-]
        (8,2) --
        (6,2) to[resistor,R=$r\+_ds_$,-*]
        (6,0)
        (8,2) to[resistor,R=$R_1$,v<=$v\+_gs2_$,*-*]
        (8,4)
        (8,4) to[short,-o]
        (10,4) node[below] {$+$}
        (8,0) to[short,-o]
        (10,0) node[above] {$-$}
        (10,2) node {$v\+_o_$}
    ;
\end{tikzpicture}
\par
\newprobheader{(1)}%
第二段电路($R_1$等)可等效为$R'_1 = r\+_ds_ + R_1 + \mu R_1$, 则\\
$\displaystyle \begin{cases}
    i = g\+_m_v\+_gs1_ + i\+_ds_, \\
    iR_2 = v\+_i_ - v\+_gs1_, \\
    iR'_1 + i\+_ds_ r\+_ds_ = g\+_s1_ - v_i
\end{cases}$\\
$\displaystyle \Rightarrow v\+_o_ = -R'_1 i = \frac{-r\+_ds_g\+_m_v\+_i_ R'_1}{r\+_ds_ + R'_1 + R_2 + r\+_ds_ g\+_m_R_2}$ \\
$\displaystyle \Rightarrow A_v = \frac{v\+_o_}{v\+_i_} = \frac{-\mu\brac{r\+_ds_ + \pare{1+\mu}R_1}}{2r\+_ds_ + \pare{1+\mu}\pare{R_1 + R_2}}.$
\par
\newprobheader{(2)}%
将$v\+_i_$置零, 则第一段电路可等效为$R'_2 = r\+_ds_ + R_2 + \mu R_2$,\\
$\displaystyle R\+_o_ = R'_1 \parallel R'_2 \Rightarrow G_o = \rec{r\+_ds_ + R_1 + \mu R_1}+\rec{r\+_ds_ + R_2 + \mu R_2}.$
\par
\newprobheader{(3)}%
$\displaystyle A_v = \boxed{-\frac{\mu}{2},}$\\
$\displaystyle R\+_o_ = \boxed{\frac{r\+_ds_ + \pare{1+\mu}R_s}{2}.}$

\end{document}
