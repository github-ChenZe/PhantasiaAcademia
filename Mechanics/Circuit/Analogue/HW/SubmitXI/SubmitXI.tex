\documentclass[hidelinks]{ctexart}

\usepackage{van-de-la-illinoise}
\usepackage[paper=b5paper,top=.3in,left=.9in,right=.9in,bottom=.3in]{geometry}
\usepackage{calc}
\usepackage{van-le-trompe-loeil}
\pagenumbering{gobble}
\setlength{\parindent}{0pt}
\pgfplotsset{compat=newest}

\newdimen\indexlen
\def\newprobheader#1{%
\def\probindex{#1}
\setlength\indexlen{\widthof{\textbf{\probindex}}}
\hskip\dimexpr-\indexlen-1em\relax
\textbf{\probindex}\hskip1em\relax
}
\def\newprob#1{%
\newprobheader{#1}%
\def\newprob##1{%
\probsep%
\newprobheader{##1}%
}%
}
\def\probsep{\vskip1em\relax{\color{gray}\dotfill}\vskip1em\relax}
\def\rD{\mathrm{D}}
\newcommand{\mathmin}[2]{(#1+#2)/2-abs(#1-#2)/2}
\newcommand{\mathmax}[2]{(#1+#2)/2+abs(#1-#2)/2}

\begin{document}

\newprob{8.1.1 (a)}%
通过$R_2$反馈. 直流/交流负反馈.
\par
\newprobheader{(b)}%
通过$R\+_f1_$, $C$, $R\+_f2_$反馈. 直流负反馈.\\
通过$T_3$的$\mathrm{E}$极到$R\+_e1_$反馈. 直流/交流负反馈.
\par
\newprobheader{(c)}%
通过$R\+_f_$反馈. 直流/交流负反馈.
\par
\newprobheader{(d)}%
通过$R_2$, $R_1$反馈. 直流/交流负反馈.
\par
\newprobheader{(e)}%
通过$\mathrm{A}_2$, $R_3$反馈. 直流/交流负反馈.
\par
\newprobheader{(f)}%
通过$R\+_L_$, $R_6$反馈. 直流/交流负反馈.
\newprob{8.1.3 (a)}%
电压串联负反馈, 故视信号源为电压源, \underline{$R\+_si_$小}.
\par
\newprobheader{(b)}%
电流并联负反馈, 故视信号源为电流源, \underline{$R\+_si_$大}.
\newprob{8.1.5 (a)}%
不能. 引入了正反馈. 应当将OP AMP的$+$端和$-$段对调.
\par
\newprobheader{(b)}%
不能. 应将$R$与$R\+_L_$互换.
\newprob{8.1.6 (1)}%
a-c, b-d, f-j, i-h.\vspace{-2\baselineskip}
\begin{center}
    \begin{tikzpicture}[yscale=0.9]
        \draw
        (0,0) node[rground] {} -- 
        (0,1) node (b) {} to[european voltage source,v<=$v\+_s_$]
        (0,3) --
        (1,3) node (a) {} to[resistor,R=$R_1$]
        (3,3) node[op amp,anchor=+,yscale=-1] (opamp) {}
        (b) to[resistor,R=$R_2$,*-*] ++(2,0) node(Rb) {} |- (opamp.-)
        (opamp.out) node[npn,anchor=G,label=right:{$\mathrm T_1$}](T1) {}
        (T1.S) --++(0.5,0) node[npn,anchor=G,label=right:{$\mathrm T_2$}](T2) {}
        (T1.D) |-
        (T2.D|-0,3.5) to[resistor,R=$R_3$,*-o]
        ++(0,2) node[right]{$+V\+_CC_$}
        (T2.D) -- (T2.D|-0,3.5)
        (Rb) to[resistor,R=$R\+_f_$,-*] (T2.S|-Rb)
        to[short,-o] ++(1,0) node[right] {$v\+_o_$}
        (T2.S) to[resistor,R=$R_4$,-o]
        ++(0,-2) node[right] {$-V\+_EE_$}
        ;
    \end{tikzpicture}
\end{center}
\par
\newprobheader{(2)}%
a-d, b-c, f-j, i-h.\vspace{-2\baselineskip}
\begin{center}
    \begin{tikzpicture}[yscale=0.9]
        \draw
        (0,0) node[rground] {} -- 
        (0,2) node (b) {}
        (1,3) node (a) {} to[resistor,R=$R_1$] node(Rb) {}
        (3,3) node[op amp,anchor=+,yscale=-1] (opamp) {}
        (opamp.-) to[resistor,R=$R_2$] ++(-2,0) node (R2in) {}
        (R2in-|0,2) ++ (0,-0.5) to[european voltage source,v<=$v\+_s_$] (0,3.5)
        (0,3.5) --++(0.5,0) to[short] (R2in)
        (R2in-|0,2) ++ (0,-0.5) to[short,*-] ++ (0.5,0) to[short] (1,3)
        (opamp.out) node[npn,anchor=G,label=right:{$\mathrm T_1$}](T1) {}
        (T1.S) --++(0.5,0) node[npn,anchor=G,label=right:{$\mathrm T_2$}](T2) {}
        (T1.D) |-
        (T2.D|-0,3.5) to[resistor,R=$R_3$,*-o]
        ++(0,2) node[right]{$+V\+_CC_$}
        (T2.D) -- (T2.D|-0,3.5)
        (opamp.-) to[short,*-] (Rb|-T2.S) to[resistor,R=$R\+_f_$,-*] (T2.S)
        to[short,-o] ++(1,0) node[right] {$v\+_o_$}
        (T2.S) to[resistor,R=$R_4$,-o]
        ++(0,-2) node[right] {$-V\+_EE_$}
        ;
    \end{tikzpicture}
\end{center}
\par
\newprobheader{(3)}%
a-d, b-c, e-j, i-g.\vspace{-2\baselineskip}
\begin{center}
    \begin{tikzpicture}[yscale=0.9]
        \draw
        (0,0) node[rground] {} -- 
        (0,2) node (b) {}
        (1,3) node (a) {} to[resistor,R=$R_1$] node(Rb) {}
        (3,3) node[op amp,anchor=+,yscale=-1] (opamp) {}
        (opamp.-) to[resistor,R=$R_2$] ++(-2,0) node (R2in) {}
        (R2in-|0,2) ++ (0,-0.5) to[european voltage source,v<=$v\+_s_$] (0,3.5)
        (0,3.5) --++(0.5,0) to[short] (R2in)
        (R2in-|0,2) ++ (0,-0.5) to[short,*-] ++ (0.5,0) to[short] (1,3)
        (opamp.out) node[npn,anchor=G,label=right:{$\mathrm T_1$}](T1) {}
        (T1.S) --++(0.5,0) node[npn,anchor=G,label=right:{$\mathrm T_2$}](T2) {}
        (T1.D) |-
        (T2.D|-0,3.5) to[resistor,R=$R_3$,*-o]
        ++(0,2) node[right]{$+V\+_CC_$}
        (T2.D) -- (T2.D|-0,3.5)
        (opamp.+) to[short,*-] (opamp.+|-0,4) to[resistor,R=$R\+_f_$] (7,4) to[short,-*] (7,3.5)
        (T2.S) to[short,-o] ++(1,0) node[right] {$v\+_o_$}
        (T2.S) to[resistor,R=$R_4$,-o]
        ++(0,-2) node[right] {$-V\+_EE_$}
        ;
    \end{tikzpicture}
\end{center}
\par
\newprobheader{(4)}%
a-c, b-d, e-j, i-g.\vspace{-2\baselineskip}
\begin{center}
    \begin{tikzpicture}[yscale=0.9]
        \draw
        (0,0) node[rground] {} -- 
        (0,1) node (b) {} to[european voltage source,v<=$v\+_s_$]
        (0,3) --
        (1,3) node (a) {} to[resistor,R=$R_1$]
        (3,3) node[op amp,anchor=+,yscale=-1] (opamp) {}
        (b) to[resistor,R=$R_2$,*-] ++(2,0) node(Rb) {} |- (opamp.-)
        (opamp.out) node[npn,anchor=G,label=right:{$\mathrm T_1$}](T1) {}
        (T1.S) --++(0.5,0) node[npn,anchor=G,label=right:{$\mathrm T_2$}](T2) {}
        (T1.D) |-
        (T2.D|-0,3.5) to[resistor,R=$R_3$,*-o]
        ++(0,2) node[right]{$+V\+_CC_$}
        (T2.D) -- (T2.D|-0,3.5)
        %(Rb) to[resistor,R=$R\+_f_$,-*] (T2.S|-Rb)
        (T2.S) to[short,-o] ++(1,0) node[right] {$v\+_o_$}
        (T2.S) to[resistor,R=$R_4$,-o]
        ++(0,-2) node[right] {$-V\+_EE_$}
        (opamp.+) to[short,*-] (opamp.+|-0,4) to[resistor,R=$R\+_f_$] (7,4) to[short,-*] (7,3.5)
        ;
    \end{tikzpicture}
\end{center}
\newprob{8.2.1}%
$\displaystyle v\+_id_ = v\+_o_/A_v = \boxed{\SI{1}{\milli\volt}.}$\\
$\displaystyle v\+_f_ = F_v v_o = \boxed{\SI{0.099}{\volt}.}$\\
$\displaystyle v\+_i_ = v\+_id_ + v\+_f_ = \boxed{\SI{0.1}{\volt}.}$
\newprob{8.2.2}%
$A_2$的闭环增益为$\displaystyle A\+_2f_ = \frac{A_2}{1+A_2F_2}$.\\
若$F_1$不存在, 则增益为$\displaystyle A\+_o_ = A_1 A\+_2f_$.\\
故总的闭环增益为$\displaystyle A\+_f_ = \frac{A\+_o_}{1+A\+_o_F_1} = \boxed{\frac{A_1A_2}{1+A_2F_2 + A_1A_2F_1}.}$
\newprob{8.2.3}%
$\displaystyle F = \frac{R_1}{R_1+R\+_f_} = \boxed{0.0979.}$\\
$\displaystyle A\+_\mathnormal{v}f_ = \frac{A\+_\mathnormal{v}o_}{1+A\+_\mathnormal{v}o_F_v} = \boxed{10.2.}$

\end{document}
