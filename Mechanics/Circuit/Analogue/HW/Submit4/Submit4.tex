\documentclass[hidelinks]{ctexart}

\usepackage{van-de-la-illinoise}
\usepackage[paper=b5paper,top=.3in,left=.9in,right=.9in,bottom=.3in]{geometry}
\usepackage{calc}
\usepackage{van-le-trompe-loeil}
\pagenumbering{gobble}
\setlength{\parindent}{0pt}

\newdimen\indexlen
\def\newprobheader#1{%
\def\probindex{#1}
\setlength\indexlen{\widthof{\textbf{\probindex}}}
\hskip\dimexpr-\indexlen-1em\relax
\textbf{\probindex}\hskip1em\relax
}
\def\newprob#1{%
\newprobheader{#1}%
\def\newprob##1{%
\probsep%
\newprobheader{##1}%
}%
}
\def\probsep{\vskip1em\relax{\color{gray}\dotfill}\vskip1em\relax}
\def\rD{\mathrm{D}}
\newcommand{\mathmin}[2]{(#1+#2)/2-abs(#1-#2)/2}
\newcommand{\mathmax}[2]{(#1+#2)/2+abs(#1-#2)/2}

\begin{document}

\newprob{5.1.1}%
工作在放大状态时时基极电位居中, 射极电位与基极电位接近. 故$V\+_C_ = \SI{-6.2}{\volt}$是基极电位. $V\+_B_ = \SI{-6}{\volt}$为射极电位, $V\+_A_ = \SI{-9}{\volt}$为集极电位. 射极电位高于基极电位故为\boxed{\text{PNP型.}\ \mathrm{C}\mapsto \mathrm{b},\ \mathrm{B}\mapsto \mathrm{e},\ \mathrm{A}\mapsto \mathrm{c}.}
\newprob{5.1.2}%
电流最小者为基极, 故\boxed{\mathrm{B}\mapsto\mathrm{b}.} 电流最大者为射极, 故\boxed{\mathrm{C}\mapsto \mathrm{e}.} 电流居中者为集极, 故\boxed{\mathrm{A}\mapsto \mathrm{c}.} 由射集电流向外知为\boxed{\text{NPN型}.}
\[ \conj{\beta} \approx \frac{I\+_Collector_}{I\+_Base_} = \frac{I\+_A_}{I\+_B_} = \boxed{50.} \]
\vspace{-\baselineskip}
\newprob{5.1.3}%
$V\+_CE_ = \SI{10}{\volt}$, $\displaystyle P\+_CM_ = \SI{150}{\milli\watt} \Rightarrow I < \frac{\SI{150}{\milli\watt}}{\SI{10}{\volt}} = \SI{15}{\milli\ampere} < I\+_CM_$, 故$I\+_C_ < \boxed{\SI{15}{\milli\ampere}.}$\\
$I\+_C_ = \SI{1}{\milli\ampere}$, $\displaystyle P\+_CM_ = \SI{150}{\milli\watt} \Rightarrow U\+_CE_ < \frac{\SI{150}{\milli\watt}}{\SI{1}{\milli\ampere}} = \SI{150}{\volt} > U\+_(BR)CEO_$, 故$U\+_CE_ < \boxed{\SI{30}{\volt}.}$
\newprob{5.1.4 (1)}%
$\displaystyle \beta \approx \frac{i\+_C_}{i\+_B_} = \boxed{85.}\quad i\+_E_ = i\+_B_ + i\+_C_ = \boxed{\SI{516}{\micro\ampere}.} \quad \alpha = \frac{i\+_C_}{i\+_E_} = \boxed{0.988.}$
\par
\newprobheader{(2)}%
$\displaystyle \beta \approx \frac{i\+_C_}{i\+_B_} = \boxed{53.}\quad i\+_E_ = i\+_B_ + i\+_C_ = \boxed{\SI{2.7}{\milli\ampere}.} \quad \alpha = \frac{i\+_C_}{i\+_E_} = \boxed{0.981.}$

\end{document}
