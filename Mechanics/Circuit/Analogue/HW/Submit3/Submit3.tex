\documentclass[hidelinks]{ctexart}

\usepackage{van-de-la-illinoise}
\usepackage[paper=b5paper,top=.3in,left=.9in,right=.9in,bottom=.3in]{geometry}
\usepackage{calc}
\usepackage{van-le-trompe-loeil}
\pagenumbering{gobble}
\setlength{\parindent}{0pt}

\newdimen\indexlen
\def\newprobheader#1{%
\def\probindex{#1}
\setlength\indexlen{\widthof{\textbf{\probindex}}}
\hskip\dimexpr-\indexlen-1em\relax
\textbf{\probindex}\hskip1em\relax
}
\def\newprob#1{%
\newprobheader{#1}%
\def\newprob##1{%
\probsep%
\newprobheader{##1}%
}%
}
\def\probsep{\vskip1em\relax{\color{gray}\dotfill}\vskip1em\relax}
\def\rD{\mathrm{D}}
\newcommand{\mathmin}[2]{(#1+#2)/2-abs(#1-#2)/2}
\newcommand{\mathmax}[2]{(#1+#2)/2+abs(#1-#2)/2}

\begin{document}

\newprob{3.4.1}%
理想模型下, $v\+_s_ > 0$时$v\+_L_ = v\+_s_$, 否则$v\+_L_ = 0$.\\
恒压降模型下, $v\+_s_ > \SI{0.7}{\volt}$时$v\+_L_ = v\+_s_ - \SI{0.7}{\volt}$, 否则$v\+_L_ = 0$.
\begin{figure}[ht]
    \centering
    \begin{subfigure}{4.2cm}
        \centering
        \begin{tikzpicture}
            \draw[->] (0,0) -- (3.4,0) node[below] {$t$};
            \draw[->] (0,-1.2) -- (0,1.2) node[left] {$v\+_s_$};
            \draw[dashed] (0,1) node[below left] {$\SI{2}{\volt}$} -- (3,1);
            \draw[domain=0:pi] plot function{sin(3*x)};
        \end{tikzpicture}
        \caption{$v\+_s_$的波形}
    \end{subfigure}
    \begin{subfigure}{4.2cm}
        \centering
        \begin{tikzpicture}
            \draw[->] (0,0) -- (3.4,0) node[below] {$t$};
            \draw[->] (0,-1.2) -- (0,1.2) node[left] {$v\+_o_$};
            \draw[dashed] (0,1) node[below left] {$\SI{2}{\volt}$} -- (3,1);
            \draw[domain=0:pi] plot function{(sin(3*x)+abs(sin(3*x)))/2};
        \end{tikzpicture}
        \caption{理想模型下$v\+_o_$的波形}
    \end{subfigure}
    \begin{subfigure}{4.2cm}
        \centering
        \begin{tikzpicture}
            \draw[->] (0,0) -- (3.4,0) node[below] {$t$};
            \draw[->] (0,-1.2) -- (0,1.2) node[left] {$v\+_o_$};
            \draw[dashed] (0,0.65) node[below left] {$\SI{1.3}{\volt}$} -- (3,0.65);
            \draw[domain=0:pi] plot function{(sin(3*x)-0.35+abs(sin(3*x)-0.35))/2};
        \end{tikzpicture}
        \caption{恒压降模型下$v\+_o_$的波形}
    \end{subfigure}
\end{figure}
\vspace{-\baselineskip}
\newprob{3.4.3 (1)}%
$v\+_s_ > 0$时$\rD_2$和$\rD_4$导通, $\rD_1$和$\rD_3$截止, 从而$v\+_L_ = v\+_s_$. 而$v\+_s_ < 0$时$D_1$和$D_3$导通, $\rD_2$和$\rD_4$截止, 从而$v\+_L_ = -v\+_s_$. 故有$v\+_L_ = \abs{v\+_s_}$.
\begin{figure}[ht]
    \centering
    \begin{tikzpicture}
            \draw[->] (0,0) -- (3.4,0) node[below] {$t$};
            \draw[->] (0,-0.2) -- (0,1.2) node[left] {$v\+_L_$};
            \draw[domain=0:pi] plot function{abs(sin(3*x))};
    \end{tikzpicture}
\end{figure}
\par
\newprobheader{(2)}$v\+_Z_ \ge \SI{220}{\volt} \times \sqrt{2} = \boxed{\SI{311}{\volt}.}$
\newprob{3.4.7}%
当$\abs{v\+_s_} < \SI{3.7}{\volt}$, 两个二极管皆处于截止状态, 故$v\+_o_ = v\+_s_$.\\
当$v\+_s_ > \SI{3.7}{\volt}$, $\rD_1$导通而$\rD_2$截止, $v\+_o_ = \SI{3.7}{\volt}.$\\
当$v\+_s_ < \SI{-3.7}{\volt}$, $\rD_2$导通而$\rD_1$截止, $v\+_o_ = \SI{-3.7}{\volt}.$
\begin{figure}[ht]
    \centering
    \begin{tikzpicture}
            \draw[->] (0,0) -- (3.4,0) node[below] {$t$};
            \draw[->] (0,-1.2) -- (0,1.2) node[left] {$v\+_L_$};
            \draw[samples=60,domain=0:pi] plot function{ \mathmax{\mathmin{sin(3*x)}{0.61666667}}{-0.61666667} };
            \draw[dashed,domain=0:pi] plot function{ sin(3*x) };
    \end{tikzpicture}
    \caption*{$v\+_o_$的波形, 虚线为$v\+_s_$}
\end{figure}
\vspace{-\baselineskip}
\newprob{3.4.8}%
$v\+_I_ < \SI{6}{\volt}$时$\rD$截止, 故$v\+_O_ = \SI{6}{\volt}$. 而$v\+_I_ > \SI{6}{\volt}$时$\rD$导通, 故$v\+_O_ = \pare{v\+_I_+\SI{6}{\volt}}/2$.\\[1em]
\centerline{{\begin{tikzpicture}
            \draw[->] (0,0) -- (3.8,0) node[below] {$t/\SI{}{\milli\second}$};
            \draw[->] (0,-0.2) -- (0,2.8) node[left] {$v\+_O_/\SI{}{\volt}$};
            \draw[dashed] (1.8,1.8) -- (1.8,0) node[below] {$3$}
                            (3,2.4) -- (3,0) node[below] {$5$}
                            (0,1.8) node[left] {$6$}
                            (3,2.4) -- (0,2.4) node[left] {$8$};
            \draw (0,1.8) -- (1.8,1.8) -- (3,2.4);
    \end{tikzpicture}}}
\newprob{3.4.12 (a)}%
导通. $V\+_AO_ = \boxed{\SI{-6}{\volt}.}$
\par
\newprobheader{(b)}%
截止. $V\+_AO_ = \boxed{\SI{-12}{\volt}.}$
\par
\newprobheader{(c)}%
$\rD_1$导通, $\rD_2$截止. $V\+_AO_ = \boxed{\SI{0}{\volt}.}$
\par
\newprobheader{(c)}%
$\rD_2$导通, $\rD_1$截止. $V\+_AO_ = \boxed{\SI{-6}{\volt}.}$
\par
\begin{wrapfigure}{r}{4cm}
    \centering
    \begin{tikzpicture}
        \draw (0,0) node[rground] {} to[european voltage source, v<=$v\+_i_$] (0,2) to [resistor,R=$\SI{1}{\kilo\ohm}$] (2,2) to [resistor,R=$3r\+_D_$] (2,0) node[rground] {};
        \draw (2,2) to[short,-o] (2.5,2) node[right] {$v\+_o_$};
    \end{tikzpicture}
    \vspace{-9em}
\end{wrapfigure}
\newprob{3.4.15 (1)}%
$V\+_DD_ = \SI{10}{\volt}$下, $Q$点$V\+_OQ_ = 3\times 0.7 = \boxed{\SI{2.1}{\volt}.}$\\
$I\+_DQ_ = \pare{10 - 2.1}/1 = \boxed{\SI{7.9}{\milli\ampere}.}$
\par
\newprobheader{(2)}%
交流通路如右图, $r\+_D_ = V\+_T_ / I\+_D_ = 26/7.9 = \SI{3.29}{\ohm}$.\\
$\displaystyle v\+_o_ = \frac{3r\+_D_}{3r\+_D_ + R}\cdot v\+_i_ = \SI{9.87e-3}{} v\+_i_$.\\
当$v\+_i_$在$\pm\SI{1}{\volt}$变化, 相应的$v\+_i_$变化为$\pm\SI{9.87}{\milli\volt}$. 故$V\+_O_$波动约$\boxed{\pm\SI{9.87}{\milli\volt}.}$
\leavevmode
\newprob{3.4.17}%
\leavevmode
正向导通压降$\SI{0.7}{\volt}$的情形下, $\displaystyle I\+_D_ = \frac{V\+_I_ - \SI{0.7}{\volt}}{R} - \frac{\SI{0.7}{\volt}}{R_L}$.\\
为了让$\SI{2}{\milli\ampere} < I_D < \SI{15}{\milli\ampere}$ \makebox[0pt][l]{在$V\+_I_ \in \pare{\SI{5}{\volt}, \SI{10}{\volt}}$时都有解, 须}
\[ \boxed{\SI{0.38}{\kilo\ohm} < R < \SI{2.15}{\kilo\ohm},\quad R_L > \SI{76}{\ohm}.} \]
\vspace{-2em}
为了让$R_L$开路时有解,
\vspace{2em}
\[ \SI{2}{\milli\ampere} \le \frac{V\+_I_ - \SI{0.7}{\volt}}{R} \le \SI{15}{\milli\ampere} \Rightarrow R\in \pare{\SI{0.62}{\kilo\ohm}, \SI{2.15}{\kilo\ohm}}. \]
带负载时, 若$R=\SI{0.62}{\kilo\ohm}$, 则
\[ \SI{2}{\milli\ampere} \le \frac{V\+_I_ - \SI{0.7}{\volt}}{R} - \frac{\SI{0.7}{\volt}}{R_L} \le \SI{15}{\milli\ampere} \Rightarrow R\+_L_ > \SI{0.14}{\kilo\ohm}. \]
因此, 为了让开路时有解, 可以取$\displaystyle \boxed{R = \SI{0.62}{\kilo\ohm},\text{\ 且\ }R_L > \SI{0.14}{\kilo\ohm}.}$
\newprob{3.5.3 (1)}%
$\displaystyle \SI{5}{\milli\ampere} \le \frac{V\+_I_ - V\+_Z_}{R} - \frac{V\+_Z_}{R\+_L_} \le \SI{50}{\milli\ampere} \Rightarrow \boxed{R\+_L_ \ge \SI{111}{\ohm}.}$
\par
\newprobheader{(2)}%
$\displaystyle P\+_OM_ = I\+_Omax_\cdot U\+_Z_ = \SI{5}{\volt}\cdot \SI{5}{\volt}/\pare{\SI{111}{\ohm}} = \boxed{\SI{0.225}{\watt}.}$
\par
\newprobheader{(3)}%
$\displaystyle P\+_ZM_ = I\+_Zmax_ \cdot V\+_Z_ = \boxed{\SI{0.25}{\watt}.}$\\
$\displaystyle P\+_RM_ = U_R^2 / R = \boxed{\SI{0.25}{\watt}.}$

\end{document}
