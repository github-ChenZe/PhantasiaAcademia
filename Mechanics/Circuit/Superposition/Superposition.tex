\documentclass{ctexart}

\usepackage{van-de-la-sehen}

\begin{document}

\section{叠加方法与网络函数} % (fold)
\label{sec:叠加方法与网络函数}

\subsection{叠加原理} % (fold)
\label{sub:叠加原理}


由线性电阻, 线性受控源及独立源组成的电路中, 每一元件的电流或电压可以看作是每一个独立源单独作用于电路时, 该元件上产生的电流与电压的代数和.
\par
某一独立源单独作用时, 其它独立源置零. 独立电压源替换为短路, 独立电流源替换为开路(等同于将独立源的符号中的圆去除).
\begin{pitfall}
    功率不可如此叠加.
\end{pitfall}
\inlinehardlink{书本例3-3, 3-4}

% subsection 叠加原理 (end)

% section 叠加方法与网络函数 (end)

\end{document}
