\documentclass{ctexart}

\usepackage{van-de-la-sehen}

\begin{document}

\section{网孔分析和节点分析} % (fold)
\label{sec:网孔分析和节点分析}

\paragraph{作业} % (fold)
\label{par:作业}

4 8 17 22 25

% paragraph 作业 (end)

\subsection{网孔方程} % (fold)
\label{sub:网孔方程}

网孔电流不能用KCL相联系, 但能根据KVL以及支路VCR为每一个网孔列出一个KVL当成. 方程中的支路电压可以通过Ohm定律用网孔电流表示.
\paragraph{网孔方程的一般表达式} % (fold)
\label{par:网孔方程的一般表达式}

如\inlinehardlink{书图2-2}, 网孔方程可表示为
\[ \begin{cases}
    R_{11} i_{M1} +  R_{12}i_{M2} + R_{13} i_{M3} = u_{S11},\\
    R_{21} i_{M1} +  R_{22}i_{M2} + R_{23} i_{M3} = u_{S22}, \\
    R_{31} i_{M1} +  R_{32}i_{M2} + R_{33} i_{M3} = u_{S33}.
\end{cases} \]
$R_11$到$R_33$是自电阻, 分别是各自网孔内所有电阻的总和. $R_{12}$是网孔$1$和网孔$2$的互电阻, 即二者的共有电阻. 若流经二者电流同向, 则$R_{12} = R_5$(正号). 若流经二者电流相反, 则$R_{13} = -R_4$(负号).
\par
等号右侧为各个网孔中电源的电压升的代数和. 由网孔方向决定电压升的符号.
\par
具有$m$个网孔的电路, 网孔方程的形式类似.
\inlinehardlink{书本例2-1}\\
\inlinehardlink{?书本例2-2 cf. lecture slides}\\
对于含电流源的情形, 电流源所在的网孔直接有$i_{M2} = i$, 只需列出另一网孔的方程. 此外, 网孔中存在电流源时, $i_{M2}$优先考虑设定为与电流源方向相同.\\
\inlinehardlink{书本例2-3}\\
在存在受控电压源的情形下, 先将受控源视为独立源, 按照规则写出网孔方程, 最后将受控源的控制量用网孔方程表示.
\begin{cenum}
    \item 网孔电流不是支路电流, 网孔电流在网孔中所有支路上是一样的;
    \item 在含自电流源的网孔中, 电流源的电流值就是网孔电流, 可以不再列此网孔方程;
    \item 对于含有受控电压源的方程, 先视为独立源再代入约束.
\end{cenum}
\inlinehardlink{书本习题2-2}\\
对于非自电流的情形, 需要假设端电压. \inlinehardlink{cf. lecture sides, et FEC super nodes}

% paragraph 网孔方程的一般表达式 (end)

% subsection 网孔方程 (end)

\subsection{节点分析} % (fold)
\label{sub:节点分析}

选择一个节点为参考点, 列出每一节点到参考点的电压降, 谓之该节点的节点电压. 一个具有$n$个节点的电路有$n-1$个独立的节点电压. 所有支路电压都可以由节点电压表示. 支路电压由节点电压及KVL表出.
\par
只含有独立电源和电阻的电路, $\+vG \+vu = \+vi$, 其中$G_{ii}$为节点$i$的自电导(汇聚于该节点电导之和), $G_{12}$是节点$1$和节点$2$之间共有电导的{\color{red}负值}. $\+vi$是电流源输送给各节点电流的代数和.\\
\inlinehardlink{书本习题2-6}\\
\inlinehardlink{书本习题2-9}\\
若存在独立电压源的情形.
\par
如果已知电源为电流源, 则网孔方便. 电源为电压源则节点方便. 网孔只能平面电路, 节点无限制. 受控电流源暂时看作独立源, 两侧电压设出.

% subsection 节点分析 (end)

\subsection{对偶电路} % (fold)
\label{sub:对偶电路}

\[ \begin{cases}
    G_{11} u_{N1} +  G_{12}u_{N2} + G_{13} u_{N3} = i_{S11},\\
    G_{21} u_{N1} +  G_{22}u_{M2} + G_{23} u_{M3} = i_{S22}, \\
    G_{31} u_{N1} +  G_{32}u_{M2} + G_{33} u_{M3} = i_{S33}.
\end{cases} \]
网孔电流替换为节点电压, 电阻换以电导, 电压源换以电流源, 得到节点方程式. 反之, 也可以由后者得到前者.

% subsection 对偶电路 (end)

\subsection{含运算放大器的电阻电路} % (fold)
\label{sub:含运算放大器的电阻电路}

\begin{figure}
    \centering
    \incfig{6cm}{OPs}
    \caption{运算放大器}
\end{figure}
运算放大器(OP)需要由外电源$U$供电, 输出$u_0$将正比于$u_+-u_-$, 但$u_0$超出$U$后即取$U$.
\par
\inlinehardlink{运算放大器作为受控源}
运算放大器可以等效为受控源. 放大倍数$A$在理想情形下应当为$\infty$, 输入电阻在理想情况下应为$\infty$, 输出电阻在理想情形下应为零. 故$u_+ = u_-$, $i_+ = i_- = 0$. 
\begin{sample}
    \begin{ex}[比例器]
        \inlinehardlink{书本例2-12} 只需写出节点2的方程, 并且考虑到运算放大器的两端$u$相等, $u_2 = 0$,
        \[ G_1u_1 + G_3u_3 = 0, \]
        \[ u_O = u_3 = -\frac{R_2}{R_1}u_1 = -\frac{R_2}{R_1}u_S. \]
    \end{ex}
    \begin{ex}
        \inlinehardlink{书本例2-13}
    \end{ex}
\end{sample}

% subsection 含运算放大器的电阻电路 (end)

\subsection{图和树} % (fold)
\label{sub:图和树}

用一线段代替电路中的每一个元件, 该线段称为支路, 线段之端点谓节点.
\par
图谓一组节点和一组支路的集合. 支路只在节点处相交. 对每一支路规定一方向, 谓之有向图. 若图中任意两节点都有支路连通之, 则谓之连通图.
\par
不含任何回路的图为树. 构成树的各支路谓树支, 剩下的谓连支.
\par
图中选定一树后, 支路或为树支, 或为连支. 若图的节点数为$n$, 则树支数必为$n-1$. 由于树支数和连支数的总和为$b$, 可知连支数的总量为$b-\pare{n-1}$. 

% subsection 图和树 (end)

\subsection{割集分析法} % (fold)
\label{sub:割集分析法}

树支电压是一组完备的独立电压变量$\pare{n-1}$个. 数不能包含回路, 故树支电压不能用KVL相联系. 树连通所有节点, 因此任何两点之间的电压都可以用树支电压表示. 

\par
若对某些支路进行切割, 就会使图形分为两部分. 但只要少割任一支路, 则图形仍然连通, 则此支路之集合谓割集. 对于网络中的任一割集, 有流经割集的电流代数和为零.
\par
恰好包含一条树支的割集, 谓基本割集. $\pare{n-1}$个树支就有$\pare{n-1}$个基本割集.
\par
选出$\pare{n-1}$个基本割集后, 列出KCL并且用树支电压表示之, 可得完备的电压方程组.
\begin{remark}
    割集的参考方向应当与相应树支的关联参考方向一致.
\end{remark}
如果把含已知电压源的支路都选为树支, 可以减少方程数目. 

% subsection 割集分析法 (end)

\subsection{回路分析法} % (fold)
\label{sub:回路分析法}

连枝电流是线性无关的, 可以作为独立电流变量. 连枝电流本身完备, 树支电流可以用连支电流表示. 
\par
选定树后, 如果每次只接上一条连支, 就可以形成一个闭合回路, 由一条连支和其它有关树支组成, 则谓之基本回路.
\par
基本回路电流可以作为独立电流变量. 对这些回路电流写出KVL方程可得完备的方程组. 
\par
使用回路分析法时, 将电流源(无论是否已知)放在连支上可减少方程数目.

% subsection 回路分析法 (end)

% section 网孔分析和节点分析 (end)

\end{document}
