\documentclass{scrartcl}
\usepackage[T1]{fontenc}
\usepackage{mathptmx}
\usepackage[scaled =.92]{helvet}
\setlength{\paperwidth}{5.2075in}
\setlength{\paperheight}{4.90in}
\renewcommand*{\familydefault}{phv}
\usepackage[pdftex,margin=0.5in]{geometry}
\usepackage{fancyhdr}
\lhead{A Sample Calculation}\chead{}
\rhead{Area of Circle}
\lfoot{}\cfoot{}\rfoot{}
\pagestyle{fancy}
\usepackage{graphicx}
\usepackage{xcolor}
\usepackage[pdftex,pdfpagelayout=SinglePage,
pdftitle={Wishing you a happy year},pdfsubject={Invest your new year improving your TeX skills}%
]{hyperref}
\setlength{\parindent}{0.0cm}
\usepackage[pdftex]{insdljs}


\begin{insDLJS}[test]{test}{JavaScript}
function doCalculation()
{
var radius=0.0 + this.getField("radius").value;
 this.getField("diameter").value=radius*2;
 this.getField("areacircle").value=  Math.PI * Math.pow(radius, 2);
}
\end{insDLJS}

%% Short hand commands
\newcommand{\textforlabel}[2]{%
\TextField[name={#1}, value={#2}, width=9em,align=2,%
               bordercolor={0.990 .980 .85},%
               readonly=true]{}%
}


%% Define the heading
\newcommand{\heading}[1]{\textsc{#1}}

\begin{document}
\begin{center}
\textbf{\Huge Calculations\\ with JavaScript\\*[4pt] and pdfLateX!}
\end{center}

\newpage

\begin{Form}

\heading{Area of Circle}

%% 
%%% Input field radius

\textforlabel{l01}{Radius:}
\TextField[name=radius,width=10em, bordercolor={0.650 .790 .94}]{}%
~m\\

%% Push button is defined here
\textforlabel{l02}{Press to calculate}  
\PushButton[name=start,onclick={doCalculation();},bordercolor={0.650 .790 .94}%
]{Calculate}\\ 

\heading{Results}\\

%% RESULTS
%% Diameter
\textforlabel{name=l04}{%
Diameter :} \TextField[name=diameter,width=10em,bordercolor={0.650 .790 .94},%
readonly=true]{}~m

\textforlabel{name=l05}{Area:}  
\TextField[name=areacircle,width=10em,%
bordercolor={0.650 .790 .94},readonly=true]{}~m$^2$\\*[-0.8em]
\end{Form}
\end{document}