\documentclass{ctexart}

\usepackage[singleton, margintoc, nova]{van-de-la-sehen}

\begin{document}

\showtitle{奇技淫巧}

\section{不等式} % (fold)
\label{sec:不等式}

\subsection{离散不等式} % (fold)
\label{sub:离散不等式}

\subsubsection{均值不等式} % (fold)
\label{ssub:均值不等式}

\paragraph{反向归纳法} % (fold)
\label{par:反向归纳法}

均值不等式可以通过Lagrange乘子法证明, 惟引入微分并未必须. 通过反向归纳法, 假设对于$n=2,2^2,\cdots,2^k$成立, 则
\[ \pare{a_1\cdots a_{2^{k+1}}}^{\rec{2^{k+1}}} \le \frac{\pare{a_1\cdots a_{2^{k}}}^{\rec{2^{k}}} + \pare{a_{2^k+1}\cdots a_{2^{k+1}}}^{\rec{2^{k}}}}{2} \le \frac{a_1+\cdots + a_{2^{k+1}}}{2^{k+1}}.  \]
故$n=2^{k+1}$之情形得证. 此时开始反向归纳, 若$n+1$之情形正确, 记$\displaystyle A = \frac{a_1+\cdots+a_n}{n}$, 则
\begin{align*}
    \frac{a_1+\cdots+a_n + A}{n+1} &\ge \pare{a_1\cdots a_nA}^{\rec{n+1}}, \\
    A &\ge A^{\rec{n+1}} \cdot \pare{a_1\cdots a_n}^{\rec{n+1}}, \\
    A &\ge \pare{a_1\cdots a_n}^{\rec{n}}.
\end{align*}

% paragraph 反向归纳法 (end)

% subsubsection 均值不等式 (end)

% subsection 离散不等式 (end)

% section 不等式 (end)

\section{极限} % (fold)
\label{sec:极限}

\subsection{存在性} % (fold)
\label{sub:存在性}

\subsubsection{\texorpdfstring{$e$}{e}} % (fold)
\label{ssub:e}

引入$\displaystyle a_n = \pare{1+\rec{n}}^{n+1}$, 由均值不等式有
\[ \rec{a_n} = \pare{\frac{n}{n+1}}^{n+1} = \pare{\frac{1+\frac{n-1}{n}\cdot n}{n+1}}^{n+1} \ge \pare{\frac{n-1}{n}}^n = \rec{a_{n-1}}. \]
从而$a_n\le a_{n-1}$, $a_n$单调递减恒正, 故极限存在.

% subsubsection e (end)

% subsection 存在性 (end)

% section 极限 (end)

\end{document}
