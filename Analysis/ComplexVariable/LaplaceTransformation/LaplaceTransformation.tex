\documentclass[../ComplexVariable.tex]{subfiles}

\begin{document}

\section{Laplace变换} % (fold)
\label{sec:laplace变换}

\subsection{Hilbert空间} % (fold)
\label{sub:hilbert空间}

在$\+bR^n$上可以定义内积
\[ \bra{a}\ket{b} = \sum_{k=1}^n a_k b_k,\quad a = \begin{pmatrix}
    a_1 & \cdots & a_n
\end{pmatrix}^T,\quad b = \begin{pmatrix}
    b_1 & \cdots & b_n
\end{pmatrix}^T. \]
由内积可以导出范数
\[ \norm{a} = \sqrt{\bra{a}\ket{a}} = \sqrt{a_1^2 + \cdots + a_n^2}. \]
可以定义垂直$a\perp b \Leftrightarrow \bra{a}\ket{b} = 0$.
\begin{remark}
    在$\+bC^n$上内积应定义为
    \[ \bra{a}\ket{b} = \sum_{k=1}^n a_k \conj{b}_k. \]
\end{remark}
设$\xi_1,\cdots,\xi_n$是$\+bR^n$的一组标准正交基, 即
\[ \bra{\xi_i}\ket{\xi_j} = \begin{cases}
    1, & i=j, \\
    0, & i\neq j
\end{cases} = \delta_{ij}. \]
且$\forall \alpha\in\+bR^n$, $\alpha = a_1 \xi_1 + \cdots + a_n\xi_n$. 且对$1\le k \le n$,
\[ \bra{\alpha}\ket{\xi_k} = \bra{a_1\xi_1 + \cdots + a_n\xi_n}\ket{\xi_k} = \bra{a_1\xi_k}\ket{\xi_k} = a_1\bra{\xi_k}\ket{\xi_k} = a_1. \]
\begin{theorem}[Pythagorean定理, Parseval定理]
    在$\+bR^n$内,
    \[ \norm{\alpha}^2 = a_1^2 + \cdots + a_n^2. \]
\end{theorem}
对于可积且平方可积函数族
\[ \+sL\brac{a,b} = \setcond{\func{f}{\brac{a,b}}{\+bC}}{f,\abs{f}^2\text{可积}}, \]
可以定义内积
\[ \bra{f}\ket{g} = \int_a^b f\pare{x}\conj{g\pare{x}}\,\rd{x}. \]
以及范数
\[ \norm{f} = \sqrt{\bra{f}\ket{f}} = \pare{\int_a^b \abs{f}^2\,\rd{x}}^{\half}. \]
并定义两个函数垂直当且仅当$\bra{f}\ket{g} = 0$.
\begin{remark}
    $\brac{a,b}\mapsto \+bC$的函数可以视为$\+bC$上的无穷维向量.
\end{remark}
设$\xi_1,\cdots,\xi_n,\cdots$为$\+sL\brac{a,b}$的标准正交基, 其存在性由选择公理保证. $\bra{\xi_i,\xi_j} = \delta_{ij}$. 任意$f$皆可有展开
\[ f = a_1\xi_1 + \cdots + a_n\xi_n + \cdots = \sum_{k=1}^\infty a_k\xi_k. \]
此时$\bra{f}\ket{\xi_k}$谓Fourier系数, 上面的展开谓Fourier展开.
\begin{theorem}[Bessel不等式, Parseval定理]
    \[ \abs{a_1}^2 + \cdots + \abs{a_n}^2 \le \norm{f}^2. \]
    取$n\rightarrow \infty$的极限,
    \[ \abs{a}_1^2 + \cdots + \abs{a_n}^2 + \cdots = \norm{f}^2. \]
\end{theorem}
标准的Fourier级数是在$\brac{-\pi,\pi}$上按
\[ \rec{\sqrt{2\pi}}, \frac{\cos x}{\sqrt{\pi}}, \frac{\sin x}{\sqrt{\pi}}, \frac{\cos 2x}{\sqrt{\pi}}, \frac{\sin 2x}{\sqrt{\pi}}, \cdots \]
展开. 这是一组标准正交基. 此时展开式为
\[ f = A_0\rec{\sqrt{2\pi}} + \sum_{k=1}^\infty \pare{A_n \frac{\cos nx}{\sqrt{\pi}} + B_n \frac{\sin nx}{\sqrt{\pi}}}. \]
从而
\[ A_n = \bra{f}\ket{\frac{\cos nx}{\sqrt{\pi}}} = \rec{\sqrt{\pi}}\int_{-\pi}^{\pi} f\pare{x} \cos nx\,\rd{x},\quad B_n = \bra{f}\ket{\frac{\sin nx}{\sqrt{\pi}}}. \]
若采用指数形式,
\[ \bra{e^{inx}}\ket{e^{imx}} = \int_{-\pi}^\pi e^{inx}\conj{e^{imx}}\,\rd{x} = \int_{-\pi}^{\pi} e^{inx - imx}\,\rd{x} = \begin{cases}
    0, & n\neq m, \\
    2\pi, & n=m.
\end{cases} \]
此时对于$\brac{-\pi,\pi}$上的函数可以有展开
\[ f\pare{x} = \sum_{n=-\infty}^{+\infty} F_ne^{inx}, \]
投影为
\[ \bra{f}\ket{e^{inx}} = \bra{\sum_{k=-\infty}^{+\infty} F_ke^{ikx}}\ket{e^{inx}} = \bra{F_ne^{inx}}\ket{e^{inx}} = 2\pi F_n. \]
从而
\[ F_n = \rec{2\pi}\bra{f}\ket{e^{inx}} = \rec{2\pi} \int_{-\pi}^\pi f\pare{x}e^{-inx}\,\rd{x}. \]
若$f$以$2l$为周期, 则
\[ f\pare{x} = \sum_{n=\infty}^{\infty} F_n e^{in\frac{\pi}{l}x}, \]
其中
\[ F_n = \rec{2l}\int_{-l}^l f\pare{x} e^{-in\frac{\pi}{l}x}\,\rd{x}. \]
将$F_n$的表达式代入,
\begin{align*}
    f\pare{x} &= \lim_{n\rightarrow\infty} \sum_{n=-\infty}^{+\infty} F_n e^{in\frac{\pi}{l}x} \\
    &= \lim_{n\rightarrow\infty} \sum_{n=-\infty}^{+\infty} \brac{\rec{2l}\int_{-l}^l f\pare{x} e^{-in\frac{\pi}{l}x}\,\rd{x}} e^{in\frac{\pi}{l}x} \\
    &\xlongequal{\lambda_n = {n\pi}/{l},\Delta\lambda = \pi/l} \rec{2\pi} \lim_{\Delta\lambda\rightarrow 0}\sum_{-\infty}^{+\infty} \brac{\int_{-\pi/\Delta\lambda}^{\pi/\Delta\lambda} f\pare{t}e^{-i\lambda_n t}\,\rd{t}\, e^{i\lambda_n x}}\Delta\lambda \\
    &= \rec{2\pi} \int_{-\infty}^{+\infty} \brac{\int_{-\infty}^{+\infty} f\pare{t} e^{-i\lambda t}\,\rd{t}}e^{i\lambda x}\,\rd{\lambda}
\end{align*}
从而
\[ f\pare{x} = \rec{2\pi} \int_{-\infty}^{+\infty} \brac{\int_{-\infty}^{+\infty} f\pare{t} e^{-i\lambda t}\,\rd{t}}e^{i\lambda x}\,\rd{\lambda}. \]
定义Fourier变换及其逆变换
\begin{align*}
    \+sF\brac{f} &= \int_{-\infty}^{+\infty} f\pare{t} e^{-i\lambda t}\,\rd{t} = F,\\
    f = \+sF^{-1}\brac{F} &= \rec{2\pi} \int_{-\infty}^{+\infty} F\pare{\lambda} e^{i\lambda x}\,\rd{\lambda}.
\end{align*}
\begin{remark}
    若定义
    \begin{align*}
        \+sF\brac{f} &= \rec{\sqrt{2\pi}}\int_{-\infty}^{+\infty} f\pare{t} e^{-i\lambda t}\,\rd{t} = F,\\
    f = \+sF^{-1}\brac{F} &= \rec{\sqrt{2\pi}} \int_{-\infty}^{+\infty} F\pare{\lambda} e^{i\lambda x}\,\rd{\lambda},
    \end{align*}
    则Fourier变换与逆变换具有更加对称的形式.
\end{remark}

% subsection hilbert空间 (end)

\subsection{Laplace变换} % (fold)
\label{sub:laplace变换}

Fourier变换的缺点在于只能对$x\rightarrow \infty$时下降足够快的函数$f\pare{x}$有效. 若约定
\[ \restr{f}{\pare{-\infty, 0}} = 0, \]
且$f$在任何有限区间上都逐段光滑, 且存在某$c\ge 0$, $k>0$使得
\begin{equation}
    \label{eq:laplace速降条件}
    \abs{f\pare{t}} \le ke^{ct},\quad \forall t\ge 0,
\end{equation}
则可以保证Laplace变换的积分式有意义.
\begin{definition}[Laplace变换]
    \[ L\pare{f} = \+sF\brac{fe^{-\sigma t}} = \int_0^\infty fe^{-\sigma t} e^{-i\lambda t}\,\rd{t} = \int_0^\infty f\cdot e^{-pt}\,\rd{t},\quad p = \sigma + it. \]
\end{definition}
\begin{proposition}
    若\eqref{eq:laplace速降条件}满足, 则$F\pare{p}$在$\Re p > c$解析.
\end{proposition}
\begin{sample}
    \begin{ex}
        $L\brac{e^{at}} = \displaystyle \rec{p-a}$, 特别地, $L\brac{1} = \displaystyle \rec{p}$. 这里要求$\Re p > \Re a$. 因为
        \begin{align*}
            L\brac{e^{at}} &= \int_0^\infty e^{at}e^{-pt}\,\rd{t} = \int_0^\infty e^{\pare{a-p}t}\,\rd{t} \\
            &= \rec{a-p}\int_0^\infty e^{\pare{a-p}t}\,\rd{e^{\pare{a-p}t}} \\
            &= \rec{a-p}\brac{e^{\pare{a-p}t}}_0^\infty \\
            &= \rec{a-p}\brac{0-1} \\
            &= \rec{p-a}.
        \end{align*}
    \end{ex}
    \begin{remark}
        定义阶跃函数
        \[ h\pare{t} = \begin{cases}
            1, & t\ge 0,\\
            0, & t<0.
        \end{cases} \]
        则$f=1$之特例应写作$L\brac{h} = 1/p$. 而所有$f\pare{t}$在变换中实际上都应写作$f\pare{t}h\pare{t}$.
    \end{remark}
\end{sample}
\begin{sample}
    \begin{ex}
        $\displaystyle L\brac{t^\alpha} = \frac{\int_0^\infty x^\alpha e^{-x}\,\rd{x}}{p^{\alpha+1}} = \frac{\Gamma\pare{\alpha+1}}{p^{\alpha+1}}$.
        特别地,$\displaystyle L\brac{t^n} = \frac{n!}{p^{n+1}}$.
    \end{ex}
    \begin{proof}[幂函数的Laplace变换]
        \begin{align*}
            L\brac{t^n} &= \int_0^\infty t^n e^{-pt}\,\rd{t} \\ &= -\rec{p}\int_0^\infty t^n \,\rd{e^{-pt}} = -\rec{p}\brac{\left. t^n e^{-pt}\right\vert_0^\infty - \int_0^\infty e^{-pt}\,\rd{t^n}} \\
            &= -\rec{p} \brac{0 - 0 - \int_0^\infty e^{-pt}nt^{n-1}\,\rd{t}} \\
            &= \frac{n}{p}L\brac{t^{n-1}}. \\
            & L\brac{t^n} = \frac{n}{p}L\brac{t^{n-1}} = \frac{n}{p}\frac{n-1}{p} L\brac{t^{n-2}} = \frac{n!}{p^n} L\brac{1} = \frac{n!}{p^{n+1}}. \qedhere
        \end{align*}
    \end{proof}
\end{sample}
约定用小写字母表示原函数名, 大写字母表示变换后的函数名, $p$表示变换后的变量. 如
\[ L\brac{f} = F\pare{p}. \]
\begin{definition}
    卷积谓
    \[ f*g\pare{t} = \int_{-\infty}^{+\infty} f\pare{t-\xi}g\pare{\xi}\,\rd{\xi}. \]
\end{definition}
\begin{proposition}[卷积的性质]
    \mbox{}
    \begin{cenum}
        \item $f*g = g*f$;
        \item $f*\pare{g*h} = \pare{f*g}*h$;
        \item $f*\pare{g+h} = f*g + f*h$;
        \item $\+sF\brac{f*g} = \+sF\brac{f}\cdot\+sF\brac{g}$.
    \end{cenum}
\end{proposition}
\begin{finale}
\begin{theorem}[Laplace变换的性质]
    \mbox{}
    \begin{cenum}
        \item 线性性:
        \begin{align*}
            L\brac{\alpha f + \beta g} &= \alpha L\brac{f} + \beta L\brac{g},\\ L^{-1}\brac{\alpha f + \beta g} &= \alpha L^{-1}\brac{f} + \beta L^{-1}\brac{g}.
        \end{align*}
        \item 相似性:
        \[ L\brac{f\pare{at}} = \rec{a}F\pare{\frac{p}{a}},\quad a>0. \]
        \item 微分法:
        \begin{cenum}
            \item 变换前微分:
            \begin{align*}
            L\brac{f'\pare{t}} &= pF\pare{p} - f\pare{0}. \\
            L\brac{f^{\pare{n}}\pare{t}} &= p^n F\pare{p} - p^{n-1}f\pare{0} - p^{n-2}f'\pare{0} - \cdots \\ & \phantom{=}\  - tf^{\pare{n-2}}\pare{0} - f^{\pare{n-1}}\pare{0}.
        \end{align*}
            \item 变换后微分:
            \begin{align*}
                \+dpd{}L\brac{f\pare{t}} &= -L\brac{tf\pare{t}}, \\
                \frac{\rd{^n}}{\rd{p^n}}L\brac{f\pare{t}} &= \pare{-1}^n L\brac{t^n f\pare{t}}.
            \end{align*}
        \end{cenum}
        \item 积分法:
        \begin{cenum}
            \item 变换前积分:
            \begin{align*}
                L\brac{\int_0^t f\pare{t}\,\rd{t}} &= \frac{F\pare{p}}{p}, \\
                L\brac{\underbrace{\int_0^t \cdots \int_0^t}_{n} f\pare{t}\,\rd{t}\cdots \,\rd{t}} &= \frac{F\pare{p}}{p}.
            \end{align*}
            \item 变换后积分:
            \begin{align*}
                \int_p^\infty F\pare{p}\,\rd{p} &= L\brac{\frac{f\pare{t}}{t}}, \\
                \underbrace{\int_p^\infty \cdots \int_p^\infty F\pare{p}\,\rd{p}\cdots \,\rd{p}}_{n} &= L\brac{\frac{f\pare{t}}{t^n}}.
            \end{align*}
        \end{cenum}
        \item 延迟性:
        \begin{align*}
            L\brac{f\pare{t-\tau}} &= e^{-p\tau} F\pare{p}.
        \end{align*}
        严格的写法应作
        \begin{align*}
            L\brac{f\pare{t-\tau}h\pare{t-\tau}} &= e^{-p\tau}L\brac{f\pare{t}}, \\
            L^{-1}\brac{e^{-p\tau}F\pare{p}} &= f\pare{t-\tau} h\pare{t-\tau}.
        \end{align*}
        \item 位移定理:
        \begin{align*}
            L\brac{e^{\lambda t}f\pare{t}} &= F\pare{p-\lambda}.
        \end{align*}
        \item 若$f$以$T$为周期, 则
        \[ L\brac{f} = \rec{1-e^{-pT}} \int_0^T f\pare{t}e^{-pt}\,\rd{t}. \]
        \item 卷积公式:
        \[ L\brac{f*g} = L\brac{f}\cdot L\brac{g}. \]
    \end{cenum}
\end{theorem}
\end{finale}
\begin{proof}[相似性的证明]
    \begin{align*}
        L\brac{f\pare{at}} &= \int_0^\infty f\pare{at}e^{-pt}\,\rd{t} \\
        &\xlongequal{\xi = at} \int_0^\infty f\pare{\xi} e^{-p\xi /a}\rec{a}\,\rd{\xi} = \rec{a}F\pare{\frac{p}{a}}. \qedhere
    \end{align*}
\end{proof}
\begin{proof}[变换前微分法的证明]
    \begin{align*}
        L\brac{f'} &= \int_0^\infty f'\pare{t} e^{-pt}\,\rd{t} = \int_0^\infty e^{-pt}\,\rd{f\pare{t}} \\
        &= \left. e^{-pt}f\pare{t}\right\vert_0^\infty - \int_0^\infty f\pare{t}\,\rd{e^{-pt}} \\
        &= 0 - e^0 f\pare{0} + p\int_0^\infty f\pare{t} e^{-pt}\,\rd{t} \\
        &= pL\brac{f} - f\pare{0}. \qedhere
    \end{align*}
\end{proof}
\begin{proof}[变换后微分法的证明]
    \begin{align*}
        \+dpd{}L\brac{f\pare{t}} &= \+dpd{} \int_0^\infty f\pare{t} e^{-pt}\,\rd{t} \\
        &= \int_0^\infty f\pare{t}\+dpd{}e^{-pt}\,\rd{t} \\
        &= \int_0^\infty f\pare{t}\pare{-t}e^{-pt}\,\rd{t} \\
        &= -\int_0^\infty tf\pare{t}e^{-pt}\,\rd{t} = -L\brac{tf\pare{t}}. \qedhere
    \end{align*}
\end{proof}
\begin{proof}[变换前积分法的证明]
    \begin{align*}
        g\pare{t} &= \int_0^t f\pare{t}\,\rd{t},\quad g\pare{0} = 0,\quad g'\pare{t} = f\pare{t}, \\
        L\brac{g'\pare{t}} &= pL\brac{g} - g\pare{0}, \\
        L\brac{f\pare{t}} &= pL\brac{\int_0^t f\pare{t}\,\rd{t}} - 0, \\
        L\brac{\int_0^t f\pare{t}\,\rd{t}} &= \frac{L\brac{f\pare{t}}}{p}. \qedhere
    \end{align*}
\end{proof}
\begin{proof}[变换后积分法的证明]
    \begin{align*}
        \int_p^\infty F\pare{p}\,\rd{p} &= \int_p^\infty \int_0^\infty f\pare{t}e^{-pt}\,\rd{t}\,\rd{p} \\
        &= \int_0^\infty f\pare{t} \pare{\int_p^{\infty} e^{-pt}\,\rd{p}}\,\rd{t}.
    \end{align*}
\end{proof}
\begin{proof}[延迟性的证明]
    \begin{align*}
        L\brac{f\pare{t-\tau}} &= \int_0^\infty f\pare{t-\tau} e^{-pt}\,\rd{t} \\
        &= \int_\tau^\infty f\pare{t-\tau} e^{-pt}\,\rd{t} \\
        &\xlongequal{\xi = t-\tau} \int_0^\infty f\pare{\xi} e^{-p\pare{\xi + \tau}}\,\rd{\xi} \\
        &= e^{-p\tau}\int_0^\infty f\pare{\xi} e^{-p\xi}\,\rd{\xi} \\
        &= e^{-p\tau} L\brac{f}. \qedhere
    \end{align*}
\end{proof}
\begin{proof}[位移定理的证明]
    \begin{align*}
        L\brac{e^{\lambda t}f\pare{t}} &= \int_0^\infty e^{\lambda t}f\pare{t} e^{-pt}\,\rd{t} \\
        &= \int_0^\infty f\pare{t}e^{-\pare{p-\lambda}t}\,\rd{t} = F\pare{p-\lambda}. \qedhere
    \end{align*}
\end{proof}
\begin{proof}[周期函数的Laplace变换]
    \begin{align*}
        L\brac{f} &= \int_0^\infty f\pare{t} e^{-pt}\,\rd{t} \\
        &= \int_0^T + \int_T^{2T} + \int_{2T}^{3T} + \cdots. \qedhere
    \end{align*}
\end{proof}
\begin{proof}[另证]
    \begin{align*}
        L\brac{f} &= \int_0^\infty f\pare{t} e^{-pt}\,\rd{t} \\ &= \int_0^T f\pare{t} e^{-pt}\,\rd{t} + \int_T^\infty f\pare{t}e^{-pt}\,\rd{t}
         \\ &= \int_0^T + \int_0^\infty f\pare{\xi + t} e^{-p\pare{\xi + T}}\,\rd{\xi} \\
        &= \int_0^T + e^{-pt}\int_0^\infty f\pare{\xi} e^{-p\xi}\,\rd{\xi}. \\
        F\pare{p} &= \int_0^T + e^{-pt}F\pare{p}. \qedhere
    \end{align*}
\end{proof}
Laplace变换中, 卷积
\begin{align*}
    f_1*f_2\pare{t} &= \int_{-\infty}^{+\infty} f_1\pare{t-\xi} f_2\pare{\xi}\,\rd{\xi} \\
    &= \cancelto{0}{\int_{-\infty}^{0}} + \int_0^t + \cancelto{0}{\int_t^\infty} \\
    &= \int_0^t f_1\pare{t-\xi} f_2\pare{\xi}\,\rd{\xi}.
\end{align*}
\begin{sample}
    \begin{ex}
        线性性表明
        \begin{align*}
            L\brac{\cos\omega t} &= L\brac{\frac{e^{i\omega t} + e^{-i\omega t}}{2}} = \half L\brac{e^{i\omega t}} + \half L\brac{e^{-i\omega t}} \\ &= \half \brac{\rec{p- i\omega} + \rec{p+i\omega}} = \frac{p}{p^2+\omega^2}. \\
            L\brac{\sin\omega t} &= \frac{\omega}{p^2+\omega^2}.
        \end{align*}
    \end{ex}
\end{sample}
\begin{sample}
    \begin{ex}
        设$\displaystyle F\pare{p} = \frac{5p-1}{\pare{p+1}\pare{p-2}}$, 求$\displaystyle f\pare{t} = L^{-1}\brac{F\pare{p}}$.
    \end{ex}
    \begin{solution}
        $\displaystyle F\pare{p} = 2\rec{p+1} + 3\rec{p-2}$, 从而
        \[ f\pare{t} = L^{-1}\brac{F} = 2L^{-1}\brac{\rec{p+1}} + 3L^{-1}\brac{\rec{p-2}} = 2e^{-t} + 3e^{2t}. \qedhere \]
    \end{solution}
\end{sample}
\begin{sample}
    \begin{ex}
        求解微分方程$\displaystyle \begin{cases}
            y''\pare{t} + \omega^2 y\pare{t} = 0, \\
            y\pare{0} = 0,\quad y'\pare{0} = \omega.
        \end{cases}$
    \end{ex}
    \begin{solution}
        令$Y\pare{p} = L\brac{y\pare{t}}$, 对微分方程作Laplace变换,
        \begin{align*}
            & p^2Y\pare{p} - py\pare{0} - y'\pare{0} + \omega^2 Y\pare{p} = 0, \\
            &\Rightarrow Y\pare{p} = \frac{\omega}{p^2 + \omega^2}. \\
            &y\pare{t} = L^{-1}\brac{Y\pare{p}} = L^{-1}\pare{\frac{\omega}{p^2+\omega^2}} = \sin\omega t. \qedhere
        \end{align*}
    \end{solution}
\end{sample}
\begin{sample}
    \begin{ex}
        幂函数$f\pare{t} = t^n$的Laplace变换还可以通过
        \begin{align*}
            L\brac{f^{\pare{n}}\pare{t}} &= p^n L\brac{f\pare{t}} - p^{n-1}f\pare{0} - \cdots - f^{\pare{n-1}}\pare{0} = p^nL\brac{t^n}\\ &= L\brac{n!} = n!L\brac{1} = \frac{n!}{p}.\\
            \Rightarrow L\brac{t^n} &= \frac{n!}{p^{n+1}}
        \end{align*}
        得到.
    \end{ex}
\end{sample}
\begin{sample}
    \begin{ex}
        \begin{align*}
            L\brac{t\sin\omega t} &= -\+dpd{}\pare{\frac{\omega}{p^2+\omega^2}} = \frac{2p\omega}{\pare{p^2+\omega^2}^2}. \\
            L\brac{t^2\cos^2\omega t} &= \half L\brac{t^2 \pare{1+\cos 2t}} = \half \frac{\rd{^2}}{\rd{p^2}}\brac{\rec{p} + \frac{p^2}{p^2+4}} \\
            &= \frac{2\pare{p^6 + 24p^2 + 32}}{p^3 \pare{p^2+4}^3}.
        \end{align*}
    \end{ex}
\end{sample}
\begin{sample}
    \begin{ex}
        \begin{align*}
            L\brac{\sin t} &= \rec{1+p^2}, \\
            L\brac{\frac{\sin t}{t}} &= \int_p^\infty \rec{1+p^2}\,\rd{p} = \frac{\pi}{2} - \arctan p. \\
            \int_0^\infty \frac{\sin t}{t} e^{-pt}\,\rd{t} &= \frac{\pi}{2} - \arctan p.
        \end{align*}
        特别地, 令$p = 0$, 则
        \[ \int_0^\infty \frac{\sin t}{t}\,\rd{t} = \frac{\pi}{2}. \]
    \end{ex}
\end{sample}
\begin{remark}
    令$p=0$, 则有
    \begin{align*}
        \int_0^\infty f\pare{t}\,\rd{t} &= F\pare{0}, \\
        \int_0^\infty tf\pare{t}\,\rd{t} &= -F'\pare{0}, \\
        \int_0^\infty \frac{f\pare{t}}{t}\,\rd{t} &= \int_0^\infty F\pare{p}\,\rd{p}.
    \end{align*}
\end{remark}
\begin{sample}
    \begin{ex}
        为了计算
        \[ \int_0^\infty \frac{e^{-at}- e^{-bt}}{t}\,\rd{t}, \]
        考虑到
        \begin{align*}
            L\brac{e^{-at} - e^{-bt}} &= \rec{p+a} - \rec{p+b}, \\
            \int_0^\infty \frac{e^{-at}- e^{-bt}}{t}\,\rd{t} &= \int_0^\infty \pare{\rec{p+a} - \rec{p+b}}\,\rd{p} = \left.\ln \frac{p+a}{p+b}\right\vert_0^\infty = \ln \frac{b}{a}.
        \end{align*}
    \end{ex}
\end{sample}
\begin{sample}
    \begin{ex}
        为了计算
        \[ \int_0^\infty e^{-3t}\cos 2t\,\rd{t}, \]
        考虑到
        \begin{align*}
            L\brac{\cos 2t} &= \frac{p}{p^2+4}, \\
        \int_0^\infty e^{-3t}\cos 2t\,\rd{t} &= \left. L\brac{\cos 2t}\right\vert_{p=3} = \left. \frac{p}{p^2+4}\right\vert_{p=3} = \frac{3}{13}.
        \end{align*}
    \end{ex}
\end{sample}
\begin{sample}
    \begin{ex}
        为了计算
        \[ \int_0^\infty \frac{1-\cos t}{t}e^{-t}\,\rd{t}, \]
        考虑到
        \begin{align*}
            L\brac{\frac{1-\cos t}{t}} &= \int_p^\infty L\brac{1-\cos t}\,\rd{p} \\
            &= \int_p^\infty \rec{p\pare{p^2+1}}\,\rd{p} \\
            &= \left. \half\ln \frac{p^2}{p^2+1}\right\vert_p^\infty = \half \ln \frac{p^2+1}{p^2}.
        \end{align*}
    \end{ex}
\end{sample}
\begin{sample}
    \begin{ex}
        $f\pare{t} = \sin t$, 求$\displaystyle L\brac{f\pare{t-\frac{\pi}{2}}}$.
    \end{ex}
    \begin{solution}
        \begin{align*}
            L\brac{f\pare{t-\frac{\pi}{2}}} &= e^{-p\pi/2}L\brac{f\pare{t}} = e^{-p\pi/2}L\brac{\sin t} = e^{-p\pi/2} \rec{1+p^2}. \qedhere
        \end{align*}
    \end{solution}
    \begin{remark}
        $\displaystyle L\brac{\sin\pare{t - \frac{\pi}{2}}} = L\brac{-\cos t} = -\rec{p^2+1}$, 但$\displaystyle f\pare{t-\frac{\pi}{2}}$在$\displaystyle \pare{0,\frac{\pi}{2}}$的取值为零.
    \end{remark}
    \begin{remark}
        采用严格写法, 则
        \begin{align*}
            L^{-1}\brac{\frac{e^{-p\pi/2}}{1+p^2}} &= \sin \pare{t-\frac{\pi}{2}} h\pare{t-\frac{\pi}{2}} \\
            &= \begin{cases}
                -\cos t, & t\ge \pi/2, \\
                0, & t<\pi/2.
            \end{cases}
        \end{align*}
    \end{remark}
\end{sample}
\begin{pitfall}
    应用延迟性时应采用严格写法.
\end{pitfall}
\begin{sample}
    \begin{ex}
        \begin{align*}
            L\brac{e^{\lambda t}t^n} &= \frac{n!}{\pare{p-\lambda}^{n+1}}, \\
            L\brac{e^{\lambda t}\sin \omega t} &= \frac{\omega}{\pare{p-\lambda}^2 + \omega^2}, \\
            L\brac{e^{\lambda t}\cos \omega t} &= \frac{p-\lambda}{\pare{p-\lambda}^2 + \omega^2}.
        \end{align*}
        任何一个有理函数都可以写成右侧三种有理函数的线性组合, 故借助这三条公式可以简化有理函数的积分.
    \end{ex}
\end{sample}
\begin{sample}
    \begin{ex}
        $f_1\pare{t} = t$, $f_2\pare{t} = \sin t$,
        \begin{align*}
            f_1*f_2 &= \int_0^t \tau \sin \pare{t-\tau}\,\rd{\tau} \\
            &= \left.\tau\cos\pare{t-\tau}\right\vert_0^t - \int_0^t \cos\pare{t-\tau}\,\rd{\tau} \\
            &= t-\sin t.
        \end{align*}
    \end{ex}
\end{sample}
\begin{sample}
    \begin{ex}
        设$\displaystyle F\pare{p} = \frac{p^2}{\pare{p^2+1}^2}$, 求$f\pare{t} = L^{-1}\brac{F}$.
    \end{ex}
    \begin{solution}
        \begin{align*}
            f\pare{t} &= L^{-1}\brac{f} = L^{-1}\brac{\frac{p}{p^2+1}\cdot \frac{p}{p^2+1}} \\
            &= L^{-1}\brac{\frac{p}{p^2+1}} * L^{-1}\brac{\frac{p}{p^2+1}} \\
            &= \cos t * \cos t \\
            &= \half \pare{t\cos t + \sin t}. \qedhere
        \end{align*}
    \end{solution}
\end{sample}

% subsection laplace变换 (end)

\subsection{Laplace逆变换} % (fold)
\label{sub:laplace逆变换}

对于Fourier变换和逆变换, 有
\[ \+sF\brac{f} = \int_{-\infty}^{+\infty} f\pare{t}e^{-i\sigma t}\,\rd{t},\quad \+sF^{-1}\brac{F} = \rec{2\pi} \int_{-\infty}^{\infty} F\pare{\sigma} e^{i\sigma x}\,\rd{\sigma}. \]
设$p=\sigma + is$, 有
\begin{align*}
    L\brac{f} = F\pare{p} &= \int_{-\infty}^{+\infty} \pare{f\pare{t}h\pare{t}e^{-\sigma t}}e^{-ist}\,\rd{t}, \\
    f\pare{t}h\pare{t}e^{-\sigma t} &= \rec{2\pi} \int_{-\infty}^{+\infty} F\pare{\sigma + is}e^{ist}\,\rd{s}.
\end{align*}
\begin{theorem}
    取定$\sigma > c$, 在$f\pare{t}$连续处, 有
    \[ f\pare{t} = \rec{2\pi i}\int_{\sigma - i\infty}^{\sigma + i\infty}F\pare{p} e^{-pt}\,\rd{p}. \]
\end{theorem}
\begin{theorem}
    若$F\pare{p}$除在半平面$\Re p\le \sigma$有奇点$p_1,p_2,\cdots,p_n$外处处解析, $p\rightarrow\infty$时$F\pare{p}\rightarrow 0$, 且
    \[ \int_{\sigma-i\infty}^{\sigma+i\infty} F\pare{p}\,\rd{p} \]
    当$\sigma > c$时绝对收敛, 则
    \begin{cenum}
        \item $F\pare{p}$为
        \[ f\pare{t} = \rec{2\pi i}\int_{\sigma-i\infty}^{\sigma+i\infty} F\pare{p}e^{pt}\,\rd{p} \]
        的像函数;
        \item $\displaystyle f\pare{t} = \sum_{k=1}^n \Residue\pare{F\pare{p}e^{pt},p_k}$, $t>0$.
    \end{cenum}
\end{theorem}
\begin{proof}
    设$p=\sigma+is$,
    \begin{align*}
        F\pare{p} &= \int_{-\infty}^{+\infty} \pare{f\pare{t}h\pare{t}e^{-\sigma t}} e^{-ist}\,\rd{t}, \\
        f\pare{t}g\pare{t}e^{-\sigma t} &= \rec{2\pi}\int_{-\infty}^{+\infty}F\pare{p}e^{ist}\,\rd{s} \\
        \Rightarrow f\pare{t} &= \rec{2\pi i} \int_{\sigma-i\infty}^{\sigma+i\infty} F\pare{p}e^{pt}\,\rd{p}. \qedhere
    \end{align*}
\end{proof}
欲求
\[ F\pare{p} = \frac{Q_m}{P_n} \]
的Laplace逆变换, 对$P_n$求根后可得部分分式分解
\[ F\pare{p} = \frac{Q_m}{P_n} = \sum c_j \rec{\pare{p-z_j}^{k_j}}. \]
欲求得实系数的部分分式分解, 将$P_n$分解为一次或二次的实系数多项式即可. 再引用
\begin{align*}
            {e^{\lambda t}t^n} &= L^{-1}\brac{\frac{n!}{\pare{p-\lambda}^{n+1}}}, \\
            {e^{\lambda t}\sin \omega t} &= L^{-1}\brac{\frac{\omega}{\pare{p-\lambda}^2 + \omega^2}}, \\
            {e^{\lambda t}\cos \omega t} &= L^{-1}\brac{\frac{p-\lambda}{\pare{p-\lambda}^2 + \omega^2}}.
\end{align*}
\begin{sample}
    \begin{ex}
        求$\displaystyle F\pare{p} = \rec{\pare{p-2}\pare{p-1}^2}$的Laplace逆变换.
    \end{ex}
    \begin{solution}
        设$F\pare{p} = \frac{a_1}{p-2} + \frac{a_2}{p-1} + \frac{a_3}{\pare{p-1}^2}$, 通过代入$p$的特殊值求解待定系数可得
        \begin{align*}
            F\pare{p} &= \rec{p-2} - \rec{p-1} - \rec{\pare{p-1}^2}. \\
            f\pare{t} &= L^{-1}\brac{\rec{p-2}} - L^{-1}\brac{\rec{p-1}} - L^{-1}\brac{\rec{\pare{p-1}^2}} \\
            &= e^{2t} - e^t - te^t. \qedhere
        \end{align*}
    \end{solution}
    \begin{proof}[第二种解法]
        $\displaystyle F_1\pare{p} = \rec{p-2}$, $\displaystyle F_2\pare{p} = \rec{\pare{p-1}^2}$, 分别有逆变换$f_1\pare{t} = L^{-1}\brac{F_1} = e^{2t}$, $f_2\pare{t}=L^{-1}\brac{F_2}=te^t$, 故
        \begin{align*}
            f\pare{t} &= L^{-1}\brac{F_1\cdot F_2} = f_1 * f_2 \\
            &= \int_0^t \tau e^\tau e^{2\pare{t-\tau}}\,\rd{\tau} \\
            &= e^{2t} - e^t - te^t. \qedhere
        \end{align*}
    \end{proof}
    \begin{proof}[第三种解法]
        \begin{align*}
            f\pare{t} &= \Residue\brac{F\pare{p}e^{pt},2} + \Residue\brac{F\pare{p}e^{pt}, 1} \\
            &= \left.\frac{e^{pt}}{\pare{p-1}^2}\right\vert_{p=2} + \left.\pare{\frac{e^{pt}}{\pare{p-2}}}'\right\vert_{p=1} \\
            &= e^{2t} - e^{t} - te^t. \qedhere
        \end{align*}
    \end{proof}
\end{sample}
\begin{sample}
    \begin{ex}
        求解$\displaystyle \begin{cases}
            x''\pare{t} - 2x'\pare{t} + 2x\pare{t} = 2e^t\cos t, \\
            x'\pare{0} = x\pare{0} = 0.
        \end{cases}$
    \end{ex}
    \begin{solution}
        令$X\pare{p} = L\brac{x\pare{t}}$,
        \begin{align*}
            & p^2X\pare{p} - 2pX\pare{p} + 2X\pare{p} = \frac{2\pare{p-1}}{\pare{p-1}^2 + 1}. \\
            & X\pare{p} = \frac{2\pare{p-1}}{\brac{\pare{p-1}^2 + 1}^2}. \\
            & x\pare{t} = L^{-1}\brac{X\pare{p}} = L^{-1}\brac{\frac{2\pare{p-1}}{\brac{\pare{p-1}^2+1}^2}} \\
            &= e^t L^{-1}\brac{\frac{2p}{\pare{p^2+1}^2}} \\
            &= e^t L^{-1}\brac{\+dpd{}\pare{\frac{-1}{p^2+1}}} \\
            &= te^t L^{-1}\brac{\rec{p^2+1}} \\
            &= te^t\sin t. \qedhere
        \end{align*}
    \end{solution}
\end{sample}
\begin{sample}
    \begin{ex}
        求解$\displaystyle \begin{cases}
            x'\pare{t} + x\pare{t} - y\pare{t} = e^t, \\
            y'\pare{t} + 3x\pare{t} - 2y\pare{t} = 2e^t, \\
            x\pare{0} = y\pare{0} = 1.
        \end{cases}$
    \end{ex}
    \begin{proof}
        令$X\pare{p} = L\brac{x\pare{t}}$, $Y\pare{p} = L\brac{y\pare{t}}$, 有
        \begin{align*}
            & \left\{ \begin{aligned}
                pX\pare{p} - 1 + X\pare{p} - Y\pare{p} = \rec{p-1}, \\
                pY\pare{p} - 1 + 3X\pare{p} - 2Y\pare{p} = \frac{2}{p-1}
            \end{aligned} \right. \Rightarrow \left\{ \begin{aligned}
                X\pare{p} = \rec{p-1}, \\
                Y\pare{p} = \rec{p-1}.
            \end{aligned} \right.\\
            &\Rightarrow x\pare{t} = y\pare{t} = e^t. \qedhere
        \end{align*}
    \end{proof}
\end{sample}
\begin{sample}
    \begin{ex}
        质量为$m$的物体静止在原点. 在$t=0$时受到$x$方向的力$F_0\delta\pare{t}$.
    \end{ex}
    \begin{solution}
        $\displaystyle \begin{cases}
            mx''\pare{t} = F_0\delta\pare{t}, \\
            x\pare{0} = x'\pare{0} = 0.
        \end{cases}$而$L\brac{\delta\pare{t}} = 1$, $F*\delta\pare{t} = F$.
        \[ mpX\pare{p} = F_0 \Rightarrow X\pare{p} = \frac{F_0}{mp^2} \Rightarrow x\pare{t} = \frac{F_0}{m}t. \qedhere \]
    \end{solution}
\end{sample}

% subsection laplace逆变换 (end)

\subsection{Mellin变换} % (fold)
\label{sub:mellin变换}

Mellin变换及其逆变换为
\begin{align*}
    \curb{\+cMf}\pare{s} &= \varphi\pare{s} = \int_0^\infty x^{s-1}f\pare{x}\,\rd{x},\\ \curb{\+cM^{-1}\varphi}\pare{x} &= f\pare{x} = \rec{2\pi i} \int_{c-i\infty}^{c+i\infty} x^{-s}\varphi\pare{s}\,\rd{s}.
\end{align*}
其与Fourier变换之间有联系
\begin{align*}
    \curb{\+sFf}\pare{-s} &= \curb{\+cMf\pare{-\ln x}}\pare{-is}, \\
    \curb{\+cMf}\pare{s} &= \curb{\+sFf\pare{e^{-x}}}\pare{-is}.
\end{align*}
\begin{sample}
    \begin{ex}
        记$f\pare{x} = e^{-x}$, 则
        \[ \curb{\+cMf}\pare{s} = \int_0^\infty x^{s-1}e^{-x}\,\rd{x} = \Gamma\pare{s}. \]
        采用$y^{-s}$的主支, 对$c>0$有
        \[ e^{-y} = \rec{2\pi i} \int_{c-i\infty}^{c+i\infty}\Gamma\pare{s}y^{-s}\,\rd{s}. \]
    \end{ex}
\end{sample}
\begin{sample}
    \begin{ex}
        记$f\pare{x} = e^{-px}$, 则
        \[ \curb{\+cMf}\pare{s} = \int_0^\infty x^se^{-px}\,\frac{\rd{x}}{x} = \rec{p^s}\Gamma\pare{s}. \]
    \end{ex}
\end{sample}
\begin{sample}
    \begin{ex}
        记$\displaystyle f\pare{x} = \rec{e^x-1}$, 则
        \[ \curb{\+cMf}\pare{s} = \int_0^\infty x^{s-1}\rec{e^x-1}\,\rd{x} = \sum_{n=1}^\infty \rec{n^s}\Gamma\pare{s} = \Gamma\pare{s}\zeta\pare{z}. \]
    \end{ex}
\end{sample}
\begin{sample}
    \begin{ex}
        对微分方程
        \[ \rec{r}\+DrD{}\pare{r\+DrDf} + \rec{r^2}\frac{\partial^2 f}{\partial\theta^2} = 0 \]
        作Mellin变换$r\mapsto s$, 有
        \[ s^2F + \frac{\partial^2 F}{\partial\theta^2} = 0. \]
        这有通解
        \[ F\pare{s,\theta} = C_1\pare{s}\cos\pare{s\theta} + C_2\pare{s}\sin\pare{s\theta}. \]
        再作逆变换即可.
    \end{ex}
\end{sample}
设有Dirichlet级数
\[ g\pare{s} = \sum_{n=1}^\infty \frac{a\pare{n}}{n^s} = s\int_1^\infty A\pare{x}x^{-\pare{s+1}}\,\rd{x}, \]
其中$\displaystyle A\pare{x} = \sideset{}{'}\sum_{n\le x} a\pare{n}$, 其中$\displaystyle \sideset{}{'}\sum$表示$n=x$的项取权重$1/2$. 由Mellin逆变换,
\[ A\pare{x} = \sideset{}{'}\sum_{n\le x} a\pare{n} = \rec{2\pi i}\int_{c-i\infty}^{c+i\infty}g\pare{z} \frac{x^z}{z}\,\rd{z}. \]
设$\Lambda\pare{n}$为von Mangoldt函数, 即若有某素数$p$满足$n=p^k$则$\Lambda\pare{n} = \log p$否则$\Lambda\pare{n} = 0$, 则
\[ \frac{\zeta'\pare{s}}{\zeta\pare{s}} = -\sum_{n=1}^\infty \frac{\Lambda\pare{n}}{n^s}. \]
从而
\[ \sideset{}{'}\sum_{n\le x} \Lambda\pare{n} = \rec{2\pi i}\int_{c-i\infty}^{c+i\infty} \pare{-\frac{\zeta'\pare{z}}{\zeta\pare{z}}}\frac{x^z}{z}\,\rd{z}. \]
$\zeta\pare{z}$在$\Re z = 1$上仅有$z=1$处有奇点且无零点之事实表明
\[ \sideset{}{'}\sum_{n\le x} \Lambda\pare{x} \sim x. \]
再由$\displaystyle \sideset{}{'}\sum_{n\le x} \Lambda\pare{x} \sim \pi\pare{x}\log x$, 可得$\displaystyle \pi\pare{x} \sim \frac{x}{\ln x}$, 即素数定理.

% subsection mellin变换 (end)

% section laplace变换 (end)

\end{document}
