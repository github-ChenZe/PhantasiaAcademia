\documentclass{ctexart}

\usepackage{van-de-la-sehen}

\begin{document}

\section{部分分式分解} % (fold)
\label{sec:部分分式分解}

设$f$是$\+bR$上的有理函数, 即存在$\+bR$上的多项式使
\[ f\pare{x} = \frac{p\pare{x}}{q\pare{x}}, \]
若$q\pare{x}$已被分解为线性和二次多项式的积, 则
\begin{align*}
    & f\pare{x} = \frac{p\pare{x}}{q\pare{x}} \\
    &= \frac{p\pare{x}}{\pare{x-a_1}^{j_1}\cdots\pare{x-a_m}^{j_m}\pare{x^2+b_1x+c_1}^{k_1}\cdots\pare{x^2+b_n x+c_n}^{k_n}} \\
    &= P\pare{x} + \sum_{i=1}^m \brac{\frac{A_{i1}}{x-a_i} + \frac{A_{i2}}{\pare{x-a_i}^2} + \cdots + \frac{A_{ij_i}}{\pare{x-a_i}^{j_i}}} \\
    &+ \sum_{i=1}^n \brac{\frac{B_{i1}x + C_{i1}}{x^2 + b_i x + c_i} + \frac{B_{i2}x + C_{i2}}{\pare{x^2 + b_i x + c_i}^2} + \cdots + \frac{B_{ik_i}x + C_{ik_i}}{\pare{x^2 + b_i x + c_i}^{k_i}}}
\end{align*}
\begin{sample}
    \begin{ex}
        设$\displaystyle f\pare{x} = \rec{x^2+2x-3} = \frac{A}{x+3} + \frac{B}{x-1}$, 则通分后有
        \[ 1 = A\pare{x-1} + B\pare{x+3}. \]
        令$x=-3$有$A = -1/4$, 令$x=1$有$B=1/4$.
    \end{ex}
\end{sample}
\begin{sample}
    \begin{ex}
        设$\displaystyle f\pare{x} = \frac{4x^2 - 8x + 16}{x\pare{x^2-4x+8}} = \frac{A}{x} + \frac{Bx+C}{x^2 - 4x + 8}$. 通分有
        \[ 4x^2 - 8x + 16 = A\pare{x^2 - 4x + 8} + \pare{Bx+C}x. \]
        令$x = 0$有$16 = 8A$, 比较二次项系数有$4 = A+B$, 比较一次项系数有$-8 = -4A+C$, 从而
        \[ A=2,\quad B=2,\quad C=0. \]
    \end{ex}
\end{sample}
\begin{sample}
    \begin{ex}
        设
        \begin{align*}
            f &= \frac{2x^6 - 4x^5 + 5x^4 - 3x^3 + x^2 + 3x}{\pare{x-1}^3\pare{x^2+1}^2} \\ &= \frac{A}{x-1} + \frac{B}{\pare{x-1}^2} + \frac{C}{\pare{x-1}^3} + \frac{Dx+E}{x^2+1} + \frac{Fx+G}{\pare{x^2+1}^2}. 
        \end{align*}
        通分有
        \begin{align*}
            & 2x^6 - 4x^5 + 5x^4 - 3x^3 + x^2 + 3x \\
            &= A\pare{x-1}^2 \pare{x^2+1}^2 + B\pare{x-1}\pare{x^2+1}^2 + C\pare{x^2+1}^2 \\
            &+\pare{Dx+E}\pare{x-1}^3\pare{x^2+1} + \pare{Fx+G}\pare{x-1}^3.
        \end{align*}
        将$x$的若干特殊值代入, 得
        \[ \begin{matrix}
            x=1 & \Rightarrow & 4=4C, \\
            x=i & \Rightarrow & 2+2i = \pare{Fi+G}\pare{2+2i}, \\
            x=0 & \Rightarrow & 0 = A-B+C-E-G,
        \end{matrix} \Rightarrow \begin{cases}
            C = 1, \\
            F = 0,\quad G = 1,\\
            E = A-B.
        \end{cases} \]
        在$x=1$处求导, 则
        \begin{align*}
            & \left. 12x^5 - 20x^4 + 20x^3 - 9x^2 + 2x + 3\right\vert_{x=1} \\ &= \left. B\pare{x^2+1}^2 + C\cdot 2\pare{x^2+1}\cdot 2x\right\vert_{x=1}.
        \end{align*}
        即$8 = 4B + 8C$, $B = 0$. 两侧取$x\rightarrow\infty$的极限, 有
        \[ 2 = A + D. \]
        取$x^5$项系数, 有
        \[ -4 = A\pare{-2\cdot 1} + \pare{E - 3D}. \]
        从而$A = D = E = 1$.
    \end{ex}
\end{sample}
\begin{sample}
    \begin{ex}
        设$\displaystyle f = \frac{x}{\pare{x^2+2x+5}\pare{x^2+4}} = \frac{Ax+B}{x^2+2x+5} + \frac{Cx+D}{x^2+4}$. 通分有
        \[ x = \pare{Ax+B}\pare{x^2+4} + \pare{x^2+2x+5}\pare{Cx+D}. \]
        令$x$取诸特殊值, 则
        \[ \begin{matrix}
            x=\infty & \Rightarrow & A+C = 0, \\
            x=0 & \Rightarrow & 4B+5D = 0, \\
            x=2i & \Rightarrow & \pare{2iC+D}\pare{1+4i} = 2i,
        \end{matrix} \Rightarrow \begin{cases}
            A+C = 0, \\
            4B + 5D = 0,\\
            D - 8C = 0,\\ 4D + 2C = 2.
        \end{cases} \]
        从而$A = -1/17$, $B = -10/17$, $C = 1/17$, $D = 8/17$.
    \end{ex}
\end{sample}

% section 部分分式分解 (end)

\section{最大模原理} % (fold)
\label{sec:最大模原理}

\begin{sample}
    \begin{ex}
        设$f$在$\+bC$上解析, $\abs{f\pare{z}} = 1$在$\abs{z} = 1$上成立. 若$f$在$\abs{z}<1$内无零点, 则$\log\abs{f}$是调和函数, 且在边界上取常数值, 故$\abs{f}$在$\abs{z}<1$内为常数. 若有零点, 则视零点是否为$z=0$逐次分解出形如$z^m$或
        \[ \frac{a_j - z}{1-\conj{a_j}z} \]
        的因子, 仍有$\abs{f\pare{z}} = 1$在边界成立, 惟零点数少$1$. 然而形如$\displaystyle \frac{a_j - z}{1-\conj{a_j}z}$的因子不可能出现, 否则$f$有极点.
    \end{ex}
    \begin{ex}
        \href{https://math.stackexchange.com/questions/626488/find-all-entire-functions-fz-such-that-fz-1-for-z-1}{另一证明}为, 考虑函数
        \[ z\mapsto \rec{\conj{f\pare{1/\conj{z}}}}, \]
        其在$\+bC\backslash\curb{0}$上解析且在单位圆上与$f$一致, 故
        \[ f\pare{z} = \rec{\conj{f\pare{1/\conj{z}}}}, \]
        因此$f$不可能有$z=0$外的零点, 否则$f$在$z=0$外有极点. $f=0$在$z=0$处成立, 则$f\pare{\infty} = \infty$, 从而$f$为多项式, 从而为单项式.
    \end{ex}
\end{sample}

% section 最大模原理 (end)

\end{document}
