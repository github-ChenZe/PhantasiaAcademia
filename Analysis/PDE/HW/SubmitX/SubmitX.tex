\documentclass[hidelinks]{ctexart}

\usepackage{van-de-la-illinoise}
\usepackage[paper=b5paper,top=.3in,left=.9in,right=.9in,bottom=.3in]{geometry}
\usepackage{calc}
\pagenumbering{gobble}
\setlength{\parindent}{0pt}
\sisetup{inter-unit-product=\ensuremath{{}\cdot{}}}
\usepackage{van-le-trompe-loeil}

\usepackage{stackengine}
\stackMath
\usepackage{scalerel}
\usepackage[outline]{contour}

\newdimen\indexlen
\def\newprobheader#1{%
\def\probindex{#1}
\setlength\indexlen{\widthof{\textbf{\probindex}}}
\hskip\dimexpr-\indexlen-1em\relax
\textbf{\probindex}\hskip1em\relax
}
\def\newprob#1{%
\newprobheader{#1}%
\def\newprob##1{%
\probsep%
\newprobheader{##1}%
}%
}
\def\probsep{\vskip1em\relax{\color{gray}\dotfill}\vskip1em\relax}

\newlength\thisletterwidth
\newlength\gletterwidth
\newcommand{\leftrightharpoonup}[1]{%
{\ooalign{$\scriptstyle\leftharpoonup$\cr%\kern\dimexpr\thisletterwidth-\gletterwidth\relax
$\scriptstyle\rightharpoonup$\cr}}\relax%
}
\def\tensor#1{\settowidth\thisletterwidth{$\mathbf{#1}$}\settowidth\gletterwidth{$\mathbf{g}$}\stackon[-0.1ex]{\mathbf{#1}}{\boldsymbol{\leftrightharpoonup{#1}}}  }
\def\onedot{$\mathsurround0pt\ldotp$}
\def\cddot{% two dots stacked vertically
  \mathbin{\vcenter{\baselineskip.67ex
    \hbox{\onedot}\hbox{\onedot}}%
}}%

\begin{document}

\newprob{5.3}%
偶関数であることに注意すれば、$\delta\pare{x}$のフーリエ係数は、
\[ A_n = \rec{\pi}\int_{-\pi}^\pi \delta\pare{x} \cos nx\,\rd{x} = \rec{\pi},\quad B_n \equiv 0 \]
となる. したがって、$\delta\pare{x}$のフーリエ級数は、
\[ \boxed{\delta\pare{x} \sim \rec{2\pi} + \rec{\pi} \sum_{n=1}^\infty \cos nx} \]
となる. リーマン·ルベーグの補題より
\[ \lim_{n\rightarrow \infty} \sin n x \xlongequal{w} 0 \]
だから$\displaystyle \int_{\epsilon<\abs{x}<\pi} \frac{f\pare{x}}{\sin \pare{x/2}}\sin \pare{N+\half }x\,\rd{x} \rightarrow 0,\quad \forall \epsilon > 0$.
\[  \int_{-\pi}^{\pi} \rec{2\pi}\pare{1+2\sum_{n=1}^N \cos nx}\,\rd{x} = 1\quad\text{及び}\quad \lim_{n\rightarrow \infty} \lim_{x\rightarrow 0} \frac{\displaystyle \sin\pare{N+\half} x}{\displaystyle 2\pi \sin \frac{x}{2}} = \infty \]
は明らか. よって
\begin{align*}
    \hspace{-2em}& \lim_{N\rightarrow \infty} \int_{-\pi}^\pi \rec{2\pi}\pare{1+2\sum_{n=1}^N \cos nx} f\pare{x} \,\rd{x}\\
    \hspace{-2em} &= \lim_{N\rightarrow \infty} \int_{-\pi}^\pi \rec{2\pi}\pare{1+2\sum_{n=1}^N \cos nx} f\pare{0}\,\rd{x}\\
    &\phantom{=\ } + \lim_{N\rightarrow \infty} \int_{-\pi}^\pi \rec{2\pi}\pare{1+2\sum_{n=1}^N \cos nx} \brac{f\pare{x} - f\pare{0}}\,\rd{x} \\
    \hspace{-2em} &= f\pare{0} + \lim_{N\rightarrow \infty} \int_{-\epsilon}^\epsilon \rec{2\pi}\pare{1+2\sum_{n=1}^N \cos nx} \brac{f\pare{x} - f\pare{0}}\,\rd{x}\\
    & \phantom{= f\pare{0}\ } + \lim_{N\rightarrow \infty} \int_{\epsilon<\abs{x}<\pi} \rec{2\pi}\pare{1+2\sum_{n=1}^N \cos nx} \brac{f\pare{x} - f\pare{0}}\,\rd{x} \\
    & = f\pare{0} + O\pare{\epsilon} + 0 \rightarrow f\pare{0} = \int_{-\pi}^\pi \delta\pare{x}f\pare{x}\,\rd{x}
\end{align*}
を得る.
\newprob{5.14 (1)}%
$\displaystyle
     \abs{\int_{\+bR} \cos nx f\pare{x}\,\rd{x}} = \abs{\left.\frac{\sin nx}{n}f\pare{x}\right\vert_{-\infty}^\infty - \int_{\+bR} \frac{\sin nx}{n}f'\pare{x}\,\rd{x}}
     \le \rec{n} \int_{\+bR} \abs{f'}\,\rd{x} \rightarrow 0.$
\par
\newprobheader{(3)}%
$\displaystyle \int_{\+bR} \rec{\sqrt{2\pi}\epsilon}e^{-x^2/2\epsilon^2}\,\rd{x} = 1$は明らか. 任意のもの$\+cE>0$に対して$\displaystyle \sup_{\abs{x}<\delta} \abs{f\pare{x} - f\pare{0}}<\+cE$となる$\delta$が存在します. $\displaystyle M = \sup_{x\in \+bR} \abs{f\pare{x}-f\pare{0}}$とする.
\begin{align*}
    & \abs{\int_{\+bR} \rec{\sqrt{2\pi}\epsilon}e^{-x^2/2\epsilon^2} f\pare{x}\,\rd{x} - f\pare{0}} = \abs{\int_{\+bR} \rec{\sqrt{2\pi}\epsilon}e^{-x^2/2\epsilon^2} \brac{f\pare{x}-f\pare{0}}\,\rd{x}} \\
    &\le \abs{\int_{\abs{x}<\delta} \rec{\sqrt{2\pi}\epsilon}e^{-x^2/2\epsilon^2} \brac{f\pare{x}-f\pare{0}}\,\rd{x}} + \abs{\int_{\abs{x}>\delta} \rec{\sqrt{2\pi}\epsilon}e^{-x^2/2\epsilon^2} \brac{f\pare{x}-f\pare{0}}\,\rd{x}} \\
    &\le \+cE + M \delta \int_{\abs{t}>\delta/\epsilon} \rec{\sqrt{2\pi}}e^{-t^2/2}\,\rd{t} = \+cE + O\pare{M\epsilon}.
\end{align*}
だから、
\[ \abs{\int_{\+bR} \rec{\sqrt{2\pi}\epsilon}e^{-x^2/2\epsilon^2} f\pare{x}\,\rd{x} - f\pare{0}} = O\pare{M\epsilon} \]
よって
\[ \lim_{\epsilon\rightarrow 0} \rec{\sqrt{2\pi}\epsilon}e^{-x^2/2\epsilon^2} \xlongequal{w} \delta\pare{x} \]
を得る.
\par
\newprobheader{(5)}\\[-2.5\baselineskip]
\begin{flalign*}
    \int_{\+bR} \frac{x}{\sqrt{2\pi}\epsilon^3} e^{-x^2/2\epsilon^2} f\pare{x}\,\rd{x} &= \int_{\+bR} \+dxd{}\brac{-\rec{\sqrt{2\pi}\epsilon} e^{-x^2/2\epsilon^2}} f\pare{x}\,\rd{x} && \\
    &= \left. -\rec{\sqrt{2\pi}\epsilon} e^{-x^2/2\epsilon^2} f\pare{x}\right\vert_{-\infty}^\infty + \int_{\+bR} \rec{\sqrt{2\pi}\epsilon} e^{-x^2/2\epsilon^2}f'\pare{x}\,\rd{x} \\
    & \rightarrow f'\pare{0} \quad \pare{\epsilon\rightarrow 0}\\
    & = -\int_{\+bR} \delta'\pare{x}f\pare{x}\,\rd{x}.
\end{flalign*}

\end{document}
