\documentclass[hidelinks]{ctexart}

\usepackage{van-de-la-illinoise}
\usepackage[paper=b5paper,top=.3in,left=.9in,right=.9in,bottom=.3in]{geometry}
\usepackage{calc}
\pagenumbering{gobble}
\setlength{\parindent}{0pt}
\sisetup{inter-unit-product=\ensuremath{{}\cdot{}}}
\usepackage{van-le-trompe-loeil}

\usepackage{stackengine}
\stackMath
\usepackage{scalerel}
\usepackage[outline]{contour}

\newdimen\indexlen
\def\newprobheader#1{%
\def\probindex{#1}
\setlength\indexlen{\widthof{\textbf{\probindex}}}
\hskip\dimexpr-\indexlen-1em\relax
\textbf{\probindex}\hskip1em\relax
}
\def\newprob#1{%
\newprobheader{#1}%
\def\newprob##1{%
\probsep%
\newprobheader{##1}%
}%
}
\def\probsep{\vskip1em\relax{\color{gray}\dotfill}\vskip1em\relax}

\newlength\thisletterwidth
\newlength\gletterwidth
\newcommand{\leftrightharpoonup}[1]{%
{\ooalign{$\scriptstyle\leftharpoonup$\cr%\kern\dimexpr\thisletterwidth-\gletterwidth\relax
$\scriptstyle\rightharpoonup$\cr}}\relax%
}
\def\tensor#1{\settowidth\thisletterwidth{$\mathbf{#1}$}\settowidth\gletterwidth{$\mathbf{g}$}\stackon[-0.1ex]{\mathbf{#1}}{\boldsymbol{\leftrightharpoonup{#1}}}  }
\def\onedot{$\mathsurround0pt\ldotp$}
\def\cddot{% two dots stacked vertically
  \mathbin{\vcenter{\baselineskip.67ex
    \hbox{\onedot}\hbox{\onedot}}%
}}%

\begin{document}

\newprob{3.6 (3)}%
$\displaystyle u\vert_{r=1} = P_1\pare{\cos\theta}$, $\displaystyle u\vert_{r=2} = \frac{2}{3}P_2\pare{\cos\theta} + \frac{4}{3}P_0\pare{\cos\theta}$,
\[ \Rightarrow u = \left\{\begin{array}{c}
    1 \\ r^{-1}
\end{array}\right\}P_0\pare{\cos\theta} + \left\{\begin{array}{c}
    r \\ r^{-2}
\end{array}\right\}P_1\pare{\cos\theta} + \left\{\begin{array}{c}
    r^2 \\ r^{-3}
\end{array}\right\}P_2\pare{\cos\theta}. \]
代入边界条件得到
\[ \boxed{u = \frac{8}{3}\pare{1-\rec{r}}P_0\pare{\cos\theta} + \rec{7}\pare{-r+\frac{8}{r^2}}P_1\pare{\cos\theta} + \frac{16}{93}\pare{r^2 - \rec{r^3}P_2\pare{\cos\theta}}.} \]
\newprob{3.7}%
由奇偶性知
\[ u = \frac{u_1+u_2}{2} + \frac{u_1-u_2}{2} \sum_{n=1,3,5,\cdots} A_n \pare{\frac{a}{r}}^{n+1}P_n\pare{\cos\theta}. \]
其中
\[ A_n = \frac{\displaystyle 2 \int_0^1 P_n\pare{x}\,\rd{x}}{\displaystyle \int_0^1 P_n^2\pare{x}\,\rd{x}} = \pare{-1}^{\frac{n-1}{2}}\frac{\pare{n-2}!!\pare{2n+1}}{\pare{n+1}!!}.\quad \pare{\text{其中$\pare{-1}!!=1$}} \]
故
\[ \boxed{u = \frac{u_1+u_2}{2}\pare{\frac{a}{r}} + \frac{u_1-u_2}{2}\sum_{n=1,3,5,\cdots} \pare{-1}^{\frac{n-1}{2}}\frac{\pare{n-2}!!\pare{2n+1}}{\pare{n+1}!!}\pare{\frac{a}{r}}^{n+1}P_n\pare{\cos\theta}.} \]
\newprob{3.9}%
注意到$\displaystyle A\sin^2\frac{\theta}{2} = \frac{A}{2}\pare{1-\cos\theta} = \frac{A}{2}\brac{P_0\pare{\cos\theta} - P_1\pare{\cos\theta}}$, 可设
\[ u = \frac{A}{2}\brac{1-\pare{A\pare{\frac{r}{R}} + B\pare{\frac{R}{r}}^2}\cos\theta}. \]
代入边界条件后有
\[ \boxed{u = \frac{A}{2}\brac{1-\pare{\frac{6r}{7R} + \frac{R^2}{7r^2}}\cos\theta}.} \]
\newprob{3.10 (2)}%
由$\displaystyle P_2^1\pare{\cos\theta} = \frac{3}{2}\sin 2\theta$, $\displaystyle P_1^1\pare{\cos\theta} = \sin\theta$, 可得
\[ f\pare{\theta,\varphi} = \sin\theta\cos\varphi + \frac{3}{2}\sin 2\theta\cos\varphi = \boxed{P_1^1\pare{\cos\theta}\cos\varphi + P_2^1\pare{\cos\theta}\cos\varphi.} \]
\newprob{3.11 (1)}%
由$\displaystyle P_2^1\pare{\cos\theta} = \frac{3}{2}\sin 2\theta$, 可设$\displaystyle u = A\pare{\frac{r}{a}}^2 P_2^1\pare{\cos\theta}\cos\varphi$, 立刻有
\[ \boxed{u = \frac{2}{3} \pare{\frac{r}{a}}^2 P_2^1\pare{\cos\theta}\cos\varphi.} \]
\newprob{3.12 (3)}%
\\[-3\baselineskip]
\begin{align*}
    \int J_3 &= -\int x^2\cdot\pare{-x^2 J_3} = -\int x^2\cdot \pare{x^{-2}J_2}' = -J_2 + 2\int \frac{J_2}{x} \\
    &= -J_2 - \frac{2J_1}{x} + C = -\pare{J_0 - 2J_1'} - \frac{2J_1}{x} + C \\
    &= -\pare{J_0 - 2\cdot \frac{x\pare{xJ_1}' - xJ_1}{x^2}} - \frac{2J_1}{x} + C \\
    &= -\brac{J_0 - \pare{2J_0 - \frac{2J_1}{x}}} - \frac{2J_1}{x} + C \\
    &= \boxed{J_0 - \frac{4J_1}{x} + C.}
\end{align*}
\par
\newprobheader{(4)}%
\\[-3\baselineskip]
\begin{align*}
    \int xJ_1 &= \int x\pare{-J_0}' = \boxed{-xJ_0\pare{x} + \int J_0\pare{x}\,\rd{x}.}
\end{align*}
\newprob{3.14}%
设$\displaystyle f\pare{x} = \sum_{n=1}^\infty A_nJ_0\pare{\omega_n x}$, 有
\[ A_n = \frac{\displaystyle \int_0^2 xJ_0\pare{\omega_n x}f\pare{x}\,\rd{x}}{\displaystyle \int_0^2 xJ_0^2\pare{\omega_n x}\,\rd{x}}. \]
其中
\begin{align*}
    & \int_0^2 xJ_0\pare{\omega_n x}f\pare{x}\,\rd{x} = \int_0^1 xJ_0\pare{\omega_n}x\,\rd{x} = \left.\frac{xJ_1\pare{\omega_n x}}{\omega_1}\right\vert_{x=0}^1 = \frac{J_1\pare{\omega_n}}{\omega_n}, \\
    & \int_0^2 xJ_0^2\pare{\omega_n x}\,\rd{x} = \frac{2^2}{2} J_1^2\pare{2\omega_n} = 2 J_1^2\pare{2\omega_n}.
\end{align*}
故
\[ \boxed{f\pare{x} = \sum_{n=1}^\infty\frac{J_1\pare{\omega_n}}{2J_1^2\pare{2\omega_n}\omega_n}J_0\pare{\omega_n x}.} \]

\end{document}
