\documentclass[hidelinks]{ctexart}

\usepackage{van-de-la-illinoise}
\usepackage[paper=b5paper,top=.3in,left=.9in,right=.9in,bottom=.3in]{geometry}
\usepackage{calc}
\pagenumbering{gobble}
\setlength{\parindent}{0pt}
\sisetup{inter-unit-product=\ensuremath{{}\cdot{}}}
\usepackage{van-le-trompe-loeil}
\usetikzlibrary{quotes,angles}
\usetikzlibrary{arrows.meta}
\usepackage{makecell}

\usepackage{stackengine}
\stackMath
\usepackage{scalerel}
\usepackage[outline]{contour}

\newdimen\indexlen
\def\newprobheader#1{%
\def\probindex{#1}
\setlength\indexlen{\widthof{\textbf{\probindex}}}
\hskip\dimexpr-\indexlen-1em\relax
\textbf{\probindex}\hskip1em\relax
}
\def\newprob#1{%
\newprobheader{#1}%
\def\newprob##1{%
\probsep%
\newprobheader{##1}%
}%
}
\def\probsep{\vskip1em\relax{\color{gray}\dotfill}\vskip1em\relax}

\newlength\thisletterwidth
\newlength\gletterwidth
\newcommand{\leftrightharpoonup}[1]{%
{\ooalign{$\scriptstyle\leftharpoonup$\cr%\kern\dimexpr\thisletterwidth-\gletterwidth\relax
$\scriptstyle\rightharpoonup$\cr}}\relax%
}
\def\tensor#1{\settowidth\thisletterwidth{$\mathbf{#1}$}\settowidth\gletterwidth{$\mathbf{g}$}\stackon[-0.1ex]{\mathbf{#1}}{\boldsymbol{\leftrightharpoonup{#1}}}  }
\def\onedot{$\mathsurround0pt\ldotp$}
\def\cddot{% two dots stacked vertically
  \mathbin{\vcenter{\baselineskip.67ex
    \hbox{\onedot}\hbox{\onedot}}%
}}%

    \tikzset{
    partial ellipse/.style args={#1:#2:#3}{
        insert path={+ (#1:#3) arc (#1:#2:#3)}
    }}

\begin{document}

\newprob{5.6 (1)}%
设$x = \alpha s$, $y=\beta t$, 则
\[ \alpha^2 \+D{x^2}D{^2u} + \beta^2 \+D{y^2}D{^2u} = \delta\pare{x,y} \Rightarrow \+D{s^2}D{^2u} + \+D{t^2}D{^2u} = \frac{\delta\pare{s,t}}{\alpha\beta}, \]
从而
\[ u = \rec{4\pi \alpha\beta} \ln \pare{s^2 + t^2} = \boxed{\rec{4\pi \alpha\beta} \ln \brac{\pare{\frac{x}{\alpha}}^2 + \pare{\frac{y}{\beta}}^2}.} \]
\par
\newprobheader{(3)}%
$\displaystyle \laplacian \laplacian u = \delta\pare{x,y,z} \Rightarrow \laplacian u = -\rec{4\pi r}$, 即
\[ \rec{r^2}\+DrD{}\pare{r^2 \+DrDu} = -\rec{4\pi r} \Rightarrow {r^2 \+DrDu} = -\frac{r^2}{8\pi} \Rightarrow \boxed{u = -\frac{r}{8\pi}.} \]
\newprob{5.8 (2)}%
按如图方式构造镜像, 则有Green函数
\[ G = \boxed{\rec{4\pi} \brac{\rec{\abs{\+vr-\+vr_1}} - \frac{R}{r'} \rec{\abs{\+vr - \+vr_2}} - \rec{\abs{\+vr - \+vr_3}} + \frac{R}{r'}\rec{\abs{\+vr - \+vr_4}}}.} \]
\begin{center}
    \incfig{12cm}{SphereImage}
\end{center}
\vspace{1em}
\par
\newprobheader{(3)}%
设单位正电荷位于$\pare{x_0,y_0,z_0}$处, 则构造电像后点电荷分布为
\begin{align*}
    +1&: \pare{x_0,y_0,z_0 + 2nH}, \quad n = 0, \pm 1, \pm 2, \cdots, \\
    -1&: \pare{x_0,y_0,-z_0 - 2nH}, \quad n = 0, \pm 1, \pm 2, \cdots.
\end{align*}
故
\[ \hspace{-6.17em} G = \boxed{\rec{4\pi} \sum_{n=-\infty}^{+\infty} \pare{\rec{\sqrt{\pare{x-x_0}^2 + \pare{y-y_0}^2 + \pare{z - z_0 - 2nH}^2}} - \rec{\sqrt{\pare{x-x_0}^2 + \pare{y-y_0}^2 + \pare{z + z_0 - 2nH}^2}} } .} \]

\end{document}
