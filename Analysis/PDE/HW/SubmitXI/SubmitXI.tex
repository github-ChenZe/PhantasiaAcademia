\documentclass[hidelinks]{ctexart}

\usepackage{van-de-la-illinoise}
\usepackage[paper=b5paper,top=.3in,left=.9in,right=.9in,bottom=.3in]{geometry}
\usepackage{calc}
\pagenumbering{gobble}
\setlength{\parindent}{0pt}
\sisetup{inter-unit-product=\ensuremath{{}\cdot{}}}
\usepackage{van-le-trompe-loeil}

\usepackage{stackengine}
\stackMath
\usepackage{scalerel}
\usepackage[outline]{contour}

\newdimen\indexlen
\def\newprobheader#1{%
\def\probindex{#1}
\setlength\indexlen{\widthof{\textbf{\probindex}}}
\hskip\dimexpr-\indexlen-1em\relax
\textbf{\probindex}\hskip1em\relax
}
\def\newprob#1{%
\newprobheader{#1}%
\def\newprob##1{%
\probsep%
\newprobheader{##1}%
}%
}
\def\probsep{\vskip1em\relax{\color{gray}\dotfill}\vskip1em\relax}

\newlength\thisletterwidth
\newlength\gletterwidth
\newcommand{\leftrightharpoonup}[1]{%
{\ooalign{$\scriptstyle\leftharpoonup$\cr%\kern\dimexpr\thisletterwidth-\gletterwidth\relax
$\scriptstyle\rightharpoonup$\cr}}\relax%
}
\def\tensor#1{\settowidth\thisletterwidth{$\mathbf{#1}$}\settowidth\gletterwidth{$\mathbf{g}$}\stackon[-0.1ex]{\mathbf{#1}}{\boldsymbol{\leftrightharpoonup{#1}}}  }
\def\onedot{$\mathsurround0pt\ldotp$}
\def\cddot{% two dots stacked vertically
  \mathbin{\vcenter{\baselineskip.67ex
    \hbox{\onedot}\hbox{\onedot}}%
}}%

\begin{document}

\newprob{5.7}%
Fourier变换后
\[ -\pare{p_x^2 + p_y^2 + p_z^2} \hat u + k^2\hat u = 1 \Rightarrow \hat u = \rec{k^2 - p^2}, \]
对$p$空间采用球坐标系.
\begin{align*}
    u &= \rec{\pare{2\pi}^3} \int_{\+bR^3} \rec{k^2-p^2} e^{i\+vr\cdot \+vp}\,\rd{^3p} \\
    &= \rec{\pare{2\pi}^3}\times 2\pi \times \int_{0}^\infty \frac{p^2}{k^2 - p^2}\,\rd{p} \int_0^\pi \sin\theta e^{irp\cos\theta}\,\rd{\theta} \\
    &= \rec{\pare{2\pi}^2} \int_0^\infty \frac{p}{k^2 - p^2}\times \frac{2}{r}\sin \pare{rp}\,\rd{p} \\
    &= \rec{8\pi^2 ri}\int_{-\infty}^\infty\pare{\rec{k-p} - \rec{k+p}}e^{irp}\,\rd{p}.
\end{align*}
注意到
\[ F\brac{1} = 2\pi\delta\pare{p} \Rightarrow F\brac{\rec{x}} = -i 2\pi \int \delta\pare{p}\,\rd{p} = -2\pi iH\pare{p}+C, \]
且$\displaystyle F\brac{\rec{x}}$应为奇函数, 故
\[ F\brac{\rec{x}} = -2\pi iH\pare{x} + \pi i. \]
从而
\[ F\brac{\rec{x-k}} = e^{-ikp}\pare{-2\pi iH\pare{p} + i\pi},\quad F\brac{\rec{x+k}} = e^{ikp}\pare{-2\pi iH\pare{p} + i\pi}. \]
故
\begin{align*}
    & \int_{-\infty}^\infty\pare{\rec{k-p} - \rec{k+p}}e^{irp}\,\rd{p} = \pare{-e^{ikr} - e^{-ikr}}\pare{-2\pi iH\pare{-r} + i\pi} = -2\pi i\cos kr. \\
    & u = \rec{8\pi^2 ri}\int_{-\infty}^\infty\pare{\rec{k-p} - \rec{k+p}}e^{irp}\,\rd{p} = \boxed{-\frac{\cos kr}{4\pi r}.}
\end{align*}

\end{document}
