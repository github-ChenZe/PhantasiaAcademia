\documentclass[hidelinks]{ctexart}

\usepackage{van-de-la-illinoise}
\usepackage{cmbright}
\usepackage{nccmath}
\usepackage[paperheight=297mm,paperwidth=240mm,top=.2in,left=.1in,right=.1in,bottom=.2in, landscape]{geometry}
\usepackage{tensor}
\usepackage{colortbl}

\definecolor{graybg}{RGB}{228,235,243}
\definecolor{titlepurple}{RGB}{150,131,104}
\definecolor{shadegray}{RGB}{102,119,136}
\definecolor{itemgray}{RGB}{163,149,128}
\definecolor{mathnormalblack}{RGB}{0,0,0}
\pagecolor{graybg}

\setCJKmainfont{STHeitiSC-Light}
\setmainfont{Arial}

\usepackage{multicol}
\setlength{\columnsep}{.1in}

\newcommand{\raisedrule}[2][0em]{\qquad}
%\leaders\hbox{\rule[#1]{1pt}{#2}}\hfill}
\newcommand{\wdiv}{\,·\,}

\setlength{\parindent}{0pt}

\setCJKfamilyfont{pfsc}{STYuanti-SC-Regular}
\newcommand{\titlefont}{\CJKfamily{ttt}}
\setCJKfamilyfont{ttt}{STFangsong}
\newcommand{\mathtextfont}{\CJKfamily{ttt}}
\def\bili#1#2{#2}

\newdimen\indexlen
\def\newheader#1{%
\def\probindex{#1}
\setlength\indexlen{\widthof{\Large\color{titlepurple} #1\qquad}}
\vspace{1em}
{\Large\color{titlepurple} #1\qquad}
\raisebox{.5em}{\tikz \fill[titlepurple,opacity=.2,path fading=east] (0,0.05em) rectangle (\dimexpr\linewidth-\indexlen\relax,0em);}
}
\def\newlongheader#1{%
\def\probindex{#1}
\setlength\indexlen{\widthof{\Large\color{titlepurple} #1\qquad}}
\vspace{1em}
{\Large\color{titlepurple} #1\qquad}
\raisebox{.5em}{\tikz \fill[titlepurple,opacity=0,path fading=east] (0,0.05em) rectangle (\dimexpr\linewidth-\indexlen\relax,0em);}
}
\def\mathitem#1{\text{\color{itemgray}#1}}
\def\mathcomment#1{\text{\color{lightgray}\quad \texttt{\#}\kern-0pt#1}}
\def\mathheadcomment#1{\text{\color{lightgray}\texttt{\#}\kern-0pt#1}}
\def\midbreak{\smash{\raisebox{1.5em}{\smash{\tikz \path[opacity=.2,left color=white,right color=white,middle color=black] (0,0.05em) rectangle (\linewidth,0em);}}}
\vspace{-4em}}
\newtcolorbox{cheatresume}{enhanced, arc=.5pt, left=.5em, frame hidden, boxrule=0pt, colback=white, fuzzy halo=.05pt with lightgray, shadow={.4pt}{-.4pt}{0pt}{fill=shadegray,opacity=0.3}}
\def\multiline#1{\begin{array}[t]{@{}l}
       #1 
\end{array}}
\newcommand{\emphbox}[1]{\colorbox{lightgray!20}{$\displaystyle #1$}}

\begin{document}

\begin{multicols*}{3}[\centerline{\titlefont 特集\wdiv 様々な展開}]
\raggedcolumns%
\newheader{$P_n$による$x^n$を展開する}\vspace{0em}
\begin{cheatresume}
\begin{flalign*}
    & \multiline{\mathitem{重要な}\\ \mathitem{関係式}} && \begin{array}[t]{ll}
        \+:m22l{\emphbox{\displaystyle P_n\pare{0} = \begin{cases}
            0,& n=2m+1 \\
            \displaystyle \pare{-1}^{n/2}\frac{\pare{n-1}!!}{n!!}, & n=2m
        \end{cases}}}\\
        \\[1.4em]
    \emphbox{nP_n = xP'_n - P'_{n-1}} & \mathheadcomment{漸化式1} \\
    \emphbox{\pare{2n+1}P_n = P'_{n+1} - P'_{n-1}} & \mathheadcomment{漸化式2}
    \end{array} && \\
    & \mathitem{例} && \int_0^1 x^2 P_n\pare{x}\,\rd{x} && \\
    & \+:c3l{$\displaystyle = \int_0^1 x^2 \frac{xP'_n - P'_{n-1}}{n} \,\rd{x} \mathcomment{漸化式1から}$} && \\
    & \+:c3l{$\displaystyle = \rec{n}\brac{-\int_0^1 3x^2 P_n\,\rd{x} + \int_0^1 2xP_{n-1}\,\rd{x}} \mathcomment{次数減少した}$} \\
    & \+:c3l{\mathheadcomment{漸化式を繰り返し使うことにより}} && \\
    & \mathitem{例} && \int_0^1 P_n\,\rd{x} && \\
    & \+:c3l{$\displaystyle = \int_0^1 \frac{P'_{n+1} - P'_{n-1}}{2n+1} \,\rd{x} \mathcomment{漸化式2から}$} && \\
    & \+:c3l{$\displaystyle = \frac{P_{n-1}\pare{0} - P_{n+1}\pare{0}}{2n+1} = \pare{-1}^{\frac{n}{2}-1}\frac{\pare{n-2}!!}{\pare{n+1}!!}$} \mathcomment{nは奇数} &&
\end{flalign*}
    
\end{cheatresume}
\newheader{$J_n$による$x^n$を展開する}\vspace{0em}
\begin{cheatresume}
\begin{flalign*}
    & \multiline{\mathitem{重要な}\\ \mathitem{関係式}} && \begin{array}[t]{ll}
    \emphbox{\pare{x^\nu J_\nu}' = x^\nu J_{\nu - 1}} & \mathheadcomment{漸化式1} \\
    \emphbox{\frac{2\nu J_\nu}{x} = J_{\nu - 1} + J_{\nu + 1}} & \mathheadcomment{漸化式2}
    \end{array} &&
\end{flalign*}
\end{cheatresume}

\end{multicols*}

\end{document}
