\documentclass[hidelinks]{ctexart}

\usepackage{van-de-la-illinoise}
\usepackage{cmbright}
\usepackage{nccmath}
\usepackage[paperheight=297mm,paperwidth=240mm,top=.2in,left=.1in,right=.1in,bottom=.2in, landscape]{geometry}
\usepackage{tensor}
\usepackage{colortbl}

\definecolor{graybg}{RGB}{241,243,245}
\definecolor{titlepurple}{RGB}{116,95,181}
\definecolor{shadegray}{RGB}{102,119,136}
\definecolor{itemgray}{RGB}{170,170,170}
\definecolor{mathnormalblack}{RGB}{0,0,0}
\pagecolor{graybg}

\setCJKmainfont{STHeitiSC-Light}
\setmainfont{Arial}

\usepackage{multicol}
\setlength{\columnsep}{.1in}

\newcommand{\raisedrule}[2][0em]{\qquad}
%\leaders\hbox{\rule[#1]{1pt}{#2}}\hfill}
\newcommand{\wdiv}{\,·\,}

\setlength{\parindent}{0pt}

\setCJKfamilyfont{pfsc}{STYuanti-SC-Regular}
\newcommand{\titlefont}{\CJKfamily{ttt}}
\setCJKfamilyfont{ttt}{STFangsong}
\newcommand{\mathtextfont}{\CJKfamily{ttt}}
\def\bili#1#2{#2}

\newdimen\indexlen
\def\newheader#1{%
\def\probindex{#1}
\setlength\indexlen{\widthof{\Large\color{titlepurple} #1\qquad}}
\vspace{1em}
{\Large\color{titlepurple} #1\qquad}
\raisebox{.5em}{\tikz \fill[titlepurple,opacity=.2,path fading=east] (0,0.05em) rectangle (\dimexpr\linewidth-\indexlen\relax,0em);}
}
\def\newlongheader#1{%
\def\probindex{#1}
\setlength\indexlen{\widthof{\Large\color{titlepurple} #1\qquad}}
\vspace{1em}
{\Large\color{titlepurple} #1\qquad}
\raisebox{.5em}{\tikz \fill[titlepurple,opacity=0,path fading=east] (0,0.05em) rectangle (\dimexpr\linewidth-\indexlen\relax,0em);}
}
\def\mathitem#1{\text{\color{itemgray}#1}}
\def\mathcomment#1{\text{\color{lightgray}\quad \texttt{\#}\kern-0pt#1}}
\def\mathheadcomment#1{\text{\color{lightgray}\texttt{\#}\kern-0pt#1}}
\def\midbreak{\smash{\raisebox{1.5em}{\smash{\tikz \path[opacity=.2,left color=white,right color=white,middle color=black] (0,0.05em) rectangle (\linewidth,0em);}}}
\vspace{-4em}}
\newtcolorbox{cheatresume}{enhanced, arc=.5pt, left=.5em, frame hidden, boxrule=0pt, colback=white, fuzzy halo=.05pt with lightgray, shadow={.4pt}{-.4pt}{0pt}{fill=shadegray,opacity=0.3}}
\newcommand*{\mysans}{\fontfamily{phv}\selectfont}
\definecolor{CJKblack}{RGB}{72,72,72}

\begin{document}

\begin{multicols*}{3}[\centerline{\titlefont 積分変換}]
\raggedcolumns%
\newheader{Fourier変換}
\begin{cheatresume}
    \begin{flalign*}
        & \mathitem{定義} && \hat f\pare{p} = F\brac{f\pare{x}} = \int_{-\infty}^\infty \rd{x}\, e^{-ipx}f\pare{x} && \\
        & \mathitem{逆変換} && f\pare{x} = F^{-1}\brac{\hat f\pare{p}} = \rec{2\pi} \int_{-\infty}^\infty \rd{p}\, e^{ipx} \hat f\pare{p} &&
    \end{flalign*}
    \midbreak
    \begin{flalign*}
        & \mathitem{線型性} && F\brac{af\pare{x} + bg\pare{x}} = a\hat f\pare{p} + b\hat g\pare{p} && \\
        & \mathitem{平行移動} && F\brac{f\pare{x-x_0}} = e^{-i x_0 p}\hat f\pare{p} && \\
        & \mathitem{変調} && F\brac{e^{ix p_0} f\pare{x}} = \hat f\pare{p - p_0} && \\
        & \mathitem{定数倍} && F\brac{f\pare{ax}} = \rec{\abs{a}}\hat f\pare{\frac{p}{a}} && \\
        & \mathitem{畳み込み} && F\brac{f*g\pare{x}} = \hat f\pare{p}\hat g\pare{p} &&
    \end{flalign*}
    \midbreak
    \begin{flalign*}
        & \mathitem{微分} && F\brac{f^{\pare{n}}\pare{x}} = \pare{ip}^n \hat f\pare{p} && \\
        & && F^{-1}\brac{\hat f^{\pare{n}}\pare{p}} = \pare{-ix}^n f\pare{x} && \\
        & \mathitem{積分} && F\brac{\int_{-\infty}^x \rd{y}\, f\pare{x}} = \frac{\hat f\pare{p}}{ip} && \\
        & && F^{-1}\brac{\int_{-\infty}^p \rd{q}\, \hat f\pare{p}} = \frac{f\pare{x}}{-ix} &&
    \end{flalign*}
    \midbreak
    \begin{flalign*}
        && 1 & \mapsto 2\pi\delta\pare{p} & x^n & \mapsto 2\pi i^n \delta^{\pare{n}}\pare{p} && \\
        && \half e^{-\abs{x}} & \mapsto \rec{1+p^2} & e^{-x^2/2} & \mapsto \sqrt{2\pi}e^{-w^2/2} && 
    \end{flalign*}
\end{cheatresume}
\newheader{\mysans{Useful Bullshit}}
\begin{cheatresume}
    \begin{flalign*}
        & \int_{0}^{+\infty} \cos tx\,\rd{x} = \pi \delta\pare{t} && \int_0^{+\infty} \sin tx\,\rd{x} = \rec{t} &&
    \end{flalign*}
\end{cheatresume}

\columnbreak

\newheader{Laplace変換}
\begin{cheatresume}
    \begin{flalign*}
        & \mathitem{定義} && \tilde{f}\pare{p} = L\brac{f\pare{x}} = \int_0^\infty \rd{t}\, e^{-pt}f\pare{t} && \\
        & \mathitem{逆変換} && f\pare{t} = L^{-1}\brac{\tilde{f}\pare{p}} = \rec{2\pi i}\int_{q-i\infty}^{q+i\infty} \rd{p}\, e^{pt}\tilde{f}\pare{p} &&
    \end{flalign*}
    \midbreak
    \begin{flalign*}
        & \mathitem{線型性} && L\brac{af\pare{t} + bg\pare{t}} = a\tilde f\pare{t} + b\tilde g\pare{t} && \\
        & \mathitem{平行移動} && L\brac{f\pare{t-t_0}} = e^{-pt_0}\tilde f\pare{p} && \\
        & \mathitem{変調} && L\brac{e^{p_0 t} f\pare{t}} = \tilde f\pare{p - p_0} && \\
        & \mathitem{定数倍} && L\brac{f\pare{at}} = \rec{a}\tilde f\pare{\frac{p}{a}} && \\
        & \mathitem{畳み込み} && L\brac{f*g\pare{t}} = \tilde f\pare{p}\tilde g\pare{p} &&
    \end{flalign*}
    \midbreak
    \begin{flalign*}
        & \mathitem{微分} && L\brac{f^{\pare{n}}\pare{t}} = p^n\tilde{f}\pare{p} - p^{n-1}f\pare{0} - \cdots - f^{\pare{n-1}}\pare{0} && \\
        & && F^{-1}\brac{\tilde{f}^{\pare{n}}\pare{p}} = \pare{-t}^n f\pare{t} && \\
        & \mathitem{積分} && F\brac{\int_0^t \,\rd{\tau}\,f\pare{\tau}} = \frac{\tilde f\pare{p}}{p} && \\
        & && F^{-1}\brac{\int_{p}^\infty \rd{p}\, \tilde f\pare{p}} = \frac{f\pare{t}}{t} && \\
        & && \int_0^\infty \frac{f\pare{\tau}}{\tau}\,\rd{\tau} = \int_0^\infty \tilde f\pare{p}\,\rd{p} &&
    \end{flalign*}
    \midbreak
    \begin{flalign*}
        \delta\pare{t-\alpha} & \mapsto e^{-\alpha p} & \eta\pare{t-\alpha} & \mapsto \rec{p}e^{-\alpha p} \\
        1 & \mapsto \rec{p} & t^n & \mapsto \frac{n!}{p^{n+1}} \\
        \sinh \omega t & \mapsto \frac{\omega}{p^2 - \omega^2} & \cosh \omega t &\mapsto \frac{p}{p^2 - \omega^2} \\
        e^{\lambda t}\sin\omega t & \mapsto \frac{\omega}{\pare{p-\lambda}^2 + \omega^2} & e^{\lambda t}\cos\omega t & \mapsto \frac{p-\lambda}{\pare{p-\lambda}^2 + \omega^2} \\
        \frac{e^{-\rec{4t}}}{2\sqrt{\pi}t^{3/2}} & \mapsto e^{-\sqrt{p}} & \+:c2c{$\displaystyle \erfc\pare{\rec{2\sqrt{t}}} \mapsto \frac{e^{-\sqrt{p}}}{p}$}
    \end{flalign*}
\end{cheatresume}

\columnbreak

\newheader{正弦\wdiv 余弦変換}
\begin{cheatresume}
    \begin{flalign*}
        & \mathitem{正弦変換} && \hat f_s\pare{p} = F_s\brac{f\pare{x}} = \int_{0}^\infty \rd{x}\, \sin px f\pare{x} && \\
        & \mathitem{逆変換} && f\pare{x} = F^{-1}_s\brac{\hat f_s\pare{p}} = \frac{2}{\pi} \int_{0}^\infty \rd{p}\, \sin px \hat f\pare{p} && \\
        & \mathitem{余弦変換} && \hat f_c\pare{p} = F_c\brac{f\pare{x}} = \int_{0}^\infty \rd{x}\, \cos px f\pare{x} && \\
        & \mathitem{逆変換} && f\pare{x} = F^{-1}_c\brac{\hat f_c\pare{p}} = \frac{2}{\pi} \int_{0}^\infty \rd{p}\, \cos px \hat f\pare{p} &&
    \end{flalign*}
    \midbreak
    \begin{flalign*}
        & \displaystyle \begin{array}[t]{@{}l!{\color{lightgray}\vrule}l}%
         \mathitem{微分}\quad F_s\brac{f'} = -p\hat f_c\pare{p} & F_s\brac{f''} = -p^2 \hat f_s\pare{p} + pf\pare{0}  \\%
         \quad F_c\brac{f'} = p\hat f_s\pare{p} - f\pare{0} & F_c\brac{f''} = -p^2 \hat f_c\pare{p} - f'\pare{0} \end{array} &&
    \end{flalign*}
\end{cheatresume}

\newheader{正弦及ビ余弦、Laplace変換ノ違イ}
\begin{cheatresume}
    \begin{flalign*}
        & \+:c5l{\color{CJKblack} Dirichlet境界条件の場合には正弦変換を使う} \\
        & \mathitem{例} && u_t = a^2 u_{xx},\ u\vert_{x=0} = \varphi\pare{t},\ u\vert_{t=0} = 0 && \\
        & && \Rightarrow \tilde{u}_t = a^2 \pare{-w^2 \tilde{u} + w\varphi\pare{t}},\ \tilde{u}\vert_{t=0} = 0 && \\
        & \+:c5l{\color{CJKblack} Neumannには余弦変換を使う} \\
        & \+:c5l{\color{CJKblack} Robinにはその線型結合を作る} \\
        & \+:c5l{\color{CJKblack} $u\vert_{t=0}$も$u_t\vert_{t=0}$も与えられた$\Rightarrow $ Laplace変換を使う}
    \end{flalign*}
\end{cheatresume}

\end{multicols*}

\end{document}
