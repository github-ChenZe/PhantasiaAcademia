\documentclass[hidelinks]{ctexart}

\usepackage{van-de-la-illinoise}
\usepackage{bbm}

\begin{document}

\section{Green函数} % (fold)
\label{sec:green函数}

\subsection{点源} % (fold)
\label{sub:点源}

\newpoint{}点电荷的静电势可以用来求得任意电荷产生的电势.
\newpoint{}点电荷符合条件
\[ \int_V \delta\pare{\+vx-\+va} = \begin{cases}
    1, & \+va\in V, \\
    0, & \+va\notin V.
\end{cases} \]
\newpoint{}一维情形有逼近
\[ \rho\pare{x} = \rec{2\epsilon}\chi\pare{\abs{x-a}\le \epsilon}. \]

% subsection 点源 (end)

\subsection{和点源解的关系} % (fold)
\label{sub:和点源解的关系}

高维空间仍有
\[ \int_V f\pare{\+vx} \delta\pare{\+vx-\+va} = \int_V f\pare{\+va}\delta\pare{\+vx-\+va}. \]
以及
\[ f\pare{\+vx} \delta\pare{\+vx - \+va} = f\pare{\+va}\delta\pare{\+vx-\+va}. \]
Laplace方程的点源形式为
\[ \laplacian_D G\pare{\+vx-\+va} = \delta\pare{\+vx-\+va}. \]

% subsection 和点源解的关系 (end)

\subsection{推广} % (fold)
\label{sub:推广}

可以设
\[ G\propto \begin{cases}
    \abs{\+vx-\+va}^{2-D}, & D\neq 2, \\
    \ln \abs{\+vx - \+va}, & n=2.
\end{cases} \]
这里的$G$谓$\laplacian_D$的Green函数. 这一概念可以推广到任意的线性微分算子$L_{\+vx}$. 若已知
\[ L_{\+vx}G\pare{\+vx,\+va} = \delta\pare{\+vx-\+va}, \]
而PDE为
\[ L_{\+vx}\phi = f\pare{\+vx}, \]
则一般非齐次项的解为
\[ \phi\pare{\+vx} = \int \rd{\+va}\, G\pare{\+vx,\+va}f\pare{\+va}, \]
因为
\[ f\pare{\+vx} = \int \rd{\+va}\, \delta\pare{\+vx,\+va}f\pare{\+vx}. \]
高维的情形同样有
\[ \delta\pare{\+vx,\+va} = \delta\pare{\+vx - \+va, 0} = \delta\pare{\+vx-\+va}. \]
另外可以看出$\delta\pare{\+vx} = 0$, 如果$\+vx\neq 0$.

% subsection 推广 (end)

\subsection{\texorpdfstring{$\delta$}{delta}函数作为基} % (fold)
\label{sub:texorpdfstring_delta函数作为基}

将任意函数$\+vf$视为无穷维向量, 有
\[ \+vf\pare{\+vx} = \int \rd{\+vx}\, \delta\pare{\+vx-\+va}f\pare{\+va}. \]
故标准基为$\delta$函数.

% subsection texorpdfstring_delta函数作为基 (end)

\subsection{一般SL问题} % (fold)
\label{sub:一般sl问题}

\newpoint{}假设内积是
\[ \braket{f}{g} = \int \rho^{x}\pare{x}\,\rd{x}\, \conj{f}\pare{x}g\pare{x}. \]
\newpoint{}设SL问题提供一组正交基
\[ \ket{e_n} = e_n\pare{x},\quad \braket{e_m}{e_n} = \abs{\ket{e_n}}^2 \delta_{m,n}. \]
\newpoint{}空间中的任意函数可以展开成
\[ \ket{\psi} = \sum_n \psi_e^n \ket{e_n}. \]

% subsection 一般sl问题 (end)

\subsection{线性代数和泛函观点} % (fold)
\label{sub:线性代数和泛函观点}

\newpoint{}引入Dirac符号,
\begin{align*}
    & \abs{\ket{e_n}}^2 = \braket{e_n}{\psi} = \int \rho^x\pare{a}\,\rd{a}\, \conj{e}^n\pare{a}\psi\pare{a}, \\
    & \ket{\psi} = \sum_n \frac{\braket{e_n}{\psi}}{\abs{\ket{e_n}}^2} \ket{e_n} = \sum_n \frac{\ket{e_n}\bra{e_n}}{\braket{e_n}{e_n}}\ket{\psi}.
\end{align*}
\newpoint{}用函数的形式可见$\delta$函数,
\begin{align*}
    \psi\pare{x} &= \sum_n \frac{e_n\pare{x}}{\abs{\ket{e_n}}^2} \int \rho^x \pare{a}\,\rd{a}\, \conj{e}_n\pare{a}\psi\pare{a} \\
    &= \int \rd{a}\,\psi\pare{a}\rho^x\pare{a} \sum_n \frac{\conj{e}_n\pare{a}e_n\pare{x}}{\abs{\ket{e_n}}^2}.
\end{align*}

% subsection 线性代数和泛函观点 (end)

\subsection{从正交基到\texorpdfstring{$\delta$}{delta}函数} % (fold)
\label{sub:从正交基到函数}

\newpoint{}对于所有$\psi$都成立有
\[ \rho^x\pare{x} \sum_n \frac{\conj{e}_n\pare{a}e_n\pare{x}}{\abs{\ket{e_n}}^2} = \delta\pare{x-a}. \]
\begin{ex}
    在$L^2\pare{\brac{0,\pi}}$上,
    \begin{align*}
        \delta\pare{x-a} &= \frac{2}{\pi}\sum_{n=1}^\infty \sin \pare{na}\sin \pare{nx} = \frac{2}{\pi} \sum_{n=0}^\infty\cos\pare{na}\cos\pare{nx} \\
        &= \frac{2}{\pi}\sum_{n=1}^\infty\sin\brac{\pare{n+\half} a}\sin\brac{\pare{n+\half}x}.
    \end{align*}
\end{ex}
\begin{ex}
    在$L^2\pare{S^1\sim \blr{0,2\pi}}$上,
    \begin{align*}
        \delta\pare{x-a} &= \rec{2\pi}\sum_{n=-\infty}^\infty e^{in\pare{a-x}} \\
        &= \rec{\pi}\pare{1+\sum_{n=1}^\infty \cos\brac{n\pare{x-a}}}.
    \end{align*}
\end{ex}
\begin{ex}
    在$L_r^2\pare{\brac{0,1}}$上,
    \[ \delta\pare{r-r'} = r\sum_{n=1}^\infty \frac{J_\nu\pare{\omega_n r'}J_\nu\pare{\omega_n r}}{\abs{\ket{J_{\nu,n}}}^2}. \]
\end{ex}
\begin{ex}
    在$L^2\pare{\brac{-1,1}}$上,
    \[ \delta\pare{x-a} = \sum_{n=m}^\infty \frac{\pare{2n+1}\pare{n-m}!}{2\pare{n+m}!}P_n^m\pare{a}P_n^m\pare{x}. \]
\end{ex}
\begin{ex}
    在$L^2\pare{\brac{0,1}}$上,
    \[ \delta\pare{x-a} = \sum_{n=0}^\infty \pare{4n+3} P_{2n+1}\pare{a}P_{2n+1}\pare{x}. \]
\end{ex}
\begin{ex}
    在$L^2\pare{S^2}$上,
    \begin{align*}
        \delta\pare{\theta-\theta'}\delta\pare{\phi-\phi'} &= \sin\pare{\theta}\sum_{l=0}^\infty \sum_{m=-l}^{+l}Y_{l,m}\pare{\theta,\phi}Y_{l,m}^* \pare{\theta'\phi'} \\
        &= \frac{\sin \theta}{4\pi} \sum_{l=0}^\infty \pare{2l+1}P_l\pare{\cos\gamma}.
    \end{align*}
\end{ex}

% subsection 从正交基到函数 (end)

\subsection{从积分变换开始} % (fold)
\label{sub:从积分变换开始}

\newpoint{}将一个$L^2_{\rho^x}$空间中的函数用$\psi_x\pare{x}$表示, 考虑变换
\[ \psi_p\pare{p} = \int \rd{x}\, \sigma_{p,x}\pare{p,x} \psi_x\pare{x}. \]
利用逆变换恢复之,
\[ \psi_x\pare{x} = \int \rd{p}\, \sigma_{x,p}\pare{x,p} \psi_p\pare{p}. \]
\newpoint{}正反变换可以表示为
\begin{align*}
    & \psi_x\pare{x} = \int \rd{p}\,\sigma_{x,p}\pare{x,p} \int\rd{a}\,\sigma_{p,x}\pare{p,a}\psi_x\pare{a}, \\
    & \Rightarrow \int \rd{p}\,\sigma_{x,p}\pare{x,p} \sigma_{p,x}\pare{p,a} = \delta\pare{x-a}. \\
    & \psi_p\pare{p} = \int \rd{x}\,\sigma_{p,x}\pare{p,x} \int\rd{q}\,\sigma_{x,p}\pare{x,p}\psi_p\pare{q}, \\
    & \Rightarrow \int \rd{x}\,\sigma_{p,x}\pare{p,x} \sigma_{x,p}\pare{x,q} = \delta\pare{p-q}.
\end{align*}
\newpoint{}课本中出现过的变换如下.
\begin{center}
    \begin{tabular}{ccc}
        变换 & $\sigma_{x,p}\pare{x,p}$ & $\sigma_{p,x}\pare{p,x}$ \\
        \hline
        Fourier & $e^{ipx}/2\pi$ & $e^{-ipx}$ \\
        正弦 & $\displaystyle \frac{2}{\pi} \sin p x$ & $\sin px$ \\[.5em]
        正弦 & $\displaystyle \frac{2}{\pi} \cos p x$ & $\cos px$
    \end{tabular}
\end{center}
\newpoint{}考虑一般的Hilbert空间$H$. 设其有基$\curb{\ket{e_n}}$, 将任意向量$\psi$展开,
\[ \psi = \sum_n \psi_e^n e_n. \]
\newpoint{}$n$既可以有离散的部分也可以有连续的部分.
\newpoint{}如果$\curl{e_n}$是正交基,
\[ \braket{e_m}{e_n} = \abs{\ket{e_n}}^2 \delta_{m,n} \Rightarrow \braket{e_m}{\psi} = \abs{\ket{e_n}}^2 \psi_e^n. \]
\newpoint{}代入$\psi$的展开立刻有
\begin{align*}
    \psi &= \sum_{n} \frac{\ket{e_n}\braket{e_n}{\psi}}{\braket{e_n}{e_n}} \\
    &\Rightarrow \resumath{\sum_n \frac{\ket{e_n}\bra{e_n}}{\braket{e_n}{e_n}} = \mathbbm{1}.}
\end{align*}

% subsection 从积分变换开始 (end)

\subsection{正交基的来源} % (fold)
\label{sub:正交基的来源}

\newpoint{}可观测量构成Hermitian算子, 其固有向量构成正交基.
\begin{ex}
    动量算子$\hat p$的固有向量是$e^{ipx}$, 位置算子的固有向量是$\delta\pare{x-a}$.
    \begin{align*}
        & \hat p = -i\partial_x,\quad \hat p e^{ipx} = pe^{ipx}, \\
        & \hat x = x,\quad \hat x\delta\pare{x-a} = a\delta\pare{x-a}.
    \end{align*}
\end{ex}

% subsection 正交基的来源 (end)

\subsection{用任意正交基表示内积} % (fold)
\label{sub:用任意正交基表示内积}

\newpoint{}$H$的内积可以展开,
\begin{align*}
    \braket{\phi}{\psi} &= \bra{\phi}\mathbbm{1}\ket{\psi} = \sum_n \frac{\braket{\phi}{e_n}\braket{e_n}{\psi}}{\braket{e_n}{e_n}} \\
    &= \sum_n \conj{\phi}_e^n \psi_e^n \abs{\ket{e_n}}^2 = \sum_n \braket{\phi_e^n e_n}{\psi_e^n e_n}.
\end{align*}
\newpoint{}用$\ket{\hat x=x}$表示$\+ux$的固有向量, 则$\curb{\ket{\hat x=x}}$构成一组固有正交基,
\[ \braket{\hat x=x_1}{\hat x=x_2} = \rho^x\pare{x}\delta\pare{x_1-x_2}. \]
\newpoint{}$\braket{\hat x = x}{\hat x = x} = \infty$. 离散时$\delta_{m,n}$项的系数为$\abs{\ket{e_n}}^2$, 这里定义$\delta\pare{x_1-x_2}$的系数为$\rho^x\pare{x}$.

% subsection 用任意正交基表示内积 (end)

\subsection{量子力学的图像} % (fold)
\label{sub:量子力学的图像}

\newpoint{}$\ket{\psi}$对这组基的展开为
\[ \int \rd{x}\,\psi_x\pare{x}\ket{\hat x=x}. \]
\newpoint{}用波函数$\psi_x\pare{x}$代表态$\ket{\psi}$则谓Schr\"odinger绘景.
\newpoint{}用$\psi_p\pare{p}$代表$\ket{\psi}$谓Heisenberg绘景.
\newpoint{}故$\delta\pare{x-a}$是$\ket{\+ux=a}$的图像.
\newpoint{}内积可展开为积分形式
\[ \braket{\psi}{\phi} = \int \rd{x}\, \rho^x\pare{x}\conj{\psi}_x\pare{x}\phi_x\pare{x}. \]
\newpoint{}因此$\rho^x\pare{x}$就是以$x$为变量$L^2$空间的密度函数.
\newpoint{}有
\[ \braket{\hat x = x_1}{\hat x = x_2} = \rho^x\pare{x}\delta\pare{x_1-x_2}. \]

% subsection 量子力学的图像 (end)

\subsection{基变换} % (fold)
\label{sub:基变换}

\newpoint{}从$\curb{\ket{f_n}}$变换为$\curb{\ket{e_n}} = \displaystyle \sum_m \sigma_n^m \ket{f_m}$, 有
\[ \ket{\phi} = \sum_n \phi_e^n \ket{e_n} = \sum_{m,n} \phi_e^n \sigma_n^m \ket{f_m} = \sum_{m}\phi_f^m \ket{f_m}. \]
其中
\[ \phi_f^m = \sum_n \phi_e^n \sigma_n^m \]
是展开系数变换.
\newpoint{}反变换用到$\sigma_n^m$的逆,
\[ \ket{f_m} = \sum_n \pare{\sigma^{-1}}_m^n \ket{e_n}. \]
\newpoint{}对于连续统基,
\begin{align*}
    & \ket{\hat x=x} = \int \rd{p}\,\ket{\hat p = p}\sigma_{p,x}\pare{p,x}, \\
    & \ket{\psi} = \int \rd{p}\, \psi_p\pare{p}\ket{\hat p = p} = \int \rd{x}\,\psi_x\pare{x}\ket{\hat x = x} \\
    &= \int \rd{x}\, \psi_x\pare{x} \int \rd{p}\, \sigma_{p,x}\pare{p,x}\ket{\hat p = p} \\
    & \Rightarrow \psi_p\pare{p} = \int \rd{x}\, \sigma_{p,x}\pare{p,x}\psi_x\pare{x}.
\end{align*}

% subsection 基变换 (end)

\subsection{正交基} % (fold)
\label{sub:正交基}

\newpoint{}正交基之间的正反变换不难计算,
\begin{align*}
    & \braket{f_m}{e_n} = \sigma_n^m \abs{\ket{f_m}}^2, \\
    & \braket{e_n}{f_m} = \pare{\sigma^{-1}}_m^n\abs{\ket{e_n}}^2, \\
    & \pare{\sigma^{-1}}_m^n = \frac{\braket{e_n}{f_m}}{\abs{\ket{e_n}}^2} = \sigma^{*m}_n \frac{\abs{\ket{f_m}}^2}{\abs{\ket{e_n}}^2}. \\
    & \braket{\hat p = p}{\hat x = x} = \rho^p\pare{p}\sigma_{p,x}\pare{p,x}, \\
    & \braket{\hat x = x}{\hat p = p} = \rho^x\pare{x}\sigma_{x,p}\pare{x,p}, \\
    & \Rightarrow \rho^x\pare{x} \sigma_{x,p}\pare{x,p} = \rho^p\pare{p}\sigma^*_{p,x}\pare{p,x}. 
\end{align*}
\newpoint{}利用自反性有
\begin{align*}
    & \int \rd{p}\, \rho^p\pare{p} \sigma^*_{p,x}\pare{p,x}\sigma_{p,x}\pare{p,y} = \rho^x\pare{x}\delta\pare{x-y}, \\
    & \int \frac{\rd{x}}{\rho^x\pare{x}}\,\sigma_{p,x}\pare{q,x}\sigma^*_{p,x}\pare{p,x} = \frac{\delta\pare{q-p}}{\rho^p\pare{p}}.
\end{align*}

% subsection 正交基 (end)

\subsection{Dirac记号} % (fold)
\label{sub:dirac记号}

\newpoint{}考虑一图像$\hat x$, 则线性算子$\hat G$的作用在该图像中为
\[ \pare{G\psi}_{\hat x}\pare{x} = \int \rd{y}\,G\pare{x,y}\psi_{\hat x}\pare{y}. \]
\newpoint{}$G\pare{x,y}$谓$G$的核.
\newpoint{}算子$\displaystyle \sum_k \ket{\alpha^k}\bra{\beta^k}$对向量$\psi$的作用为
\begin{align*}
    \ket{\psi} &\mapsto \sum_k \ket{\alpha^k}\braket{\beta^k}{\psi}, \\
    \psi_{\hat x}\pare{x} &\mapsto \sum_k \alpha_{\hat x}^k\pare{x} \int \rho^{x}\pare{y}\,\rd{y}\, \conj{\beta}^k\pare{y}\psi\pare{y},
\end{align*}
对应的核为
\[ \rho^x\pare{y}\sum_k \alpha_{\hat x}^k\pare{x} \conj{\beta}^k\pare{y}. \]
\newpoint{}连续统下恒等运算可分解为
\[ \mathbbm{1} = \int \rd{x}\, \frac{\ket{\hat x = x}\bra{\hat x = x}}{\rho^{x}\pare{x}}. \]
\newpoint{}可以从矩阵成分得到$G$的核,
\begin{align*}
    \pare{G\psi}_{\hat x}\pare{x} &= \frac{\braket{\hat x=x}{G\psi}}{\rho^{x}\pare{x}} \\
    &= \int \rd{y}\, \frac{\bra{\hat x=x}G\ket{\hat x=y}\braket{\hat x=y}{\psi}}{\rho^x\pare{x}\rho^y\pare{y}} \\
    &= \int \rd{y}\, \bra{\hat x=x}G\ket{\hat x=y} \frac{\psi_{\hat x}\pare{x}}{\rho^{x}\pare{x}}. \\
    \Rightarrow G\pare{x,y} &= \frac{\bra{\hat x=x}G\ket{\hat x=y}}{\rho^{x}\pare{x}}.
\end{align*}
特别地, $\mathbbm{1}$的核就是$\delta\pare{x-y}$.

% subsection dirac记号 (end)

\subsection{线性泛函} % (fold)
\label{sub:线性泛函}

\newpoint{}广义函数为良好的普通函数的线性泛函
\[ \func{f}{\phi}{\braket{f}{\phi}}. \]
\newpoint{}普通函数通过内积自动成为广义函数.
\newpoint{}但广义函数不必是普通函数, 可以具有奇异性.

% subsection 线性泛函 (end)

\subsection{泛函观点} % (fold)
\label{sub:泛函观点}

\newpoint{}广义函数是作用在普通函数上的连续线性泛函.
\newpoint{}广义函数总涉及两个函数空间, 基本函数空间极其连续对偶广义函数空间.
\newpoint{}物理上广义函数是荷密度分布的抽象化, 因此也叫分布.

% subsection 泛函观点 (end)

\subsection{常用的基本和广义函数空间} % (fold)
\label{sub:常用的基本和广义函数空间}

\newpoint{}不同基本函数空间的广义函数空间也不同.
\newpoint{}普通函数性质和空间拓扑决定广义函数空间.
\newpoint{}经常要求基本函数光滑并且由紧支撑.

% subsection 常用的基本和广义函数空间 (end)

\subsection{弱收敛} % (fold)
\label{sub:弱收敛}

\newpoint{}作为线性泛函的一个序列$\curb{f_n}$的极限谓弱极限,
\[ \braket{\lim_{n\rightarrow \infty} f_n}{\phi} = \lim_{n\rightarrow \infty}\braket{f_n}{\phi}. \]
\newpoint{}但这不一定是一个广义函数, 因为极限不一定存在, 存在也不一定连续.
\newpoint{}如果存在且连续, 就通过线性泛函作用定义广义函数的收敛与极限. 若$\exists f$满足
\[ \braket{f}{\phi} = \lim_{n\rightarrow \infty}\braket{f_n}{\phi},\quad \forall \phi, \]
则谓$\braket{f_n}$弱收敛于$f$.
\newpoint{}一个普通函数的序列可能在普通空间里没有极限函数, 但在广义函数空间里有奇异广义函数的极限.
\newpoint{}$\delta$函数作为普通函数的弱极限即为一例.
\begin{ex}
    在弱收敛意义下,
    \begin{align*}
        \delta\pare{x} &= \lim_{t\rightarrow 0} \frac{e^{-x^2/2t^2}}{t\sqrt{2\pi}} \\
        &= \lim_{t\rightarrow 0} \frac{t}{\pi\pare{x^2 + t^2}} \\
        &= \lim_{N\rightarrow \infty} \frac{\sin Nx}{\pi x}.
    \end{align*}
\end{ex}

% subsection 弱收敛 (end)

\subsection{广义函数的微积分} % (fold)
\label{sub:广义函数的微积分}

\newpoint{}用分布积分公式的形式成立定义广义函数的导数.
\newpoint{}$\displaystyle \resumath{ \braket{f^{\pare{n}}}{\phi} = \int f^{\pare{n}}\phi = \pare{-1}^n \braket{f}{\phi^{\pare{n}}}. }$
\newpoint{}这要求边界项可以忽略. 所以$\phi$需要满足一定的速降或紧支撑条件.
\newpoint{}反导数$\displaystyle \int_{-\infty}^x \rd{y}\,\delta\pare{y} = H\pare{x}$.

% subsection 广义函数的微积分 (end)

\subsection{卷积} % (fold)
\label{sub:卷积}

\newpoint{}对于一般函数, 有
\[ f*g\pare{x} = \braket{f\pare{x-y}}{g\pare{y}}_y = \braket{f\pare{y}}{g\pare{y}}_y. \]
对于广义函数, 有
\begin{align*}
    & \braket{f*g\pare{x}}{\phi\pare{x}}_x = \iint \rd{x}\,\rd{y}\, f\pare{y}g\pare{x-y}\phi\pare{x} \\
    &= \bra{f\pare{y}}\ket{\braket{g\pare{x-y}}{\phi\pare{x}}_x}_y = \braket{f\pare{x-y}}{\braket{g\pare{y}}{\phi\pare{x}}_x}_y \\
    &= \iint \rd{y}\,\rd{x} \,f\pare{y}g\pare{x}\phi\pare{x+y} = \braket{f\pare{y}}{\braket{g\pare{x}}{\phi\pare{x+y}}_x}_y \\
    &= \resumath{\braket{g\pare{x}}{\braket{f\pare{y}}{\phi\pare{x+y}}_y}_x.}
\end{align*}
\newpoint{}$\displaystyle \resumath{\delta\pare{x-a}*g\pare{x} = g\pare{x-a}.}$
\newpoint{}$\displaystyle \resumath{\begin{cases}
    f^{\pare{n}}*g = \pare{f*g}^{\pare{n}} = f*g^{\pare{n}}, \\
    \delta^{\pare{n}}*f = \pare{\delta * f}^{\pare{n}} = f^{\pare{n}}.
\end{cases}}$

% subsection 卷积 (end)

\subsection{微分算子和卷积} % (fold)
\label{sub:微分算子和卷积}

\newpoint{}$\displaystyle \resumath{L\pare{f*g} = \pare{Lf}*g = f*\pare{Lg}.}$
\newpoint{}可以定义正常函数和广义函数的乘积$\braket{f\psi}{\phi} = \braket{f}{\psi\phi}$, 但无法一般地定义两个广义函数的乘积.

% subsection 微分算子和卷积 (end)

\subsection{积分变换} % (fold)
\label{sub:积分变换}

\newpoint{}定义$\displaystyle \pare{f,\phi} = \int f\phi$, 对于普通广义函数,
\begin{align*}
    K\brac{f}\pare{x} &= \int \rd{y}\, K\pare{x,y} f\pare{y}, \\
    \pare{K\brac{f},\phi} &= \int \rd{x}\int\rd{y}\, K\pare{x,y}f\pare{y}\phi\pare{x} \\
    &= \int \rd{y}\int \rd{x}\, f\pare{y} K\pare{x,y}\phi\pare{x} \\
    &= \pare{f,K^T\brac{\phi}}.
\end{align*}
\newpoint{}$K^T$表示$K$的转置, $K^T\pare{x,y} = K\pare{y,x}$.
\newpoint{}对于奇艺的广义函数, 这是积分变换的定义.
\newpoint{}对于Fourier变换,
\begin{align*}
    & F\pare{p,x} = e^{-ipx} = F\pare{x,p} = F^T\pare{x,p}, \\
    & F = F^T \Rightarrow \begin{cases}
        \pare{F\brac{f},\phi} = \pare{f,F\brac{\phi}}, \\
        \pare{F^{-1}\brac{f},\phi} = \pare{f,F^{-1}\brac{\phi}}.
    \end{cases}
\end{align*}
\newpoint{}$\displaystyle F\brac{\delta\pare{\+vx-\+va}} = \int \rd{x}\, e^{-i\+vp\cdot \+vx} \delta\pare{\+vx-\+va} = e^{-i\+vp\cdot \+va}$.
\newpoint{}由之前的积分变换公式,
\begin{align*}
    & \int \frac{\rd{\+vx}}{\pare{2\pi}^D} e^{i\pare{\+vq-\+vp}\cdot \+vx} = \delta\pare{\+vq - \+vp} \\
    & \Rightarrow F\brac{e^{i\+vq\cdot \+vx}} = \pare{2\pi}^D \delta\pare{\+vx-\+vp}.
\end{align*}
\newpoint{}取$D=1$,
\begin{equation*}
    \resumath{\begin{cases}
        F\brac{\cos px} = \pi\brac{\delta\pare{q+p} + \delta\pare{q-p}}, \\
        F\brac{\sin px} = i\pi\brac{\delta\pare{q+p} - \delta\pare{q-p}}.
    \end{cases}}
\end{equation*}

% subsection 积分变换 (end)

\subsection{解PDE的线性代数观点} % (fold)
\label{sub:解pde的线性代数观点}

\newpoint{}解线性方程时,
\[ Ax = y,\quad A^{-1}A = \mathbbm{1} \Rightarrow x = A^{-1}y. \]
\newpoint{}解微分方程时,
\[ L_{\+vx}u\pare{\+vx} = f\pare{\+vx},\quad L_{\+vx}G\pare{\+vx,\+vy} = \delta\pare{\+vx-\+vy}. \]
\newpoint{}用Green函数,
\begin{align*}
    f\pare{\+vx} &= \int \rd{\+vy}\, \delta\pare{\+vx,\+vy} f\pare{\+vy} \\
    &= \int \rd{\+vy}\, L_{\+vx}G\pare{\+vx,\+vy}f\pare{\+vy} \\
    &= L_{\+vx}\int \rd{\+vy}\,G\pare{\+vx,\+vy}f\pare{\+vy}.
\end{align*}
故解为
\[ u = \int \rd{\+vy}\, G\pare{\+vx,\+vy}f\pare{\+vy}. \]
\newpoint{}可写为
\[ LG = \mathbbm{1},\quad Gf = u,\quad Lu = L\pare{Gf} = \pare{LG}f = \mathbbm{1}f = f. \]
\newpoint{}通过叠加原理可以从点源解获得任意源的解.
\newpoint{}一般情况的点源解必须在给定的边界条件下求.

% subsection 解pde的线性代数观点 (end)

\subsection{物理直观解释} % (fold)
\label{sub:物理直观解释}

\newpoint{}Green函数是点源的解, $LG = \mathbbm{1}$.
\newpoint{}任意源是点源的线性叠加, $\mathbbm{1} f = f$.
\newpoint{}其解是对应的Green函数的线性叠加, $L\pare{Gf} = \pare{LG}f = \mathbbm{1}f = f$, $\Rightarrow u = Gf$.

% subsection 物理直观解释 (end)

\subsection{基本解与Green函数} % (fold)
\label{sub:基本解与green函数}

\newpoint{}Green函数由$L$和边界条件共同决定.
\newpoint{}全空间里的Green函数称为对应算子$L$的基本解.
\newpoint{}静电学里$L$是Laplace算子, 基本解是真空中点电荷的电势, Green函数是在给定边界条件下点电荷产生的电势.
\newpoint{}带边界的Green函数应当视为由电源和感应电荷的场叠加而成.
\newpoint{}当$L$是常系数微分算子, 基本解具有平移对称性,
\begin{align*}
    & L_{\+vx}u\pare{\+vx} = f\pare{\+vx}, \\
    & \Rightarrow f\pare{\+vx - \+va} = L_{\+vx-\+va}u\pare{\+vx-\+va} = L_{\+vx}u\pare{\+vx-\+va}, \\
    & L_{\+vx}G\pare{\+vx} = \delta\pare{\+vx} \Rightarrow L_{\+vx}G\pare{\+vx-\+va} = \delta\pare{\+vx-\+va} \Rightarrow G\pare{\+vx,\+vy} = G\pare{\+vx-\+vy}, \\
    & \Rightarrow u = \int \rd{\+vy}\, G\pare{\+vx-\+vy}f\pare{\+vy} = G*f, \\
    & \Rightarrow L\pare{G*f} = \pare{LG}*f = \delta*f = f.
\end{align*}

\begin{sample}
    \begin{ex}
        最简单的微分算子$d_x$的基本解$H$满足$U'\pare{x} = \delta\pare{x}$, 故
        \[ U\pare{x} = H\pare{x} + \const, \]
        而
        \begin{align*}
            & f\pare{x} = d_x \int_{-\infty}^x \rd{y}\,f\pare{y} \\
            &= d_x \brac{\int \rd{y}\, \pare{H\pare{x-y} + \const}f\pare{y}}.
        \end{align*}
        再考虑
        \[ y' + ay = e^{-ax}\pare{e^{ax}y}', \]
        有$e^{-ax}\pare{H\pare{x} + \const}$是算子$d_x + a$的基本解. 从而$y' + ay = f\pare{x}$的通解为
        \[ y\pare{x} = \int \rd{z}\,e^{-az}\pare{H\pare{z} + \const}f\pare{x-z}. \]
    \end{ex}
\end{sample}

% subsection 基本解与green函数 (end)

\subsection{Green函数的一般特征函数展开} % (fold)
\label{sub:green函数的一般特征函数展开}

\newpoint{}设有一可对角化算子
\[ L = \sum_i \lambda_i \frac{\ket{e_i}\bra{e_i}}{\abs{e_i}^2} = \int \rd{p}\, \lambda\pare{p} \frac{\ket{\sigma_p}\bra{\sigma_p}}{\rho^\sigma\pare{p}}. \]
满足本征值方程
\[ L\ket{e_i} = \lambda_i \ket{e_i},\quad L\ket{E_p} = \lambda\pare{p}\ket{E_p}. \]
\newpoint{}可对角化的算子有逆
\[ L^{-1} = \sum_i \frac{\ket{e_i}\bra{e_i}}{\lambda_i \abs{e_i}^2} = \int \rd{p}\,\frac{\ket{\sigma_p}\bra{\sigma_p}}{\lambda\pare{p}\rho^\sigma\pare{p}}. \]
从而
\begin{align*}
    G\pare{a,b} &= \bra{\delta\pare{x-a}}L^{-1}\ket{\delta\pare{x-b}} \\
    &= \sum_i \frac{e_i\pare{a}e_i\pare{b}}{\lambda_i\abs{e_i}^2} \\
    &= \int \rd{p}\, \frac{\sigma_p\pare{a}\sigma_p\pare{b}}{\lambda\pare{p}\rho^\sigma\pare{p}}.
\end{align*}
这是用$L$的固有向量做的展开的积分变换.
\begin{sample}
    \begin{ex}
        将Laplace算子推广, 由于$\laplacian_D$的固有函数为$e^{i\+vp\cdot \+vx}$, 用Fourier变换解
        \[ \laplacian_D G\pare{\+vx,\+va} = \delta\pare{\+vx - \+va}. \]
        由平移对称性, $G\pare{\+vx,\+va} = G\pare{\+vx-\+va}$. 只需要解
        \[ \laplacian_D G\pare{\+vx} = \delta\pare{\+vx}. \]
        有
        \[ F\brac{\laplacian_D} = -p^2 F\brac{G} \Rightarrow F\brac{\delta} = 1 \Rightarrow F\brac{G} = -p^{-2}. \]
        从而
        \[ G\pare{\+vx} = \int \rd{\+vp}\, \frac{-\abs{\+vp}^2}{\pare{2\pi}^D}e^{i\+vp\cdot \+vx}. \]
        用$D$维球坐标积分, 被积式仅依赖于$\theta$和$p = \abs{\+vp}$. 假设$D$维单位球的表面积为$A_D$, 积分微元为$A_{D-2} \,\rd{p}\, p^{D-1}\,\rd{\theta}\,\sin^{D-2}\theta$.
        \begin{align*}
            G\pare{\+vx} &= A_{D-2}\int_0^\infty \rd{p}\, \frac{p^{D-1}}{-p^2} \int_0^\pi\, e^{ip\abs{x}\cos\theta}\pare{\sin\theta}^{D-2} \\
            &= \frac{A_{D-2}C_D}{-\abs{x}^{D-2}},\quad A_{D-1} = \frac{2\pi^{D/2}}{\Gamma\pare{D/2}}. \\
            C_D &= \int_0^\infty \rd{p}\, p^{D-3}\int_0^\pi \rd{\theta}\, e^{ip\cos\theta}\sin^{D-2}\theta.
        \end{align*}
        可用物理方法得到最终表达式, 记$\+vE = \grad G$, 则
        \[ \grad \rec{\abs{\+vx}^n} = -\frac{n\+ux}{\abs{\+vx}^{n+1}}. \]
        由Green公式,
        \begin{align*}
            & \oint_{\partial V} \rd{\+vS}\cdot \+vE = \int_V \div \+vE = 1, \\
            & \oint_{\partial V} \rd{\+vS} \cdot \grad \rec{\abs{\+vx}^{D-2}} = -\pare{D-2}A_{D-1} \\
            & \Rightarrow G\pare{x} = -\rec{\pare{D-2}A_{D-1}\abs{\+vx}^{D-2}}.
        \end{align*}
    \end{ex}
\end{sample}

% subsection green函数的一般特征函数展开 (end)

\subsection{Ostrogradsky-Gauss-Stokes定理} % (fold)
\label{sub:ostrogradsky_gauss_stokes定理}

\newpoint{}设$\partial V$是$V$的边界, $\rd{S}$是边界表面法向元素, $\rd{V}$是体积元素.
\[ \oint_{\partial V} \rd{\+vS}\cdot \+vE = \int_V \rd{V}\, \div \+vE. \]
\newpoint{}代入$E = u\grad v$得到Green公式,
\begin{align*}
    &\oint_{\partial V}\rd{\+vS}\cdot u\grad v = \iiint \rd{V}\,\pare{\grad u \cdot \grad v + u \laplacian v}, \\
    &\oint_{\partial V}\rd{\+vS}\cdot \pare{u\grad v-v\grad u} = \int_V \rd{V}\pare{u\laplacian v - v\laplacian u}.
\end{align*}
\newpoint{}二维情形下之特例为Green公式
\[ \oint_{\partial \Sigma} A_x\,\rd{x} + A_y\,\rd{y} = \iint_{\Sigma}\pare{\partial_y A_x - \partial_x A_y}\,\rd{x}\,\rd{y}. \]
\begin{pitfall}
    注意边界的定向. 此处和惯例不同.
\end{pitfall}
\newpoint{}任意维下有Kelvin-Stokes定理,
\[ \oint_{\partial \Sigma} A_\mu\,\rd{x^\mu} = \oiint_\Sigma\pare{\partial_\nu A_\mu - \partial_\mu A_\nu}\,\rd{x^\nu}\,\rd{x^\mu}. \]
\newpoint{}一般情形下有Stokes定理
\[ \int_{\partial V} \omega = \int_V \,\rd{\omega}. \]

% subsection ostrogradsky_gauss_stokes定理 (end)

\subsection{变形Poisson方程的基本解} % (fold)
\label{sub:变形poisson方程的基本解}

\newpoint{}$\pare{\laplacian - k^2} G\pare{\+vx} = \delta\pare{\+vx}$.
\newpoint{}和Helmholtz方程相差一个符号.
\newpoint{}在球坐标下,
\[ \laplacian = \rec{r^{D-1}}\partial_r\pare{r^{D-1}\partial_r} + \frac{\laplacian_{S^{D-1}}}{r^2}. \]
\newpoint{}解仅有对$r$的依赖性,
\[ \pare{\laplacian_D - k^2} G\pare{r} = G_{rr} + \frac{D-1}{r} G_r - k^2G = 0, \]
除了$r=0$处.
\newpoint{}做变换$x=kr$, $y\pare{x} = G\pare{r} = G\pare{x/k}$,
\[ x^2 y'' + \pare{D-1}xy' - x^2y = 0, \]
除了$x=0$.
\newpoint{}令$y=x^\alpha z$, 则
\[ x^2y'' + \pare{D-1}xy' - x^2y = x^\alpha\curb{x^2z'' + \brac{\pare{D-1}+2\alpha}xz' + \brac{\alpha\pare{\alpha+D-2}-x^2}z}. \]
选择适当的$\alpha$可以将方程转化为柱Bessel方程.
\newpoint{}选择$\alpha = 1/D/2$,
\begin{align*}
    & z\pare{x} = x^{D/2-1}y\pare{x}, \\
    & x^2z'' + xz' + \brac{x^2 - \pare{D/2-1}^2}z = 0,\quad x\neq 0.
\end{align*}
这是$\pare{D/2-1}$阶变形Bessel方程的标准型.
\newpoint{}通解为
\[ G = \pare{kr}^{1-D/2}\brac{AI_{D/2-1}\pare{kr} + BK_{D/2-1}\pare{kr}}. \]
\newpoint{}例如$D=3$,
\[ I_{1/2} = \sqrt{\frac{2}{\pi x}}\sinh \pare{x},\quad K_{1/2} = \sqrt{\frac{\pi}{2x}}e^{-x}. \]
物理要求$G\pare{r\rightarrow \infty} = 0$, $\Rightarrow A=0$. 可以得到三维空间下的解.
\newpoint{}根据广义函数微分的定义,
\begin{align*}
    & \pare{\pare{\laplacian - k^2}\frac{e^{-kr}}{r},\phi} = \pare{\frac{e^{-kr}}{r},\pare{\laplacian - k^2}\phi} \\
    &= \int_0^\infty \rd{r}\, r^2 \frac{e^{kr}}{r}\oiint_{S^2} \curb{r^{-2}\brac{\partial_r \pare{r^2\partial_r} + \laplacian_{S^2}} - k^2}\phi, \\
    & \oiint_{S^2} = \int_0^\pi \rd{\theta}\,\sin\theta\int_0^{2\pi}\rd{\phi},\\
    & \laplacian_{S^2} = \rec{\sin^2\theta}\brac{\pare{\sin \theta\,\partial_\theta}^2 + \partial_\phi^2}.
\end{align*}
对单值的球函数分部积分后, 边界项为零.
\[ \forall \phi,\quad \oiint_{S^2}\laplacian_{S^2}\phi = 0. \]
从而
\begin{align*}
    &\pare{\pare{\laplacian - k^2}\frac{e^{-kr}}{r},\phi} \\
    &= \int_0^\infty \rd{r}\, r^2 \frac{e^{kr}}{r}\oiint_{S^2} \curb{r^{-2}\brac{\partial_r \pare{r^2\partial_r} + \laplacian_{S^2}} - k^2}\phi \\
    &= \oiint_{S^2} \curb{re^{-kr} \partial_r \phi \vert_{r=0}^{r=\infty} + \int_0^\infty\rd{r}\, e^{-kr}\brac{\pare{kr+1}\partial_r - k^2 r}\phi} \\
    &= \oiint_{S^2} e^{-kr}\pare{kr+1}\phi\vert_{r=0}^{r=\infty} \\
    &= -4\pi \phi\pare{0}.
\end{align*}
从而
\[ G\pare{\+vr} = -\frac{e^{-kr}}{r}. \]

% subsection 变形poisson方程的基本解 (end)

\subsection{二维空间中Laplace方程的基本解} % (fold)
\label{sub:二维空间中laplace方程的基本解}

\newpoint{}考虑$\laplacian G = \delta$, 点源在原点, 则$G=G\pare{r}$且
\[ \pare{r\partial_r}^2 G = 0 \Rightarrow G = A + B \ln r. \]
\newpoint{}定义
\[ \pare{\ln r,\phi} = \int r\,\rd{r}\,\rd{\theta}\,\ln r\, \phi\pare{r,\theta}, \]
则这一积分收敛. $\ln r$可视为一广义函数.
\newpoint{}$A$不影响Laplacian, 故设$G = B\ln r$,
\begin{align*}
    \pare{\laplacian \ln r,\phi} &= \pare{\ln r,\laplacian \phi} \\
    &= \int_0^\infty\,\rd{r}\, r\ln r \int_0^{2\pi} \rd{\theta}\, \rec{r^2}\brac{\pare{r\partial_r}^2 + \partial_\theta^2}\phi \\
    &= \int_0^{2\pi} \rd{\theta}\int_0^\infty\rd{r}\,\ln r \,\partial_r\pare{r\partial_r \phi} \\
    &= \int_0^{2\pi} \rd{\theta}\,\pare{r\ln r\partial_r\phi \vert_{r=0}^{r=\infty} - \int_0^\infty\rd{r}\,\partial_r\phi}
    &= 2\pi\phi\pare{0}. \\
    &\Rightarrow G = \frac{\ln r}{2\pi}.
\end{align*}
\newpoint{}从复分析来看, $z = x+iy$, $\conj{z} = x-iy$,
\[ \Rightarrow \partial_z \partial_{\conj{z}} = \frac{\laplacian_2}{4} \Rightarrow \partial_{\conj{z}} \rec{z} = \partial_z \rec{\conj{z}} = \pi\delta\pare{x}\delta\pare{y}. \]

% subsection 二维空间中laplace方程的基本解 (end)

\subsection{Green函数} % (fold)
\label{sub:green函数}

\newpoint{}考虑二阶线性微分算子$\hat M$, 求解区域$V$, 边界条件
\[ \alpha u + \beta\partial_n u\vert^{\partial V} = 0. \]
$\alpha$, $\beta$不一定是常熟, $\partial_n$表示法向导数.
\newpoint{}对应的Green函数满足
\[ \hat M_{\+vx}G\pare{\+vx,\+va} = \delta\pare{\+vx-\+va}, \]
其中$\hat M_{\+vx}$表示微分算子对$\+vx$求导. 这里$G$不再具备平移对称性, 不只依赖于$\+vx-\+va$.
\newpoint{}考虑实数域上的$L^2$空间,
\[ \braket{f}{g} = \int_V \rd{\+vx}\,f\pare{\+vx}g\pare{\+vx}. \]
\newpoint{}更一般的情况可用复$L^2$和密度函数.
\newpoint{}这里假设了简化公式
\[ \pare{*,*} = \bra{*}\ket{*}. \]
\newpoint{}假设$\hat M$是Hermitian算子, 是客观测量.
\newpoint{}当$V$的维数超过$1$, 单个算子及其固有之不足以完整描述固有向量和基, 有无穷多的简并.
\newpoint{}需要一个对易可观测量的完整集合(CSCO).
\begin{ex}
    三维空间中, 单个位置的坐标算子$\hat x_1$不足, 单个坐标的动量算子$\hat p_2$也不足.
\end{ex}
\begin{ex}
    CSCO的例子如$\hat {\+vx}$, $\curb{\hat p_1,\hat p_2, \hat x^3}$, $\curb{\hat L_3,\abs{\+vL}^2,\hat p_3}$. 它们都对易.
\end{ex}
\begin{ex}
    不是CSCO的例子如$\curb{\hat p_1,\hat p_2,\hat x^2}$, $\curb{\hat L_3,\hat L_1,\hat p_3}$, 这些相互间都不对易.
\end{ex}
\newpoint{}假设$\hat M$是一组CSCO$\+uM$的成员,
\[ \hat M\ket{\+uM = m} = m\ket{\+uM = m} \]
是本征值问题. $\+vm$包括所有$\+uM$的固有之, $m$是其中$\hat M$的固有之.
\begin{align*}
    & \hat M = \int \rd{\+vm}\, m \frac{\ket{\+uM = \+vm}\bra{\+uM = \+um}}{\rho^{\+uM}\pare{\+um}}, \\
    & \mathbbm{1} = \int \rd{\+vm}\, \frac{\ket{\+uM = \+vm}\bra{\+uM = \+vm}}{\rho^{\+uM}\pare{\+vm}}.
\end{align*}
\newpoint{}如果没有本征值零, 则$\hat M$可逆, $\hat M^{-1} = \hat G$,
\[ \hat G = \int \rd{\+vm}\, \frac{\ket{\+uM = \+vm}\bra{\+uM = \+vm}}{m\rho^{\+uM}\pare{\+vm}}. \]
\newpoint{}如果有$m=0$, 则$M$不可逆, 需要特别处理.
\newpoint{}谱是离散时, 积分改为求和, $\rho^{\+uM}\pare{\+vm}$改为$\abs{\ket{\+uM = \+vm}}^2$.

% subsection green函数 (end)

\subsection{边界条件和算子的对称性} % (fold)
\label{sub:边界条件和算子的对称性}

\newpoint{}可以将这里的讨论视为SL定理的推广, 令$\hat M = -\laplacian + q$, 则
\begin{align*}
    & \braket{\hat M u}{v} - \braket{u}{\hat M v} \\
    &= \int_V \pare{u\laplacian v - \laplacian\pare{uv}} \\
    &= \oint_{\partial V}\rd{\+vS}\cdot \pare{u\grad v - v\grad u} \\
    &= \oint_{\partial V} \pare{u\partial_n v - v\partial_n u} \\
    &= \oint_{\partial V} W.
\end{align*}
\newpoint{}引入Wronskian
\[ W = \det \begin{pmatrix}
    u & v \\
    \partial_n u & \partial_n v
\end{pmatrix}. \]

% subsection 边界条件和算子的对称性 (end)

\subsection{互反性} % (fold)
\label{sub:互反性}

\newpoint{}若$u,v$满足相同的边界条件, 即有$W = 0$. 因此$\hat M$是对称算子, 在复$L^2$空间中为Hermitian算子. 如果$u_i$是非齐次项$\rho_i$的解, 则
\begin{align*}
    & \hat M u_i = \rho_i, \\
    & \int_V u\sigma - v\rho = \braket{u}{\sigma} - \braket{\rho}{v} = \braket{u}{\hat M v} - \braket{\hat Mu}{v} = 0.
\end{align*}
\newpoint{}这证明了互反定理
\[ \braket{u}{\sigma} = \braket{\sigma}{u} = \braket{\rho}{v}. \]

% subsection 互反性 (end)

\subsection{对称性} % (fold)
\label{sub:对称性}

\newpoint{}考虑两个弱收敛于$\delta$函数的普通函数序列
\[ \lim_{n\rightarrow \infty} \rho_n \xlongequal{w} = \delta\pare{\+vx-\+va_1},\quad \lim_{n\rightarrow \infty}\sigma_n \xlongequal{w} = \delta\pare{\+vx - \+va_2}. \]
相应的非齐次项的解为
\[ \hat u_n = \rho_n,\quad \hat M v_n = \sigma_n. \]
\newpoint{}极限和$\hat M$的作用次序可以交换,
\[ \lim_{n\rightarrow \infty} u_n \xlongequal{w} = G\pare{\+vx;\+va_1},\quad \lim_{n\rightarrow \infty}v_n \xlongequal{w} = G\pare{\+vx;\+va_2}. \]
\newpoint{}这证明了Green函数的对称性,
\begin{align*}
    \lim_{n\rightarrow \infty}\braket{\rho_n}{v_n} \xlongequal{w} \braket{\rho}{v} = G\pare{\+va_1,\+va_2} \\
    =\lim_{n\rightarrow \infty}\braket{\sigma_n}{u_n} \xlongequal{w} \braket{\sigma}{u} = G\pare{\+va_2,\+va_1}.
\end{align*}

% subsection 对称性 (end)

\subsection{用Green函数解边值问题} % (fold)
\label{sub:用green函数解边值问题}

\newpoint{}设$v$为待求解, $v$为Green函数, 满足的边界条件的系数相同,
\begin{align*}
    & \phi_1 = \alpha u + \beta \partial_n u\vert^{\partial V} = \phi, \\
    & \phi_2 = \alpha v + \beta \partial_n v\vert^{\partial V} = 0, \\
    & W = u_1 \partial_n u_2 - u_2 \partial_n u_1 \\
    & = \pare{\phi_2 u_1 - \phi_1 u_2}/\beta \\
    & = \pare{\phi_1 \partial_n u_2 - \phi_2 \partial_n u_1}/\alpha. \\
    & \rho = \hat K u,\quad \sigma = \hat K v = \delta\pare{\+vx-\+va}, v\pare{\+vx} = G\pare{\+vx;\+va}, \\
    & W = \phi\pare{b} \partial_{nb} G\pare{b,a}/\alpha = \phi\pare{b}G\pare{b,a}/\beta.
\end{align*}
\newpoint{}由
\[ \braket{\rho}{v} - \braket{\sigma}{u} = \oint_{\partial V}W, \]
有
\begin{align*}
    u\pare{a} &= \braket{\sigma}{u} = \braket{\rho}{v} - \oint_{\partial V}\rd{^2x}\, W \\
    &= \int_V \rd{\+vx}\, G\pare{\+va;\+vx}\rho\pare{\+vx} - \oint_{\partial V} \frac{\rd{\+vx}}{\alpha}\phi\pare{\+vx}\partial_{n\+vx}G\pare{\+va,\+vx} \\
    &= \int_V \rd{\+vx}\,G\pare{\+va,\+vx}\rho\pare{\+vx} + \oint_{\partial V} \frac{\rd{\+vx}}{\beta}\phi\pare{\+vx}G\pare{\+va,\+vx}.
\end{align*}
\newpoint{}有些边界条件不允许标准Green函数存在.
\newpoint{}例如$q=0$且边界条件为Neumann时,
\[ \alpha = 0,\quad \beta = 1,\quad \laplacian u = \rho,\quad \partial_n u\vert^{\partial_v} = \phi, \]
必须满足Gau\ss 定理
\[ \oint_{\partial V}\phi = \int_V \rho \]
才有解. 这也说明标准的Green函数不存在,
\[ \oint_{\partial V}\rd{\+vx}\,\partial_{nx}G\pare{\+vx,\+va} = \int_V \rd{^3\+vx}\laplacian_{\+vx}G\pare{\+vx;\+va} = -1. \]
而$\partial_{nx}G\pare{\+vx,\+va} = 0$与
\[ -\laplacian_{\+vx}G\pare{\+vx,\+va} = \delta\pare{\+vx-\+va} \]
矛盾.
\newpoint{}此时可以修改Green函数的定义, 设
\[ \laplacian_{\+vx} \tilde{G}\pare{\+vx;\+va} = \omega\pare{\+vx} - \delta\pare{\+vx-\+va}, \]
且$\partial_{xn}\tilde{G}\pare{\+vx,\+va}\vert^{\+vx\in \partial V} = 0$. 则$w$只需满足
\[ \int_V \rd{\+vx}\,\omega\pare{\+vx} = 1. \]
因为$W = -\phi \tilde{G}$,
\begin{align*}
    & u\pare{\+va} = \int_V \rd{^3\+vx}\, u\pare{\+vx}\brac{\omega\pare{\+vx} - \laplacian_b \tilde{G}\pare{\+vx,\+va}} \\
    &= \int_V \rd{^3 \+vx}\, \rho\pare{\+vx}\tilde{G}\pare{\+vx;\+va} - \oint_{\partial V}\rd{^2\+vx}\,\phi\pare{\+vx}\tilde{G}\pare{\+vx;\+va}+C.
\end{align*}
其中
\[ C = \int_V \rd{^3 \+vx}\,\omega\pare{\+vx}u\pare{\+vx}. \]
\newpoint{}Neumann边界条件本身就导致一个不确定的积分常熟, 从而$C$不重要.
\newpoint{}$\omega$可根据解题方便选取, 例如取常数.
\newpoint{}设$\rho = \omega\pare{\+vx} - \delta\pare{\+vx-\+va},\phi = 0$, 则
\begin{align*}
    & u\pare{\+vx} = G\pare{\+vx,\+va'}, \\
    & G\pare{\+va,\+va'} - G\pare{\+va',\+va} = \int_V \rd{\+vx}\,\omega\pare{\+vx}\brac{G\pare{\+vx,\+va'} - G\pare{\+vx,a}}.
\end{align*}
\newpoint{}因此修改后的$G$一般不对称.

% subsection 用green函数解边值问题 (end)

\subsection{镜像法求解Green函数} % (fold)
\label{sub:镜像法求解green函数}

\newpoint{}Dirichlet条件: 选取大小相等, 符号相反, 处于镜像位置的电荷.
\newpoint{}Neumann条件: 选取正电荷.
\newpoint{}在复平面, 考虑上半复平面, 点源在$w$, 镜像在$\conj{w}$,
\[ G\pare{z,w} = \rec{2\pi} \ln \abs{\frac{z-w}{z-\conj{w}}}. \]
\newpoint{}球内: 猜测镜像电荷在通过球心和源电荷的轴上.
\newpoint{}球外类似.
\newpoint{}园内: 猜测镜像电荷在通过圆心和源电荷的轴上.
\newpoint{}设待定位置和电荷, 计算总电荷, 要求满足边界条件即可确定.
\newpoint{}圆外空间类似.
\newpoint{}可以推广到高维球.

% subsection 镜像法求解green函数 (end)

\subsection{二维情形} % (fold)
\label{sub:二维情形}

\newpoint{}链式法则表明
\[ \+DzD{} = \half \pare{\partial_x - i\partial_y},\quad \+D{\conj{z}}D{} = \half \pare{\partial_x + i\partial_y}, \]
从而
\[ \laplacian = \partial_x^2 + \partial_y^2 = 4\partial_z\partial_{\conj{z}}. \]
\newpoint{}考虑亚纯函数
\[ \tilde{x} + i\tilde{y} = f\pare{z} \Rightarrow \+D{\conj{z}}Df = 0 \Rightarrow \resumath{\partial_x \tilde{x} = \partial_y \tilde{y},\quad \partial_x \tilde{y} = -\partial_y \tilde{x}.} \]
\newpoint{}因此
\[ \partial_x^2 \tilde{x} = -\partial_y^2 \tilde{x},\quad \partial_x^2 \tilde{y} = -\partial_y^2 \tilde{y}. \]
故亚纯函数的实部和虚部都是调和函数.
\newpoint{}由Cauchy-Riemann方程可得出
\[ f' = \partial_x \pare{\tilde{x} + i\tilde{y}} = \partial_y \pare{\tilde{y} - i\tilde{x}} \Rightarrow \resumath{\abs{f'}^2 = {\+D{\pare{x,y}}D{\pare{\tilde{x},\tilde{y}}}}.} \]
\newpoint{坐标变换} 设$f$是双全纯或亚纯函数, 有反函数$g$, 则
\[ \tilde{z} = f\pare{z},\quad g\pare{f\pare{z}} = z,\quad z = g\pare{\tilde{z}},\quad f\pare{g\pare{\tilde{z}}} = \tilde{z}. \]
从而
\[ \+DzD{} = \+DzD{\tilde{z}} \+D{\tilde{z}}D{} = f'\pare{z}\+D{\tilde{z}}D{} = \rec{g'\pare{z}}\+D{\tilde{z}}D{},\quad \+D{\tilde{z}}D{} = \rec{f'\pare{z}}\+DzD{}. \]
反全纯变量$\conj{z}$和$\conj{\tilde{z}}$之间的关系类似.
\[ \laplacian = 4\partial_z \partial_{\conj{z}} = \frac{\partial_w \partial_{\conj{w}}}{\abs{g'\pare{z}}^2}. \]
\newpoint{}设
\[ \pare{\tilde{x},\tilde{y}} = \hat f\pare{x,y},\quad \pare{x,y} = \hat g\pare{\tilde{x},\tilde{y}}, \]
则
\[ \laplacian u = \pare{\partial_x^2 + \partial_y^2} u = \rho \]
表明
\[ \tilde{\laplacian} \pare{u\comp \hat g} = \rho\abs{g'\pare{\tilde{z}}}^2. \]
因此$\tilde{u} = u\comp \hat{g}$也是Poisson方程的解. 非齐次项变为
\[ \tilde{\rho} = \rho\comp \hat g\abs{g'\pare{\tilde{z}}}^2. \]
\newpoint{}考虑$\rho = \tilde{\rho} = 0$之特例,  有$u$和$\tilde{u}$都是Laplace方程的解. 对于$\rho = \delta\pare{x-x_0}\delta\pare{y-y_0}$,
\begin{align*}
    \tilde{\rho} &= \frac{\delta\pare{x-x_0}\delta\pare{y-y_0}}{\abs{f'\pare{z}}^2} \\
    &= \delta\pare{x-x_0}\delta\pare{y-y_0}/\+D{\pare{x,y}}D{\pare{\tilde{x},\tilde{y}}} \\
    &= \delta\pare{\tilde{x} - \tilde{x}_0}\delta\pare{\tilde{y} - \tilde{y}_0}.
\end{align*}
因此$u$和$\tilde{u}$都是Poisson方程的基本解.
\newpoint{}因此Laplace方程和Poisson方程在$\hat f$的作用下有协变性.
\[ u \mapsto \tilde{u} = u\comp \hat g,\quad \rho \mapsto \tilde{\rho} = \rho \comp \hat g \abs{g'\pare{\tilde{z}}}^2, \]
因为$\rho$是张量.
\newpoint{}用不同的$f$可以从一个解得到无穷多的其它解.
\newpoint{}将$\hat f$和$\hat g$视为几何映射, 在每个点$\hat f$通过Jacobi矩阵对切向量作用,
\[ \+D{\pare{x,y}}D{\pare{\tilde{x},\tilde{y}}} = \abs{f'\pare{z}}^2 \begin{pmatrix}
    \cos \phi & \sin \phi \\
    -\sin\phi & \cos\phi
\end{pmatrix}. \]
而
\[ \tan \phi = \frac{\partial_y \tilde{x}}{\partial_x \tilde{x}}. \]
\newpoint{}Cauchy-Riemann方程决定了这一矩阵的形状. 这是一个缩放与旋转的复合.
\newpoint{}故亚纯函数构成保角变换.
\newpoint{}$f$解析且$f' \neq 0$时$\hat f$是保角映射.
\newpoint{}质量为零的场论有保角对称. 而Helmholtz方程和质量不为零的场论没有保角对称.
\begin{sample}
    \begin{ex}
        二维空间的Poisson方程有基本解
        \[ u = \frac{\ln \abs{z}^2}{4\pi}. \]
        $u$在单位圆边界满足Dirichlet条件. 从这个解出发可得其它区域满足Dirichlet条件的Green函数. 设$f$将单位圆映射到求解区域$V$, 反函数$g\pare{z} = c\pare{z-z_0} + O\pare{z-z_0}^2$, 则
        \[ G\pare{z,z_0} = \frac{\ln \abs{g\pare{z}}}{2\pi} \]
        是$V$内的Green函数.
        \begin{align*}
            G\pare{z,z_0} &= \frac{\ln \abs{g\pare{z}}}{2\pi} = \rec{4\pi} \abs{\ln\abs{z-z_0}^2 + 2\Re \ln \frac{g\pare{z}}{z-z_0}} \\
            &= \rec{4\pi}\curb{\ln \abs{z-z_0}^2 + 2\Re \ln \brac{c + O\pare{z-z_0}}}.
        \end{align*}
        而$\displaystyle \ln \frac{g\pare{z}}{z-z_0}$在$V$内解析, 故
        \[ \laplacian G\pare{z,z_0} = \frac{\ln \abs{z}}{2\pi} = \delta\pare{x-x_0}\delta\pare{y-y_0}. \]
        且$G\pare{z,z_0}\vert_{\partial V} = 0$满足边界条件.
    \end{ex}
\end{sample}
\newpoint{}注意到
\[ g\pare{z} = \frac{z-w}{1-z\conj{w}} \]
满足$g\pare{w} = 0$, 且将单位圆映射为单位圆. 因此
\[ G\pare{z,w} = \rec{2\pi} \ln \abs{\frac{z-w}{1-z\conj{w}}} \]
是荷在$w$处的Green函数.
\newpoint{}注意到$\displaystyle g\pare{z} = \frac{z-w}{z - \conj{w}}$将上班平面映射为单位圆, 故
\[ G\pare{z,w} = \rec{2\pi}\ln \abs{\frac{z-w}{z-\conj{w}}} \]
是上半平面荷在$w$处的Green函数.
\begin{ex}
    设$V = \curb{z\vert \Im z\in \pare{0,\pi}}$, 而$\func{e^z}{V}{\curb{z\vert \Im z>0}}$, 和上半平面的Green函数组合, 有
    \[ G\pare{z,w} = \rec{2\pi} \ln \abs{\frac{e^z-e^z}{e^z - e^{\conj{w}}}} \]
    是$V$内的Green函数.
\end{ex}

% subsection 二维情形 (end)

\subsection{基本解和叠加原理求解Green函数} % (fold)
\label{sub:基本解和叠加原理求解green函数}

\newpoint{}设$\delta$输入对应的Green函数为$G_0$.
\newpoint{}设$G_1$是齐次方程的解, 且在边界上
\[ \pare{\alpha + \beta \partial_n} G_1 = -\pare{\alpha + \beta \partial_n} G_0, \]
则$G = \pare{G_0 + G_1}$即为相应边界条件下对$\delta$输入的响应.

% subsection 基本解和叠加原理求解green函数 (end)

\subsection{Fourier方法求Green函数} % (fold)
\label{sub:fourier方法求green函数}

\newpoint{}对每个坐标做正交基展开(有界区间)或积分变换(无界或半无界区间), 注意边界条件.
\newpoint{}将方程展开(求导变为乘以动量).
\newpoint{}将Delta函数展开(三角/指数函数).
\newpoint{}变换后的代数方程通常容易解.
\newpoint{}反变换可得Green函数.
\newpoint{}本质上是本征向量展开,
\[ \hat G = \int \rd{\+vm}\, \frac{\ket{\+uM = \+vm}\bra{\+uM = \+vm}}{\lambda_{\+vm}\rho^{\hat M}\pare{\+vm}} \Leftrightarrow \hat G = \sum \rd{\+vm}\, \frac{\ket{e_n}\bra{e_n}}{\lambda_n \abs{\ket{e_n}}^2}. \]

\begin{sample}
    \begin{ex}
        求解区域是圆心在原点的单位球内.
        \[ \begin{cases}
            \laplacian u = -\delta\pare{\+vx - \+vx_0}, \\
            ku_r + hu\vert_{r=1} = 0,
        \end{cases} \]
        考虑CSCO$\curb{-\laplacian, \abs{\+vL}^2, L_3}$, 对应的固有值为
        \[ k_{n;l}^2,\quad l\pare{l+1},\quad m. \]
        其中$x = k_{n;l}$是$\displaystyle hj_l\pare{x} + \frac{k}{k_{n;l}}j'_l\pare{x}$的第$n$个正根. 用球坐标则
        \begin{align*}
            \ket{n,l,m} &= \psi_{n,l,m}\pare{r,\theta,\phi} \\
            &= j_l\pare{k_{n;l}r}P_l\pare{\cos\theta}e^{im\phi}, \\
            & l = 0,1,\cdots; m = -l,\cdots,l.
        \end{align*}
        相应的Green函数为
        \[ u = \sum_{l=0}^\infty \sum_{n=1}^\infty \sum_{m=-l}^l \frac{\ket{k,l,m}\bra{n,l,m}}{k_{n;l}^2 \abs{\ket{n,l,m}}^2}. \]
        使用球坐标, 让$\+vx_0$在$+z$轴上, $m\neq 0$的项依赖于$\phi$, 故可以丢弃,
        \[ u = \sum_{l=0}^\infty \sum_{n=1}^\infty \frac{j_l\pare{k_{n;l}r} P_l\pare{\cos\theta} j_l\pare{k_{n;l}\abs{\+vx_0}}}{k_{n;l}^2 \abs{\ket{n,l,m}}^2}. \]
    \end{ex}
\end{sample}
\begin{sample}
    \begin{ex}
        也可以使用基本解和叠加原理.
        \begin{align*}
            & u = U + g,\quad U = \rec{4\pi\abs{\+vx - \+vx_0}},\\ &\laplacian g=0,\quad kg_r + hg\vert_{r=1} = -kU_r + hU\vert_{r=1}.
        \end{align*}
        用正交基展开, $g$满足Laplace方程故使用球谐函数. 再用加法定理将$kU_r + hU$用球谐函数展开, 确定$g$. 可以选择$\+vx_0$在$z$轴上, 故只需使用Legendre多项式.
    \end{ex}
\end{sample}
\begin{sample}
    \begin{ex}
        在$\pare{0,\pi}\times \pare{0,\pi}$上对Dirichlet条件求解$\laplacian G = -\delta\pare{x-x_0}\delta\pare{y-y_0}$, Laplacian的固有向量为
        \[ \ket{m,n} = \sin mx \sin ny, \]
        固有值
        \[ \lambda_{m,n} = m^2 + n^2, \]
        模平方
        \[ \abs{\ket{m,n}}^2 = \frac{\pi^2}{4}. \]
        相应的
        \[ G = \frac{4}{\pi^2}\sum_{m=1}^\infty\sum_{n=1}^\infty \rec{m^2 + n^2} \sin mx_1 \sin ny_1 \sin mx_2 \sin ny_2. \]
    \end{ex}
\end{sample}
\begin{sample}
    \begin{ex}
        在$\pare{0,\pi}\times \pare{0,\infty}$上对Dirichlet条件(且$y\rightarrow \infty$时$G$有界)求解$\laplacian G = -\delta\pare{x-x_0}\delta\pare{y-y_0}$, Laplacian的固有向量为
        \[ \ket{m,p} = \sin mx \sin py, \]
        固有值
        \[ \lambda_{m,n} = m^2 + p^2, \]
        此处连续谱和离散谱同时存在,
        \[ \braket{m,p}{n,q} = \frac{\pi^2}{4}\delta_{m,n}\delta\pare{p,q}. \]
        从而
        \[ G = \frac{4}{\pi^2} \sum_{m=1}^\infty \int_0^\infty\rd{p}\, \rec{m^2 + p^2} \sin mx_1 \sin py_1 \sin mx_2 \sin py_2. \] 
    \end{ex}
\end{sample}

% subsection fourier方法求green函数 (end)

\subsection{Duhamel原理} % (fold)
\label{sub:duhamel原理}

\newpoint{}设$L_t$是微分算子, 最高阶时间导数为$a\pare{t}\partial_t^m$, Duhamel原理用到的$w\pare{t;\tau}$满足
\begin{align*}
    & t\ge \tau,\quad L_t w = 0, \\
    & \partial_t^{m-1}w\pare{t;\tau}\vert_{t=\tau} = \frac{f\pare{\tau}}{a\pare{\tau}}, \\
    & \partial_t^n w\pare{t=\tau;\tau}\vert_{t=\tau} = 0,\quad 0\le n < m-1.
\end{align*}
现在将$w$的定义域扩大到$t,\tau \ge 0$并规定$w\pare{t<\tau;\tau} \equiv 0$, 当$t>\tau$和$\tau >t$有$L_t w = 0$, $t$经过$\tau$时$\partial_t^{n<m-1} w\pare{t;\tau}$保持连续, 但是$\partial_t^{m-1} w\pare{t;\tau}$有阶梯状不连续性, 阶梯高度$\displaystyle \frac{f\pare{\tau}}{a\pare{\tau}}$. 从而
\[ L_t w = a\pare{t} \partial_t^m \brac{\frac{f\pare{\tau}}{a\pare{\tau}} H\pare{t-\tau}} = f\pare{\tau}\delta\pare{t-\tau}. \]

% subsection duhamel原理 (end)

\subsection{冲量} % (fold)
\label{sub:冲量}

\newpoint{}因此$\displaystyle \frac{w\pare{t;\tau}}{f\pare{\tau}}$的意义是$\Delta$函数所描述的瞬间冲量的解.
\newpoint{}而$L_tu = f\pare{t}$的齐次初始条件的定解就是
\[ u = \int_0^\infty \rd{\tau}\, \frac{w\pare{t;\tau}}{f\pare{\tau}}f\pare{\tau}. \]
$f,w,L_t$都可以包含其它自变量, $L_t$可以包含其它导数.

% subsection 冲量 (end)

\subsection{一阶发展方程} % (fold)
\label{sub:一阶发展方程}

\newpoint{}考虑一阶发展方程的基本解.
\[ \partial_t u = L_{x}u + f,\quad t>0,\quad u\pare{t=0} = \phi. \]
$L_x$不对时间求导. 之前的冲量原理将方程拆开后求解.
\begin{align*}
    & u = u_1 + u_2, \\
    & \partial_t u_1 = L_x u_1,\quad u_1\pare{t=0} = \phi, \\
    & \partial_t u_2 = L_x u_2 + f,\quad u_2\pare{t=0} = 0.
\end{align*}
相应的基本解为
\begin{align*}
    & u_1\pare{t,\+vx} = \int_0^t \rd{\tau}\,w\pare{t,\+vx;\tau}, \\
    & \partial_t w\pare{t,\+vx;\tau} = L_x w\pare{t,\+vx;\tau}, \\
    & w\pare{t,\+vx;\tau} = f\pare{\tau,\+vx}.
\end{align*}
\newpoint{}或者将定解问题的非齐次数据写成$\delta$函数的叠加,
\[ f = \delta * f,\quad \phi = \delta * \phi, \]
考虑任意时间起点的齐次方程的初值问题的解,
\begin{align*}
    & \partial_t U\pare{t,\+vx;\tau,\+vy} = L_x U\pare{t,\+vx;\tau,\+vy}, \\
    & U\pare{t=\tau,\+vx;\tau,\+vy} = \delta\pare{\+vx-\+vy}.
\end{align*}
可得
\begin{align*}
    w\pare{t,\+vx;\tau} &= \int \rd{\+vy}\, U\pare{t,\+vx;\tau,\+vy} f\pare{\tau,\+vy}, \\
    u_1\pare{t,\+vx} &= \int \rd{\+vy}\, U\pare{t,\+vx;0,\+vy} \phi\pare{\+vy}.
\end{align*}
如果$L$不含时, 则
\[ U\pare{t,\+vx;\tau,\+vy} = U\pare{t-\tau,\+vx;0,\+vy}. \]
如果$L$平移不变, 则
\[ U\pare{t,\+vx;\tau,\+vy} = U\pare{t,\+vx-\+vy;\tau,0}. \]
综合两个假设, $L$为常系数, 则
\[ U\pare{t,\+vx;\tau,\+vy} = U\pare{t-\tau,\+vx-\+vy;0,0} = U\pare{t-\tau,\+vx-\+vy}. \]
略去$x$坐标,
\begin{align*}
    u_1\pare{t} &= U\pare{t} * \phi, \\
    w\pare{t;\tau} &= U\pare{t-\tau} * f\pare{\tau}, \\
    \partial_t U\pare{t} &= LU\pare{t}, \\
    U\pare{t=0} &= \delta.
\end{align*}
原问题的解为
\[ u = u_1 + u_2 = U\pare{t}*\phi + \int_0^t \rd{\tau}\, U\pare{t-\tau}*f\pare{\tau}. \]

% subsection 一阶发展方程 (end)

\subsection{二阶发展方程} % (fold)
\label{sub:二阶发展方程}

拆开非齐次项, $u = u_1 + u_2 + u_3$,
\begin{align*}
    \partial_t u_1 = L_x u_1,\quad u_1\pare{t=0} = \phi,\quad \partial_t u_1\pare{t=0} = 0, \\
    \partial_t u_2 = L_x u_2,\quad u_2\pare{t=0} = 0,\quad \partial_t u_1\pare{t=0} = \psi, \\
    \partial_t u_3 = L_x u_3+f,\quad u_2\pare{t=0} = 0,\quad \partial_t u_1\pare{t=0} = 0.
\end{align*}
基本解为
\begin{align*}
    \partial_t^2 U\pare{t} &= LU\pare{t}, \\
    U\pare{t=0} &= 0,\quad U_t\pare{t=0} = \delta.
\end{align*}
\newpoint{}$u_2$和$u_3$明显可以用$U$表示. 下面求$u_1$.
\newpoint{}假设$L$无时间导数,
\[ \partial_t^2 v = Lv,\quad v\pare{t=0} = 0,\quad \partial_t v\pare{t=0} = \phi \Rightarrow u_2 = \partial_t \Rightarrow u_2 \pare{t=0} = \phi. \]
\newpoint{}$L$为常系数, 则
\[ \partial_t^2 u_1 = \partial_t Lv = Lu_2. \]
\newpoint{}由于$L$不含时间导数,
\[ \partial_t u_2\pare{t=0} = \partial_t^2 v\pare{t=0} = Lv\pare{t=0} = 0. \]
从而
\[ u = \partial_t U\pare{t} * \phi + U\pare{t} * \phi + \int_0^t \rd{\tau}\, U\pare{t-\tau}*f\pare{\tau}. \]
\newpoint{}若$f$有时间导数, 需要调节其它叠加项抵消之.

\begin{sample}
    \begin{ex}
        热方程
        \[ U_t = \laplacian U,\quad U\pare{t=0} = \delta\pare{\+vx} \]
        可通过Fourier变换变为
        \[ \hat U_t\pare{t,p} = -p^2\hat U,\quad \hat U\pare{t=0,p} = 1,\quad \hat U\pare{t,p} = e^{-\+vp^2 t}. \]
        反变换可得
        \begin{align*}
            U\pare{t,x} &= \rec{2\pi} \int \rd{\+vx}\, \exp\pare{i\+vx\cdot \+vp - t\+vp^2}, \\
            U\pare{t,x} &= \rec{\pare{2\sqrt{\pi t}}^D} \exp\pare{-\frac{\+vx^2}{4t}}.
        \end{align*}
        因此$\pare{\partial_t - \laplacian}u = f,u\pare{t=0}=\phi$的解为
        \begin{align*}
            u &= U\pare{t,x}*\phi\pare{x} + \int_0^t \rd{\tau}\,U\pare{\tau,x}* f\pare{t-\tau,x}.
        \end{align*}
    \end{ex}
\end{sample}

\begin{sample}
    \begin{ex}
        考虑二维单向波
        \begin{align*}
            & U_t + U_x = 0,\quad x\in \+bR,\quad t\ge 0, \\
            & U\pare{t=0,x} = \delta\pare{x}.
        \end{align*}
        将定解问题Fourier变换得到
        \begin{align*}
            & \hat U_t + ip\hat U = 0, \\
            & \hat U\pare{t=0} = 1, \\
            & \Rightarrow \hat U = \exp\pare{-ipt}, \\
            & \Rightarrow U = \delta\pare{x-t}.
        \end{align*}
        因此
        \[ \pare{\partial_t + \partial_x}u = f,\quad u\pare{t=0} = \phi \]
        的解为
        \begin{align*}
            u\pare{t,x} &= \delta\pare{x-t}*\phi\pare{x} + \int_0^t \rd{\tau}\,\delta\pare{x-\tau} * f\pare{t-\tau} \\
            &= \phi\pare{x-t} + \int_0^t \rd{\tau}\, f\pare{t-\tau,x-\tau}.
        \end{align*}
        对$u\pare{t,x}$有影响的是以波速从$\tau = 0$到$\tau = t$并结束在$\pare{t,x}$线段上的$f$及在线段起点的$\phi$.
    \end{ex}
\end{sample}

\begin{sample}
    \begin{ex}
        波动方程
        \begin{align*}
            & U_{tt} = \laplacian U, \\
            & U\pare{t=0} = 0,\quad U_t\pare{t=0} = \delta\pare{\+vx}, \\
            & \Rightarrow \hat U_{tt} = -p^2\hat U, \\
            & \hat U\pare{t=0} = 0,\quad \hat U_t\pare{t=0} = 1, \\
            & \Rightarrow \hat U = \frac{\sin\pare{\abs{\+vp}t}}{\abs{\+vp}}, \\
            & \Rightarrow U\pare{t,\+vx} = \frac{\delta\pare{\abs{\+vx} - t}}{4\pi\abs{\+vx}}.
        \end{align*}
        反变换可以如下得到:
        \begin{align*}
            U\pare{t,x} &= \rec{2\pi^3} \int \rd{\+vp}\, \exp\pare{i\+vx\cdot \+vp} \frac{\sin\pare{\abs{\+vp}t}}{\abs{\+vp}} \\
            &= \rec{2\pi^3}\int_0^\infty \rd{p}\, p^2 \frac{\sin pt}{p} \int_0^\pi \rd{\theta} \rd{\theta}\, 2\pi \sin\theta \exp\pare{i\abs{\+vx}p\cos\theta} \\
            &= \frac{-i}{2\pi^2 \abs{x}} \int_0^\infty \rd{p}\, \frac{\sin pt}{p}\exp\pare{i\abs{\+vx}pc}\vert_{c=-1}^{c=1} \\
            &= \rec{2\pi^2\abs{\+vx}} \int_0^\infty \rd{p}\,\sin\pare{p\abs{\+vx}}\sin \pare{pt} \\
            &= \frac{4\pare{\abs{\+vx} - t}}{4\pi\abs{\+vx}}.
        \end{align*}
        从而卷积为
        \begin{align*}
            U\pare{t,\+vx} * \psi\pare{\+vx} &= \int \rd{\+vy}\, \frac{\delta\pare{\abs{\+vy} - t}}{4\pi\abs{\+vy}}\psi\pare{\+vx-\+vy} \\
            &= \frac{t}{4\pi t^2} \oint_{\abs{\+vy-\+vx} = t}\rd{\+vt}\,\psi\pare{\+vy} \\
            &= t\expc{\psi}_{\abs{\+vy-\+vx} = t}. \\
            \expc{\abs{\+vy-\+vx} = t} &= \rec{4\pi t^2}\oint_{\abs{\+vy-\+vx} = t}\rd{\+vy}\,\psi\pare{\+vy}.
        \end{align*}
        因此完整的解为
        \[ u\pare{t,x} = t\expc{\psi}_{\abs{\+vx} = t} + \partial_t \pare{t\expc{\phi}_{\abs{\+vx} = t}} + \int_0^t \rd{\tau}\,\tau \expc{f\pare{t-\tau}}_{\abs{\+vx} = \tau}. \]
    \end{ex}
\end{sample}
\begin{remark}
    二维的波动方程的解为
    \[ U\pare{t,\+vx} = \frac{H\pare{t-\abs{\+vx}}}{2\pi\sqrt{t^2 - \abs{\+vx}^2}}. \]
    可见奇数维的$\delta$输入是即时作用, 而偶数维不是.
\end{remark}

% subsection 二阶发展方程 (end)

% section green函数 (end)

\end{document}
