\documentclass[hidelinks]{article}

\usepackage[sensei=Nakahara,gakka=Geometry\ in\ Physics,section=Quantum,gakkabbr=QM]{styles/kurisuen}
\usepackage{sidenotes}
\usepackage{van-de-la-sehen-en}
\usepackage{van-de-environnement-en}
\usepackage{boite/van-de-boite-en}
\usepackage{van-de-abbreviation}
\usepackage{van-de-neko}
\usepackage{van-le-trompe-loeil}
\usepackage{cyanide/van-de-cyanide}
\setlength{\parindent}{0pt}
\usepackage{enumitem}
\newlist{citemize}{itemize}{3}
\setlist[citemize,1]{noitemsep,topsep=0pt,label={-},leftmargin=1em}

\usepackage{mathtools}
\usepackage{ragged2e}

\DeclarePairedDelimiter\abs{\lvert}{\rvert}%
\DeclarePairedDelimiter\norm{\lVert}{\rVert}%

% Swap the definition of \abs* and \norm*, so that \abs
% and \norm resizes the size of the brackets, and the 
% starred version does not.
\makeatletter
\let\oldabs\abs
\def\abs{\@ifstar{\oldabs}{\oldabs*}}
%
\let\oldnorm\norm
\def\norm{\@ifstar{\oldnorm}{\oldnorm*}}
\makeatother

\newcommand*{\Value}{\frac{1}{2}x^2}%

\usepackage{fancyhdr}
\usepackage{lastpage}

\fancypagestyle{plain}{%
\fancyhf{} % clear all header and footer fields
\fancyhead[R]{\smash{\raisebox{2.75em}{{\hspace{1cm}\color{lightgray}\textsf{\rightmark\quad Page \thepage/\pageref{LastPage}}}}}} %RO=right odd, RE=right even
\renewcommand{\headrulewidth}{0pt}
\renewcommand{\footrulewidth}{0pt}}
\pagestyle{plain}

\newtheorem*{experiment*}{Measurement}
\newtheorem{example}{Example}
\newtheorem{remark}{Remark}

\def\elementcell#1#2#3#4#5#6#7{%
    \draw node[draw, regular polygon, regular polygon sides=4, minimum height=2cm, draw=cyan, line width=0.4mm, fill=cyan!15!white, #1, inner sep=-2mm](#3) {\Large\textbf{\textsf{\color{cyan!50!black}#4}}};
    \draw (#3.corner 1) node[below left] {\footnotesize{\phantom{Hj}#5}};
    \draw (#3.corner 2) node[below right] {\small{\textsf{#6}}};
    \draw (#3.side 3) node[above] {\footnotesize #7};
    \draw (#3.corner 2) ++ (0,-0.4mm) node(nw#3) {};
    \tcbsetmacrotowidthofnode{\elementcellwidth}{#3}
    \node [fill=cyan, line width=0mm, rectangle, rounded corners=1.8mm, rectangle round south east=false, rectangle round south west=false, anchor=south west, minimum width=\elementcellwidth] at (nw#3) {\small\textsf{\color{white}#2}};
}

\DeclareSIUnit\Dq{Dq}
\usepackage{physics}
\usepackage{bbm}
\newtheorem{lemma}{Lemma}
\newtheorem{proposition}{Proposition}

\DeclareMathOperator{\Pfaffian}{Pf}
\DeclareMathOperator{\sign}{sign}

\usepackage{slashed}
\usepackage{tensor}

\begin{document}

\section{Quantum Physics} % (fold)
\label{sec:quantum_physics}

\subsection{Analytical Mechanics} % (fold)
\label{sub:analytical_mechanics}

\subsubsection{Lagranian Formalism} % (fold)
\label{ssub:lagranian_formalism}

\begin{termdef}{Configuration Space}
    The manifold $M$ which consists of the $N$ parameters $\curb{q_i}$ describing the state of a system.
\end{termdef}
The parameters $\curb{q_i}$ are called \gloss{generalized coordinates}. The \gloss[2\baselineskip]{generalized velocity} is defined by $\dot{q}_i = \rd{q_i}/\rd{t}$. The \gloss[2\baselineskip]{generalized momentum} is $\displaystyle p_k = \+D{\dot{q}_k}DL$.
\begin{termdef}[4\baselineskip]{Action}
    The functional
    \[ S\brac{q\pare{t},\dot{q}\pare{t}} = \int_{t_i}^{t_f} L\pare{q,\dot{q}}\,\rd{t}. \]
\end{termdef}
\begin{termdef}{Functional Derivative}\vspace{-\baselineskip}
    \[ \frac{\delta S\brac{q,\dot{q}}}{\delta q\pare{s}} = \lim_{\epsilon \rightarrow 0} \frac{\displaystyle \curb{S\brac{q\pare{t} + \epsilon \delta\pare{t-s}, \dot{q}\pare{t} + \epsilon \+dtd{} \delta\pare{t-s}}} - S\brac{q\pare{t},\dot{q}\pare{t}}}{\epsilon}. \]
\end{termdef}
\begin{finaleq}{Hamilton's Principle, Principle of the Least Action}
    The physically realized trajectory corresponds to an extremum of the action.
\end{finaleq}
\begin{finaleq}{Euler-Lagrange Equation}
    \centerline{$\displaystyle \frac{\delta S}{\delta q\pare{s}}\+DqDL - \+dtd{}\+D{\dot{q}}DL = 0,$\quad i.e.\quad $\displaystyle \+dtd{p_k} = \+D{q_k}DL.$}
\end{finaleq}
\begin{termdef}{Cylic}
    A coordinate is called cylic if $\displaystyle \+D{q_k}DL = 0$.
\end{termdef}
The momentum conjugate to a cylic coordinate is conserved.
More generally, if $S$ in invariant under $q_k\pare{t} \rightarrow q_k\pare{t} + \delta q_k\pare{t}$, the quantity $\displaystyle \sum_k \delta q_k\pare{t}p_k\pare{t}$ is independent of $t$.

% subsubsection lagranian_formalism (end)

\subsubsection{Hamiltonian Formalism} % (fold)
\label{ssub:hamiltonian_formalism}

\begin{termdef}{Hamiltonian}
    \[ H\pare{q,p} = \sum_k p_k \dot{q}_k - L\pare{q,\dot{q}}. \]
\end{termdef}
The space with coordinates $\pare{q_k, p_k}$ is called the \gloss{phase space}.
\begin{finaleq}{\\Hamilton's Equations of Motion}
    \centerline{$\displaystyle \dot{q}_k = \+D{p_k}DH,\quad \dot{p}_k = -\+D{q_k}DH.$}
\end{finaleq}
\begin{termdef}{Poisson Bracket}
    \[ \brac{A,B} = \sum_k \pare{\+D{q_k}DA \+D{p_k}DB - \+D{p_k}DA\+D{q_k}DB}. \]
\end{termdef}
The poisson bracket is a \gloss{Lie bracket}, i.e.
\begin{align*}
    \brac{A,c_1B_1 + c_2B_2} &= c_1\brac{A,B_1} + c_2\brac{A,B_2}, \\
    \brac{A,B} &= -\brac{B,A}, \\
    \brac{\brac{A,B},C} + \brac{\brac{C,A},B} + \brac{\brac{B,C},A} &= 0.
\end{align*}
The time development of physical quantity $A\pare{q,p}$ is expressed in terms of Poisson bracket as
\[ \+dtdA = \brac{A,H}. \]
\begin{finaleq}{Noether's Theorem}
    If $H\pare{q_k,p_k}$ is invariant under an infinitesimal coordinate transformation $q_k \mapsto q'_k = q_k + \epsilon f_k\pare{q}$, then
    \[ Q = \sum_k p_k f_k\pare{q} \]
    is conserved.
\end{finaleq}
The conserved quantity $Q$ is the generator of the transformation, i.e.
\[ \brac{q_i,Q} = f_i\pare{q} \Rightarrow \delta q_i = \epsilon\brac{q_i,Q}. \]

% subsubsection hamiltonian_formalism (end)

% subsection analytical_mechanics (end)

\subsection{Canonical Quantization} % (fold)
\label{sub:canonical_quantization}

\subsubsection{Bracket Notation and Axioms} % (fold)
\label{ssub:bracket_notation_and_axioms}

\begin{termdef}{Ket, Ket Vector}
    An element in a complex Hilbert space $\+cH = \curb{\ket{\phi}, \ket{\psi}, \cdots}$.
\end{termdef}
\begin{termdef}{Dual Vector Space}
    The set of linear functions $\func{\alpha}{\+cH}{\+bC}$, denoted by $\+cH^*$.
\end{termdef}
\begin{termdef}{Bra, Bra Vector}
    An element of $\+cH^*$.
\end{termdef}
The \gloss{dual basis} of $\curb{\ket{e_k}}$ is defined by
\[ \bra{\varepsilon_i}\ket{e_j} = \delta_{ij}. \]
Under the expansion
\[ \ket{\psi} = \sum_k \psi_k \ket{e_k},\quad \bra{\phi} = \sum_k \phi_k^* \bra{\varepsilon_k}, \]
the \gloss{inner product} is
\[ \bra{\phi}\ket{\psi} = \sum_k \phi_k^* \psi_k. \]
The \gloss{norm} is defined by
\[ \norm{\ket{\psi}} = \sqrt{\braket{\psi}{\psi}}, \]
where
\[ \bra{\psi} = \sum_k \psi_k^* \bra{\varepsilon_k}. \]
\begin{finaleq}{Completeness Relation}
    If $\ket{e_k}$ is an orthonormal basis, then
    \[ \sum_k \bra{e_k}\ket{e_k} = \mathbbm{1}. \]
\end{finaleq}
The axioms of canonical quantization are listed below.
\begin{cenum}
    \item The \gloss[-\baselineskip]{state} or \gloss{state vector} is a vector $\ket{\psi}\in \+cH$. Moreover, for any $c\in \+bC\backslash\curb{0}$, $\ket{\psi}$ and $c\ket{\psi}$ describe the same state. The state is a \gloss{ray representation} of $\+cH$.
    \item An \gloss{observable} is a Hermitian operator $\hat{A}$ acting on $\+cH$.
    \item The Poisson bracket is replaced by \gloss{commutator}
    \[ \brac{\hat{A},\hat{B}} = \rec{i}\pare{\hat{A}\hat{B} - \hat{B}\hat{A}}. \]
    \item The expectation value of the result of an observation of $A$ is
    \[ \expc{A}_t = \frac{\expval{\hat{A}\pare{t}}{\psi}}{\bra{\psi}\ket{\psi}}. \]
    \item For any $\ket{\psi} \in \+cH$, there exists an operator for which $\ket{\psi}$ is one of the eigenstates.
\end{cenum}

% subsubsection bracket_notation_and_axioms (end)

\subsubsection{Quantum Dynamics} % (fold)
\label{ssub:quantum_dynamics}

\begin{finaleq}{Heisenberg's Equation of Motion}
    \centerline{$\displaystyle \+dtd{\hat{A}} = \rec{i}\brac{\hat{A},\hat{H}}.$}
\end{finaleq}
The probability of $\ket{\psi}$ being in $\ket{n}$ is
\[ \abs{\psi_n}^2 = \abs{\bra{n}\ket{\psi}}^2. \]
$\braket{n}{\psi}$ is called the \gloss{probability amplitude}.
If $\hat{A}$ has a continuous spectrum $a$, the state $\ket{\psi}$ is expanded as
\[ \ket{\psi} = \int \rd{a}\,\psi\pare{a}\ket{a}. \]
The completeness relation is now
\[ \int \rd{a}\,\ket{a}\bra{a} = \mathbbm{1}, \]
ans the normalization is
\[ \bra{a'}\ket{a} = \delta\pare{a'-a}, \]
where $\delta$ is the \textbf{Dirac $\delta$-function}\jgloss{Dirac $\delta$-Function}.
\begin{termdef}[\baselineskip]{Heisenberg Picture}
    Operators depend on $t$ while states do not.
\end{termdef}
In this formalism,
\[ \hat{A}\pare{t} = \hat{U}^\dagger\pare{t}\hat{A}\pare{0}\hat{U}\pare{t}, \]
where
\[ \hat{U}\pare{t} = e^{-i\hat{H}t}. \]
\begin{termdef}{Schr\"odinger Picture}
    The states depend on $t$ while operators do not.
\end{termdef}
The evolution of the state vector is described by \gloss{Schr\"odinger equation}
\[ i\+dtd{}\ket{\psi\pare{t}}_S = \hat{H}\ket{\psi\pare{t}}_S. \]
$\psi\pare{x} = \bra{x}\ket{\psi}$ is called the \gloss{wavefunction}.
\begin{lemma}
    $\hat{U}\pare{a} = e^{-ia\hat p}$ satisfies
    \[ \hat{U}\pare{a}\ket{x} = \ket{x+a}. \]
\end{lemma}
The action of operators on wavefunctions are
\begin{align*}
    \bra{x}\hat{x}\ket{\psi} &= x\bra{x}\ket{\psi} = x\psi\pare{x}, \\
    \bra{x}\hat{p}\ket{\psi} &= -i\+dxd{}\bra{x}\ket{\psi} = -i\+dxd{}\psi\pare{x}.
\end{align*}
The latter equation is obtained noting that\begin{marginwarns}
    The $\mathrm{d}/\mathrm{d}x$ here is the difference between two eigenstates.
\end{marginwarns}
\[ \hat{U}\pare{\epsilon}\ket{x} = \ket{x+\epsilon} \sim \pare{1-i\epsilon \hat{p}}\ket{x} \Rightarrow \hat{p}\ket{x} = i\+dxd{}\ket{x}. \]
\begin{proposition}
    $\displaystyle \braket{x}{p} = \rec{\sqrt{2\pi}}e^{ipx}$,\quad $\displaystyle \braket{p}{x} = \rec{\sqrt{2\pi}}e^{-ipx}$.
\end{proposition}
\begin{finaleq}{Time-Dependent Schr\"odinger Equation}
    For a Hamiltonian of the type $\displaystyle \hat H = \frac{\hat p^2}{2m} + V\pare{\hat{x}}$,
    \[ i\+dtd{}\psi\pare{x,t} = -\rec{2m}\+d{x^2}d{^2}\psi\pare{x,t}+V\pare{x}\psi\pare{x,t}. \]
\end{finaleq}
With separation of variables, one obtains the \gloss[-2\baselineskip]{time-independent Schr\"odinger equation}, i.e. the \gloss[0\baselineskip]{stationary state Schr\"odinger equation}, or the \gloss[3\baselineskip]{Schr\"odinger equation}
\[ -\rec{2m}\laplacian \phi\pare{\+vx} + V\pare{\+vx}\phi\pare{\+vx} = E\phi\pare{\+vx}. \]
\begin{sample}
    \begin{example}
        For a harmonic oscillator $\displaystyle \hat H = \frac{\hat p^2}{2m} + \half m\omega^2 \hat{x}^2$, with $\xi = \sqrt{m\omega}x$, $\varepsilon = E/\hbar\omega$, one arrives at
        \[ \psi'' + \pare{\varepsilon - \xi^2}\psi = 0, \]
        and $\displaystyle E = E_n = \pare{n+\half}\omega$,
        \[ \psi\pare{\xi} = \sqrt{\frac{m\omega}{2^n n! \sqrt{\pi}}} H_n\pare{\xi}e^{-\xi^2/2}, \]
        where
        \[ H_n\pare{\xi} = \pare{-1}^n e^{\xi^2/2}\+d{\xi^n}d{^n}e^{-\xi^2/2}. \]
    \end{example}
\end{sample}
With the \gloss{annihilation operator} $\hat a$ and the \gloss[2\baselineskip]{creation operator} $\hat a^\dagger$ defined by
\[ \hat a = \sqrt{\frac{m\omega}{2}}\hat x + i\sqrt{\rec{2m\omega}}\hat p,\quad \hat a^\dagger = \sqrt{\frac{m\omega}{2}}\hat x - i\sqrt{\rec{2m\omega}}\hat p, \]
and the number operator by
\[ \hat N = \hat a^\dagger \hat a, \]
one obtains
\[ \hat N\hat a\ket{n} = \pare{n-1}\hat a\ket{n},\quad \hat N\hat a^\dagger\ket{n} = \pare{n+1}\hat a\dagger\ket{n}. \]
Since $n = \bra{n}\hat N\ket{n} \ge 0$ and the equiality holds if and only if $\hat a\ket{n} = 0$, $n$ must be a non-negative integer. With $\displaystyle \hat H = \pare{\hat N+\half}\omega$, one finds $\displaystyle E = \pare{n+\half}\omega$.

% subsubsection quantum_dynamics (end)

% subsection canonical_quantization (end)

\subsection{Path Integral Quantization of a Bose Particle} % (fold)
\label{sub:path_integral_quantization_of_a_bose_particle}

\subsubsection{Path Integral Quantization} % (fold)
\label{ssub:path_integral_quantization}

Let $\ket{x_i,t_i}$ and $\ket{x_f,t_f}$ denote the eigenvector of position in Heisenberg picture, respectively, i.e.
\[ \hat x\pare{t_i}\ket{x_i,t_i} = x_i\ket{x_i,t_i},\quad \hat x\pare{t_f}\ket{x_f,t_f} = x_f\ket{x_f,t_f}, \]
with which we define the transition amplitude.
\begin{termdef}{Transition Amplitude}
    The probability amplitude $\bra{x_f,t_f}\ket{x_i,t_i}$.
\end{termdef}
Writing in terms of the Schr\"odinger picture,\begin{margintips}
    $\begin{array}{@{}l}
        \displaystyle \int_{-\infty}^\infty e^{-iap^2}\,\rd{p}\\
        \displaystyle  = \sqrt{\frac{\pi}{ia}}.
    \end{array}$
\end{margintips}
\[ \ket{x_i,t_i} = e^{i\hat H t_i}\ket{x_i},\quad \ket{x_f,t_f} = e^{i\hat H t_f}\ket{x_f}, \]
we found
\[ \bra{x_f,t_f}\ket{x_i,t_i} = \bra{x_f}e^{-i\hat H\pare{t_f - t_i}}\ket{x_i} = h\pare{x_f,x_i;i\pare{t_f - t_i}}, \]
where the function
\[ h\pare{x,y;\beta} = \bra{x}e^{-\hat H\beta}\ket{y} \]
is called the \gloss[-\baselineskip]{heat kernel} of $\hat H$.\begin{margintips}[2\baselineskip]
    $\begin{array}{@{}l}
        \bra{x}A\pare{\hat x,\hat p}\ket{p}\\
        = A\pare{\hat x,-i\partial_x}\bra{x}\ket{p}.
    \end{array}$
\end{margintips}
\begin{finaleq}{Infinitesimal Transition Amplitude}
    \begin{proposition}
        For $\displaystyle \hat H = \frac{\hat p^2}{2m} + V\pare{\hat x}$ and $\epsilon$ an infinitesimal positive number,
        \begin{align*}
            \bra{x}e^{-i\hat H\epsilon}\ket{y} &= \rec{\sqrt{2\pi i\epsilon}}\exp\brac{i\epsilon\curb{\frac{m}{2} \pare{\frac{{x-y}}{\epsilon}}^2 - V\pare{\frac{x+y}{2}}} + O\pare{\epsilon^2} + O\pare{\epsilon\pare{x-y}^2}}.
        \end{align*}
    \end{proposition}
\end{finaleq}
\marginnote[-\baselineskip]{\inlinehardlink{cf. \S 2.6, Sakurai.}}
The prescription that the average value $\pare{x+y}/2$ is taken as the variable of $V$ is called the \gloss{Weyl ordering}.
\par
The transition amplitude for a finite time interval is
\begin{align*}
    \bra{x_f,t_f}\ket{x_i,t_i} &= \bra{x_f,t_f} \int \rd{x_{n-1}}\,\ket{x_{n-1},t_{n-1}}\bra{x_{n-1},t_{n-1}} \\
    &\phantom{=\ }\times \int \rd{x}_{n-2}\,\ket{x_{n-2},t_{n-2}}\cdots \int \rd{x_1}\,\ket{x_1,t_1}\bra{x_1,t_1}\ket{x_0,t_0} \\
    &= \lim_{n\rightarrow \infty}\pare{\frac{m}{2\pi i\epsilon}}^{n/2} \int \prod_{j=1}^{n-1}\rd{x_j}\, \exp\pare{i\sum_{k=1}^n \Delta S_k},
\end{align*}
where
\[ \Delta S_k = \epsilon\brac{\frac{m}{2}\pare{\frac{x_k-x_{k-1}}{\epsilon}}^2 - V\pare{\frac{x_{k-1}+x_k}{2}}}, \]
which may be written symbolically as
\begin{align*}
    \bra{x_f,t_f}\ket{x_i,t_i} &= \int \+cDx\,\exp\brac{i\int_{t_i}^{t_f}\rd{t}\,\pare{\frac{m}{2}v^2 - V\pare{x}}} \\
    &= \int \+cDx\,\exp\brac{i\int_{t_i}^{t_f}\rd{t}\,L\pare{x,\dot{x}}},
\end{align*}
where $\displaystyle \int \+cDx$ represents summation over all paths $x\pare{t}$ with $x\pare{t_i} = x_i$ and $x\pare{t_f} = x_f$.
\begin{sample}
    \begin{example}
        The propagator of a free particle is obtained via
        \begin{align*}
            \bra{x_f,t_f}\ket{x_i,t_i} &= \lim_{n\rightarrow \infty}\pare{\frac{m}{2\pi i\epsilon}}^{n/2}\int \rd{x_1}\,\cdots\,rd{x_{n-1}} \\
            &\phantom{=}\ \exp\brac{i\epsilon \sum_{k=1}^n \frac{m}{2}\pare{\frac{x_k - x_{k-1}}{\epsilon}}^2} \\
            &= \sqrt{\frac{m}{2\pi iT}}\exp\brac{\frac{im\pare{x_f - x_i}^2}{2T}},
        \end{align*}
        where $\epsilon = T/n$ and $T = t_f - t_i$.
    \end{example}
\end{sample}

% subsubsection path_integral_quantization (end)

\subsubsection{Imaginary Time and Partition Function} % (fold)
\label{ssub:imaginary_time_and_partition_function}

The \gloss{Wick rotation} is the replacement
\[ t = -i\tau,\quad \tau\in\+bR_+, \]
with which the path integral is expressed as
\begin{align*}
    \bra{x_f,\tau_f}\ket{x_i,\tau_i} &= \bra{x_f}e^{-\hat H\pare{\tau_f - \tau_i}}\ket{x_i} \\
    &= \int \conj{\+cD}x\,\exp\brac{-\int_{\tau_i}^{\tau_f} \rd{\tau}\, \brac{\half m\pare{\+d\tau dx}^2 + V\pare{x}}}.
\end{align*}
\begin{termdef}{Partition Function}
    The {partition function} for a given Hamiltonian $\hat H$ is defined by
    \[ Z\pare{\beta} = \trace e^{-\beta \hat H} = \sum_n e^{-\beta E_n}, \]
\end{termdef}
which may also be written as
\[ Z\pare{\beta} = \int \rd{x}\,\bra{x}e^{-\beta\hat H}\ket{x}. \]
Putting $\beta = iT$, we find
\begin{align}
    Z\pare{\beta} &= \int \rd{y}\int_{x\pare{0} = x\pare{\beta} = y}\conj{\+cD}x\,\exp\curb{-\int_0^\beta \rd{\tau}\,\pare{\half m\dot{x}^2 + V\pare{x}}}. \notag
\end{align}\vspace{-\baselineskip}
\begin{finaleq}{Partition Funtion via Path Integral}
    \begin{equation}
    \label{eq:partition_function_by_path_integral}
        Z\pare{\beta} = \int_{\mathrm{periodic}}\conj{\+cD}x\,\exp\curb{-\int_0^\beta \rd{\tau}\,\pare{\half m\dot{x}^2 + V\pare{x}}}.
    \end{equation}
\end{finaleq}

% subsubsection imaginary_time_and_partition_function (end)

\subsubsection{Time-Ordered Product and Generating Functional} % (fold)
\label{ssub:time_ordered_product_and_generating_functional}

\begin{termdef}{$T$-product}
    The $T$-product of Heisenberg operators $A\pare{t}$ and $B\pare{t}$ is defined by
    \[ T\brac{A\pare{t_1}B\pare{t_2}} = A\pare{t_1}B\pare{t_2}\Theta\pare{t_1 - t_2} + B\pare{t_2}A\pare{t_1}\Theta\pare{t_2 - t_1}. \]
\end{termdef}
For the case with more than three operators, the result is an arragement of operators in an order such that the time parameters decrease from the left to the right.
\par
Suppose $t_i < t_1\le t_2\le \cdots\le t_n<t_f$, inserting $\mathbbm{1}$ between the operators one gets
\begin{align*}
    \bra{x_f,t_f}\hat x\pare{t_n}\cdots \hat x\pare{t_1}\ket{x_i,t_i} & = \int \rd{x_1}\,\cdots \,\rd{x_n}\, x_1\cdots x_n \bra{x_f,t_f}\ket{x_n,t_n}\cdots\bra{x_1,t_1}\ket{x_i,t_i} \\
    &= \int \+cD x\,x\pare{t_1}\cdots x\pare{t_n}e^{iS}.
\end{align*}
The ordering of time parameters on the RHS is arbitrary, therefore the equality holds as long as the ordering on the LHS is the one prescribed by the $T$-product, i.e.
\[ \bra{x_f,t_f}T\brac{\hat x\pare{t_n}\cdots \hat x\pare{t_1}}\ket{x_i,t_i} = \int \+cD x\,x\pare{t_1}\cdots x\pare{t_n}e^{iS}. \]
\par
With a coupling $x\pare{t}J\pare{t}$ added into the Lagrangian, the functional derivative of the transition amplitude with respect to $J\pare{t}$ is
\[ \frac{\delta}{\delta J\pare{t}}\bra{x_f,t_f}\ket{x_i,t_i}_J = \int \+cD{x}\, ix\pare{t}\exp\brac{i\int_{t_i}^{t_f}\rd{t}\,\pare{\half m\dot{x}^2 - V\pare{x} + xJ}}. \]
Therefore,
\[ \bra{x_f,t_f}T\brac{\hat x\pare{t_n}\cdots \hat x\pare{t_1}}\ket{x_i,t_i} = \left.\pare{-i}^n \frac{\delta^n}{\delta J\pare{t_1} \cdots \delta J\pare{t_n}}\int \+cD x\, e^{iS\brac{x\pare{t},J\pare{t}}}\right\vert_{J=0}. \]
\begin{termdef}{Generating Functional}
    With $U^J\pare{b,a}$ being the time evolution operator of $H^J = H - x\pare{t}J\pare{t}$ and $J = 0$ outside $\brac{a,b}$, the generating functional is defined by
    \[ Z\brac{J} = \bra{0}U^J\pare{b,a}\ket{0} = \lim_{\stackrel{t_f \rightarrow \infty}{t_i \rightarrow -\infty}}\bra{0}U^J\pare{t_f,t_i}\ket{0}, \]
\end{termdef}
where the ground state energy have been shifted to zero. Under this definition, the transition amplitude may be written as
\begin{align*}
    \bra{x_f,t_f}\ket{x_i,t_i}_J &= \sum_{m,n} e^{-iE_m\pare{t_f - b}}e^{-iE_n\pare{a-t_i}}\bra{x_f}\ket{m}\bra{n}\ket{x_i}\bra{m}U^J\pare{b,a}\ket{n},
\end{align*}
where only the ground state contributes to the summation, i.e.
\[ \lim_{\stackrel{t_f \rightarrow \infty}{t_i \rightarrow -\infty}}\bra{x_f,t_f}\ket{x_i,t_i}_J = \bra{x_f}\ket{0}\bra{0}\ket{x_j}Z\brac{J}. \]
\vspace{-\baselineskip}
\begin{finaleq}{Generating Functional}
    With $\+cN$ chosen such that $Z\brac{0} = 1$ i.e. $\displaystyle \+cN^{-1} = \int \+cD x\, e^{iS\brac{x,0}}$, we have
    \[ Z\brac{J} = \lim_{\stackrel{t_f \rightarrow \infty}{t_i\rightarrow -\infty}} \frac{\bra{x_f,t_f}\ket{x_i,t_i}_J}{\bra{x_f}\ket{0}\bra{0}\ket{x_i}} = \+cN \int \+cD x\, e^{iS\brac{x,J}}. \]
\end{finaleq}
\begin{finaleq}{$T$-product Between the Ground States}
    \centerline{$\displaystyle \bra{0}T\brac{x\pare{t_1}\cdots x\pare{t_n}}\ket{0} = \left.\pare{-i}^n \frac{\delta^n}{\delta J\pare{t_1}\cdots \delta J\pare{t_n}}Z\brac{J}\right\vert_{J=0}.$}
\end{finaleq}

% subsubsection time_ordered_product_and_generating_functional (end)

% subsection path_integral_quantization_of_a_bose_particle (end)

\subsection{Harmonic Oscillator} % (fold)
\label{sub:harmonic_oscillator}

\subsubsection{Transition Amplitude} % (fold)
\label{ssub:transition_amplitude}

The action of the classical path of motion is
\begin{equation}
    \label{eq:SHO_action_classical_path}
    S_c = \frac{m\omega}{2\sin\omega T}\brac{\pare{x_f^2 + x_i^2}\cos\omega T - 2x_f x_i},
\end{equation}
which has the functional derivative
\begin{align*}
    \frac{\delta^2 S}{\delta x\pare{t_1} \delta x\pare{x_2}} &= -m\pare{\+d{t_1^2}d{^2}+\omega^2}\delta\pare{t_1 - t_2},
\end{align*}
hence the new action of a path around $x_c$
\begin{align*}
    S\brac{x_c+y} &= S\brac{x_c} + \rec{2!}\int \rd{t_1}\,\rd{t_2}\, y\pare{t_1}y\pare{t_2}\left.\frac{\delta^2 S\brac{x}}{\delta x\pare{x_1}\delta\pare{x_2}}\right\vert_{x=x_c} \\
    &= S\brac{x_c} + \frac{m}{2}\int \rd{t}\,\pare{\dot{y}^2 - \omega^2 y^2} \\
    &= S\brac{x_c} + \frac{m}{2}\cdot \frac{T}{2}\sum_{n\in\+bN} a_n^2\brac{\pare{\frac{n\pi}{T}}^2 - \omega^2},
\end{align*}
with the expansion
\[ y\pare{t} = \sum_{n\in\+bN} a_n \sin \frac{n\pi t}{T}. \]
The transition amplitude is
\begin{align*}
    \bra{x_f,T}\ket{x_i,0} &= e^{iS\brac{x_c}}\int_{y\pare{t_i} = y\pare{t_f} = 0}\+cD y\, e^{i\frac{m}{2}\int_{0}^{T} \rd{t}\,\pare{\dot{y}^2 - \omega^2 y^2}} \\
    &= \lim_{N\rightarrow \infty} J_N \pare{\frac{1}{2\pi i\epsilon}}^{N/2} e^{iS\brac{x_c}}\int \rd{a_1}\,\cdots \,\rd{a_{N-1}}\,\exp\brac{i\frac{mT}{4}\sum_{n=1}^{N-1}a_n^2 \curb{\pare{\frac{n\pi}{T}}^2 - \omega^2}},
\end{align*}
where
\[ J_N = N^{-N/2}2^{-\pare{N-1}/2}\pi^{N-1}\pare{N-1}! \]
is the Jacobian $\det \partial y_k/\partial a_n$. The integrals could be evaluated to yield the final form of the propagator.
\begin{finaleq}{Transition Amplitude of the Harmonic Oscillator}
    \centerline{$\displaystyle \bra{x_f,t_f}\ket{x_i,t_i} = \pare{\frac{\omega}{2\pi i\sin \omega T}}^{1/2} \exp\brac{\frac{i\omega}{2\sin\omega T}\curb{\pare{x_f^2 + x_i^2}\cos\omega T - 2x_i x_f}}.$}
\end{finaleq}

% subsubsection transition_amplitude (end)

\subsubsection{Partition Function} % (fold)
\label{ssub:partition_function}

\paragraph{Method 1} % (fold)
\label{par:method_1}

Putting $\beta = it$, \marginnote{\inlinehardlink{\Cref{ssub:imaginary_time_and_partition_function}}}\begin{marginwarns}[1.5\baselineskip]{Factors of $m$ missing.}\end{marginwarns}
\begin{align*}
    Z\pare{\beta} &= \int \rd{x}\,\bra{x}e^{-\beta \hat H}\ket{x} \\
    &= \pare{\frac{\omega}{2\pi i \pare{-i\sinh \beta \omega}}}^{1/2}\int \rd{x}\,\exp i\brac{\frac{\omega}{-2i\sinh \beta\omega}\pare{2x^2 \cosh \beta \omega - 2x^2}} \\
    &= \rec{2\sinh \pare{\beta\omega/2}}.
\end{align*}

% paragraph method_1 (end)

\paragraph{Method 2} % (fold)
\label{par:method_2_of_partition_function_of_harmonic_oscillator}

The flucuation part in \eqref{eq:partition_function_by_path_integral} is evaluated as
\begin{align*}
    \int_{y\pare{0} = y\pare{\beta} = 0} \conj{\+cD}y\,\exp\brac{-\half \int\rd{\tau}\, y\pare{-m\+d{\tau^2}d{^2} + m\omega^2}y}
    &= -\rec{\sqrt{\det\+_D_ \brac{\pare{m/\pi}\cdot\pare{-\mathrm{d}^2/\mathrm{d}\tau^2 + \omega^2}}}},
\end{align*}
where the subscript $\mathrm{D}$ implies that the eigenvalues are evaluated with Dirichlet boundary condition $y\pare{0} = y\pare{\beta} = 0$.
\begin{termdef}{Spectral $\zeta$-function}
    \[ \zeta_{\+cO}\pare{s} = \sum_n \rec{\lambda_n^s}. \]
\end{termdef}
With the above definition we arrive at\begin{margintips}
    $\begin{array}{@{}l}
        \displaystyle \prod_{n=1}^\infty \pare{1+\frac{x^2}{n^2}}\\
        \displaystyle  = \frac{\sinh \pi x}{\pi x}.
    \end{array}$
\end{margintips}
\[ \det \+cO = \exp\brac{\left.\+dsd{\zeta_{\+cO}\pare{s}}\right\vert_{s=0}}. \]
Taking $\+cO = -\pare{m/\pi}\cdot\pare{-\mathrm{d}^2/\mathrm{d}^2\tau^2}$, we get
\[ \zeta_{-\pare{m/\pi}\mathrm{d}^2/\mathrm{d}\tau^2}\pare{s} = \sum_{n\ge 1} \pare{\frac{mn}{\beta}}^{-2s} = \pare{\frac{\beta}{m}}^{2s}\zeta\pare{2s}. \]
Now
\[ {\det}_D \pare{-\frac{\mathrm{d^2}}{\mathrm{d}\tau^2} + \omega^2} = \prod_{n=1}^\infty \pare{\frac{n \pi}{\beta}}^2 \prod_{p=1}^\infty \brac{1+\pare{\frac{\beta \omega}{p\pi}}^2}, \]
and the partition function is obtained after evaluating the infinite products multiplied by the integral of the non-fluctuation part \eqref{eq:SHO_action_classical_path} with $\tau = it$.

% paragraph method_2 (end)

% subsubsection partition_function (end)

% subsection harmonic_oscillator (end)

\subsection{Path Integral Quantization of a Fermi Particle} % (fold)
\label{sub:path_integral_quantization_of_a_fermi_particle}

\subsubsection{Fermionic Harmonic Oscillator} % (fold)
\label{ssub:fermionic_harmonic_oscillator}

\begin{termdef}{Fermionic Harmonic Oscillator}
    The Hamiltonian is defined by
    \[ H = \half\pare{c^\dagger c - cc^\dagger} \omega. \]
\end{termdef}
Where $c$ and $c^\dagger$ satisfy the anti-commutation relations
\[ \curb{c,c^\dagger} = 1,\quad \curb{c,c} = \curb{c^\dagger,c^\dagger} = 0, \]
and the Hamiltonian takes the form
\[ H = \pare{N-\half}\omega. \]
where $N = c^\dagger c$. Now $N^2 = c^\dagger c c^\dagger c = N$, we obtain the Pauli principle $N\pare{N-1} = 0$, and that
\[ c^\dagger\ket{0} = \ket{1},\quad c\ket{0} = 0,\quad c^\dagger\ket{1} = 0,\quad c\ket{1} = \ket{0}. \]

% subsubsection fermionic_harmonic_oscillator (end)

\subsubsection{Calculus of Grassmann numbers} % (fold)
\label{ssub:calculus_of_grassmann_numbers}

With generators $\curb{\theta_1,\cdots,\theta_n}$ satisfying \begin{margindef}{C-Number}
    `C' stands for `commuting'.
\end{margindef} the anti-commutation relations
\[ \curb{\theta_i,\theta_j} = 0,\quad \forall i,j \]
we could define the Grassman number and the Grassmann algebra.
\begin{termdef}[\baselineskip]{Grassmann Number}
    The set of linear combinations of $\curb{\theta_i}$ with c-number coefficients.
\end{termdef}
\begin{termdef}{Grassmann Algebra}
    The algebra generated by $\curb{\theta_i}$, denoted by $\Lambda^n$, i.e. the algebra consisting of
    \[ f\pare{\theta} = f_0 + \sum_{i=1}^n f_i\theta_i + \sum_{i<j}f_{ij}\theta_i\theta_j + \cdots = \sum_{0\le k\le n} \rec{k!}\sum_{\curb{i}}f_{i_1\cdots i_k}\theta_{i_1}\cdots \theta_{i_k}. \]
    Where the $\curb{f_{i_1\cdots i_l}}$ are c-numbers and anti-symmetric under exchange of any two indices.
\end{termdef}
The elements $f\pare{\theta}$ may also be written as
\[ f\pare{\theta} = \sum_{k_i = 0,1} \tilde{f}_{k_1\cdots k_n}\theta_1^{k_1}\cdots \theta_n^{k_n}. \]
\begin{sample}
    \begin{example}
        For a Grassmann number $\theta$, $e^\theta = 1+\theta$.
    \end{example}
\end{sample}
$\Lambda^n$ may be written as a direct sum
\[ \Lambda^n = \Lambda^n_+ \oplus \Lambda^n_-, \]
where $\Lambda^n_+$ and $\Lambda^n_-$ denotes the subset generated by monomials of even power and odd power, respectively. This separation is called \textbf{$\+bZ_2$-grading}\jgloss{Z\textsubscript{2}-Grading}.
\par
The generators satisfy the following relations:
\begin{align*}
    \theta_k^2 &= 0, \\
    \theta_{k_1}\cdots \theta_{k_n} &= \epsilon_{k_1 \cdots k_n}\theta_1\cdots\theta_n, \\
    \theta_{k_1}\cdots \theta_{k_m} &= 0,\quad m>n.
\end{align*}

% subsubsection calculus_of_grassmann_numbers (end)

\subsubsection{Differentiation} % (fold)
\label{ssub:differentiation}

The differential operator anti-commutes with $\theta_k$, and the Leibniz rule takes the form
\[ \+D{\theta_i}D{}\pare{\theta_j \theta_k} = \delta_{ij}\theta_k - \theta_{ik}\theta_j, \]
and therefore the differential operator is nilpotent:
\[ \+D{\theta_i^2}D{^2} = 0. \]

% subsubsection differentiation (end)

\subsubsection{Integration} % (fold)
\label{ssub:integration}

\begin{termdef}{Integration of Grassmann Numbers}
    \[ \int \rd{\theta}\, f\pare{\theta} = \+D\theta D{f\pare{\theta}}. \]
\end{termdef}
For $n$ generators we have \begin{marginwarns}
    The ordering of $\curb{\rd{\theta_i}}$ and $\curb{\partial/\partial \theta_i}$ matters.
\end{marginwarns}
\[ \int \rd{\theta_1}\,\rd{\theta_2}\cdots \rd{\theta_n}\, f\pare{\theta_1,\theta_2,\cdots,\theta_n} = \+D{\theta_1}D{} \+D{\theta_2}D{} \cdots \+D{\theta_n}D{} f\pare{\theta_1,\theta_2,\cdots,\theta_n}. \]
Under the change of variables $\theta' = a\theta$,
\[ \int \rd{\theta_1}\,\rd{\theta_2}\cdots \rd{\theta_n}\, f\pare{\theta} = \det a \int \rd{\theta_1'}\,\rd{\theta_2'}\cdots \rd{\theta_n'}\, f\pare{a^{-1}\theta}. \]
\begin{sample}
    \begin{example}
        $\displaystyle \int \rd{\theta} = 0,\ \int \rd{\theta}\,\theta = 1$.
    \end{example}
\end{sample}

% subsubsection integration (end)

\subsubsection{Delta-Function} % (fold)
\label{ssub:delta_function}

The $\delta$ function is given by
\[ \delta\pare{\theta-\alpha} = \theta-\alpha. \]
It may also be written in the form of Fourier transformation as
\[ \delta\pare{\theta} = \theta = -i\int \rd{\xi}\,e^{i\xi\theta}. \]

% subsubsection delta_function (end)

\subsubsection{Gaussian Integral} % (fold)
\label{ssub:gaussian_integral}

For two independent Grassmann variables $\curb{\theta_i}$ and $\curb{\omega_i}$, and an $n\times n$ c-number matrix $M$,
\begin{equation}
    \label{eq:gaussian_integral_to_det}
    \boxed{\int\rd{\omega_1}\,\rd{\theta_1}\cdots \rd{\omega_n}\,\rd{\theta_n}\, e^{-\sum_{ij}\omega_i M_{ij} \theta_j} = \det M.}
\end{equation}\vspace{-\baselineskip}
\begin{termdef}{Pfaffian}
    \[ \Pfaffian\pare{a} = \rec{2^n n!}\sum_{\substack{\mathrm{Permutations of}}{\curb{i_1,\cdots,i_{2n}}}} \sign\pare{P} a_{i_1i_2}\cdots a_{i_{2n-1}i_{2n}}. \]
\end{termdef}
\begin{proposition}
    $\det a = \Pfaffian\pare{a}^2$.
\end{proposition}

% subsubsection gaussian_integral (end)

\subsubsection{Functional Derivative} % (fold)
\label{ssub:functional_derivative}

Let $\psi\pare{t}$ be a Grassmann variable depending on a c-number $t$ and $F\brac{\psi\pare{t}}$ be a functional of $\psi$, we define \begin{marginwarns}
    Division by a Grassmann number is not well defined.
\end{marginwarns}
\[ \frac{\delta F\brac{\psi\pare{t}}}{\delta \psi\pare{s}} = \rec{\epsilon}\curb{F\brac{\psi\pare{t}} + \epsilon \delta\pare{t-s} - F\brac{\psi\pare{t}}}, \]
where the numerator is evaluated via Taylor expansion.

% subsubsection functional_derivative (end)

\subsubsection{Complex Conjugation} % (fold)
\label{ssub:complex_conjugation}

The complex conjugation of Grassmann numbers is defined by $\pare{\theta_i}^* = \theta_i^*$ and $\pare{\theta_i^*}^* = \theta_i$, and that
\[ \pare{\theta_i\theta_j}^* = \theta_j^*\theta_i^*. \]

% subsubsection complex_conjugation (end)

\subsubsection{Coherent States and Completeness Relation} % (fold)
\label{ssub:coherent_states_and_completeness_relation}

\begin{termdef}{Coherent States}
    The Coherent States of a fermionic oscillator is of the form
    \[ \ket{\theta} = \ket{0} + \ket{1}\theta,\quad \bra{\theta} = \bra{0} + \theta^*\bra{1}, \]
    where $\theta$ and $\theta^*$ are Grassmann numbers.
\end{termdef}
\begin{sample}
    \begin{example}
        With $\ket{f} = \ket{0}f_0 + \ket{1}f_1$, we have $\displaystyle \bra{\theta}c\ket{f} = \bra{\theta}\ket{0}f_1 = \+D{\theta^*}D{}\bra{\theta}\ket{f}$.
    \end{example}
\end{sample}
\begin{finaleq}{Completeness Relation of a Fermionic Oscillator}
    \begin{equation}
        \label{eq:completeness_relation_of_a_fermionic_oscillator}
        \int \rd{\theta^*}\,\rd{\theta}\, \ket{\theta}\bra{\theta}e^{-\theta^*\theta} = \mathbbm{1}.
    \end{equation}
\end{finaleq}

% subsubsection coherent_states_and_completeness_relation (end)

\subsubsection{Parition function of a Fermionic Oscillator} % (fold)
\label{ssub:parition_function_of_a_fermionic_oscillator}

The partition function is given by
\[ Z\pare{\beta} = \trace e^{-\beta H} = \sum_{n=0}^1 \bra{n}e^{-\beta H}\ket{n} = 2\cosh\pare{\beta \omega/2}, \]
where $\ket{n}$ denotes an eigenvector of $N$.
\begin{lemma}
    Let $H$ be the Hamiltonian of a fermionic harmonic oscillator, then the partition function may be written as
    \[ Z\pare{\beta} = \trace e^{-\beta H} = \int \rd{\theta^*}\,\rd{\theta}\,\bra{-\theta}e^{-\beta H}\ket{\theta} e^{-\theta^*\theta}. \]
\end{lemma}

\paragraph{Method 1} % (fold)
\label{par:method_1}

With the expansion $\displaystyle e^{-\beta H} = \lim_{N\rightarrow \infty} \pare{1-\beta H/N}^N$ and inserting the completeness relation \eqref{eq:completeness_relation_of_a_fermionic_oscillator} at each step we find
\begin{align*}
    Z\pare{\beta} &= \lim_{N\rightarrow \infty}\int \prod_{k=1}^N \rd{\theta_k^*}\,\rd{\theta_k}\, e^{-\sum_{n=1}^Nb} \\
    &\phantom{=\ } \times \bra{\theta_N}\pare{1-\epsilon H}\ket{\theta_{N-1}}\bra{\theta_{N-1}}\cdots \ket{\theta_1}\bra{\theta_1}\pare{1-\epsilon H}\ket{-\theta_N},
\end{align*}
where $\epsilon = \beta/N$ and $\theta = -\theta_N = \theta_0$, $\theta^* = -\theta^*_N = \theta^*_0$. Each matrix element is evaluated as
\[ \bra{\theta_k}\pare{1-\epsilon H}\ket{\theta_{k-1}} = e^{\epsilon \omega/2}e^{\pare{1-\epsilon\omega}\theta_k^*\theta_{k-1}}. \]
The partition function is now expressed as
\begin{equation}
    \label{eq:gaussian_integral_form_of_partition_function_of_fermionic_oscillator}
    Z\pare{\beta} = e^{\beta\omega/2}\lim_{N\rightarrow \infty} \prod_{k=1}^N \int \rd{\theta_k^*}\,\rd{\theta_k}\, e^{-\theta^\dagger\cdot B\theta},
\end{equation}
where
\[ \theta = \begin{pmatrix}
    \theta_1 \\ \theta_2 \\ \vdots \\ \theta_N
\end{pmatrix},\quad B_N = \begin{pmatrix}
    1 & 0 & \cdots & 0 & -y \\
    y & 1 & 0 & \cdots & 0 \\
    0 & y & 1 & 0 & \cdots \\
    \vdots & & \ddots & & \vdots \\
    0 & 0 & \cdots & y & 1
\end{pmatrix},\quad y = -1+\epsilon\omega. \]
Therefore, with \eqref{eq:gaussian_integral_to_det} we find
\[ Z\pare{\beta} = e^{\beta\omega/2}\lim_{N\rightarrow \infty} \det B_N = e^{\beta\omega/2}\lim_{N\rightarrow \infty} \brac{1+\pare{1-\beta\omega/N}^N} = 2\cosh \frac{\beta \omega}{2}. \]

% paragraph method_1 (end)

\paragraph{Method 2} % (fold)
\label{par:method_2}

Starting from \eqref{eq:gaussian_integral_form_of_partition_function_of_fermionic_oscillator} and expanding the $\theta$ in the path integral into Fourier series, \marginnote{\inlinehardlink{\Cref{par:method_2_of_partition_function_of_harmonic_oscillator}}}
we find
\begin{align*}
    Z\pare{\theta} &= e^{\beta\omega/2}\lim_{N\rightarrow \infty}\prod_{k=1}^N \int \rd{\theta_k^*}\,\rd{\theta_k} e^{-\sum_n\brac{\pare{1-\epsilon\omega}\theta_n^*\pare{\theta_n - \theta_{n-1}}/\epsilon + \omega\theta_n^*\theta_n}} \\
    &= e^{\beta\omega/2}\int \+cD\theta^*\,\+cD\theta \exp\brac{-\int_0^\beta \rd{\tau}\,\theta^*\pare{\pare{1-\epsilon\omega}\+d\tau d{} + \omega}\theta} \\
    &= e^{\beta\omega/2}{\det}\+_APBC_\pare{\pare{1-\epsilon\omega}\+d\tau d{} + \omega},
\end{align*}
where the subscript APBC implies that the eigenvalues should be evaluated for the solutions subjected to the boundary condition $\theta\pare{\theta} = -\theta\pare{0}$.
\par
Note that the coherent states are overcomplete and the actual number of freedom is $N$ since $\epsilon = \beta/N$, and the product should be truncated with only $N/2$ eigenvalues left since each complex variable has two real degrees of freedom. Therefore,
\begin{align*}
    Z\pare{\beta} &= e^{\beta\omega/2}\prod_{n=-N/4}^{N/4}\brac{i\pare{1-\epsilon\omega} \frac{\pi\pare{2n-1}}{\beta} + \omega} \\
    &= \underbrace{\prod_{k=1}^\infty \brac{\frac{\pi\pare{2k-1}}{\beta}}^2}_P \prod_{n=1}^\infty \brac{1+\pare{\frac{\beta\omega}{\pi\pare{2n-1}}}^2}.
\end{align*}
The first infinite product is divergent but may be assigned a value via the \textbf{Hurwitz $\zeta$-function}\jgloss{Hurwitz Zeta-Function}
\[ \zeta\pare{s,a} = \sum_{k=1}^\infty \rec{\pare{k+a}^s},\quad 0<a<1. \]
We obtain\begin{margintips}
    $\begin{array}{@{}l}
        \displaystyle \prod_{n=1}^\infty \pare{1+\frac{x^2}{n^2}}\\
        \displaystyle  = \frac{\sinh \pi x}{\pi x}.
    \end{array}$
\end{margintips}
\[ P = e^{-2\zeta'\pare{s,1/2}\vert_{s=0}} = 2. \]
Finally we have
\[ Z\pare{\beta} = 2\prod_{n=1}^\infty \brac{1+\pare{\frac{\beta\omega}{\pi\pare{2n-1}}}^2} = 2\cosh \frac{\beta\omega}{2}. \]

% paragraph method_2 (end)

\begin{finaleq}{Partition Function of a Fermionic Oscillator}
    \[ Z\pare{\beta} = 2\cosh\pare{\beta\omega/2}. \]
\end{finaleq}

% subsubsection parition_function_of_a_fermionic_oscillator (end)

% subsection path_integral_quantization_of_a_fermi_particle (end)

\subsection{Quantization of a Scalar Field} % (fold)
\label{sub:quantization_of_a_scalar_field}

\subsubsection{Free Scalar Field} % (fold)
\label{ssub:free_scalar_field}

\paragraph{Real Scalar Field} % (fold)
\label{par:real_scalar_field}

Let $\phi\pare{x^0,\+vx}$ denoting a scalar field, and the action
\[ S = \int \rd{x}\, \+cL\pare{\phi,\partial_\mu \phi}, \]
where $\+cL$ is the Lagrangian density.
\begin{finaleq}{Euler-Lagrange Equation}
    \[ \+D{x^\mu}D{}\pare{\+D{\pare{\partial_\mu \phi}}D{\+cL}} - \+D\phi D{\+cL} = 0. \]
\end{finaleq}
\jgloss{Klein-Gordon Equation}
\begin{sample}
    \begin{example}
        For $\displaystyle \+cL_0\pare{\phi,\partial_\mu \phi} = -\half\pare{\partial_\mu \partial^\mu \phi + m^2\phi^2}$, the Euler-Lagrange equation is exactly the Klein-Gordon equation
        \[ \pare{\Box^2 - m^2}\phi = 0 \]
        where $\Box^2 = -\partial_0^2 + \laplacian$.
    \end{example}
\end{sample}
The vacuum-to-vacuum amplitude in the presence of a source $J$ is
\begin{align*}
    Z_0\brac{J} &= \int \+cD\phi\,\exp\brac{i\int \rd{x}\,\pare{\+cL_0 + J\phi + \frac{i}{2}\epsilon \phi^2}} \\
    &= \int \+cD\phi\,\exp\brac{i\int \rd{x}\, \pare{\half \curb{\phi\pare{\Box^2 - m^2}\phi + i\epsilon \phi^2} + J\phi}}.
\end{align*}
The integral may be written in terms of the Feynman propagator.
\begin{finaleq}{Feynman Propagator}
    The Feynman propagator
    \[ \Delta\pare{x-y} = \frac{-1}{\pare{2\pi}^d}\int \rd{^d k}\, \frac{e^{ik\pare{x-y}}}{k^2 + m^2 - i\epsilon} \]
    is the Green's function of the inhomogeneous Klein-Gordon equation
    \[ \pare{\Box^2 - m^2 + i\epsilon}\phi_c = -J, \]
    i.e. $\phi_c\pare{x} = -\Delta * J$.
\end{finaleq}
\begin{remark}
    The Feynman propagator may be obtained by
    \[ \Delta\pare{x-y} = \frac{i}{Z_0\brac{0}}\left. \frac{\delta^2 Z_0\brac{J}}{\delta J\pare{x}\delta J\pare{y}} \right\vert_{J=0}. \]
\end{remark}
\begin{finaleq}{Generating Functional of a Scalar Field}
    \[ Z_0\brac{J} = Z_0\brac{0} \exp\brac{-\frac{i}{2}\int \rd{x}\,\rd{y}\, J\pare{x}\Delta\pare{x-y}J\pare{y}} \]
\end{finaleq}
Now we evaluate $Z_0\brac{J}$. With the imaginary time $x^4 = \tau = ix^0$,
\begin{align*}
    Z_0\brac{0} &= \int \conj{\+cD}\phi\,\exp\brac{\half \int \rd{x}\,\phi\pare{\conj{\Box}^2 - m^2}\phi} \\
    &= \brac{\det\pare{\conj{\Box}^2 - m^2}}^{-1/2},
\end{align*}
where $\conj{\Box}^2 = \partial_\tau^2 + \laplacian$.

% paragraph real_scalar_field (end)

\paragraph{Complex Scalar Field} % (fold)
\label{par:complex_scalar_field}

The Lagrangian density is now
\[ \+cL_0 = -\partial_\mu \phi^* \partial^\mu \phi - m^2\phi^*\phi + J\phi^* + J^*\phi, \]
the propagator
\[ \Delta\pare{x-y} = \frac{i}{Z_0\brac{0,0}}\left.\frac{\delta^2 Z_0\brac{J,J^*}}{\delta J^*\pare{x}\delta J\pare{y}}\right\vert_{J=J^* = 0}, \]
the generating funtional
\begin{align*}
    Z_0\brac{J,J^*} &= \int \+cD \phi \,\+cD \phi^*\, \exp\brac{i\int \rd{x}\,\pare{\+cL_0 + i\epsilon \phi^*\phi}} \\
    &= \int \+cD\phi\,\+cD\phi^*\, \exp\brac{i\int \rd{x}\,\curb{\phi^*\pare{\Box^2 - m^2 + i\epsilon}\phi + J^*\phi + J\phi^*}} \\
    &= Z_0\brac{0,0}\exp\brac{-i\int \rd{x}\,\rd{y}\, J^*\pare{x}\Delta\pare{x-y}J\pare{y}},
\end{align*}
where
\begin{align*}
    Z_0\brac{0,0} &= \int \+cD\phi \,\+cD\phi^* \,\exp\brac{i\int \rd{x}\, \phi^* \pare{\Box^2 - m^2 + i\epsilon}\phi} \\
    &= \brac{\det \pare{\conj{\Box}^2 - m^2}}^{-1}.
\end{align*}

% paragraph complex_scalar_field (end)

% subsubsection free_scalar_field (end)

\subsubsection{Interacting Scalar Field} % (fold)
\label{ssub:interacting_scalar_field}

Adding an interaction term to the free field Lagrangian\marginnote{\inlinehardlink{cf. \S 9.2, Peskin.}}
\[ \+cL\pare{\phi,\partial_\mu \phi} = \+cL_0\pare{\phi,\partial_\mu \phi} - V\pare{\phi},\quad V\pare{\phi} = \frac{g}{n!}\phi^n,\quad n\ge 3. \]
Treating $V$ as a perturbation, we may expand the $Z\brac{J}$ into Dyson series
\begin{align*}
    Z\brac{J} &= \int \+cD \phi \,\exp\brac{i\int \rd{x}\,\curb{\half \phi\pare{\Box^2 - m^2}\phi - V\pare{\phi} + J\phi}} \\
    &= \sum_{k=0}^\infty \int \rd{x_1}\,\cdots \,\int \rd{x_k}\, \frac{\pare{-i}^k}{k!} V\pare{\rec{i}\frac{\delta}{\delta J\pare{x_1}}}\cdots V\pare{\rec{i}\frac{\delta}{\delta J\pare{x_k}}}Z_0\brac{J},
\end{align*}
where
\[ Z_0\brac{J} = \int\+cD\phi\,\exp\brac{i\int \rd{x}\,\curb{\+cL_0 + J\phi}}. \]
The vacuum expectation value of the $T$-product of field operators is given by
\[ G_n\pare{x_1,\cdots,x_n} = \bra{0}T\brac{\phi\pare{x_1}\cdots\phi\pare{x_n}}\ket{0} = \left. \frac{\pare{-i}^n \delta^n}{\delta J\pare{x_1}\cdots \delta J\pare{x_n}}Z\brac{J} \right\vert_{J=0}, \]
also known as the \gloss{Green's function}.
\par
The functional Taylor expansion of $Z\brac{J}$ is
\[ Z\brac{J} = \sum_{n=1}^\infty \rec{n!}\brac{\prod_{i=1}^n \int \rd{x_i}\, J\pare{x_i}}\bra{0}T\brac{\phi\pare{x_1}\cdots \phi\pare{x_n}}\ket{0} = \bra{0}T\exp\curb{\int \rd{x}\,J\pare{x}\phi\pare{x}}\ket{0}. \]
\vspace{-\baselineskip}
\begin{termdef}{Effective Action}
    The effective action $\Gamma\brac{\phi\+_cl_}$ is defined by the Legendre transformation
    \[ \Gamma\brac{\phi\+_cl_} = W\brac{J} - \int \rd{\tau}\,\rd{\+vx}\,J\phi\+_cl_, \]
    where
    \[ \phi\+_cl_ = \expc{\phi}^J = \frac{\delta W\brac{J}}{\delta J}, \]
    and $W\brac{J}$ is defined by $Z\brac{J} = e^{-W\brac{J}}$.
\end{termdef}
The functional generates \gloss{one-particle irreducible} diagrams.

% subsubsection interacting_scalar_field (end)

% subsection quantization_of_a_scalar_field (end)

\subsection{Quantization of a Dirac Field} % (fold)
\label{sub:quantization_of_a_dirac_field}

\begin{margindef}[1.5\baselineskip]{Feynman Slash}
    $\slashed{A} = \gamma^\mu A_\mu$.
\end{margindef}
The Lagrangian of the free \gloss{Dirac field} $\psi$ is
\[ \+cL_0 = \conj{\psi}\pare{i\slashed{\partial} - m}\psi. \]
Varitaion with respect to $\conj{\psi}$ yields the Dirac equation.
\begin{finaleq}{Dirac Equation}
    \[ \pare{i\slashed{\partial} - m}\psi = 0. \]
\end{finaleq}
The Dirac field satisfies the anti-commutation relation
\[ \curb{\conj{\psi}\pare{x^0,\+vx}, \psi\pare{x^0,\+vy}} = \delta\pare{\+vx-\+vy}. \]
The generating functional is
\[ Z_0\brac{\conj{\eta},\eta} = \int \+cD \conj{\psi}\, \+cD\psi \, \exp\brac{i\int \rd{x}\, \pare{ \conj{\psi}\pare{i\slashed{\partial} - m}\psi + \conj{\psi}\eta + \conj{\eta}\psi} } = Z_0\brac{0,0}\exp\brac{-i\int \rd{x}\,\rd{y}\, \conj{\eta}\pare{x}S\pare{x-y}\eta\pare{y}}, \]
where $\eta$ and $\conj{\eta}$ are Grassmannian sources, and
\begin{align*}
    S\pare{x-y} &= -\frac{\delta^2 Z_0\brac{\conj{\eta},\eta}}{\delta\conj{\eta}\pare{x}\delta\eta\pare{y}} = \rec{\pare{2\pi}^d}\int \rd{^d k} \frac{e^{ik\pare{x-y}}}{\slashed{k}-m-i\epsilon} = \pare{i\slashed{\partial} + m + i\epsilon}\Delta\pare{x-y}
\end{align*}
is the propagator, where $\Delta\pare{x-y}$ is the scalar field propagator, and the normalization factor is obtained as
\[ Z_0\brac{0,0} = \det\pare{i\slashed{\partial} - m} = \prod_i \lambda_i, \]
where $\lambda_i$ is the $i$th eigenvalue of the Dirac operator.

% subsection quantization_of_a_dirac_field (end)

\subsection{Gauge Theories} % (fold)
\label{sub:gauge_theories}

\subsubsection{Abelian Gauge Theories} % (fold)
\label{ssub:abelian_gauge_theories}

Maxwell's equations are invariant under the \gloss{gauge transformation}
\[ A_\mu \rightarrow A_\mu + \partial_\mu \chi. \]
The Lagrangian
\[ \+cL = \conj{\psi}\brac{i\gamma^\mu\pare{\partial_\mu - ieA_\mu} + m}\psi \]
is invariant under
\[ \psi \rightarrow e^{-ie\alpha\pare{x}}\psi,\quad \conj{\psi} \rightarrow \conj{\psi}e^{ie\alpha\pare{x}}, \]
whence we have
\[ \conj{\psi}\pare{i\gamma^\mu\partial_\mu + m}\psi \rightarrow \conj{\psi}\pare{i\gamma^\mu\partial_\mu + e\gamma^\mu \partial_\mu \alpha + m}\psi, \]
and
\[ A_\mu \rightarrow A'_\mu = A_\mu - \partial_\mu \alpha\pare{x}. \]
After introducing the \gloss{covariant derivatives}
\[ \nabla_\mu = \partial_\mu - ieA_\mu,\quad \partial'_\mu = \partial_\mu - ieA'_\mu, \]
we have
\[ \nabla'_\mu \psi' = e^{-ie\alpha\pare{x}}\nabla_\mu \psi. \]
\vspace{-\baselineskip}
\begin{finaleq}{QED Lagrangian}
    The total quantum electrodynamic Lagrangian is
    \[ \+cL\+_QED_ = -\rec{4}F^{\mu\nu}F_{\mu\nu} + \conj{\psi}\pare{i\gamma^\mu \nabla_\mu + m}\psi. \]
\end{finaleq}

% subsubsection abelian_gauge_theories (end)

\subsubsection{Non-Abelian Gauge Theories} % (fold)
\label{ssub:non_abelian_gauge_theories}

\begin{termdef}{Structure Constants}
    The coefficients $\tensor{f}{_\alpha_\beta^\gamma}$ in the commutation relations of anti-Hermitian generators $\curb{T_\alpha}$
    \[ \brac{T_\alpha,T_\beta} = \tensor{f}{_\alpha_\beta^\gamma}T_\gamma. \]
\end{termdef}
An element $U$ of a Lie group $G$ near the unit element can be expressed as
\[ U = \exp{-\theta^\alpha T_\alpha}. \]
We suppose a Dirac field $\psi$ transforms under $U\in G$ as
\[ \psi \rightarrow U\psi,\quad \conj{\psi} \rightarrow \conj{\psi}U^\dagger. \]
Consider the Lagrangian
\[ \+cL = \conj{\psi}\brac{i\gamma^\mu\pare{\partial_\mu + g\+cA_\mu} + m}\psi, \]
where the \gloss[-\baselineskip]{Yang-Mills gauge field} $\+cA_\mu$ takes its values in the Lie algebra of $G$, i.e. $\+cA_\mu = \tensor{A}{_\mu^\alpha}T_\alpha$.
\begin{finaleq}{Gauge Invariance of Yang-Mills Lagrangian}
    The $\+cL$ defined above is invariant under
    \begin{align*}
        & \psi \rightarrow \psi' = U\psi,\quad \conj{\psi} \rightarrow \conj{\psi}' = \conj{\psi}U^\dagger, \\
        & \+cA_\mu \rightarrow \+cA'_\mu = U\+cA_\mu U^\dagger + g^{-1}U\partial_\mu U^\dagger.
    \end{align*}
    The covariant derivative defined by $\nabla_\mu = \partial_\mu + g\+cA_\mu$ transforms covariantly under the gauge transformation,
    \[ \nabla'_\mu \psi' = U\nabla_\mu \psi. \]
\end{finaleq}
\begin{termdef}{Yang-Mills Field Tensor}
    \[ \+cF_{\mu\nu} = \partial_\mu \+cA_\nu - \partial_\nu \+cA_\mu + g\brac{\+cA_\mu,\+cA_\nu}. \]
\end{termdef}
The Yang-Mills field tensor has components \begin{margindef}{Dual Field Tensor}
    $\displaystyle *\+cF_{\mu\nu} = \half \epsilon_{\mu\nu\kappa\lambda}\+cF^{\kappa\lambda}$.
\end{margindef}
\[ \tensor{F}{_\mu_\nu^\alpha} = \partial_\mu \tensor{A}{_\nu^\alpha} - \partial_\nu \tensor{A}{_\mu^\alpha} + g\tensor{f}{_\beta_\gamma^\alpha}\tensor{A}{_\mu^\beta}\tensor{A}{_\nu^\gamma}. \]
\vspace{-\baselineskip}
\begin{finaleq}{Bianchi Identity}
    \[ \+cD_\mu * \+cF^{\mu\nu} = \partial_\nu * \+cF^{\mu\nu} + g\brac{\+cA_\mu,*\+cF^{\mu\nu}} = 0. \]
\end{finaleq}
It can be shown that $\+cF_{\mu\nu}$ transforms as \begin{marginwarns}
    \raggedright
    Calligraphic font stands for matrix here.
\end{marginwarns}
\[ \+cF_{\mu\nu} \rightarrow U\+cF_{\mu\nu} U^\dagger, \]
from which we find a gauge-invariant action
\[ \+cL\+_YM_ = -\half \trace{\+cF^{\mu\nu}\+cF_{\mu\nu}}, \]
or, in component form,
\[ \+cL\+_YM_ = -\half \tensor{F}{^\mu^\nu^\alpha}\tensor{F}{_\mu_\nu^\beta}\trace\pare{T_\alpha T_\beta} = \rec{4}\tensor{F}{^\mu^\nu^\alpha}\tensor{F}{_\mu_\nu_\alpha}, \]
where $\curb{T_\alpha}$ is normalized such that $\displaystyle \trace\pare{T_\alpha T_\beta} = -\half \delta_{\alpha\beta}$. The field equation is
\[ \+cD_\mu \+cF_{\mu\nu} = \partial_\mu \+cF_{\mu\nu} + g\brac{\+cA_\mu,\+cF_{\mu\nu}} = 0. \]

% subsubsection non_abelian_gauge_theories (end)

\subsubsection{Higgs Fields} % (fold)
\label{ssub:higgs_fields}

For a complex scalar field $\phi$ whose Lagrangian is given by
\[ \+sL = -\rec{4}F^{\mu\nu}F_{\mu\nu} + \pare{\nabla_\mu \phi}^\dagger \pare{\nabla_\mu \phi} - \lambda\pare{\phi^\dagger\phi - v^2}^2. \]
The Lagrangian in invariant under the local gauge transformation
\[ A_\mu \rightarrow A_\mu - \partial_\mu \alpha,\quad \phi \rightarrow e^{-ie\alpha}\phi,\quad \phi^\dagger \rightarrow e^{ie\alpha}\phi^\dagger. \]
Take the \gloss{unitary gauge} such that $\phi$ has only the real part and expand $\phi$ as
\[ \phi\pare{x} = \rec{\sqrt{2}}\pare{v+\rho\pare{x}}, \]
the equations of motion derived from the Lagrangian now become
\[ \partial^\nu F_{\nu\mu} + 2e^2v^2 A_\mu = 0,\quad \partial_\mu \partial^\mu \rho + 2\lambda v^2\rho = 0. \]
From the first equation we find $A^\mu$ must satisfy the Lorentz condition $\partial_\mu A^\mu = 0$. The degrees of freedom of the original Lagrangian is $2\text{(photon)}+2\text{(real scalar)} = 4$, and if the \gloss[\baselineskip]{vacuum expectation value} \begin{margindef}[-4\baselineskip]{VEV}
    Vacuum expectation value.
\end{margindef}
 of the \gloss{Higgs field} $\phi$ is not zero, we have $3\text{(massive vector)}+1\text{(real scalar)}=4$. The field $A_0$ has the wrong sign and so cannot be a physical degree of freedom. The creation of massive fields out of a gauge field is called the \gloss{Higgs mechanism}.

% subsubsection higgs_fields (end)

% subsection gauge_theories (end)

\subsection{Magnetic Monopoles} % (fold)
\label{sub:magnetic_monopoles}

\subsubsection{Dirac Monopole} % (fold)
\label{ssub:dirac_monopole}

The vector potential of a magnetic monopole is
\[ \+vA^{\mathrm{N}}\pare{\+vr} = \frac{g\pare{1-\cos\theta}}{r\sin\theta}\+u\phi,\quad \+vA^{\mathrm{S}}\pare{\+vr} = -\frac{g\pare{1+\cos\theta}}{r\sin\theta}\+u\phi. \]
The two expressions break down either on the positive $z$-axis or the negative one. The singularity is called the \gloss{Dirac string}.

% subsubsection dirac_monopole (end)

\subsubsection{The Wu-Yang Monopole} % (fold)
\label{ssub:the_wu_yang_monopole}

The vector potentials on the northern and southern hemispheres are related by the gauge transformation
\[ \+vA^{\mathrm{N}} - \+vA^{\mathrm{S}} = \nabla\pare{2g\phi}, \]
from which we deduce the total flux
\[ \Phi = \oint_S \curl \+vA\cdot \rd{\+vS} = \oint_{\mathrm{equator}} \nabla\pare{2g\phi}\cdot \rd{\+vs} = 4g\pi. \]

% subsubsection the_wu_yang_monopole (end)

\subsubsection{Charge Quantization} % (fold)
\label{ssub:charge_quantization}

A point particle with electric charge $e$ and mass $m$ moving in the field of a magnetic monopole of charge $g$ is described by the Schr\"odinger equation
\[ \rec{2m}\pare{\+vp - \frac{e}{c}\+vA}^2 \psi\pare{\+vr} = E\psi\pare{\+vr}. \]
The values of the wave function on the northern and southern hemisphere are related by
\[ \psi^{\mathrm{S}}\pare{\+vr} = \exp\pare{-\frac{ie\Lambda}{\hbar c}}\psi^{\mathrm{N}}\pare{\+vr}. \]
On the equator, we found the \gloss{Dirac quantization condition}
\[ \frac{2eg}{\hbar c} = n. \]

% subsubsection charge_quantization (end)

% subsection magnetic_monopoles (end)

\subsection{Instantons} % (fold)
\label{sub:instantons}

The vacuum-to-vacuum amplitude in the Euclidean theory is \marginnote{\inlinehardlink{ \Cref{ssub:time_ordered_product_and_generating_functional} }}
\[ Z = \bra{0}\ket{0} \propto \int \+cD\phi\, e^{-S\brac{\phi,\partial_\mu \phi}}. \]
The local minima of $S\brac{\phi,\partial_\mu \phi}$ in non-Abelian gauge theories are called \gloss{instantons}.

\subsubsection{The SU(2) Case} % (fold)
\label{ssub:the_su}

Consider the $\mathrm{SU}\pare{2}$ gauge theory defined in four-dimensional Euclidean space $\+bR^4$, with
\[ \+cA_\mu = \tensor{A}{_\mu^\alpha}\frac{\sigma_\alpha}{2i},\quad \+cF_{\mu\nu} = \tensor{F}{_\mu_\nu^\alpha}\frac{\sigma_\alpha}{2i}, \]
The field equation in \cref{ssub:non_abelian_gauge_theories} becomes
\[ \+cD_\mu \+cF_{\mu\nu} = \partial_\mu \+cF_{\mu\nu} + g\brac{\+cA_\mu,\+cF_{\mu\nu}} = 0. \]
Suppose $\+cA_\mu$ satisfies
\[ \+cA_\mu \rightarrow U\pare{x}^{\dagger}\partial_\mu U\pare{x},\quad \abs{x}\rightarrow \infty, \]
where $U\in \mathrm{SU}\pare{2}$, it can be shown that $\+cF_{\mu\nu}$ vanishes for this asymptotic part. We require that on $S^3$ of large radius, the gauge potential be given by the above expression.

% subsubsection the_su (end)

\subsubsection{The (Anti-)Self-Dual Solution} % (fold)
\label{ssub:the_anti_self_dual_solution}

In the inequality
\[ \int \rd{^4 x}\,\trace\pare{\+cF_{\mu\nu} \pm * \+cF_{\mu\nu}}^2 \ge 0, \]
$\+cF$ is \gloss{self-dual} (or \gloss[\baselineskip]{anti-self-dual}) if the positive (or negative) sign in chosen. In either case, the field equation is automatically satisfied, which can be seen from the Bianchi identity. With \begin{margintips}
    $\displaystyle \begin{array}{@{}l}
        *\+cF_{\mu\nu}*\+cF^{\mu\nu}\\ = \+cF_{\mu\nu}\+cF^{\mu\nu}
    \end{array}$
\end{margintips}
\[ Q = -\rec{16\pi^2}\int \rd{^4 x}\,\trace{\+cF_{\mu\nu} * \+cF^{\mu\nu}} \]
being an integer characterizing the way $S^3$ is mapped to $\mathrm{SU}\pare{2}$, we find from the above inequality that
\[ S\ge 8\pi^2 \abs{Q}. \]
The inequality is saturated in self-dual and anti-self-dual cases. Looking for solutions of the form
\[ \+cA_\mu = f\pare{r}U\pare{x}^{-1}\partial_\mu U\pare{x} \]
where $r = \abs{x}$ and
\[ f\pare{r} \rightarrow 1,\quad r\rightarrow \infty,\quad U\pare{x} = \rec{r}\pare{x_4 + ix_i\sigma_i}, \]
we find
\[ \+cA_\mu\pare{x} = \frac{r^2}{r^2+\lambda^2}U\pare{x}^{-1}\partial_\mu U\pare{x},\quad \+cF_{\mu\nu}\pare{x} = \frac{4\lambda^2}{r^2+\lambda^2}\sigma_{\mu\nu}, \]
where
\[ \sigma_{ij} = \rec{4i}\brac{\sigma_i,\sigma_j},\quad \sigma_{i0} = \half\sigma_i = -\sigma_{0i}. \]
This solution gives $Q=+1$ and $S=8\pi^2$.

% subsubsection the_anti_self_dual_solution (end)

% subsection instantons (end)

% section quantum_physics (end)

\end{document}
