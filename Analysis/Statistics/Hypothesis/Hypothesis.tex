\documentclass[../Statistics.tex]{subfiles}

\begin{document}

\section{假设检验} % (fold)
\label{sec:假设检验}

\subsection{基本概念} % (fold)
\label{sub:基本概念}

假设谓关于总体分布的概率性质的假定.

\subsubsection{基本概念} % (fold)
\label{ssub:基本概念}

假设检验的问题是研究如何根据抽样后的样本来检验抽样前的假设是否合理.
\begin{sample}
    \begin{ex}
        某厂产品出厂检验规定, 每批产品次品率$p$不超过$4\%$才能出厂. 现在从某批产品$10000$件中任意抽查$12$件发现$4$件次品, 问该批产品能否出厂? 若结果是$1$件呢?
    \end{ex}
    \begin{solution}
        设$p$表示次品率, 假设$p\le 4\%$, 并记$Y$为$12$件中的次品数. 由于总产品数很大, 可认为$Y\sim B\pare{12,p}$. 当$p\le 0.04$,
        \[ P\pare{Y=4} = \binom{12}{4} p^4\pare{1-q}^8 < 0.000913. \]
        从而$p\le 0.04$, 抽样发生这一结果的概率过低, 故可怀疑假设$p\le 0.04$的正确性. 而
        \[ P\pare{Y=1} \binom{12}{1}p^1\pare{1-p}^11 = 0.306, \]
        从而$Y=1$并非小概率事件, 没有足够证据表明原假设不成立.
    \end{solution}
\end{sample}
\begin{sample}
    \begin{ex}
        某饮料厂在自动流水线上罐装饮料. 正常情况下, 每瓶饮料的容量$X$服从$N\pare{500,10^2}$. 抽取$9$件样品, 平均值$\overbar{X} = 492$, 问每瓶饮料的容量仍是$500$还是已经减小至$490$? 假设标准差仍为$10$.
    \end{ex}
\end{sample}
统计假设为罐装饮料容量$X\sim N\pare{\mu,10^2}$. 问题为根据样本在$\mu=500$和$\mu=490$之间作判断. 前者谓原假设$H_0$, 后者谓备择假设或对立假设$H_1$或$H_a$. 检验就是要在
\[ H_0: \mu=500 \leftrightarrow H_1: \mu = 490 \]
中选择何者成立. 断言「$H_0$成立」则谓「不能拒绝$H_0$」. 断言「$H_0$不能成立」则谓「拒绝$H_0$」.
\par
为了检验假设, 需要通过样本$X_1,\cdots,X_n$构造极大似然估计$T=\overbar{X}$来估计$\mu$. 当$T$绝对值和$\mu$偏差过大时将不利于$H_0$, 从而应拒绝$H_0$. 实现取定常数$\tau$, 谓临界值, 当
\[ W = \curb{\overbar{X} < \tau} \]
时拒绝$H_0$, 谓$W$为拒绝域. 即样本取值落在$W$中就拒绝$H_0$, 反之不能拒绝.
\par
\begin{cenum}
    \item 谓「实际上$H_0$成立但是被拒绝」为第I类错误(弃真);
    \item 谓「实际上$H_0$不成立但是没有被拒绝」为第II类错误(存伪).
\end{cenum}
要让第I类错误概率小, 应该让$\tau$小; 要让第II类错误小, 则不能让$\tau$太小. 实用中应在控制第I类错误的基础上, 减少犯第II类错误的概率. 这种只限制第I类错误的原则下的检验方法, 谓显著性检验(Significance Test).
\par
给定一个允许犯第一类错误概率最大值的$\alpha$, 选择$\tau$使得$P_{H_0}\pare{T<\tau}\le \alpha$连续场合下, 等号可以达到. 这样的$\tau$可以通过在$T$在$H_0$下的分布及上式条件求得. $\alpha$谓显著性水平, 通常取$0.1,0.05,0.01$等.
\par
将问题一般化:
\begin{cenum}
    \item 提出假设检验问题
    \[ H_0:\theta \in \Theta_0 \leftrightarrow H_1: \theta\in\Theta_1. \]
    其中$H_0$谓零假设, $H_1$谓对立假设或或备择假设.
    \item 根据参数估计的方法构造适当的\emph{检验统计量}$T = T\pare{X_1,\cdots,X_n}$, 其中$X_1,\cdots,X_n$是一个样本.
    \item 根据对立假设的形状构造一检验的拒绝域$W = \curb{T\pare{X_1,\cdots,X_n}\in A}$, 其中$A$为一个集合, 通常是一个区间. 例如拒绝域可以取为
    \[ T\pare{X_1,\cdots,X_n}>\tau, \]
    谓$\tau$临界值.
    \item 对任意$\theta\in\Theta_0$, 犯第I类错误的概率
    \[ P_\theta\pare{T\pare{X_1,\cdots,X_n}\in A} \le \alpha, \]
    则谓$\alpha$显著性水平.
    \item 结合$T$在$H_0$下的分布定出$A$.
\end{cenum}
谓$\beta\pare{\theta} = P_\theta\pare{\text{$H_0$被拒绝}}$为检验的功效函数. 如果检验的显著性水平为$\alpha$, 则$\beta\pare{\theta}<\alpha$. 当$\theta\in\Theta_1$时, 功效值越大越好, 可以作为评价一个检验优劣的准则.

% subsubsection 基本概念 (end)

\subsubsection{原假设的提法} % (fold)
\label{ssub:原假设的提法}

将受保护的对象置为零假设. 这包括已存在的事实, 或者错误拒绝会带来严重后果的情形.
\begin{ex}
    无罪推定, 批准新药等. 后者零假设为「新药不比安慰剂效果好」.
\end{ex}
\par
如果希望「证明」某命题, 则取相反解决或者其中一部分作为零假设.
\par
假设检验的「拒绝零假设」结果比「不能拒绝零假设」更有保证.

% subsubsection 原假设的提法 (end)

\subsubsection{检验统计量的选取及假设检验的步骤} % (fold)
\label{ssub:检验统计量的选取及假设检验的步骤}

\begin{sample}
    \begin{ex}
        能否在显著性水平$0.05$下认为饮料的平均容量确实减少到$490$毫升?
    \end{ex}
    \begin{solution}
        根据问题, 考虑假设
        \[ H_0: \mu = 500 \leftrightarrow H_1: \mu = 490. \]
        由于$\overbar{X}$为$\mu$的无偏估计, 则合理的检验准则为$\overbar{X}<\tau$时$RH_0$, 否则不能$RH_0$. 第一类错误的概率为
        \begin{align*}
            & P\pare{\overbar{X}<\mu \vert \mu = 500} \le \alpha \\
            &\Leftrightarrow P_{\mu = 500} \pare{\frac{\overbar{X}-500}{10} < 3\cdot \frac{\tau - 500}{10}} \le \alpha \\
            &\Leftrightarrow 3\cdot \frac{\tau - 500}{10} \le -u_\alpha.
        \end{align*}
        取显著性水平为$0.05$, 则$u_{0.05}\approx 1.645$, $\overbar{X} = 492$, $n=9$, $T_1 = -2.4 < -1.645$, 故样本落在拒绝域中, 可以拒绝零假设.
    \end{solution}
\end{sample}
几种常见的假设检验问题如
\begin{cenum}
    \item $H_0: \theta = \theta_0 \leftrightarrow H_1: \theta = \theta_1$;
    \item $H_0: \theta = \theta_0 \leftrightarrow H_1: \theta \neq \theta_0$;
    \item $H_0: \theta = \theta_0 \leftrightarrow H_1: \theta>\theta_0$, 或者$H_0: \theta \le \theta_0 \leftrightarrow H_1: \theta>\theta_0$.
    \item $H_0: \theta = \theta_0 \leftrightarrow H_1: \theta<\theta_0$, 或者$H_0: \theta \ge \theta_0 \leftrightarrow H_1: \theta<\theta_0$.
\end{cenum}
第一种为简单假设, 第二种为双侧假设, 第三和第四种为单侧假设.
\par
显著性检验的一般步骤为
\begin{cenum}
    \item 求出未知参数的一个较优的点估计$\hat\theta = \hat\theta\pare{X_1,\cdots,X_n}$, 例如极大似然估计;
    \item 以$\hat\theta$为基础, 寻找统计量
    \[ T = t\pare{X_1,\cdots,X_n} \]
    使得$\theta = t_0$时分布已知;
    \item 以$T$为基础, 根据$H_1$的实际意义, 寻找拒绝域;
    \item 当零假设成立时, 犯第I类错误的概率$\le \alpha$, 得到临界值;
    \item 给定样本观测值, 给出检验统计量的样本观测值, 若落在拒绝域中则可拒绝零假设.
\end{cenum}

% subsubsection 检验统计量的选取及假设检验的步骤 (end)

% subsection 基本概念 (end)

\subsection{一样本和两样本的总体参数检验} % (fold)
\label{sub:一样本和两样本的总体参数检验}

\subsubsection{一样本正态总体参数检验} % (fold)
\label{ssub:一样本正态总体参数检验}

设总体$X\sim N\pare{\mu,\sigma^2}, -\infty<\mu<\infty, \sigma^2>0$, $X_1,\cdots,X_n$是$X$的一个样本, 取显著性水平为$\alpha$, 可以考虑参数$\mu$和$\sigma^2$.

\paragraph{方差已知时的均值检验} % (fold)
\label{par:方差已知时的均值检验}

$X_1, \cdots, X_n$ i.i.d $\sim N\pare{\mu,\sigma^2}$, $\sigma^2$已知.
\[ H_0: \mu = \mu_0 \leftrightarrow H_1: \mu\neq \mu_0. \]
$\mu$的MLE为$\overbar{X}$, 且$X\sim N\pare{\mu, \displaystyle \frac{\sigma^2}{n}}$. 合理的检验为$\overbar{X} < c$或$\overbar{X}>d$时$RH_0$, 否则不能$RH_0$. 第I类错误的概率为
\begin{align*}
    &P_{\mu = \mu_0} \pare{\overbar{X}<c \ \text{或} \ \overbar{X}>d} \\
    &= P_{\mu = \mu_0}\pare{\overbar{X}<c} + P_{\mu = \mu_0}\pare{\overbar{X}>d} \\
    &= P_{\mu=\mu_0} P\pare{\sqrt{n} \frac{\overbar{X} - \mu_0}{\sigma} < \sqrt{n}\frac{c-\mu_0}{\sigma}} + P\pare{\sqrt{n} \frac{\overbar{X} - \mu_0}{\sigma} > \sqrt{n}\frac{d-\mu_0}{\sigma}} \\
    &= \Phi\pare{\sqrt{n} \frac{c-\mu_0}{\sigma}} + \brac{1-\Phi\pare{\sqrt{n}\frac{d-\mu_0}{\sigma}}}.
\end{align*}
使其$\le \alpha$, 取
\begin{align*}
    \Phi\pare{\sqrt{n} \frac{c-\mu_0}{\sigma}} &= \alpha/2,\quad 1-\Phi\pare{\sqrt{n} \frac{d-\mu_0}{\sigma}} = \alpha/2, \\ \Rightarrow c &= \mu_0 - \frac{\sigma}{\sqrt{n}}u_{\alpha/2},\quad d = \mu_0 + \frac{\sigma}{\sqrt{n}}u_{\alpha/2}. 
\end{align*}
其中
\[ Z = Z\pare{X_1,\cdots,X_n} = \sqrt{n} \frac{\overbar{X} - \mu_0}{\sigma}. \]
\begin{remark}
    这和置信区间估计的思路是类似的, 如果$p=0.95$确定$\mu$落在某个区间内, 而$\mu$未落在该区间内, 则可以在$\alpha=0.05$下拒绝之.
\end{remark}
\par
对于右侧假设,
\[ H_0: \mu = \mu_0\leftrightarrow H_1: \mu>\mu_0 \]
或者
\[ H_0: \mu \le \mu_0 \leftrightarrow H_1: \mu > \mu_0, \]
构造相同的$Z$, 则拒绝域为$Z>u_\alpha$. 例如
\begin{align*}
    &P_{\mu\le \mu_0}\pare{\overbar{X} > d} \\
    &= P_{\mu \le \mu_0} \pare{\sqrt{n}\frac{\overbar{X}-\mu}{\sigma} > \sqrt{n}\frac{d-\mu}{\sigma}}.
\end{align*}
欲使
\[ \mu\le \mu_0 \Rightarrow 1-\Phi\pare{\sqrt{n}\frac{d-\mu}{\sigma}} \le \alpha, \]
则由单调性,
\[ 1-\Phi\pare{\sqrt{n} \frac{d-\mu_0}{\sigma}} = \alpha \Rightarrow d = \mu_0 + \frac{\sigma}{\sqrt{n}}u_\alpha. \]
\begin{remark}
    显著性水平越小, 零假设被保护得越好从而更不容易被拒绝.
\end{remark}

% paragraph 方差已知时的均值检验 (end)

\begin{sample}
    \begin{ex}
        对正态总体$N\pare{\mu,\sigma^2}$(其中$\sigma^2$已知)下的假设检验问题
        \[ H_0: \mu \ge \mu_0 \leftrightarrow H_1: \mu<\mu_0, \]
        如果要求第II类错误的概率小于某$\beta > 0$, 应如何?
    \end{ex}
    \begin{solution}
        $\beta\pare{\theta} = P_\theta\pare{RH_0}$, 而
        \begin{align*}
            P\pare{\text{犯第I型错误}} &= \begin{cases}
                \beta\pare{\theta}, & H_0\text{为真},\\
                0, & H_0\text{为伪}.
            \end{cases} \\
            P\pare{\text{犯第II型错误}} &= \begin{cases}
                0, & H_1\text{为真}, \\
                1-\beta\pare{\theta}, & H_1\text{为伪}.
            \end{cases}
        \end{align*}
        第I类错误的约束要求$\beta\pare{\mu}\le \alpha$, 当$\mu\ge \mu_0$. 第II类错误的约束要求$1-\beta\pare{\mu} \ge \beta$, 当$\mu < \mu_0$. 由
        \begin{align*}
            \beta\pare{\mu} &= P_\mu \pare{T<-u_\alpha} \\
            &= P_\mu \pare{\sqrt{n} \frac{\pare{\overbar{X} - \mu_0}}{\sigma} < -u_\alpha} \\
            &= P_\mu \pare{\frac{\sqrt{n}\pare{\overbar{X} - \mu + \mu - \mu_0}}{\sigma} < -u_\alpha} \\
            &= \Phi\pare{-u_\alpha + \sqrt{n} \frac{\mu_0 - \mu}{\sigma}}.
        \end{align*}
        这满足第I类错误的约束, 但对于第II类错误,
        \[ \mu < \mu_0 \Rightarrow \Phi\pare{-u_\alpha + \sqrt{n}\frac{\mu_0 - \mu}{\sigma}} \ge 1-\beta. \]
        当$\mu < \mu_0$, 而$\mu$接近$\mu_0$时, $\beta\pare{\mu} \approx \alpha$. 而一般$\alpha < 1-\beta$, 故这一要求难以达到.
        \par
        实际中要求会放松一些, 要求对某个指定的$\mu_1 < \mu_0$, 有
        \[ \beta\pare{\mu} \ge 1-\beta,\quad \mu < \mu_1. \]
        因为$\beta\pare{\mu}$为$\mu$的减函数, 因此等价于要求
        \[ \beta\pare{\mu_1} \ge 1-\beta \Rightarrow \Phi\pare{\sqrt{n} \frac{\mu_0-\mu_1}{\sigma} - u_\alpha} \ge 1-\beta. \]
        等价地得到
        \[ n \ge \sigma^2 \pare{u_\alpha + u_\beta}^2 / \pare{\mu_0 - \mu_1}^2. \qedhere \]
    \end{solution}
\end{sample}

\paragraph{方差未知时的均值检验} % (fold)
\label{par:方差未知时的均值检验}

考虑
\[ H_0: \mu = \mu_0 \leftrightarrow \mu\neq \mu_0, \]
由于方差未知, 可以在将$\overbar{X}$标准化的过程中用样本方差代替总体方差$\sigma^2$得检验统计量
\[ T = \sqrt{n} \frac{\overbar{X} - \mu_0}{S}. \]
在$H_0$下, $T\sim t_{n-1}$, 于是取拒绝域
\[ \curb{\abs{T} > t_{n-1}\pare{\alpha/2}}. \]
此检验谓$t$检验.

% paragraph 方差未知时的均值检验 (end)

\paragraph{方差的检验} % (fold)
\label{par:方差的检验}

考虑检验问题
\[ H_0: \sigma^2 = \sigma_0^2 \leftrightarrow H_1: \sigma^2 \neq \sigma_0^2. \]
对于均值已知的情形, $\sigma^2$有MLE
\[ \hat\sigma^2 = \rec{n}\sum_{i=1}^n \pare{X_i - \mu}^2. \]
可以构造检验统计量
\[ \chi^2 = \rec{\sigma_0^2} \sum_{i=1}^n \pare{X_i - \mu}^2 = \frac{n\hat\sigma^2}{\sigma_0^2}. \]
在$H_0$下, $\chi^2\sim \chi_n^2$, $\chi^2$的平均值为$n$, 而在$H_1$下$\displaystyle \chi^2 = \frac{\sigma^2}{\sigma_0^2}\frac{n\hat\sigma^2}{\sigma^2}$的均值为$\displaystyle \frac{\sigma^2}{\sigma_0^2}n \neq 0$. 因此当$\chi^2$的值偏离$n$过大时应$RH_0$, 于是拒绝域为
\[ \curb{\chi^2 < \chi_n^2\pare{1-\alpha/2}\lor \chi^2 > \chi_n^2\pare{\alpha/2}}. \]
\par
对于均值未知的情形, 构造检验统计量
\[ \chi^2 = \frac{\pare{n-1}S^2}{\sigma_0^2}, \]
其中$S^2$为样本方差. 在$H_0$下$\chi^2\sim \chi_{n-1}^2$, 拒绝与为
\[ \curb{\chi^2 < \chi_{n-1}^2\pare{1-\alpha/2}\lor \chi^2 > \chi_{n-1}^2\pare{\alpha/2}}. \]

% paragraph 方差的检验 (end)

\paragraph{作业} % (fold)
\label{par:作业}

8.3, 24, 29, 30

% paragraph 作业 (end)

\par
显著性检验之一般步骤谓
\begin{cenum}
    \item 构造基于$\eta$的MLE$\hat\eta = \pare{\hat\theta,\hat\xi}$.
    \item 找一个量$S\pare{\theta,\hat\eta}$, 其分布在$H_0$下完全已知.
    \item 根据对立假设的形式, 取恰当的拒绝域, 例如$S\pare{\theta,\hat{\theta}}\le a$.
\end{cenum}
\begin{sample}
    \begin{ex}
        $X\sim N\pare{\mu,\sigma^2}$, $H_0:\mu\le \mu_0\leftrightarrow \mu > \mu_0$, 而$\pare{\mu,\sigma^2}$的MLE为$\pare{\overbar{X},\displaystyle \rec{n}\displaystyle \sum\pare{X_i - \overbar{X}}^2}$. 注意到
        \[ \sqrt{n} \frac{\overbar{X} - \mu}{S} \sim t_{n-1}, \]
        令$\displaystyle \frac{\sqrt{n}\pare{\overbar{X} - \mu_0}}{S} > a$,
        \begin{align*}
            \alpha &\ge P_{\mu\le\mu_0}\pare{\frac{\sqrt{n}\pare{\overbar{X} - \mu_0}}{S} > a} \\
            &= P_{\mu\le\mu_0} \pare{\frac{\sqrt{n}\pare{\overbar{X}-\mu}}{S} > a + \sqrt{n} \frac{\mu_0-\mu}{S}}.
        \end{align*}
        但
        \[ P_{\mu = \mu_0}\pare{\frac{\sqrt{n}\pare{\overbar{X} - \mu_0}}{S}> a} \le  P_{\mu \le \mu_0}\pare{\frac{\sqrt{n}\pare{\overbar{X} - \mu}}{S}> a} = \alpha, \]
        只需令上式右侧$=\alpha$, 即$a=t_\alpha\pare{n-1}$.
    \end{ex}
\end{sample}
\begin{sample}
    \begin{ex}
        设$X\sim U\pare{0,\theta}$, $\theta>0$, $X_1,\cdots,X_n$为样本, 求
        \[ H_0:\theta \le \theta_0 \leftrightarrow H_1:\theta > \theta_0 \]
        的水平$\alpha$的检验.
    \end{ex}
    \begin{solution}
        $\theta$的MLE为$\hat\theta = X_{\pare{n}}$, 且有pdf
        \[ f_n\pare{x} = \frac{nx^{n-1}}{\theta^n}I\pare{0<x<\theta}. \]
        故可得
        \[ \frac{\hat\theta}{\theta} = \frac{X_{\pare{n}}}{\theta}. \]
        其有pdf
        \[ g_n\pare{t} = nt^{n-1} I\pare{0<t<1}. \]
        对假设, 由$H_1$的形式知一个合理的检验区间拒绝域为$X_{\pare{n}}/\theta_0 > c$. 令
        \begin{align*}
            \alpha &\ge P_{\theta \le \theta_0}\pare{\frac{X_{\pare{n}}}{\theta_0}>c} \\
            &= P_{\theta\le \theta_0}\pare{\frac{X_{\pare{n}}}{\theta} > \frac{c\theta_0}{\theta}}. \\
            P_{\theta\le\theta_0}\pare{\frac{X_{\pare{n}}}{\theta} > c\frac{\theta_0}{\theta}}  &\le P_{\theta \le \theta_0}\pare{\frac{X_{\pare{n}}}{\theta} > c} \\
            &= \int_c^1 g_n\pare{t}\,\rd{t} \\
            &= 1-c^n.
        \end{align*}
        因此拒绝域取$c = \sqrt[n]{1-\alpha}$.
    \end{solution}
\end{sample}
\begin{sample}
    \begin{ex}
        设$X\sim \Exp\pare{\lambda}$, $\lambda > 0$, 求
        \[ H_0: \lambda \le \lambda_0 \leftrightarrow H_1: \lambda > \lambda_0 \]
        的水平$\alpha$检验.
    \end{ex}
    \begin{solution}
        $\lambda$的MLE为$\hat\lambda = 1/\overbar{X}$, 直观上合理的拒域为$\hat\lambda > c$.
        \begin{align*}
            \alpha &\ge P_{\lambda \le \lambda_0}\pare{\hat\lambda > c} \\
            &= P_{\lambda \le \lambda_0} \pare{\rec{\overbar{X}} > c} \\
            &= P_{\lambda \le \lambda_0} \pare{2n\lambda \overbar{X} < \rec{c}\cdot 2n\lambda} \\
            &\le P_{\lambda \le \lambda_0}\pare{2n\lambda \overbar{X} < \rec{c}\cdot 2n\lambda_0} \\
            \Rightarrow \alpha &= P_{\lambda\le \lambda_0} \pare{2n\lambda \overbar{X} < \rec{c}2n\lambda_0} \\
            &\Rightarrow \rec{c} 2n\lambda_0 = \chi_{1-\alpha}^2\pare{2n}. \qedhere
        \end{align*}
    \end{solution}
    \begin{remark}
        $\lambda$的MLE通过
        \[ L\pare{\lambda} = \lambda^n e^{-\lambda \sum x_i} \rightarrow n\log\lambda -\lambda \sum x_i \rightarrow \frac{n}{\lambda} = \sum x_i \]
        求得.
    \end{remark}
\end{sample}
假设检验时, 用一个检验统计量来度量之, 估计参数取值范围的偏离程度. 例如$w_0 = \lbr{-\infty,\theta_0}$, $d\pare{\hat\theta,w_0} = d\pare{\hat\theta,\theta_0}$.
\par
检验统计量的构造需要观察$\hat\theta$的形式得出. 采用MLE的原因在于MLE具有渐进正态性.

% subsubsection 一样本正态总体参数检验 (end)

\subsubsection{两样本正态总体的情形} % (fold)
\label{ssub:两样本正态总体的情形}

设总体$X\sim N\pare{\mu_1,\sigma_1^2}$, $Y\sim N\pare{\mu_2,\sigma_2^2}$, $-\infty<\mu_1,\mu_2<\infty$, $\sigma_1^2>0$, $\sigma_2^2>0$. $X_1,\cdots,X_n$是从总体$X$中抽取的一个样本, $Y_1,\cdots,Y_n$是从总体$Y$中抽取的一个样本. 设来自不同总体的样本相互独立. 下面考虑有关均值差$\mu_1 - \mu_2$和方差比$\sigma_1^2/\sigma_2^2$的检验, 取显著性水平为$\alpha$.
\[ H_0: \mu_1 - \mu_2 \le \delta_0 \leftrightarrow H_1: \mu_1-\mu_2 > \delta_0. \]
注意参数为$\mu_1,\mu_2,\sigma^2$, 则
\begin{align*}
    l\pare{\mu_1,\mu_2,\sigma^2} &\propto -\frac{n}{2}\log \sigma^2 + \rec{2\sigma^2}\sum\pare{X_i - \mu_1}^2 \\
    &\phantom{\propto}\ -\frac{m}{2}\log\sigma^2 + \rec{2\sigma^2}\sum\pare{Y_j - \mu_2}^2 \\
    &= -\frac{n+m}{2}\log \sigma^2 + \rec{2\sigma^2}\brac{\sum\pare{X_i - \mu_1}^2 + \sum\pare{Y_i - \mu_2}^2}. \\
    & \+D{\mu_1}D{l} = 0,\quad \+D{\mu_2}D{l} = 0,\quad \+D{\sigma_2^2}D{l} = 0 \\
    & \Rightarrow \hat\mu_1 = \overbar{X},\quad \hat\mu_2 = \overbar{Y},\quad \hat\sigma^2 = \rec{n+m} \brac{\sum\pare{X_i - \hat\mu_1}^2 + \sum\pare{Y_j- \hat\mu_2}^2}.
\end{align*}
一个直观上合理的拒绝域为$\hat\mu_1 - \hat\mu_2 > c$. 注意到
\[ \hat\mu_1 - \hat\mu_2 = \overbar{X} - \overbar{Y} = N\pare{\mu_1 - \mu_2, \frac{\sigma^2}{n} + \frac{\sigma^2}{m}}. \]
为使
\begin{align*}
    \alpha &\ge P_{\mu_1 - \mu_2 \le \delta_0} \pare{\hat\mu_1 - \hat\mu_2 > c} \\
    &= P_{\mu_1 - \mu_2 \le \delta_0} \pare{\hat\mu_1 - \hat\mu_2 - \pare{\mu_1 - \mu_2} > c - \pare{\mu_1 - \mu_2}} \\
    &\le P_{\mu_1 - \mu_2 \le \delta_0} \pare{\hat\mu_1 - \hat\mu_2 - \pare{\mu_1 - \mu_2} > c-\delta_0} \\
    &= P_{\mu_1 - \mu_2 \le \delta_0}\pare{\frac{\hat\mu_1 - \hat\mu_2 - \pare{\mu_1 - \mu_2}}{\sqrt{\frac{\sigma^2}{n} + \frac{\sigma^2}{m}}} > \frac{c-\delta_0}{\sqrt{\frac{\sigma^2}{n} + \frac{\sigma^2}{m}}}} \quad \text{(分布仍未知)} \\
    &= P_{\mu_1 - \mu_2 \le \delta_0} \pare{ \underbrace{\frac{\frac{\hat\mu_1 - \hat\mu_2 - \pare{\mu_1 - \mu_2}}{\sqrt{\frac{\sigma^2}{n} + \frac{\sigma^2}{m}}}}{\sqrt{\rec{n+m-2}\brac{\frac{\pare{n+m}\hat\sigma^2}{\sigma^2}}}} }_{\sim t_{n+m-2}} > \frac{ \frac{c-\delta_0}{\sqrt{\frac{\sigma^2}{n} + \frac{\sigma^2}{m}}} } {\sqrt{\rec{n+m-2}\brac{\frac{\pare{n+m}\hat\sigma^2}{\sigma^2}}}} } \\
    &= P_{\mu_1 - \mu_2 \le \delta_0} \pare{T_{n+m-2} > \frac{ \frac{c-\delta_0}{\sqrt{\frac{\sigma^2}{n} + \frac{\sigma^2}{m}}} } {\sqrt{\rec{n+m-2}\brac{\frac{\pare{n+m}\hat\sigma^2}{\sigma^2}}}}}.
\end{align*}
利用$T$与$\sigma^2$的独立性, 可视不等式右边为常数. 代入原拒绝域中,
\[ \hat\mu_1 - \hat\mu_2 > \delta_0 + \pare{\cdots} t_\alpha\pare{n+m-2}. \]
\begin{sample}
    \begin{ex}
        检验磷肥的有效性.
    \end{ex}
    \begin{solution}
        磷肥对玉米产量有效果等价于$\mu_1>\mu_2$, 故将其作为对立假设. 假设检验问题是
        \[ H_0: \mu_1 \le \mu_2 \leftrightarrow H_1:\mu_1 > \mu_2. \]
        构造
        \[ T = \frac{\overbar{X} - \overbar{Y}}{S_w \sqrt{\rec{m} + \rec{n}}}, \]
        若$H_0$成立则$T\sim t_{n+m-2}$, 拒绝域为
        \[ \curb{T>t_{n+m-2}\pare{\alpha}}. \qedhere \]
    \end{solution}
\end{sample}
为了检验方差的齐性
\[ H_0: \frac{\sigma_1^2}{\sigma_2^2} \le \delta_0 \leftrightarrow H_1: \frac{\sigma_1^2}{\sigma_2^2} > \delta_0. \]
考虑到$\mu_1$, $\mu_2$, $\sigma_1^2$, $\sigma_2^2$的MLE为
\[ \hat{\mu}_1 = \overbar{X},\quad \hat{\mu}_2 = \overbar{Y},\quad \hat\sigma_1^2 = \rec{n}\sum\pare{X_i - \overbar{X}}^2,\quad \hat{\sigma}_2^2 = \rec{m} \sum\pare{Y_j - \overbar{Y}}^2. \]
一个合理的拒绝域为$\hat\sigma_1^2 / \hat\sigma_2^2 > c$, 为使
\begin{align*}
    \alpha & \ge P_{\sigma_1^2/\sigma_2^2 \le \delta_0} \pare{\frac{\hat\sigma_1^2}{\hat\sigma_2^2} > c} \\
    & P_{\sigma_1^2/\sigma_2^2 \le \delta_0} \pare{\frac{\rec{n-1}n\hat\sigma_1^2/\sigma_1^2}{\rec{m-1}m\hat\sigma_2^2/\sigma_2^2} > c\cdots \frac{\sigma_2^2}{\sigma_1^2}\times \cdots} \\
    &\le P_{\sigma_1^2/\sigma_2^2\le \delta_0} \pare{* > \underbrace{c\cdot\rec{\delta_0} *}_{=F_\alpha\pare{n-1,m-1}}}.
\end{align*}
设
\[ F = \frac{S_1^2}{S_2^2} = \frac{\pare{m-1}\hat\sigma_1^2 /m}{\pare{n-1}\hat\sigma_2^2 /n}, \]
则拒绝域为
\[ \curb{F < F_{m-1,n-1}\pare{\alpha/2}\lor F>F_{m-1,n-1}\pare{1-\alpha/2}}. \]

% subsubsection 两样本正态总体的情形 (end)

\subsubsection{成对数据} % (fold)
\label{ssub:成对数据}

对于成对数据
\[ \curb{\pare{X_1,Y_1},\cdots,\pare{X_n,Y_n}}, \]
数据对之间通常可认为独立, 但数据对内的两个观测通常不独立. 对数据内作差, 构造新总体$Z$和样本
\[ Z_1 = X_1 - Y_1,\cdots,Z_n = X_n - Y_n. \]
通常假设$Z$服从正态分布, 则假设检验转化为样本正态检验.
\begin{remark}
    T检验有多种类型, 例如单样本T检验用于方差未知时正态均值的假设检验; 二样本T检验用于方差未知时二正态总体均值之差的假设; 成对T检验用于连续数据对的检验.
\end{remark}
\paragraph{作业} % (fold)
\label{par:作业}

35, 42, 49, 52

% paragraph 作业 (end)

% subsubsection 成对数据 (end)

\begin{sample}
\begin{ex}[截尾试验]
    设$X\sim \Exp\pare{\lambda}$, 设计两种类型的样本:
    \begin{cenum}
        \item 试验进行至第$r$个失效即停止;
        \item 实验进行至某个时刻$T_0$为止.
    \end{cenum}
    前者谓定数截尾, 后者谓定时截尾. 据此考虑假设
    \[ H_0: \lambda \le \lambda_0 \leftrightarrow H_1: \lambda > \lambda_0. \]
    \begin{cenum}
        \item 对于定数截尾, 首先应当找到$\lambda$的似然估计, 设样本为$X_1,\cdots,X_n$, 排序后为
        \[ X_{\pare{1}} \le X_{\pare{2}} \le\cdots \le X_{\pare{n}}. \]
        实验进行到第$r$个失效, 故依据实验方式, $X_{\pare{1}},\cdots,X_{\pare{r}}$的值均有观测, 剩下的$X_{\pare{r+1}},\cdots,X_{\pare{n}}$并无具体值, 仅知道其皆$>X_{\pare{r}}$. 因此
        \begin{align*}
            & L\pare{\lambda} \\ &= P\pare{X_{\pare{1}} \in x_{\pare{1}}\pm\delta,\cdots,X_{\pare{r}}\in x_{\pare{r}}\pm\delta,X_{\pare{i}}>x_{\pare{r}}, r<i\le n} \\
            &= \frac{n!}{1!\cdots 1!\pare{n-r}!}P\pare{X_1\in x_{\pare{1}}\pm\Delta,\cdots,X_r\in x_{\pare{r}}\pm \Delta, X_i \cdots} \\
            &= \frac{n!}{\pare{n-r}!} \prod_{i=1}^r P\pare{X_i\in x_{\pare{i}}\pm \Delta_i} \prod_{j=r+1}^n P\pare{X_j \ge x_{\pare{r}} + \Delta_r} \\
            &= \frac{n!}{\pare{n-r}!}\prod_{i=1}^r \brac{F\pare{x_{\pare{i}}+\Delta_i} - F\pare{x_{\pare{i}}-\Delta_i}}\cdot \\ &\phantom{=}\ \prod_{j=r+1}^n \brac{1-F\pare{x_{\pare{r}} + \Delta_r}} \\
            &= \frac{n!}{\pare{n-r}!}\prod_{i=1}^r f\pare{x_{\pare{i}}}\pare{2\Delta_i}\brac{1-F\pare{x_{\pare{r}}+\Delta_r}}^{n-r}.
        \end{align*}
        故可以取似然函数
        \begin{align*}
            L\pare{\lambda} &= \prod_{i=1}^r f\pare{x_{\pare{i}}}\brac{1-F\pare{x_{\pare{r}}}}^{n-r} \\
            &= \lambda^r \exp{-\lambda \brac{\sum_1^r x_{\pare{i}} + \pare{n-r}x_{\pare{r}}}} \\
            &= \lambda^r e^{-\lambda T},\quad T = \sum_1^r x_{\pare{i}} + \pare{n-r}x_{\pare{r}}.
        \end{align*}
        于是MLE为$\hat\lambda = -r/T$. 对于假设, 合理的拒绝域为$T<c$. 由于
        \[ 2\lambda T \sim \chi_{2r}^2, \]
        为使
        \begin{align*}
            \alpha &\ge P_{\lambda\le \lambda_0}\pare{T<c} \\
            &= P_{\lambda \le \lambda_0}\pare{2\lambda T \le 2\lambda c} \\
            &\le P_{\lambda\le\lambda_0} \pare{2\lambda T < 2\lambda_0 c}.
        \end{align*}
        只需取$P_{\lambda\le\lambda_0} \pare{2\lambda T < 2\lambda_0 c} = \alpha$, 得到$c$即可.
    \end{cenum}
\end{ex}
\end{sample}

\subsubsection{\texorpdfstring{$0$-$1$}{0-1}分布中未知参数\texorpdfstring{$p$}{p}的假设检验} % (fold)
\label{ssub:0_1分布中未知参数}

设$\pare{X_1,\cdots,X_n}$是取自总体$X$的一个样本, 该总体服从$0$-$1$分布. 取$1$的概率为$p$. $X_1,\cdots,X_n$ i.i.d. $\sim B\pare{1,p}$, $p$的MLE为
\[ \hat p \rec{n} \sum_1^n X_i \sim B\pare{n,p}. \]
对于
\[ H_0: p=p_0 \leftrightarrow H_1: p\neq p_0, \]
合理的拒绝域为
\[ \hat p < c_1 \lor \hat p \ge c_2 \quad \pare{\abs{\hat p - p_0} > c}. \]
对于
\[ H_0: p = p_0 \leftrightarrow H_1: p > p_0\quad\text{或}\quad H_0: p \le p_0 \leftrightarrow H_1: p > p_0. \]
合理的拒绝域为
\[ \hat p > c. \]
对于
\[ H_0: p = p_0 \leftrightarrow H_1: p < p_0\quad\text{或}\quad H_0: p \ge p_0 \leftrightarrow H_1: p < p_0. \]
合理的拒绝域为
\[ \hat p < c. \]
对于第一种$H_0 \leftrightarrow H_1$, 为使
\begin{align*}
    \alpha &\ge P_{p_0} \pare{\hat p < c_1 \lor \hat p > c_2} \\
    &= P_{p_0}\pare{\hat p < c_1} + P_{p_0}\pare{\hat p > c_2} \\
    &= \sum_{k=0}^{nc_1-1} \binom{n}{k}p_0^kq_0^{n-k} + \sum_{k=nc_2+1}^n \binom{n}{k} p_0^{k}q_0^{n-k}.
\end{align*}
令两项概率同时$<\alpha/2$, 则
\begin{align*}
    \sum_{k=0}^{nc_1-1} \binom{n}{k}p_0^kq_0^{n-k} &\le \alpha/2, \\
    \sum_{k=nc_2+1}^n \binom{n}{k} p_0^{k}q_0^{n-k} & \le \alpha/2.
\end{align*}
选取$c_1$为使不等式成立的最大的$c_1$, $c_2$为使不等式成立的最小的$c_2$.
\par
对于第二种$H_0 \leftrightarrow H_1$, 为使
\begin{align*}
    \alpha &\ge P_{p\le p_0} \pare{\hat p > c} \\
    &= P_{p\le p_0} \pare{\sum_1^n X_i > nc} \\
    &= \sum_{nc+1}^n \binom{n}{k}p^k q^{n-k},\quad \forall p\le p_0,\quad \text{且是$p$的增函数} \\
    & \le \sum_{nc+1}^n \binom{n}{k}p_0^k q_0^{n-k}.
\end{align*}
若样本量$n$较大, 取显著性水平$\alpha$, 由于$p$的MLE为$\overbar{X}$, 则
\[ T = \sqrt{n} \frac{\overbar{X} - p_0}{\sqrt{p_0 \pare{1-p_0}}}, \]
其中$p_0$和$p_0\pare{1-p_0}/n$分别为$\overbar{X}$在零假设下的期望和方差. 当$H_0$成立, 由CLT近似成立$T\sim N\pare{0,1}$. 于是上述三种检验的拒绝域分别为
\[ \curb{\abs{T} > u_{\alpha/2}},\quad \curb{T>u_\alpha},\quad \curb{T<-u_\alpha}. \]
\begin{sample}
    \begin{ex}
        某产品不合格率通常为$0.05$. 原料产地改变后, 抽取$80$个样品检验, 发现$5$个不合格品. 在$\alpha = 0.1$下, 可得出什么结论?
    \end{ex}
    \begin{solution}
        总体$X\sim B\pare{1,p}$, 其中$p$未知. 在显著性水平$\alpha=0.1$下,
        \[ H_0:p=0.05 \leftrightarrow H_1:p\neq 0.05. \]
        由$\overbar{x} =5/80$,
        \[ T = \sqrt{n}\frac{\overbar{X} - p_0}{\sqrt{p_0\pare{1-p_0}}} = 0.513 < u_{0.05} = 1.645, \]
        故不能拒绝$H_0$.
    \end{solution}
\end{sample}

% subsubsection 0_1分布中未知参数 (end)

\subsubsection{置信区间和假设检验之间的关系} % (fold)
\label{ssub:置信区间和假设检验之间的关系}

若要求参数$\theta$的$1-\alpha$置信区间为$\brac{\ubar{\theta}\le \theta\le \overbar{\theta}} \ge 1-\alpha$ 而对于
\[ H_0: \theta = \theta_0\leftrightarrow H_1:\theta\neq \theta_0, \]
在原假设下, 有
\[ P\pare{\ubar{\theta}\le\theta_0\le\overbar{\theta}} \ge 1-\alpha \Leftrightarrow P\pare{\theta_0 > \ubar{\theta}} + P\pare{\theta_0 < \ubar{\theta}} \le \alpha. \]
假设检验要求$\ubar{\theta}\le\theta_0\le\overbar{\theta}$则接受$H_0$, 否则就拒绝.
\par
因此, 为了求出参数$\theta$的$1-\alpha$置信区间, 可以先找到$\theta$的双边检验
\[ H_0: \theta = \theta_0\leftrightarrow H_1:\theta\neq \theta_0 \]
的检验函数, 其接受域即为$1-\alpha$置信区间. 反过来, 为了求
\[ H_0: \theta = \theta_0\leftrightarrow H_1:\theta\neq \theta_0 \]
的检验, 可以先求出参数$\theta$的$1-\alpha$置信区间, 即得到该假设的接受域.
\par
类似地, 置信水平为$1-\alpha$的单侧置信区间$\pare{\ubar{\theta},\infty}$或者$\pare{-\infty,\overbar{\theta}}$与显著性水平为$\alpha$的单边即检验问题有类似对应关系.

\paragraph{LRT检验} % (fold)
\label{par:lrt检验}

设
\[ H_0: \theta\in w_0\leftrightarrow \theta\in w-w_0, \]
设$L\pare{\theta}$为似然函数, 一个合理的检验为对于
\[ \Lambda = \frac{\max_{\theta\in w_0}L\pare{\theta}}{\max_{\theta \in w}L\pare{\theta}}, \]
当$\Lambda<c$时拒绝之. 可以证明当$H_0$为真,
\[ -2\log \Lambda \xrightarrow{d} \chi^2\+_{df}_. \]
\begin{remark}
    为了比较两种检验方法,
    \begin{cenum}
        \item 设均为水平$\alpha$的测试;
        \item 比较第II型错误的概率, 更小的更好.
    \end{cenum}
    为了比较(在同一种检验方法下)两组样本支持$H_0$的证据强弱, 引入$p$值为$P\pare{\text{在$H_0$下出现像检验统计量的观测值那么糟糕或者更极端值的可能性}}$. 例如否定域的形式为$T>c$, 则$p=P_{H_0}\pare{T>T\+_obs_}$. 检验原则为$p\le\alpha$时$RH_0$, 且$p$值越小, 否定$H_0$的理由越充分.
\end{remark}

% paragraph lrt检验 (end)

% subsubsection 置信区间和假设检验之间的关系 (end)

% subsection 一样本和两样本的总体参数检验 (end)

\subsection{拟合优度检验} % (fold)
\label{sub:拟合优度检验}

为了检验
\[ H_0: \text{$X$服从某种分布$F$}, \]
可以采用Karl Pearson提出的$\chi^2$拟合优度检验. 基本想法为基于样本得到$F$的估计$\hat F_n$, 计算某种偏差$D\pare{\hat F_n, F}$. 当$H_0$正确时, $\hat F_n$是$F$的相合估计, 偏差$D\pare{\hat F_n,F}$应该很小.

\subsubsection{离散总体情形} % (fold)
\label{ssub:离散总体情形}

设总体$X$服从一个离散分布$P\pare{a_i} = p_i$, $i = 1,\cdots,k$. 现在取得样本量为$n$的样本, 落在$a_1,\cdots,a_k$的观测数分别为$n_1,\cdots,n_k$.
\begin{cenum}
    \item $p_i$有估计
    \[ \hat p_i = \frac{n_i}{n},\quad n_i = \sum_{j=1}^n I\pare{X_j = i}. \]
    \item 构造权重$w_i$, 使得
    \[ \sum_{i} w_i\pare{\hat p_i - p_i}^2 \]
    分布已知.
\end{cenum}
根据LLN, 在零假设成立时$n_i/n$依概率收敛于$p_i$, 故理论频数$np_i$与观测频数$n_i$接近. 通过检验统计量
\[ T = \sum \frac{\pare{n_i - np_i}^2}{np_i} = \sum \frac{\pare{O-E}^2}{E}. \]
若$H_0$成立, $T$的极限分布就是$\chi^2_{k-1}$. 从而有拒绝域$T>\chi_\alpha^2\pare{k-1}$.
\begin{sample}
    \begin{ex}
        有人制造含$6$个面的骰子, 投掷$600$次, 六个面的频数分别为
        \[ 97, 104, 82, 110, 93, 114, \]
        在骰子均匀的零假设下
        \[ T = \sum \frac{\pare{x_i - 100}^2}{100} = 6.94 < \chi_5^2\pare{0.2}\approx 7.29, \]
        故不能拒绝零假设.
    \end{ex}
\end{sample}
\begin{sample}
    \begin{ex}
        Mendel的豌豆杂交实验在Mendel第一定律的假设下二代豌豆应该得到$75\%$黄色和$25\%$绿色. 观测值$n_1 = 6022$, $n_2 = 2001$, 这批数据是否支持Mendel第一定律?
    \end{ex}
    \begin{solution}
        要检验的假设为
        \[ H_0:\quad \pi_1 = 0.75,\quad \pi_2 = 0.25. \]
        期望的$\mu_1 = n\pi_1 = 6017.25$, $\mu_2 = n\pi_2 = 2005.75$, 从而
        \[ Z = \sum \frac{\pare{O-E}^2}{E} = 0.015. \]
        自由度$\mathrm{df} = 1$, $p$值为$0.903$, 可以认为服从Mendel第一定律.
    \end{solution}
\end{sample}

\paragraph{作业} % (fold)
\label{par:作业}

56, 57, 58

% paragraph 作业 (end)

\paragraph{存在未知参数的情形} % (fold)
\label{par:存在未知参数的情形}

若$p_i = p_i\pare{\theta}$, $\theta$未知, 则上文的$T$不可用. 令$\theta$的MLE为$\hat\theta$,
\begin{align*}
    L\pare{\theta} &= P\pare{X_1 = x_1,\cdots,X_n = x_n} \\
    &= \prod_{i=1}^n P\pare{X_i = x_i},
    \hat\theta = \arg\max L\pare{\theta}.
\end{align*}
从而$p_i = p_i\theta$的MLE为$\hat{p}_i = p_i\pare{\theta}$. 构造
\[ \chi^2 = \sum_{i=1}^k \frac{\pare{n_i - n\hat{p}_i}^2}{n\hat{p}_i}, \]
则这一变量是自由度为$k-1-r$的$\chi^2$变量, 其中$r$是独立参数的个数.

% paragraph 存在未知参数的情形 (end)

\begin{sample}
    \begin{ex}
        取$100$人的血液检测某位点的基因型, 假设仅有$A$和$a$两个等位基因, $AA$, $Aa$, $aa$的个数分别为$30$, $40$, $30$, 能否在$0.05$的水平下认为该群体此位点达到Hardy-Weinberg稳态?
    \end{ex}
    \begin{solution}
        取$H_0$为稳态成立, 设$A$的基因频率为$p$, 则
        \[ H_0: P\pare{AA} = p^2,\quad P\pare{Aa} = 2p\pare{1-p},\quad P\pare{aa} = \pare{1-p}^2. \]
        在$H_0$下, 三个理论频数为$100\times\hat p^2, 100\times 2\times \hat p\pare{1-\hat p}, 100\times\pare{1-\hat p}^2$. 其中$\hat p$为估计的等位基因频率$0.5$, 从而$\chi^2 = 4$, 大于$\chi^2_{0.05}\pare{3-1-1} = 3.84$, 故可以在$0.05$的水平下未达到.
    \end{solution}
\end{sample}

% subsubsection 离散总体情形 (end)

\subsubsection{列表独立性与齐一性检验} % (fold)
\label{ssub:列表独立性与齐一性检验}

设属性$A$和$B$分别有分类$a_1,\cdots,a_k$与$b_1,\cdots,b_r$, 并在各个$a_i$下有各个$b_j$的数据. 将
\[ \pare{a_1,b_1},\cdots,\pare{a_1,b_r},\cdots,\pare{a_k,b_r} \]
的频数列出, 若$A$和$B$独立, 则应有
\[ p_{11} = u_1v_1,\quad \cdots\quad p_{1r} = u_1v_r,\quad \cdots\quad p_{kr} = u_kv_r. \]
记$\theta = \pare{u_1,\cdots,u_{k-1};v_1,\cdots,v_{k-1}}$, 则
\[ p_{ij} = p_{ij}\pare{\theta}. \]
在$H_0$下, $\theta$的似然函数为
\begin{align*}
    L\pare{\theta} &= \prod_{i=1}^k \prod_{j=1}^r p_{ij}^{n_{ij}} \xlongequal{H_0} \prod_{i=1}^k \prod{j=1}^r \pare{u_i v_j}^{n_{ij}} \\
    &= P\pare{X_1 = x_1,\cdots,X_n = x_n} = \prod_{i=1}^n P\pare{X_i = x_i}.
\end{align*}
令$\displaystyle \+D{u_i}D{\ln L\pare{\theta}} = 0$, $\displaystyle \+D{v_j}D{\ln L\pare{\theta}} = 0$.
\[ \sum_{i=1}^k \sum_{j=1}^r n_{ij}\brac{\ln\pare{u_i} + \ln\pare{v_j}} = \sum_{i=1}^k n_{i+}\ln u_i + \sum_{j=1}^r n_{+j}\ln v_j. \]
考虑到$\sum u_i = 1$, $\sum v_j = 1$, 有
\[ \left\{\begin{aligned}
    \frac{n_{i+}}{u_i} - \frac{n_{k+}}{u_k} = 0, \\
    \frac{n_{+j}}{v_j} - \frac{n_{+r}}{v_r} = 0
\end{aligned}\right. \Rightarrow \left\{ \begin{aligned}
    u_i = \frac{n_{i+}}{u}, \\
    v_j = \frac{n_{+j}}{n}.
\end{aligned} \right. \]
由Pearson $\chi^2$检验,
\begin{align*}
    T &= \sum \frac{\pare{O-\hat E}^2}{\hat E^2} \\
    &= \sum_{i=1}^k \sum_{j=1}^r \frac{\pare{n_{ij} - n\hat p_{ij}}^2}{n\hat p_{ij}} \\
    &= \sum_{i=1}^k \sum_{j=1}^r \frac{\pare{n_{ij} - n_{i+}n_{+j}/n}^2}{\pare{n_{i+}n_{+j}/n}}.
\end{align*}
在$H_0$下, $\hat T \rightarrow \chi^2$, 有自由度
\[ kr - 1 - \pare{k-1 + r - 1} = \pare{k-1}\pare{r-1}. \]
从而$T$的极限分布是自由度$\pare{k-1}\pare{r-1}$的$\chi^2$分布. 特别地, 对于四格表, 自由度为$1$.

\begin{sample}
    \begin{ex}
        调查收入$A$和文化支出$B$的关系.\\
        \centerline{
        \begin{tabular}{ccccc}
            \toprule
            \diagbox{B}{A} & $1$ & $2$ & $3$ & 和 \\
            \midrule
            $1$ & $63$ & $37$ & $60$ & $160$ \\
            \midrule
            $2$ & $16$ & $17$ & $8$ & $41$ \\
            \midrule
            和  & $79$ & $54$ & $68$ & $201$ \\
            \bottomrule
        \end{tabular}
        }
    \end{ex}
    \begin{solution}
        取$H_0$为$A$和$B$无关. 计算$T$中的各个项, 如第一个项为
        \[ \frac{\pare{n_{11} - n_{1+}n_{+1}/n}^2}{n_{1+}n_{+1}/n} = \frac{\pare{63 - 160 \times 79/201}^2}{160\times 79 / 201} = 0.0002. \]
        可得$T = 7.2078$, $p = 0.0207$.
    \end{solution}
\end{sample}
\par
齐一性检验要求检验属性$A$的各个水平对应属性$B$的分布是否相同. 此种假设检验不同于独立性检验. 例如下例中医院并非一随机变量. 实验中的个体数目在「医院」的每个水平下是固定的. 尽管如此, 检验的形式仍然是一样的. 齐一性要求
\begin{align*}
    P\pare{Y=1\vert X=1} &= P\pare{Y=1\vert X=2}, \\ P\pare{Y=2\vert X=1} &= P\pare{Y=2\vert X=2}.
\end{align*}
视$X$为随机变量, 则
\begin{align*}
    \frac{P\pare{Y=1,X=1}}{P\pare{X=1}} &= \frac{P\pare{Y=1,X=2}}{P\pare{X=2}} \\ \Leftrightarrow \frac{P\pare{X=1,Y=1}}{P\pare{X=1}} &= \frac{P\pare{X=2,Y=1}}{1-P\pare{X=1}}. \\
    \Leftrightarrow P\pare{X=1,Y=1} &= P\pare{X=1}\brac{P\pare{X=2,Y=1} + P\pare{X=1,Y=1}} \\
    \Leftrightarrow P\pare{X=1,Y=1} &= P\pare{X=1}P\pare{Y=1}.
\end{align*}
类似地,
\[ P\pare{X=i,Y=j} = P\pare{X=i}P\pare{Y=j},\quad i,j = 1,2. \]
\begin{sample}
    \begin{ex}
        两个医院的治疗效果如下. \\
        \centerline{
        \begin{tabular}{cccc}
            \toprule
            & 生存 & 死亡 & 合计 \\
            \midrule
            甲院 & $150$ & $88$ & $238$ \\
            \midrule
            乙院 & $36$ & $18$ & $54$ \\
            \midrule
            合计 & $186$ & $106$ & $292$ \\
            \bottomrule
        \end{tabular}
        }
    \end{ex}
    \begin{solution}
        治疗水平无差异相当于
        \[ H_0: P\pare{\text{生存} \vert \text{甲院}} = P\pare{\text{生存} \vert \text{乙院}}. \qedhere \]
    \end{solution}
\end{sample}
\begin{sample}
    \begin{ex}
        设工厂生产质量如下. \\
        \centerline{
        \begin{tabular}{ccccc}
            \toprule
            \diagbox{质量}{工厂} & $1$ & $2$ & $3$ & 和 \\
            \midrule
            $1$ & $58$ & $38$ & $32$ & $138$ \\
            \midrule
            $2$ & $28$ & $44$ & $45$ & $117$ \\
            \midrule
            $3$ & $23$ & $18$ & $14$ & $55$ \\
            \midrule
            和  & $109$ & $100$ & $91$ & $300$ \\
            \bottomrule
        \end{tabular}
        }
    \end{ex}
    \begin{solution}
        可以证明, 各工厂产品生产质量一致等价于
        \[ H_0: P\pare{X=i, Y = j} = P\pare{X=i}P\pare{Y=j},\quad i,j = 1,2,3. \qedhere \]
    \end{solution}
\end{sample}

% subsubsection 列表独立性与齐一性检验 (end)

\subsubsection{连续总体的情形} % (fold)
\label{ssub:连续总体的情形}

设有样本$X_1,\cdots,X_n$. 记$X$的分布函数为$F\pare{x}$, 要在显著性水平$\alpha$下检验
\[ H_0: F\pare{x} = F_0\pare{x;\theta_1,\cdots,\theta_r}. \]
可以先将总体分布离散化后用拟合优度的方法来检验. 把实数轴分为$k$个子区间$\lbr{a_{j-1},a_j}$, $j=1,\cdots,k$, 其中$a_0$可以取$-\infty$, $a_k$可以取$\infty$. 这构造了一个离散总体, 记
\[ p_j = P_{H_0}\pare{a_{j-1} < X \le a_j} = F_0\pare{a_j;\theta} - F_0\pare{a_{j-1};\theta}. \]
如果$H_0$成立, 则概率$p_j$应与数据落在$\blr{a_{j-1},a_j}$的频率$f_j = n_j/n$接近, 其中$n_j$表示相应的频数. 当$p_i$的取值不含未知参数, 取检验统计量
\[ \chi^2 = \sum_{j=1}^k \frac{\pare{n_j - np_j}^2}{np_j}. \]
否则取
\[ \chi^2 = \sum_{j=1}^k \frac{\pare{n_j - n\hat p_j}^2}{n\hat p_j}, \]
其中$\hat p_j$是$p_i$的估计. 拒绝域取
\[ \curb{\chi^2 > \chi^2_{k-r-1}\pare{\alpha}}, \]
其中$r$是未知参数的数目.
\begin{sample}
    \begin{ex}
        从某个连续总体中抽取样本量为$100$的样本, 发现样本均值和样本标准差分别为$-0.225$和$1.282$, 频数如下. \\
        \centerline{
        \begin{tabular}{ccccccc}
            \toprule
            区间 & $\pare{-\infty,-1}$ & $\blr{-1,-0.5}$ & $\blr{-0.5,0}$ & $\blr{0,0.5}$ & $\blr{0.5,1}$ & $\blr{1,\infty}$ \\
            \midrule
            观测 & $25$ & $10$ & $18$ & $24$ & $10$ & $13$ \\
            \midrule
            理论 & $27$ & $14$ & $16$ & $14$ & $12$ & $17$ \\
            \bottomrule
        \end{tabular}
        }
    \end{ex}
\end{sample}

% subsubsection 连续总体的情形 (end)

% subsection 拟合优度检验 (end)

% section 假设检验 (end)

\end{document}
