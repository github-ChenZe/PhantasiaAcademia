\documentclass[hidelinks]{ctexart}

\usepackage[margintoc, singleton, epu, nova]{van-de-la-sehen}

\newcommand{\gpattern}[1]{\begin{tcolorbox}[enhanced,tcbox raise base,boxrule=0.4pt,top=0mm,bottom=0mm,
  right=0mm,left=0.15mm,arc=1pt,boxsep=2pt,
  colframe=gray,coltext=gray!25!black,colback=gray!10!white,box align=center,
    halign=center,
    valign=center]#1\end{tcolorbox}
}
\newcommand{\gcomp}[1]{\textit{#1}}
\robustify{\gpattern}
\newcommand{\gvocabcline}{\cline{1-4}}


\begin{document}

\showtitle{Vorlesungsmitschrift: Tag 1}

\section{惯用} % (fold)
\label{sec:惯用}

\subsection{日常} % (fold)
\label{sub:日常}

\subsubsection{问候} % (fold)
\label{ssub:问候}

\begin{longtable}{|c|c|}
    \hline
    Guten Morgen & 早上好 \\
    \hline
    Guten Tag & 你好 \\
    \hline
    Guten Abend & 晚上好 \\
    \hline
    Hallo & 你好 \\
    \hline
\end{longtable}

% subsubsection 问候 (end)

\subsubsection{告别} % (fold)
\label{ssub:告别}

\leavevmode \InsertBoxR{0}{\parbox{5cm}{\begin{mtips}
    wieder表示「再一次」, sehen表示「见」, auf表示期待.
\end{mtips}}}[1]
\begin{tabular}{|c|c|}
    \hline
    正式 & Auf Wiedersehen \\
    \hline
    \+:r4{非正式} & Wiedersehen \\
    & Tsch\"us \\
    & Tschau \\
    & Ciao \\
    \hline
\end{tabular}
\leavevmode

% subsubsection 告别 (end)

% subsection 日常 (end)

\subsection{人称} % (fold)
\label{sub:人称}

\subsubsection{T-V对立} % (fold)
\label{ssub:t_v对立}

家庭成员/亲属/朋友/学生/关系较好的同事之间或对上帝用du, 成年人之间/儿童对非亲属成年人用Sie. 多数情形下对等, 除了儿童对成年人.\\

% subsubsection t_v对立 (end)

\subsubsection{称呼} % (fold)
\label{ssub:称呼}

Frau可用于称呼已婚和未婚.\\
\gpattern{Sie关系: 用Frau/Herr + \gcomp{Familienname}称呼之.}
\begin{sample}
    \begin{linguaex}
        Guten Tag, Frau Bauer!
    \end{linguaex}
\end{sample}
\gpattern{du关系: 用\gcomp{Vorname}称呼之.}
\begin{sample}
    \begin{linguaex}
        Hallo, Anna!
    \end{linguaex}
\end{sample}

% subsubsection 称呼 (end)

\subsubsection{自我介绍} % (fold)
\label{ssub:自我介绍}
\gpattern{Ich hei\ss e/Mein Name ist + \gcomp{Name}/\gcomp{Vorname}/\gcomp{Familienname}.\\
Ich bin + \gcomp{Name}/\gcomp{Vorname}.}
\begin{sample}
    \begin{linguaex}
        \label{langex:ich_heisse}
        Ich hei\ss e Anna Bauer./ Mein Name ist Anna Bauer.
    \end{linguaex}
\end{sample}
\gpattern{Ich komme aus \ldots/Ich bin aus \ldots}
\begin{sample}
    \begin{linguaex}
        \label{langex:ich_komme}
        Ich komme/bin aus China.
    \end{linguaex}
\end{sample}
\gpattern{Ich wohne in \ldots/Ich bin in \ldots}
\begin{sample}
    \begin{linguaex}
        \label{langex:ich_wohne}
        Ich wohne in Guangzhou.
    \end{linguaex}
\end{sample}

% subsubsection 自我介绍 (end)

% subsection 人称 (end)

% section 惯用 (end)

\section{语法} % (fold)
\label{sec:语法}

\subsection{冠词} % (fold)
\label{sub:冠词}

\subsubsection{定冠词} % (fold)
\label{ssub:定冠词}

\paragraph{dem} % (fold)
\label{par:dem}

阳性名词与格定冠词. \inlinehardlink{参考\cref{langex: an_dem}.}

% paragraph dem (end)

% subsubsection 定冠词 (end)

% subsection 冠词 (end)

\subsection{介词} % (fold)
\label{sub:介词}

\subsubsection{介词枚举} % (fold)
\label{ssub:介词枚举}

\begin{longtable}{|c|c|c}
    \cline{1-2}
    in & 在\ldots 内\\
    \cline{1-2}
    aus & 从\ldots 出来\\
    \cline{1-2}
    nach & 去\\
    \cline{1-2}
\end{longtable}

% subsubsection 介词枚举 (end)

\subsubsection{用法} % (fold)
\label{ssub:用法}

\paragraph{in} % (fold)
\label{par:in}

相当于in.
\gpattern{in + \gcomp{中性地名}.}
\begin{hardlink}
    中性地名用法参考\cref{langex:ich_wohne}.
\end{hardlink}
\begin{pitfall}
    对于阴性名词, 应当使用in der.
\end{pitfall}
\begin{sample}
    \begin{linguaex}
        Wo wohnen Sie? In der Schweiz.
    \end{linguaex}
\end{sample}

% paragraph in (end)

\paragraph{aus} % (fold)
\label{par:aus}

相当于from.
\gpattern{aus + \gcomp{中性地名}.}
\begin{hardlink}
    中性地名用法参考\cref{langex:ich_komme}.
\end{hardlink}
\begin{pitfall}
    对于阴性名词, 应当使用aus der.
\end{pitfall}
\begin{sample}
    \begin{linguaex}
        Woher kommen Sie? Aus der Schweiz.
    \end{linguaex}
\end{sample}

% paragraph aus (end)

\paragraph{nach} % (fold)
\label{par:nach}

相当于to.
\gpattern{nach + \gcomp{方向或不带冠词的地名}.}
\begin{sample}
    \begin{linguaex}
        \label{langex:wir_fahren_nach}
        Wir fahren nach S\"uden./Wie fahren nach Deutschland.
    \end{linguaex}
\end{sample}
\begin{pitfall}
    对于阴性地名不能使用nach, 应当使用in die.
\end{pitfall}
\begin{sample}
    \begin{linguaex}
        Wohin fahren Sie? In die Schweiz.
    \end{linguaex}
\end{sample}

% paragraph nach (end)

% subsubsection 用法 (end)

\subsubsection{缩合} % (fold)
\label{ssub:缩合}

\paragraph{an的缩合} % (fold)
\label{par:an的缩合}

an dem + (m.) $\Longrightarrow$ am \$1

\begin{sample}
    \begin{linguaex}
        \label{langex: an_dem}
        an dem Morgen $\Longrightarrow$ am Morgen
    \end{linguaex}
\end{sample}

% paragraph an的缩合 (end)

% subsubsection 缩合 (end)

% subsection 介词 (end)

\subsection{形容词} % (fold)
\label{sub:形容词}

\subsubsection{变格} % (fold)
\label{ssub:变格}

形容词做表语时无需变格, 做定语时根据所修饰的变格.

\begin{sample}
    \begin{linguaex}
        \label{langex: gut_und_guten}
        Guten Morgen $\longleftrightarrow$ Das Film ist gut.
    \end{linguaex}
\end{sample}

% subsubsection 变格 (end)

% subsection 形容词 (end)

\subsection{代词} % (fold)
\label{sub:代词}

\subsubsection{人称代词} % (fold)
\label{ssub:人称代词}

Sie的复数仍为Sie. du的复数为ihr.
\begin{longtable}{|c|c|c|c|}
    \hline
    \+:c2{|c|}{人称} & 单数 & 复数 \\
    \hline
    \+:c2{|c|}{第一} & ich & wir \\
    \hline
    \+:r2{第二} & 一般 & du & ihr \\
    & 敬称 & Sie & Sie \\
    \hline
    \+:r3{第三} & 阳性 & er & \+:r3{sie} \\
    & 阴性 & sie & \\
    & 中性 & es & \\
    \hline
\end{longtable}
\begin{pitfall}
    第三人称单数代词的性应从语法性而非自然性.
\end{pitfall}

% subsubsection 人称代词 (end)

\subsubsection{一般代词} % (fold)
\label{ssub:一般代词}

Das除了做冠词, 还可以表示「这个」.
\begin{sample}
    \begin{linguaex}
        Das ist Herr Miller./Das ist Tee aus China.
    \end{linguaex}
\end{sample}

% subsubsection 一般代词 (end)

% subsection 代词 (end)

\subsection{动词} % (fold)
\label{sub:动词}

\subsubsection{sein} % (fold)
\label{ssub:sein}

\begin{longtable}{|c|c|c|c|}
    \hline
    \+:c2{|c|}{人称} & 单数 & 复数 \\
    \hline
    \+:c2{|c|}{第一} & ich bin & wir sind \\
    \hline
    \+:r2{第二} & 一般 & du bist & ihr seid \\
    & 敬称 & Sie sind & Sie sind \\
    \hline
    \+:c2{|c|}{第三}  & er/sie/es ist & sie sind \\
    \hline
\end{longtable}

% subsubsection sein (end)

% subsection 动词 (end)

\subsection{陈述} % (fold)
\label{sub:陈述}

\subsubsection{陈述句语序} % (fold)
\label{ssub:陈述句语序}

\begin{pitfall}
    陈述句在任何情形下动词皆出现于第二位. 特别地, auch不得提于动词前.
\end{pitfall}


% subsubsection 陈述句语序 (end)

% subsection 陈述 (end)

\subsection{询问} % (fold)
\label{sub:询问}

\subsubsection{疑问词} % (fold)
\label{ssub:疑问词}

\paragraph{Wie} % (fold)
\label{par:wie}
相当于how.
\begin{sample}
    \begin{linguaex}
        \label{langex:wie_heissen}
        Wie hei\ss en Sie?\inlinehardlink{回答参考\cref{langex:ich_heisse}}/Wie ist Shanghai?
    \end{linguaex}
\end{sample}

% paragraph wie (end)

\paragraph{Woher} % (fold)
\label{par:woher}
相当于from where, 用aus回答.
\begin{sample}
    \begin{linguaex}
        Woher kommen Sie?\inlinehardlink{回答参考\cref{langex:ich_komme}}/Woher kommen die Produkte?
    \end{linguaex}
\end{sample}

% paragraph woher (end)

\paragraph{Wo} % (fold)
\label{par:wo}
相当于where, 用in回答.
\begin{sample}
    \begin{linguaex}
        Wo wohnen Sie?\inlinehardlink{回答参考\cref{langex:ich_wohne}}/Wo bist du?
    \end{linguaex}
\end{sample}

% paragraph wo (end)

\paragraph{Wohin} % (fold)
\label{par:wohin}

相当于to where, 用nach回答.
\begin{sample}
    \begin{linguaex}
        Wohin fahren Sie?\inlinehardlink{回答参考\cref{langex:wir_fahren_nach}}.
    \end{linguaex}
\end{sample}

% paragraph wohin (end)

\paragraph{Wer} % (fold)
\label{par:wer}

相当于who.
\begin{sample}
    \begin{linguaex}
        Wer ist Fotograf.
    \end{linguaex}
\end{sample}

% paragraph wer (end)

\paragraph{Was} % (fold)
\label{par:was}

相当于what.
\begin{sample}
    \begin{linguaex}
        Was machen Sie?/Was ist das?/Was sind Sie?
    \end{linguaex}
\end{sample}
特别地, 询问职业用
\gpattern{Was sind Sie?}

% paragraph was (end)

% subsubsection 疑问词 (end)

\subsubsection{W-Frage} % (fold)
\label{ssub:w_frage}

特殊疑问句之形式如
\gpattern{\gcomp{W} + \gcomp{V} + \gcomp{S} + \ldots?}
发音用降调.
\begin{hardlink}
    参考\cref{langex:wie_heissen}及以下.
\end{hardlink}
% subsubsection w_frage (end)

\subsubsection{Ja-/Nein-Frage} % (fold)
\label{ssub:ja_nein_frage}

一般疑问句之形式如
\gpattern{\gcomp{V} + \gcomp{S} + \ldots?}
发音用升调.
\begin{sample}
    \begin{linguaex}
        Hei\ss en Sie Bauer?/Kommen Sie aus Shanghai?/Wohnen Sie in Beijing?
    \end{linguaex}
\end{sample}
其回答类似英语, 如
\begin{sample}
    \begin{linguaex}
        Liegt China auch in Europa? Nein, in Asien.
    \end{linguaex}
\end{sample}

% subsubsection ja_nein_frage (end)

% subsection 询问 (end)

\subsection{祈使句} % (fold)
\label{sub:祈使句}

\subsubsection{形式} % (fold)
\label{ssub:形式}

祈使句形式如
\gpattern{\gcomp{V} + Sie bitte!}
\begin{sample}
    \begin{linguaex}
        H\"oren Sie bitte!/Sprechen Sie bitte!
    \end{linguaex}
\end{sample}

% subsubsection 形式 (end)

% subsection 祈使句 (end)

% section 语法 (end)

\section{词汇} % (fold)
\label{sec:词汇}

\subsection{动词} % (fold)
\label{sub:动词}

\subsubsection{用法} % (fold)
\label{ssub:用法}

\paragraph{finden} % (fold)
\label{par:finden}

找到.
\begin{sample}
    \begin{linguaex}
        In Deutschland finden Sie Tee aus China.
    \end{linguaex}
\end{sample}

% paragraph finden (end)

\paragraph{spielen} % (fold)
\label{par:spielen}

玩. 通常搭配球类、乐器. 而出去玩不能用spielen.

% paragraph spielen (end)

\paragraph{arbeiten} % (fold)
\label{par:arbeiten}

工作, 也可以表示学习.

% paragraph arbeiten (end)

\paragraph{reisen} % (fold)
\label{par:reisen}

旅游, 通常用作
\gpattern{reisen + nach + \gcomp{地名}.}

% paragraph reisen (end)

\paragraph{fehlen} % (fold)
\label{par:fehlen}

缺少, 主语为缺少的东西.

% paragraph fehlen (end)

% subsubsection 用法 (end)

% subsection 动词 (end)

% section 词汇 (end)

\clearpage

\section*{词汇表}

\subsection*{通用}

\begin{longtable}{lllll}
    \gvocabcline
    der Morgen & - & 早晨 & m. \\
    \gvocabcline
    der Tag & -e & 天, 白天 & m. \\
    \gvocabcline
    der Abend & -e & 晚上 & m. \\
    \gvocabcline
    gut & & 好 & adj. & Der Film ist gut. \\
    \gvocabcline
    die Anrede & -n & 称呼 & f. \\
    \gvocabcline
    die Frau & -en & 女士 & f. \\
    \gvocabcline
    der Herr & -en & 先生 & m. \\
    \gvocabcline
    def Name & -n & 姓名 & m. & Anna Bauer ist ein Name \\
    \gvocabcline
    der Familienname & -n & 姓 & m. \\
    \gvocabcline
    def Vorname & -n & 名 & m. \\
    \gvocabcline
    ich & & 我 & Pron. \\
    \gvocabcline
    mein & & 我的 & Pron. \\
    \gvocabcline
    hei\ss en & & 叫, 称 & Vi. \\
    \gvocabcline
    sein & & 是 & Vi. \\
    \gvocabcline
    kommen & & 来 & Vi.\\
    \gvocabcline
    wohnen & & 住 & Vi.\\
    \gvocabcline
    die Frage & -n & 问题 & f.\\
    \gvocabcline
    der Antwort & -en & 回答 & f.\\
    \gvocabcline
    h\"oren & & 听 & Vt. & Ich h\"ore Musik. \\
    \gvocabcline
    sprechen & & 说 & Vt. & Ich spreche Deutsch und Englisch. \\
    \gvocabcline
    lesen & & 读 & Vt. & Lesen Sie bitte! \\
    \gvocabcline
    schreiben & & 写 & Vt. & Schreiben Sie bitte Dialoge.\\
    \gvocabcline
    die Sprache & & 语言 & f. & Ich spreche viele Sprachen. \\
    \gvocabcline
    markienren & & 标出 & Vt. & Markieren Sie bitte! \\
    \gvocabcline
    nummerieren & & 编号 & Vt. & Nummerieren Sie bitte! \\
    \gvocabcline
    die Welt & -en & 世界 & f. \\
    \gvocabcline
    die Karte & -n & 卡片 & f. \\
    \gvocabcline
    die Weltkarte & -n & 世界地图 & f. \\
    \gvocabcline
    dir Landkarte & -n & 国家地图 & f. \\
    \gvocabcline
    der Kontinent & -e & 州 & m. \\
    \gvocabcline
    das Land & \"{}er & 国家 & n. \\
    \gvocabcline
    das Alphabet & -e & 字母表 & n. & H\"oren Sie das Alphabet. \\
    \gvocabcline
    der Mensch & -en & 人 & m. & Viele Menschen sprechen Deutsch. \\
    \gvocabcline
    der Text & -e & 课文 & m. & Wie lesen die Texte. \\
    \gvocabcline
    hier & & 这里 & Adv. & Ich bin hier. \\
    \gvocabcline
    auch & & 也 & Adv. & Ich hei\ss e auch Wang.\\
    \gvocabcline
    oder & & 或者 & Konj. & Sprechen Sie Deutsch oder Englisch? \\
    \gvocabcline
    noch & & 还 & Adv. & Ich wohne noch in Guangzhou.\\
    \gvocabcline
    mehr & & 更多 & Adj. & Ich spreche mehr Sprachen.\\
    \gvocabcline
    \+:r2{liegen} & & 平躺 & \+:r2{Vi.} & Ich liege hier. \\
    & & 位于 & & China liegt in Asien.\\
    \gvocabcline
    suchen & & 寻找 & Vt. & Was suchen Sie?\\
    \gvocabcline
    finden & & 找到 & Vt. & Wo finde ich die Toiletten?\\
    \gvocabcline
    kennen & & 认识 & Vt. & Kennst du Claudia?\\
    \gvocabcline
    lernen & & 学习 & Vt. & Ich lerne auch Deutsch. \\
    \gvocabcline
    weiter/machen & & 接着做 & Vi. & Machen Sie weiter.\\
    \gvocabcline
    das Produkt & -e & 产品 & n. & Woher kommen die Produkte? \\
    \gvocabcline
    der Jaffee & Sg. & 咖啡 & m. & \+:r4{\parbox{4cm}{酒精饮料大多为阳性, 除了啤酒. 此外, 饮料通常用单数, 复数仅用于种类.}} \\
    \gvocabcline
    der Tee & Sg. & 茶 & m. \\
    \gvocabcline
    das Bier & -e & 啤酒 & n. \\
    \gvocabcline
    der Wein & -e & 葡萄酒 & m. \\
    \gvocabcline
    die Banane & -n & 香蕉 & f. \\
    \gvocabcline
    die Tomate & -n & 西红柿 & f. \\
    \gvocabcline
    die Zitrone & -n & 柠檬 & f. \\
    \gvocabcline
    die Schokolade & Sg. & 巧克力 & f. \\
    \gvocabcline
    die Zucker & Sg. & 砂糖 & f. \\
    \gvocabcline
    das Auto & -s & 汽车 & n. \\
    \gvocabcline
    das Foto & -s & 照片 & n. \\
    \gvocabcline
    der Apparat & -e & 一起 & m. \\
    \gvocabcline
    der Fotoapparat & -e 照相机 & m. \\
    \gvocabcline
    der Computer & - & 计算机 & m. \\
    \gvocabcline
    der Zug & \"{}e & 火车 & m. \\
    \gvocabcline
    der Norden & Sg. & 北 & m. \\
    \gvocabcline
    der Westen & Sg. & 西 & m. \\
    \gvocabcline
    der S\"uden & Sg. & 南 & m. \\
    \gvocabcline
    def Osten & Sg. & 东 & m. \\
    \gvocabcline
    mitten & & 正中 & Adv. \\
    \gvocabcline
    mitten in & & 位于\ldots 中部 & & Deutschland liegt mitten in Europa. \\
    \gvocabcline
    jeden Tag & & 每天 & & Jeden Tag lerne ich Deutsch. \\
    \gvocabcline
    fahren & & 开车, 坐车 & Vt. & Ich fahre nach Frankfurt. \\
    \gvocabcline
    vielleicht & & 也许 & Adv. & Hans kommt vielleicht aus Berlin. \\
    \gvocabcline
    kombinieren & & 连接 & Vt. & Kombinieren Sie bitte. \\
    \gvocabcline
    das Beispiel & -e & 例子 & n. & Lesen Sie Beispiele. \\
    \gvocabcline
    zum Beispiel (z.B.) & & 例如 & & \\
    \gvocabcline
    der Punkt & -e & 点, 句号 & m. \\
    \gvocabcline
    das Fragezeichen & - & 问号 & n. \\
    \gvocabcline
    die Situation & -en & 情景 & n. \\
    \gvocabcline
    richtig & & 正确的 & Adj. & Die Antwort ist richtig. \\
    \gvocabcline
    falsch & & 错误的 & Adj. & Die Antwort ist falsch. \\
    \gvocabcline
    schlafen & & 睡觉 & Vi. & Herr Meier Schl\"aft. \\
    \gvocabcline
    spielen & & 玩 & Vi. & Wie spielen Karten. \\
    \gvocabcline
    der Urlaub & -e & 假期 & m. \\
    \gvocabcline
    machen & & 做 & Vt. \\
    \gvocabcline
    arbeiten & & 工作 & Vi. & Ich arbeite in Shanghai. \\
    \gvocabcline
    reisen & & 旅游 & Vi. & Martin reist nach Wien. \\
    \gvocabcline
    sehr & & 非常 & Adv. & Der Wein ist sehr gut. \\
    \gvocabcline
    viel & & 多 & Adv.& Ich reise sehr viel. \\
    \gvocabcline
    heute & & 今天 & Adv. & Heute arbeite ich. \\
    \gvocabcline
    morgen & & 明天 & Adv. & Morgen arbeite ich nicht. \\
    \gvocabcline
    der Journalist & -en & 记者 & & Martin Miller ist Journalist. \\
    \gvocabcline
    schon & & 已经 & & Ich lerne schon Deutsch. \\
    \gvocabcline
    verstehen & & 理解 & & Verstehen Sie Deutsch? \\
    \gvocabcline
    ein bisschen & & 一点点  & & Ich spreche ein bisschen Deutsch. \\
    \gvocabcline
    S\"uddeutschland & & 南德 \\
    \gvocabcline
    Norddeutschland & & 北德 \\
    \gvocabcline
    Westdeutschland & & 南德 \\
    \gvocabcline
    Ostdeutschland & & 南德 \\
    \gvocabcline
    der Fotograf & -en & 男摄影师 \\
    \gvocabcline
    die Fotografin & -nen & 女摄影师 \\
    \gvocabcline
    fragen & & 问 & Vt. & Ich frage. \\
    \gvocabcline
    antworten & & 回答 & Vt. & Antworten Sie bitte. \\
    \gvocabcline
    aber & & 但是 & Konj. & Ich spreche Deutsch, aber nur ein. \\
    \gvocabcline
    der Satz & \"{}e & 句子 & m. & Lesen Sie bitte die S\"atze. \\
    \gvocabcline
    der Vokal & -e & 元音 & m. \\
    \gvocabcline
    kurz & & 短的 & Adj. & Der Satz ist kurz. \\
    \gvocabcline
    lang & & 长的 & Adj. & Der Vokal ist lang. \\
    \gvocabcline
    fehlen & & 缺少 & Vi. & Was fehlt? \\
    \gvocabcline
    Wie bitte? & & 再说一遍? & \\
    \gvocabcline
    leider & & 可惜 & Adv. & Anna kommt leider nicht. \\
    \gvocabcline
    dann & & 然后 & Adv. & Wir h\"oren Texte. Dann schreiben wir S\"tze. \\
    \gvocabcline
    noch & & 还 & Adv. & Wir fahren noch nach Wien. \\
    \gvocabcline
\end{longtable}

\begin{finale}
    auch总是放在欲强调部分之前.
\end{finale}
\begin{finale}
    suchen和finden之区别在于suchen重过程, finden重结果.
\end{finale}
\begin{finale}
    -tion, -ssion, -sion结尾的单词都是阴性, 复数多为-en.
\end{finale}
\begin{pitfall}
    morgen(明天)是小写开头, 而der Morgen(早上)是大写开头.
\end{pitfall}

\subsection*{地名}

\begin{longtable}{cccc}
    Afrika & 非洲 & \"Osterreich & 奥地利\\
    Amerika & 美洲 & Polen & 波兰\\
    Asien & 亚洲 & Russland & 俄罗斯\\
    Australien & 大洋洲 & die Schweiz & 瑞士\\
    Europa & 欧洲 & Spanien & 西班牙\\
    Antarktis & 南极洲 & Tunesien & 突尼斯\\
    Argentinien & 阿根廷 & Ungarn & 匈牙利\\
    Belgien & 比利时 & Vietnam & 越南\\
    China & 中国 & Zypern & 塞浦路斯 \\
    D\"anemark & 丹麦 & Kuba & 古巴\\
    Deutschland & 德国 & Italien & 意大利\\
    Ecuador & 厄瓜多尔\\
    Frankreich & 法国\\
    Gro\ss britannien & 大不列颠\\
    Honduras & 洪都拉斯\\
    Indien & 印度\\
    Japan & 日本\\
    Kennia & 肯尼亚\\
    Luxemburg & 卢森堡\\
    Marokko & 摩洛哥\\
    Norwegen & 挪威\\
    Oman & 阿曼\\
\end{longtable}
\begin{finale}
    少数国家(如die Schweiz)非中性.
\end{finale}

\end{document}
