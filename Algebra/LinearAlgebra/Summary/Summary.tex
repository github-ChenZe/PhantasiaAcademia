\documentclass{ctexart}

\usepackage{van-de-la-sehen}
\usepackage{lipsum}

\begin{document}

\section{线性空间, 线性变换, 欧氏空间, 二次型} % (fold)
\label{sec:线性空间_线性变换_欧氏空间_二次型}

\begin{theorem}
    向量在基$\vB$下的坐标为$X$, 则在新的基$\vC = \vB P$下坐标为
    \[ X' = P^{-1}X. \]
\end{theorem}

\hrule

\begin{paracol}{2}

    \begin{theorem}
        设线性空间有基
        \[ \vB=\pare{v_1,v_2,\cdots,v_n}. \]
        线性变换$\sA$在基$\vB$下的矩阵$\cA$为
        \[ \begin{pmatrix}
            \vert & \vert & \cdots & \vert \\
            \sA v_1 & \sA v_2 & \cdots & \sA v_n \\
            \vert & \vert & \cdots & \vert
        \end{pmatrix}. \]
    \end{theorem}

\switchcolumn

    \begin{theorem}
        设Euclid空间有基
        \[ \vB=\pare{v_1,v_2,\cdots,v_n}. \]
        二次型$Q$在基$\vB$下的矩阵$\cA$为
        \[ \begin{pmatrix}
            \bra{v_1}\ket{v_1} & \bra{v_1}\ket{v_2} & \cdots & \bra{v_1}\ket{v_n} \\
            \bra{v_2}\ket{v_1} & \bra{v_2}\ket{v_2} & \cdots & \bra{v_2}\ket{v_n} \\
            \vdots & \vdots & \ddots & \vdots \\
            \bra{v_n}\ket{v_1} & \bra{v_n}\ket{v_2} & \cdots & \bra{v_n}\ket{v_n}
        \end{pmatrix}. \]
    \end{theorem}

\end{paracol}

\hrule

\begin{paracol}{2}

    \begin{theorem}
        设线性变换在基$\vB$下的矩阵为$\cA$, 在基
        \[ \vC = \vB P \]
        下的矩阵为$\cA'$, 则
        \[ \cA' = P^{-1}\cA P. \]
    \end{theorem}

    \switchcolumn

    \begin{theorem}
        设二次型在基$\vB$下的矩阵为$\cA$, 在基
        \[ \vC = \vB P \]
        下的矩阵为$\cA'$, 则
        \[ \cA' = P^T \cA P. \]
    \end{theorem}

\end{paracol}

\hrule

\begin{paracol}{2}
    
    \begin{theorem}
        线性变换属于不同特征值的向量
        \[ \ket{\lambda_1}, \ket{\lambda_2}, \cdots, \ket{\lambda_k} \]
        是线性无关的.
    \end{theorem}

    \switchcolumn

    \begin{theorem}
        对称矩阵属于不同特征值的向量
        \[ \ket{\lambda_1}, \ket{\lambda_2}, \cdots, \ket{\lambda_k} \]
        是彼此正交的.
    \end{theorem}

\end{paracol}

\hrule

\begin{paracol}{2}

    \begin{theorem}
        对复方阵$\cA$, 存在可逆矩阵$P$使得
        \[ P^{-1}\cA P = \Lambda \]
        为对角阵的充要条件为代数重数与几何重数相等.
    \end{theorem}
    
    \switchcolumn

    \begin{theorem}
        对对称矩阵$\cA$, 总存在正交矩阵$P$使得
        \[ P^{-1}\cA P = \Lambda \]
        为对角阵.
    \end{theorem}

\end{paracol}

% section 线性空间_线性变换_欧氏空间_二次型 (end)



\end{document}