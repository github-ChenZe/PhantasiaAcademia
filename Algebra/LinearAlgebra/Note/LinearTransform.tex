\documentclass[hidelinks]{ctexart}

\usepackage[margintoc, singleton]{van-de-la-sehen}

\begin{document}

\showtitle{线性变换}

\section{线性变换与矩阵} % (fold)
\label{sec:线性变换与矩阵}

\subsection{矩阵的对角化} % (fold)
\label{sub:矩阵的对角化}

\subsubsection{对角化判准} % (fold)
\label{ssub:对角化判准}

\begin{finale}
    $A$ of $n\times n$ is similar to a diagonal matrix iff $A$ has $n$ eigenvec of linear independent.
\end{finale}
\begin{corollary}[crit for diag]
    \begin{cenum}
        \item if $A$ has diff $n$ eigenvals;
        \item solve for eigensubsp of $\lambda_i$s'.
    \end{cenum}
\end{corollary}
Alg index defined as multiplixity of $\lambda$ as a root. Geo index defined as dim the space ass.
\begin{finale}
    diagonalizable iff Alg index equals to Geo index of each $\lambda_i$.
\end{finale}

% subsubsection 对角化判准 (end)

% subsection 矩阵的对角化 (end)

\subsection{不变子空间} % (fold)
\label{sub:不变子空间}

\subsubsection{核与像} % (fold)
\label{ssub:核与像}

\begin{proposition}
    $\ker\+sA=\curb{0}\Leftrightarrow \img\+sA = V \Leftrightarrow \+sA$为双射.
\end{proposition}
\begin{proposition}
    $\+sA$可逆当且仅当$A$可逆.
\end{proposition}
\begin{definition}
    [线性同构]线性双射谓线性同构.
\end{definition}

% subsubsection 核与像 (end)

% subsection 不变子空间 (end)

% section 线性变换与矩阵 (end)

\end{document}
