\documentclass[hidelinks]{ctexart}

\usepackage[singleton, margintoc]{van-de-la-sehen}

\begin{document}

\showtitle{二次型}

\section{正定矩阵} % (fold)
\label{sec:正定矩阵}

\subsection{正定矩阵的性质} % (fold)
\label{sub:正定矩阵的性质}

\subsubsection{正定矩阵的判定} % (fold)
\label{ssub:正定矩阵的判定}

\begin{theorem}[Sylvester判准]
    对称矩阵$A$是正定的当且仅当$A$的各阶顺序主子式大于零.
\end{theorem}

\begin{proposition}
    对称的$A$正定当且仅当$A$的所有特征值大于零.
\end{proposition}

\begin{pitfall}
    不能认为「所有特征值大于零」的矩阵都是正定的 --- 需要保证对称.
\end{pitfall}

% subsubsection 正定矩阵的判定 (end)

\subsubsection{正定矩阵的分解} % (fold)
\label{ssub:正定矩阵的分解}

\begin{ex}
    设$A$正定, 证明存在正定矩阵使得$A=P^2$.
\end{ex}
\begin{proof}
    $\lambda_i>0$, $\exists Q\in SO$ s.t. $A = Q^T \diag\curb{\lambda_i} Q$ $\Rightarrow$ $\sqrt{A} = Q^T \diag\curb{\sqrt{\lambda_i}} Q$正定, $A = \sqrt{A}^2$.
\end{proof}
\begin{ex}
    设$A$和$B$正定, 则
    \[ \det \pare{AB} \le \pare{\rec{n}\tr \pare{AB}}^{1/n}. \]
\end{ex}
\begin{proof}
    设$A=P^T P$, $B = Q^T Q$, 则$\det \pare{AB} = \det \pare{PQ^T QP^T}$且$\tr \pare{AB} = \tr \pare{PQ^T QP^T}$, 且$PQ^TQP^T=\pare{PQ^T}\pare{PQ^T}^T$, 引用AG不等式即可.
\end{proof}

% subsubsection 正定矩阵的分解 (end)

% subsection 正定矩阵的性质 (end)

\subsection{半定和负定矩阵} % (fold)
\label{sub:半定和负定矩阵}

\subsubsection{负定矩阵} % (fold)
\label{ssub:负定矩阵}

\begin{pitfall}
    不能认为负定矩阵的Sylvester准则为「\sout{各阶顺序主子式小于零}」.
\end{pitfall}

\begin{proposition}
    $A$正定当且仅当$-A$负定.
\end{proposition}
\begin{proposition}
    $A$正定当且仅当负惯性指数为零.
\end{proposition}

% subsubsection 负定矩阵 (end)

\subsubsection{半定矩阵} % (fold)
\label{ssub:半定矩阵}

\begin{theorem}[Sylvester判准]
    对称矩阵$A$是半正定的当且仅当$A$的各阶主子式非负.
\end{theorem}
\begin{proposition}
    对称矩阵$A$半正定当且仅当$A=P^T P$, $P$对称但不可逆.
\end{proposition}
\begin{pitfall}
    半正定的Sylvester判准要求各阶主子式, 而非仅仅顺序主子式.
\end{pitfall}
\begin{proof}[半正定矩阵各个判准的证明]
    先证正定$\Leftrightarrow$存在不可逆的$P$满足$A=P^TP$,
    后证正定$\Leftrightarrow$Sylvester条件.
    \begin{cenum}
        \item 若$A$半正定, 则存在可逆的$P$满足
        \[ A = P^T I_r I_r P, \]
        考虑$I_r P$即可.
        \item 若$A=P^TP$, 则
        \[ v^T A v = v^TP^T Pv = \norm{Pv}^2 \ge 0. \]
        $P$不可逆, 故存在非零的$v$使等号成立.
        \item 若$A$半正定, $Q_k\pare{x_1,\cdots,x_k} = Q\pare{x_1,\cdots,x_k,0,\cdots}$为正定或半正定的, 归纳可知.
        \item 若$A$的主子式皆非负, 考虑
        \[ D_n = \det\pare{\lambda I +A} = \+m[
        | \lambda+a_{12} | a_{12}         | \cdots | a_{1n}         |
        | a_{21}         | \lambda+a_{22} | \cdots | a_{2n}         |
        | \vdots         | \vdots         | \ddots | \vdots         |
        | a_{n1}         | a_{n2}         | \cdots | \lambda+a_{nn} |
        ],
        \]
        各行加零后拆分, 有$2^n$个行列式. $D\pare{i_1,i_2,\cdots,i_k}$即为$2^k$个行列式中的一个, 例如
        \leavevmode \InsertBoxR{2}{\parbox{4cm}{\begin{mtips}
            如果将行列式完全按列(或者行)展开, 就会得到那$2^n$个项.
        \end{mtips}}}[1]
        \[ D\pare{1,2,3} = \det\+m[
        | a_{11} | a_{12} | a_{13} | 0       |
        | a_{21} | a_{22} | a_{23} | 0       |
        | a_{31} | a_{32} | a_{33} | 0       |
        | a_{41} | a_{42} | a_{43} | \lambda |
        ],
        \]
        \leavevmode
        \[ D\pare{1,3} = \det\+m[
        | a_{11} | 0       | a_{13} | 0       |
        | a_{21} | \lambda | a_{23} | 0       |
        | a_{31} | 0       | a_{33} | 0       |
        | a_{41} | 0       | a_{43} | \lambda |
        ] \]
        显然得到$A$的主子式. 取$A$的$\curb{i_1,i_2,\cdots,i_k}$列,
        \[ \det\pare{\lambda I_k +A_k} = \lambda^k + \sum_{t=1}^k\sum_{t_1<t_2<\cdots<t_k} D\pare{i_1,i_2,\cdots,i_k}. \]
        由上面的观察,
        \[ \det\pare{\lambda I_k + A_k} = \lambda^k + \sum_{t=1}^k\lambda^{k-t} \sum D_t, \]
        求和号内为$\le k$阶的主子式的和. 故只有$\lambda \le 0$的解.
        \qedhere
    \end{cenum}
\end{proof}
\begin{corollary}
    对$\lambda>0$, $\lambda I+A$正定, 则$A$半正定.
\end{corollary}

% subsubsection 半定矩阵 (end)

% subsubsection 定义 (end)

% subsection 半定和负定矩阵 (end)

% section 正定矩阵 (end)


\end{document}
