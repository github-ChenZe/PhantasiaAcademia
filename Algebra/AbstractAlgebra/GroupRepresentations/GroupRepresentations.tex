\documentclass[hidelinks]{article}

\usepackage{ctex}
\usepackage{van-de-la-illinoise}

\DeclareMathOperator{\GL}{GL}
\let\oldgloss\gloss
\def\gloss#1{\textnormal{\textbf{\oldgloss{#1}}}}
\DeclareMathOperator{\vol}{vol}
\DeclareMathOperator{\im}{im}

\begin{document}

\section{Group Representations} % (fold)
\label{sec:group_representations}

\subsection{Definitions} % (fold)
\label{sub:definitions}

\begin{definition}
    A \gloss{matrix representation} of a group $G$ is a homomorphism
    \[ \func{R}{G}{\GL_n} \]
    from $G$ to one of the complex general linear groups. The number $n$ is the \gloss{dimension} of the representation.
\end{definition}
With $R_g$ denoting the image of a group element $g$, we have
\[ R_{gh} = R_g R_h. \]
With $c = \cos 2\pi/3$ and $s = \sin 2\pi/3$, we found that
\[ A_x = \begin{pmatrix}
    c & -s \\
    s & c
\end{pmatrix},\quad A_y = \begin{pmatrix}
    1 & 0 \\
    0 & -1
\end{pmatrix} \]
satisfy all the generator relations of $D_3$ and $S_3$. We call this the \gloss{standard representation} of $D_3$ and $S_3$.
\begin{definition}
    A representation $R$ is \gloss{faithful} if the homomorphism $\func{R}{G}{\GL_n}$ is injective, and therefore maps $G$ isomorphically to its image, a subgroup of $\GL_n$.
\end{definition}
The standard representation of $S_3$ is faithful.
\par
The \gloss{sign representation} of $S_3$ is generated by
\[ \Sigma_x = \brac{1},\quad \Sigma_y = \brac{-1}, \]
which is not faithful.
\par
Every group has the \gloss{trivial representation}, taking value $1$ identically, i.e.
\[ T_x = \brac{1},\quad T_y = \brac{1}. \]
\begin{definition}
    The \gloss{character} $\chi_R$ of a matrix representation $R$ is the complex-valued function whose domain is the group $G$, defined by $\chi_R\pare{g} = \trace R_g$.
\end{definition}
\begin{ex}
    The characters of $S_3$ is listed below.
    \[ \begin{array}{c|cccccc}
        & 1 & x & x^2 & y & xy & x^2y \\
        \hline
        \chi_T & 1 & 1 & 1 & 1 & 1 & 1 \\
        \chi_\Sigma & 1 & 1 & 1 & -1 & -1 & -1 \\
        \chi_A & 2 & -1 & - 1 & 0 & 0 & 0
    \end{array} \]
\end{ex}
\hl{$\chi_R\pare{1}$ is the dimension of the representation,} also called the \gloss{dimension} of the character.
\par
\hl{The characters are constant on conjugacy classes.}
\begin{definition}
    A \gloss{representation} of a group $G$ on a complex vector space $V$ is a homomorphism
    \[ \func{\rho}{G}{\GL\pare{V}}, \]
    where $\GL\pare{V}$ is the group of invertible linear operators on $V$, and $V$ is a finite-dimensional complex vector space, and not the zero space.
\end{definition}
The \gloss{standard representation} of the finite rotation group (finite subgroup of $SO_3$) is the mapping to their corresponding orthogonal operators.
\par
With $\rho_g$ denoting the image of a group element $g$, we have
\[ \rho_{gh} = \rho_g \rho_h. \]
The choice of a basis $\mathbf{B} = \pare{v_1,\cdots,v_n}$ for a vector space $V$ defines an isomorphism from $\GL\pare{V}$ to $\GL_n$,
\[ \begin{array}{ccc}
    \GL\pare{V} & \mapsto & \GL_n \\
    T & \mapsto & \text{matrix of } T \\
    \rho_g & \mapsto & R_g.
\end{array} \]
\par
A change of basis in $V$ by a matrix $P$ changes the matrix representation $R$ associated to $\rho$ to a \gloss{conjugate representation} $R' = P^{-1}RP$, i.e.
\[ R'_g = P^{-1}R_g P. \]
\par
An \gloss{operation} of a group $G$ \gloss{by linear operators} on a vector space $V$ is an operation on the underlying set,
\[ 1v = 1,\quad \pare{gh}v = g\pare{hv}, \]
and inaddition every group element acts as a linear operator,
\[ g\pare{v+v'} = gv + gv',\quad g\pare{cv} = cgv. \]
Given a representation $\rho$ of $G$ on $V$, we can define an operation of $G$ on $V$ by
\[ gv = \rho_g\pare{v}. \]
Conversely, given an operation, the same formula can be used to define the operator $\rho_g$.
\par
\begin{definition}
    An \gloss{isomorphism} from one representation $\func{\rho}{G}{\GL\pare{V}}$ of a group $G$ to another representation $\func{\rho'}{G}{\GL\pare{V'}}$ is an isomorphism of vector spaces $\func{T}{V}{V'}$, an invertible linear transformation, that is compatible with the operations of $G$:
    \[ T\pare{gv} = gT\pare{v}. \]
\end{definition}
\hl{If $\func{T}{V}{V'}$ is an isomorphism and if $\+vB$ and $\+vB'$ are corresponding bases of $V$ and $V'$, the assocaited matrix representations $R_g$ and $R'_g$ will be equal.}

% subsection definitions (end)

\subsection{Irreducible Representations} % (fold)
\label{sub:irreducible_representations}

\begin{definition}
    A vector $v$ is $G$-\gloss{invariant} if the operation of every group element fixes the vector:
    \[ \rho_g\pare{v} = gv = v,\quad \forall g\in G. \]
\end{definition}
The $G$-inveriant averaged vector is
\[ \resumath{\tilde{v} = \rec{\abs{G}}\sum_{g\in G}gv.} \]
\begin{definition}
    A subspace $W\subset V$ is called $G$-\gloss{invariant} if $gw$ is in $W$ for all $w\in W$ and $g$ in $G$, i.e.
    \[ \rho_g W = gW \subset W,\quad \forall g. \]
\end{definition}
\begin{lemma}
    If $W$ is an invariant subspace of $V$, then $gW = W$ for all $g\in G$.
\end{lemma}
\begin{proof}
    $gW$ is also invariant, as $hgw = g\pare{h'gw} \in gW$. Therefore $g^{-1}gW \subset gW \subset W$, hence $gW = W$.
\end{proof}
\begin{definition}
    If $V$ is the direct sum of $G$-invariant subspaces, say $V = W_1 \oplus W_2$, the representation on $V$ is called the \gloss{direct sum} of its restrictions to $W_1$ and $W_2$, and we write
    \[ \rho = \alpha \oplus \beta, \]
    where $\alpha$ and $\beta$ are the restrictions of $\rho$ to $W_1$ and $W_2$, respectively.
\end{definition}
In this case, the matrix of $\rho_g$ will have the form
\[ R_g = \begin{pmatrix}
    A_g & 0 \\
    0 & B_g
\end{pmatrix}. \]
A similar definition of direct sum goes for matrix representations.
\begin{ex}
    $S_3$ is isomorphic to $D_3 \subset SO_3$, which has the three-dimensional matrix reprentation $M$:
    \[ M_x = \begin{pmatrix}
        c & -s \\
        s & c \\
        & & 1
    \end{pmatrix},\quad M_y = \begin{pmatrix}
        1 \\ & -1  \\ & & -1
    \end{pmatrix}, \]
    where $c = \cos 2\pi/3$ and $s = \sin 2\pi/3$. This is the direct sum $A\oplus \Sigma$.
\end{ex}
\begin{definition}
    If $\rho$ is a representation of a group $G$ on $V$ and if $V$ has no proper $G$-invariant subspace, $\rho$ is called an \gloss{irreducible} representation. If $V$ has a proper G-invariant subspace, $\rho$ is \gloss{reducible}.
\end{definition}
The standard representation of $S_3$ is irreducible.
\begin{remark}
    The matrices in $SO_2$ have eigenvectors in $\+bC^2$. But in order for the presentation to be reducible, a common eigenvector should be shared by all $\rho_g$ in this case, which doesn't exist.
\end{remark}
\begin{resume}
\begin{theorem}[Maschke]
    Every representation of a finite group $G$ on a nonzero, finite-dimensional complex vector space is a direct sum of irreducible represtentations.
\end{theorem}
\end{resume}

% subsection irreducible_representations (end)

\subsection{Unitary Representations} % (fold)
\label{sub:unitary_representations}

Let $V$ be a Hermitian space with a positive definite Hermitian form $\expc{,}$. A unitary operator $T$ on $V$ is a linear operator with the property
\[ \expc{Tv,Tw} = \expc{v,w},\quad \forall v,w\in V. \]
$T$ is unitary if and only if its matrix with respect to and orthonormal basis is a unitary matrix: $A^* = A^{-1}$.
\begin{definition}
    A representation $\func{\rho}{G}{\GL\pare{V}}$ on a Hermitian space $V$ is call \gloss{unitary} if $\rho_g$ is a unitary operator for every $g$, i.e.
    \[ \expc{\rho_gv,\rho_gw} = \expc{gv,gw} = \expc{v,w},\quad \forall v,w\in V,\quad \forall g\in G. \]
    In this case, we also say that the form is $G$-\gloss{invariant}.
\end{definition}
A similar definition goes for matrix representations. A unitary matrix representation is a homomorphism
\[ \func{R}{G}{U_n}. \]
\begin{lemma}
    Let $\rho$ be a unitary representation of $G$ on a Hermitian space $V$, let $W$ be a G-inveriant subspace. The orthogonal complement $W^\perp$ is also G-inveriant, and $\rho$ is the direct sum of ites restrictions to the Hermitian spaces $W$ and $W^\perp$. These restrictions are also unitary representations.
\end{lemma}
\begin{proof}
    It is true that $V = W\oplus W^\perp$. Since $\rho$ is unitary, it preserves orthogonality: If $W$ is invariant and $u\perp W$, then $gu\perp gW = W$.
\end{proof}
\begin{corollary}
    Every unitary representation $\func{\rho}{G}{GL\pare{V}}$ on a Hermitian vector space $V$ is and orthogonal sum of irreducible representations.
\end{corollary}
\begin{theorem}
    Let $\func{\rho}{G}{\GL\pare{V}}$ be a representation of a finite group on a vector space $V$. There exists a $G$-invariant, positive definite Hermitian form on $V$.
\end{theorem}
\begin{proof}
    Begining with an arbitrary positive definite Hermitian form $\curb{,}$ on $V$, then
    \[ \expc{v,w} = \rec{\abs{G}}\sum_{g\in G} \curb{gv,gw} \]
    is $G$-invariant and positive definite.
\end{proof}
\begin{corollary}
    \mbox{}
    \begin{cenum}
        \item (Maschke's Theorem): Every representation of a finite group $G$ is a direct sum of irreducible representations.
        \item Let $\func{\rho}{G}{\GL\pare{V}}$ be a representation of a finite group $G$ on a vector space $V$. There exists a basis $\mathbf{B}$ of $V$ such that the matrix representation $R$ obtained from $\rho$ using this basis is unitary.
        \item Let $\func{R}{G}{\GL_n}$ be a matrix representation of a finite group $G$. There is an invertible matrix $P$ such that $R'_g = P^{-1}R_gP$ is unitary for all $g$, i.e. such that $R'$ is a homomorphism from $G$ to the unitary group $U_n$.
        \item \hl{Every finite subgroup of $\GL_n$ is conjugate to a subgroup of the unitary group $U_n$.}
    \end{cenum}
\end{corollary}
\begin{corollary}
    Every matrix $A$ of finite order in $\GL\pare{\+bC}$ is diagonalizable.
\end{corollary}

% subsection unitary_representations (end)

\subsection{Characters} % (fold)
\label{sub:characters}

The character of the vector space $V$ is called the \gloss{dimension} of the character $\chi$. The character of an irreducible representation is called an \gloss{irreducible character}.
\begin{proposition}
    Let $\chi$ be the character of a representation $\rho$ of a finite group $G$.
    \begin{cenum}
        \item $\chi\pare{1}$ is the dimension of $\chi$.
        \item The character is constant on conjugacy classes: If $g' = hgh^{-1}$, then $\chi\pare{g'} = \chi\pare{g}$.
        \item Let $g$ be an element of $G$ of order $k$. The roots of the characteristic polynomial of $\rho_g$ are powers of teh $k$-th root of unity $\zeta = e^{2\pi i/k}$. If $\rho$ has dimension $d$, then $\chi\pare{g}$ is a sum of $d$ such powers.
        \item $\chi\pare{g^{-1}}$ is the complex conjugate $\conj{\chi\pare{g}}$ of $\chi\pare{g}$.
        \item The character of a direct sum $\rho \oplus \rho'$ of representations is the sum $\chi + \chi'$ of their characters.
        \item Isomorphic representation have the same character.
    \end{cenum}
\end{proposition}
\begin{proof}[Proof of 3]
    The trace of $\rho_g$ is the sum of ites eigenvalues. If $\lambda$ is an eigenvalue of $\rho$, then $\lambda^k$ is an eigenvalue of $\rho_g^k$, and if $g^k = 1$, then $\rho_g^k = 1$, then $\rho_g^k = I$ and $\lambda^k = 1$.
\end{proof}
\begin{proof}[Proof of 4]
    Eigenvalues of $g^{-1}$ are the reciprocals of eigenvalues of $g$.
\end{proof}
\begin{definition}
    The \gloss{Hermitian product} on characters is
    \[ \resumath{\expc{\chi,\chi'} = \rec{\abs{G}}\sum_g \conj{\chi\pare{g}}\chi'\pare{g}.} \]
\end{definition}
By numbering the conjugacy classes as $C_1$, $\cdots$, $C_r$, each of order $\curb{C_i}$ respectively, we found
\[ \resumath{\expc{\chi,\chi'} = \rec{\abs{G}}\sum_{i=1}^r c_i\conj{\chi\pare{g_i}}\chi'\pare{g_i}.} \]
\begin{ex}
    The class equation of $S_3$ is $6=1+2+3$, therefore
    \[ \expc{\chi,\chi'} = \rec{6}\pare{\conj{\chi\pare{1}}\chi'\pare{1} + 2\conj{\chi\pare{x}}\chi'\pare{x} + 3\conj{\chi\pare{y}}\chi'\pare{y}}. \]
    We found
    \[ \expc{\chi_A,\chi_A} = \rec{6}\pare{4+2+0} = 1,\quad \expc{\chi_A,\chi_\Sigma} = \rec{6}\pare{2-2+0} = 0. \]
\end{ex}
\begin{resume}
    \begin{theorem}[Main Theorem]
        Let $G$ be a finite group.
        \begin{cenum}
            \item (orthogonality relations) The irreducible characters of $G$ are orthonormal: If $\chi_i$ is the character of an irreducible representation $\rho_i$, then $\expc{\chi_i,\chi_i} = 1$. If $\chi_i$ and $\chi_j$ are the characters of nonisomorphic irreducible representations $\rho_i$ and $\rho_j$, then $\expc{\chi_i,\chi_j} = 0$.
            \item There are finitely many isomorphism classes of irreducible representations, the same number as the number of conjugacy classes in the group.
            \item Let $\rho_1$, $\cdots$, $\rho_r$ represent the isomorphism classes of irreducible representations of $G$, and let $\chi_1$, $\cdots$, $\chi_r$ be their characters. The dimension $d_i$ of $\rho_i$ divides the order $\abs{G}$ of the group, and $\abs{G} = d_1^2 = \cdots + d_r^2$.
        \end{cenum}
    \end{theorem}
\end{resume}
We have for any representation $\rho$ that
\[ \rho \approx n_1 \rho_1 \oplus \cdots \oplus n_r \rho_r, \]
where $\rho_1$, $\cdots$, $\rho_r$ are irreducible representations.
\begin{resume}
    \begin{theorem}
        Let $\rho_1$, $\cdots$, $\rho_r$ represent the isomorphism classes of irreducible representations of a finite group $G$, and let $\rho$ be any representation of $G$. Let $\chi_i$ and $\chi$ be the characters of $\rho_i$ and $\rho$, respectively, and let $n_i = \expc{\chi,\chi_i}$. Then
        \begin{cenum}
            \item $\chi = n_1\chi_1 + \dots + n_r\chi_r$.
            \item $\rho$ is isomorphic to $n_1\rho_1\oplus \cdots \oplus n_r \rho_r$.
            \item Two representations $\rho$ and $\rho'$ of a finite group $G$ are isomorphic if and only if their characters are equal.
        \end{cenum}
    \end{theorem}
\end{resume}
\begin{corollary}
    For any characters $\chi$ and $\chi'$, $\expc{\chi,\chi'}$ is an integer.
\end{corollary}
Also note that
\[ \resumath{\expc{\chi,\chi} = n_1^2 + \cdots + n_r^2.} \]
\par
A complex-valued function on the group that is constant on each conjugacy class is called a \gloss{class function}. The complex vector space $\+cH$ of class functions has dimension equal to the number of conjugacy classes.
\begin{corollary}
    The irreducible characters form on orthonormal basis of the space $\+cH$ of class functions.
\end{corollary}
We could see that $T$, $\Sigma$, $A$ represent all of the isomprphism classes of irreducible representations of the group $S_3$.
\par
We assemble the irreducible characters into the \gloss{character table} as follows:
\[ \begin{array}{lc|cccl}
    \+:c2{c}{} & \+:c3{c}{\text{conjugacy}} \\[-.3em]
    \+:c2{c}{} & \+:c3{c}{\text{class}} \\
    \+:c2{c}{} & \pare{1} & \pare{2} & \pare{3} & \text{order of the class} \\
    & & 1 & x & y & \text{representative element} \\
    \cline{2-5}
    \text{irreducible} & \chi_1 & 1 & 1 & 1 \\
    \text{character} & \chi_2 & 1 & 1 & -1 & \text{value of the character} \\
    & \chi_3 & 2 & -1 & 0
\end{array} \]

% subsection characters (end)

\subsection{One-Dimensional Characters} % (fold)
\label{sub:one_dimensional_characters}

A one-dimensional character $\chi$ is a homomorphism from $G$ to $\GL_1 = \+bC^\times$. If $\chi$ is one-dimensional and $G$ an element of order $k$, then $\chi\pare{g}$ is a power of $\zeta = e^{2\pi i/k}$.
\par
The kernel of one-dimensional character $\chi$ is the union of the conjugacy classes $C\pare{g}$ such that $\chi\pare{g} = 1$. Normal subgroups could be determined by looking at the kernel of the character table.
\begin{theorem}
    Let $G$ be a finite abelian group.
    \begin{cenum}
        \item Every irreducible character of $G$ is one-dimensional. The number of irreducible characters is equal to the order of the group.
        \item Every matrix representation $R$ of $G$ is diagonalizable: There is an invertible matrix $P$ such that $P^{-1}R_g P$ is diagonal for all $g$.
    \end{cenum}
\end{theorem}

% subsection one_dimensional_characters (end)

\subsection{The Regular Representation} % (fold)
\label{sub:the_regular_representation}

Let $S=\pare{s_1,\cdots,s_n}$ be a finite ordered set on which a group $G$ operates, and let $R_g$ denote the permutation matrix that describes the operation of a group element $g$ on $S$. If $g$ operates on $S$ as the permutation $p$, i.e. if $gs_i = s_{pi}$, that matrix is
\[ R_g = \sum_i e_{pi,i},\quad R_g e_i = e_{pi}. \]
The map $g\mapsto R_g$ defines a matrix representation $R$ of $G$ that we call a \gloss{permutation representation}.
\par
We could get rid of the order of the elements in $S$ by introducing a vector space $V_S$ that has the unordered basis $\curb{e_s}$ indexed by elements of $S$. Elements of $V_S$ are linear combinations $\sum_g c_g e_g$ with complex coefficients $c_g$. If we are given an operation of $G$ on the set $S$, the associated \gloss{permutation representation} of $G$ on $V_S$ is defined by
\[ \rho_g\pare{e_s} = e_{gs}. \]
\begin{lemma}
    Let $\rho$ be the permutation representation associated to an operation of a group $G$ on a nonempty finite set $S$. For all $g$ in $G$, $\chi\pare{g}$ is equal to the number of elements of $S$ that are fixed by $g$.
\end{lemma}
\begin{lemma}
    Let $R$ be the permutation representation associated to an operation of $G$ on a finite nonempty ordered set $S$. When its character $\chi$ is written as an integer combination of the irreducible characters, the trivial character $\chi_1$ appears.
\end{lemma}
\begin{proof}
    The vector $\sum_g v_g = \pare{1,1,\cdots,1}$ spans a $G$-invariant space of dimension $1$.
\end{proof}
\begin{definition}
    The \gloss{regular representation} $\rho^{\mathrm{reg}}$ of a group $G$ is the representation associated to the operation of $G$ on iteself by left multiplication. It is a representation on the vector space $V_G$ that has a basis $\curb{e_g}$ indexed $\curb{e_g}$ indexed by elements of $G$. If $h$ is an element of $G$, then
    \[ \rho^{\mathrm{reg}}_g\pare{e_h} = e_{gh}. \]
\end{definition}
We have
\[ \chi^{\mathrm{reg}}\pare{1} = \abs{G},\quad \chi^{\mathrm{reg}}\pare{g} = 0,\quad \mathrm{if\ } g\neq 1. \]
Therefore
\[ \resumath{\expc{\chi^{\mathrm{reg}},\chi} = \dim \chi.} \]
\begin{corollary}
    Let $\chi_1$, $\cdots$, $\chi_r$ be the irreducible characters of a finite group $G$, let $\rho_i$ be a representation with character $\chi_i$, and let $d_i = \dim \chi_i$. Then $\resumath{\chi^{\mathrm{reg}} = d_1\chi_1 + \cdots + d_r\chi_r,}$ and $\rho^{\mathrm{reg}}$ is isomorphic to $d_1\rho_1 \oplus \cdots \oplus d_r\rho_r$.
\end{corollary}
Counting dimensions, we have
\[ \abs{G} = \dim \chi^{\mathrm{reg}} = \sum_{i=1}^r d_i \dim \chi_i = \sum_{i=1}^r d_i^2. \]

% subsection the_regular_representation (end)

\subsection{Schur's Lemma} % (fold)
\label{sub:schur_s_lemma}

Let $\rho$ and $\rho'$ be representations of a group $G$ on vector spaces $V$ and $V'$. A linear transformation $\func{T}{V'}{V}$ is called $G$-\gloss{invariant} if
\[ T\pare{gv'} = gT\pare{v'},\quad \text{or}\quad T\comp \rho'_g = \rho_g\comp T,\quad \forall g \in G. \]
A bijective $G$-invariant linear transformation is an isomorphism of representations.
\par
The condition could be rewritten as
\[ T\pare{v'} = g^{-1}T\pare{gv'},\quad \text{or}\quad \rho_g^{-1} T \rho'_g = T. \]
With a basis selected, it becomes
\[ MR'_g = R_gM,\quad \text{or}\quad R_g^{-1}MR'_g = M,\quad \forall g\in G. \]
A matrix $M$ is called $G$-\gloss{invariant} if it satisfied this condition.
\begin{lemma}
    The kernel and the image of a $G$-invariant linear transformation $\func{T}{V'}{V}$ are $G$-invariant subspaces of $V'$ and $V$, respectively.
\end{lemma}
\par
Similarly, if $\rho$ is a representation of $G$ on $V$, a linear operator on $V$ is $G$-\gloss{invariant} if
\[ T\pare{gv} = gT\pare{v},\quad \text{or}\quad \rho_g\comp T = T\comp \rho_g,\quad \forall g\in G, \]
which means that $T$ commutes with each of the operators $\rho_g$. The matrix form of this condition is
\[ R_g M = M R_g,\quad \text{or}\quad M = R_g^{-1}MR_g,\quad \forall g\in G. \]
\begin{resume}
    \begin{theorem}[Schur's Lemma]
        \mbox{}
        \begin{cenum}
            \item Let $\rho$ and $\rho'$ be irreducible representaions of $G$ on vector spaces $V$ and $V'$, respectively, and let $\func{T}{V'}{V}$ be a $G$-invariant transformation. Either $T$ is an isomorphism, or else $T = 0$.
            \item Let $\rho$ be an irreducible representations of $G$ on a vector space $V$, and let $\func{T}{V}{V}$ be a $G$-invariant linear operator. Then $T$ is multiplication by a scalar: $T = cI$.
        \end{cenum}
    \end{theorem}
\end{resume}
Therefore, if $\rho$ and $\rho'$ are irreducible and not isomorphic, $T$ must be zero.
\begin{proof}[Proof of 1]
    Suppose that $T$ is non the zero map. Since $\rho'$ is irreducible and since $\ker T$ is G-invariant, $\ker T = V'$ or $\curb{0}$. $\ker T$ cannot be $V'$ and thus can only be $\curb{0}$. Since $\rho$ is irreducible and $\im T$ is G-invariant, $im T$ is either $\curb{0}$ or $\curb{V}$. It will not be $\curb{0}$ if $T$ is not the zero map.
\end{proof}
\begin{proof}[Proof of 2]
    Suppose that $T$ is a $G$-invariant linear operator on $V$. We choose an eigenvalue $\lambda$ of $T$. The linear operator $S = T-\lambda I$ is also $G$-invariant. The kernel of $S$ isn't zero because it contains an eigen vector of $T$. Therefore $S$ is not an isomorphism. By the previous statement we have $S = 0$ and $T = \lambda I$.
\end{proof}
Starting from any linear transformation $\func{T}{V'}{V}$ and its matrix $M$, via the averaging process we found
\[ \tilde{T}\pare{v'} = \rec{\abs{G}}\sum_{g\in G} g^{-1}\pare{T\pare{gv'}} = \rec{\abs{G}}\sum_{g\in G}\rho_g^{-1}T\rho'_g,\quad \tilde{M} = \rec{\abs{G}}\sum_{g\in G}R_g^{-1}MR'_g \]
is $G$-invariant.
\begin{lemma}
    With the above notation, $\tilde{T}$ is a $G$-invariant linear transformation, and $\tilde{M}$ is a $G$-invariant matrix. If $T$ is $G$-invariant, then $\tilde{T} = T$, and if $M$ is $G$-invariant, then $\tilde{M} = M$.
\end{lemma}
\begin{proposition}
    Let $\rho$ be an irreducible representation of $G$ on a vector space $V$. Let $\func{T}{V}{V}$ be a linear operator, and let $\tilde{T}$ be as in the expression above, with $\rho' = \rho$. Then $\trace \tilde{T} = \trace T$. If $\trace T\neq 0$, then $\tilde{T} \neq 0$.
\end{proposition}

% subsection schur_s_lemma (end)

\subsection{Proof of the Orthogonality Relations} % (fold)
\label{sub:proof_of_the_orthogonality_relations}

Let $\+cM$ denote the space $\+bM^{m\times n}$ of $m\times n$ matrices.
\begin{lemma}
    Let $A$ and $B$ be $m\times m$ and $n\times n$ matrices respectively, and let $F$ be the linear operator on $\+cM$ defined by $F\pare{M} = AMB$. The trace of $F$ is the product $\pare{\trace A}\pare{\trace B}$.
\end{lemma}
\begin{proof}
    $M = X_i Y_j^T$ is an eigenvector of $F$ if $X_i$ is an eigenvector of $A$ and $Y_j$ of $B$. Then argue by continuity for the general cases.
\end{proof}
Let $\rho'$ and $\rho$ be representations of dimensions $m$ and $n$, with characters $\chi'$ and $\chi$ respectively, and let $R'$ and $R$ be their matrix representations with respect to some arbitrary bases. Define a linear operator $\Phi$ on the space $\+cM$ by
\[ \Phi\pare{\+cM} = \rec{\abs{G}} \sum_g R_g^{-1}MR'_g = \tilde{M}. \]
We have $\tilde{M}$ as a $G$-invariant matrix, and that $\tilde{M} = M$ is $M$ is invariant. Therefore the image of $\Phi$ is the space of $G$-invariant matrices, denoted by $\tilde{\+cM}$.
\begin{lemma}
    With the above notation,
    \begin{cenum}
        \item $\trace \Phi = \expc{\chi,\chi'}$.
        \item $\trace \Phi = \dim \tilde{\+cM}$.
        \item If $\rho$ is an irreducible representation, $\expc{\chi,\chi} = 1$, and if $\rho$ and $\rho'$ are non-isomorphic irreducible representations, $\expc{\chi,\chi'} = 0$.
    \end{cenum}
\end{lemma}
Hence part (1) of the Main Theorem.
\begin{proof}[Proof of 1]
    With $F_g$ denoting the linear operator $F_g\pare{M} = R_g^{-1}MR'_g$, we have
    \begin{align*}
        \trace \Phi &= \rec{\abs{G}}\sum_g \trace F_g = \rec{\abs{g}}\sum_g \pare{\trace R_g^{-1}}\pare{\trace R_g} \\
        &= \rec{\abs{G}}\sum_g \chi\pare{g^{-1}}\chi'\pare{g} = \rec{\abs{G}}\sum_g \conj{\chi\pare{g}}\chi'\pare{g} = \expc{\chi,\chi'}. \qedhere
    \end{align*}
\end{proof}
\begin{proof}[Proof of 2]
    With $\+cN$ denoting the kernel of $\Phi$. If $M$ is in the intersection $\tilde{M}\cap \+cN$, then $\Phi\pare{M} = M$ also $\Phi\pare{M} = 0$, so $M=0$, and the intersection is the zero space. Therefore $\+cM = \tilde{\+cM}\oplus \+cN$. Choosing a basis for $\+cM$ by appending basis of $\tilde{\+cM}$ and $\+cN$. Since $\tilde{M} = M$ if $M$ is invariant, $\Phi$ is the identity on $\tilde{\+cM}$. So the matrix $\Phi$ will have the block form
    \[ \begin{pmatrix}
        I & \\
        & 0
    \end{pmatrix}, \]
    where $I$ is the identity matrix of size $\dim \tilde{M}$.
\end{proof}
\begin{proof}[Proof of 3]
    Now $\expc{\chi,\chi'} = \dim \tilde{\+cM}$. If $\rho'$ and $\rho$ are irreducible and not isomorphic, Schur's Lemma tells us that the only $G$-invariant operator is zero, and so the only $G$-invariant matrix is the zero matrix. Therefore $\tilde{M} = \curb{0}$ and $\expc{\chi,\chi'} = 0$. If $\rho' = \rho$, Schur's Lemma says that the $G$-invariant matrices have the form $cI$. Then $\+cM$ has dimension $1$, and $\expc{\chi,\chi'} = 1$.
\end{proof}
We now go over to the proof of part (2) of the Main Theorem. Let $\+cH$ denote the space of class functions. Its dimension is equal to the number of conjugacy classes. Let $\+cC$ denote the subspace of $\+cH$ spanned by the characters. We show that $\+cC = \+cH$ by showint that the orthogonal space to $\+cC$ in $\+cH$ is zero.
\begin{lemma}
    \mbox{}
    \begin{cenum}
        \item Let $\varphi$ be a class function on $G$ that is orthogonal to every character. For any representation $\rho$ of $G$, $\displaystyle \rec{\abs{G}}\sum_g \conj{\varphi\pare{g}}\rho_g$ is the zero operator.
        \item Let $\rho^{\mathrm{reg}}$ be the regular representation of $G$. The operators $\rho_g^{\mathrm{reg}}$ with $g$ in $G$ are linearly independent.
        \item The only classs function $\varphi$ that is orthogonal to every character is the zero function.
    \end{cenum}
\end{lemma}
\begin{proof}[Proof of 1]
    Since any representation is a direct sum of irreducible representatins, we may assume that $\rho$ is irreducible. Let $\displaystyle T = \rec{\abs{G}}\sum_g \conj{\varphi\pare{g}}\rho_g$. $T$ is obviously a $G$-invariant operator, i.e. $T = \rho_h^{-1}T\rho_h$ for every $h\in G$.
    \par
    Let $\chi$ be the character of $\rho$, we have $\displaystyle \trace T = \rec{\abs{G}}\sum_g \conj{\varphi\pare{g}}\chi\pare{g} = \expc{\varphi,\chi} = 0$ because $\varphi$ is orthogonal to $\chi$. But Schur's Lemma tells us that $T = \lambda I$, therefore $T=0$.
\end{proof}
\begin{proof}[Proof of 2]
    We have $\rho_g^{\mathrm{reg}}\pare{e_1} = e_g$. Since $e_g$ are independent elements of $V_G$, the operators $\rho_g^{\mathrm{reg}}$ are independent too.
\end{proof}
\begin{proof}[Proof of 3]
    From part (1) we know $\displaystyle \sum_g \conj{\varphi\pare{g}}\rho_g^{\mathrm{reg}} = 0$, which is a linear relation among $\rho_g^{\mathrm{reg}}$, hence $\varphi\pare{g} \equiv 0$.
\end{proof}

% subsection proof_of_the_orthogonality_relations (end)

\subsection{Representations of \texorpdfstring{$SU_2$}{SU2}} % (fold)
\label{sub:representations_of_su2}

Let $H_n$ denote the complex vector space of homogeneous polynomials of degree $n$ of the form
\[ f\pare{u,v} = c_0 u^n + c_1 u^{n-1}v + \cdots + c_{n-1} uv^{n-1} + c_nv^n. \]
We define a representation
\[ \func{\rho_n}{SU_2}{\GL\pare{H_n}} \]
as follows: for $P \in SU_2$,
\[ \brac{Pf}\pare{u,v} = f\pare{ua+vb,-u\conj{b} + v\conj{a}},\quad \mathrm{where}\quad P = \begin{pmatrix}
    a & -\conj{b} \\
    b & \conj{a}
\end{pmatrix}. \]
Therefore
\[ \brac{Pu^iv^j} = \pare{ua+vb}^i\pare{-u\conj{b}+v\conj{a}}^j. \]
For the special case where $P$ is diagonal, let $\alpha = e^{i\theta}$ and
\[ A_\theta = \begin{pmatrix}
    e^{i\theta} & \\
    & e^{-\theta}
\end{pmatrix} = \begin{pmatrix}
    \alpha & \\
    & \conj{\alpha}
\end{pmatrix} = \begin{pmatrix}
    \alpha & \\
    & \alpha^{-1}
\end{pmatrix}, \]
we found $\brac{A_\theta u^i v^j} = \pare{u\alpha}^i \pare{v\conj{\alpha}}^j = u^i v^j \alpha^{i-j}$. So $A_\theta$ acts on the basis $\pare{u^n, u^{n-1}v,\cdot, uv^{n-1},v^n}$ of the space $H_n$ as the diagonal matrix
\[ \begin{pmatrix}
    \alpha^n & & & \\
    & \alpha^{n-1} \\
    & & \ddots \\
    & & & \alpha^{-n}
\end{pmatrix}. \]
\par
With the character $\chi_n$ defined as $\chi_n\pare{g} = \trace \rho_{n,g}$ as before, which is a constant on each conjugacy class, i.e. the latitudes on the sphere $SU_2$, we found it sufficient to compute the characters $\chi_n$ on one matrix in each latitude, and we choose $A_\theta$. With $\chi_n\pare{\theta} = \chi_n\pare{A_\theta}$, we have
\begin{align*}
    & \chi_0\pare{\theta} = 1, \\
    & \chi_1\pare{\theta} = \alpha + \alpha^{-1}, \\
    & \chi_2\pare{\theta} = \alpha^2 + 1 + \alpha^{-2}, \\
    & \cdots \\
    & \chi_n\pare{\theta} = \alpha^n + \alpha^{n-2} + \cdots + \alpha^{-n} = \frac{\alpha^{n+1} - \alpha^{-\pare{n+1}}}{\alpha - \alpha^{-1}}.
\end{align*}
\par
The Hermitian product is defined as
\[ \expc{\chi_m,\chi_n} = \rec{\abs{G}}\int_G \conj{\chi_m\pare{g}}\chi_n\pare{g}\,\rd{V}, \]
where $G$ stands for $SU_2$, the unit $3$-sphere, and $\abs{G}$ for its three-dimensional volume, $\rd{V}$ stands for the integral with respect to three-dimensional volume.
\begin{theorem}
    The characters of $SU_2$ that are defined above are orthonormal: $\expc{\chi_m,\chi_n} = 0$ if $m\neq n$, and $\expc{\chi_n,\chi_n} = 1$.
\end{theorem}
\begin{theorem}
    Every continuous representation of $SU_2$ is isomorphic to a direct sum of the representations $\func{\rho_n}{SU_2}{\GL\pare{H_n}}$.
\end{theorem}

% subsection representations_of_su2 (end)

% section group_representations (end)

\end{document}
