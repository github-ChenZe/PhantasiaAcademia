\documentclass{ctexart}

\usepackage{van-de-la-sehen}

\begin{document}

\section{千年第一思想家马克思} % (fold)
\label{sec:千年第一思想家马克思}

马克思主义是一门科学.\\
\par
\begin{cenum}
    \item M是马克思/恩格斯, 创立的基本理论, 基本观点和学说的体系.
    \item 经济领域: 剩余价值理论; 历史领域: 唯物史观.
    \item 关于自然, 社会和人类思维发展一般规律的学说, 是关于社会主义必然代替资本主义, 最终实现共产主义的学说.
    \item 无产阶级解放, 全人类解放与每个人自由全面发展的学说.
    \item 1840s创立.
\end{cenum}
\paragraph{社会根源} % (fold)
\label{par:社会根源}

工业革命 -> 资本主义生产力+经济发展. 资本主义发展. -> 生产过剩危机周期性爆发 + 两极分化.

% paragraph 社会根源 (end)

\paragraph{阶级基础} % (fold)
\label{par:阶级基础}

无产阶级反抗剥削压迫过程中科学理论指导.

% paragraph 阶级基础 (end)

\paragraph{思想渊源} % (fold)
\label{par:思想渊源}

三大先进思潮: 德国古典哲学, 形而上学唯物主义(偏离, 而唯心主义是方向错误), 古典政治经济学(对资本主义生产关系的分析和关于劳动创造价值的思想).
\par
劳动是财富之父. 劳动创造财富.

% paragraph 思想渊源 (end)

\paragraph{自然科学基础} % (fold)
\label{par:自然科学基础}

细胞学说, 
\par
马克思分析了资本主义经济历程.
《政治经济学批判》: 资本, 雇佣劳动, 土地所有制, 国家, 对外贸易, 世界市场. -> 转为《资本论》.

\begin{ex}[论据]
    看不见的手, 1929--1933经济危机.
\end{ex}

\begin{finale}
    剩余价值理论.
\end{finale}

% paragraph 自然科学基础 (end)

% section 千年第一思想家马克思 (end)

\section{整体上理解与把握马克思主义} % (fold)
\label{sec:整体上理解与把握马克思主义}

\paragraph{三个组成部分} % (fold)
\label{par:三个组成部分}

马克思主义哲学, 马克思主义政治经济学和科学社会主义是马克思主义的三个基本组成部分. 无产阶级解放和人类解放 + 两个必然 <=>科学社会主义.

% paragraph 三个组成部分 (end)

\par
马克思主义哲学: 惟物史观: 人类社会发展过程具有唯物性 -> 客观存在. 世界是物质的, 物质是运动的, 运动时有规律的, 规律是可以认识的.
\par
马克思主义经济学: 无产阶级革命且胜利: 胜利的社会革命需要生产力的高度发展和无产阶级准备成熟.

\begin{ex}
    由生产关系判断封建/资本主义社会, 由生产资料所有制判断生产关系.
\end{ex}

\paragraph{科学性} % (fold)
\label{par:科学性}

马克思主义对自然, 社会和人类思维发展本质和规律的正确反映. 是在社会实践中产生的.

% paragraph 科学性 (end)

革命性, 发展性……

\begin{ex}
    马克思主义是科学, 根本上在于{\color{red}始终严格地以客观事实为依据}.
\end{ex}
\begin{ex}
    M之所以能创立科学社会主义, 因为社会历史条件和个人努力的相互作用.
\end{ex}

% section 整体上理解与把握马克思主义 (end)

\section{哲学与哲学的基本问题} % (fold)
\label{sec:哲学与哲学的基本问题}

\subsection{哲学与哲学的基本问题} % (fold)
\label{sub:哲学与哲学的基本问题}

世界观+方法论. 世界观是人们对整个世界的根本观点和看法. 方法论是人们在一定世界观指导下认识与处理问题的根本方法.
\begin{finale}
    世界观决定方法论, 方法论反映世界观.
\end{finale}
哲学是系统化, 理论化的世界观, 哲学既是世界观又是方法论, 是世界观和方法论的统一.
\par
哲学的基本问题: 物质和意识的关系问题.
\begin{cenum}
    \item 意识物质何者为第一性的问题.
    \item 我们的思维能不能认识现实世界, 即思维和存在的同一性问题.
\end{cenum}
\begin{sample}
    \begin{ex}
        唯物主义一元论同唯心主义一元论的根本对立在于{\color{red}对世界本原的不同回答}.
    \end{ex}
    \begin{ex}
        哲学史上第一次... 完整... 是恩格斯.
    \end{ex}
    \begin{ex}
        科学家没有学习Mism也能取得伟大成就, 是由于他们{\color{red}在科学实践过程中摆脱了社会主义}, {\color{red}自觉不自觉地贯彻了唯物主义, 辩证法的原则}, {\color{red}相信独立于...存在}.
    \end{ex}
\end{sample}

% subsection 哲学与哲学的基本问题 (end)

\subsection{辩证唯物主义的物质观, 运动观和时空观} % (fold)
\label{sub:辩证唯物主义的物质观_运动观和时空观}

\paragraph{朴素唯物主义} % (fold)
\label{par:朴素唯物主义}

物质归结为具体的实物.

% paragraph 朴素唯物主义 (end)

\paragraph{形而上学唯物主义} % (fold)
\label{par:形而上学唯物主义}

物质归结为近代自然科学发现的不可再分的基本粒子.

% paragraph 形而上学唯物主义 (end)

\paragraph{辩证唯物主义中的物质} % (fold)
\label{par:辩证唯物主义中的物质}

物质是标志客观实在的哲学范畴, 客观实在是人通过感觉而感知的, 不依赖于感觉而存在.
\par
人的思维, 意识也是物质的. 物质和意识的对立只有在非常有限的范围内才有意义.
\par
M物质观的意义:
\begin{cenum}
    \item 唯物主义一元论, 批判唯心的物质观和二元论的物质观;
    \item 坚持反映论/可知论, 批判先验论/不可知论;
    \item 坚持辩证法, 克服以往唯物主义的物质观用个别、个性代替一般、共性的形而上学的缺陷;
    \item 坚持历史唯物主义的观点, 反对历史相对主义的观点.
\end{cenum}
\begin{sample}
    \begin{ex}
        彻底的唯物主义一元论的根本要求是{\color{red}坚持一切从实际出发}.
    \end{ex}
\end{sample}
\begin{finale}
    运动是物质的存在方式和根本属性. 时间和空间是物质运动的存在形式.
\end{finale}
不可分割.
\begin{sample}
    \begin{ex}
        世界上唯一不变的是变, {\color{red}变与不变是相对的}.
    \end{ex}
    \begin{ex}
        运动应当从它的反面即从静止找到它的量度, {\color{red}?}.
    \end{ex}
    \begin{ex}
        桔子洲... 天心阁... 运动和静止是相互联系的. {\color{red}静止是运动的衡量尺度}.
    \end{ex}
    \begin{ex}
        不是风动, 不是幡动, 仁者心动. 表明观点{\color{red}精神是运动的主体}.
    \end{ex}
    \begin{ex}
        "气凝为形, ... 皆气也." 在哲学上表达了{\color{red}万物都是物质的不同表现形式}, 同埋{\color{red}朴素辩证法的思想}.
    \end{ex}
    \begin{ex}
        “人生代代无穷, 江月年年相似”蕴含相同哲理...
    \end{ex}
    \begin{ex}
        世界上除了运动着的物质之外什么也没有的观点, 属于{\color{red}辩证唯物主义}.
    \end{ex}
\end{sample}
意识是物质的产物, 是物质世界在人脑中的主观映象, 这是物质、意识关系问题上的唯物主义.
\begin{sample}
    \begin{ex}
        观念的东西不外是移入人的头脑并在人的头脑中改造过的物质的东西而已. 揭示了{\color{red}观念的东西是对物质的能动反映}, {\color{red}观念的东西和物质的东西是对立统一的}.
    \end{ex}
\end{sample}
\begin{sample}
    \begin{ex}
        "你未看此花时... 你来看此花时"错误之处在于{\color{red}把人的感觉与花的存在等同起来}, {\color{red}把人对花的感觉夸大成脱离花的独立实体}.
    \end{ex}
\end{sample}
\begin{finale}
    「同一性」指由差别的联系.
\end{finale}
\begin{sample}
    \begin{ex}
        符合意识能动性原理的是{\color{red}?}.
    \end{ex}
\end{sample}

% paragraph 辩证唯物主义中的物质 (end)

% subsection 辩证唯物主义的物质观_运动观和时空观 (end)

% section 哲学与哲学的基本问题 (end)

\end{document}
