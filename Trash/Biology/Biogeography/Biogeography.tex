\documentclass{ctexart}

\usepackage{van-de-la-sehen}

\begin{document}

\section{生物地理学} % (fold)
\label{sec:生物地理学}

\subsection{生物地理学的概念和发展} % (fold)
\label{sub:生物地理学的概念和发展}

生物地理学时研究生物分布的科学, 试图说明物种和种以上的分类群分布格局的形成过程, 以及造成各个区域生物类群组成差别的原因.
\begin{theorem}[Buffon定律]
    环境类似而地理上隔离的地区生物种类并不一样.
\end{theorem}
Wallace线: 动物区系中澳大利亚区和东洋区的分界.
\par
生物地理学的发展: 经典Mendel遗传学, 种群遗传学, 生物系统学, Darwin进化论修正和发展, 现代综合进化论, 地理隔离观点.
\par
隔离分化案例: Darwin地雀. 14中地雀分布在13个岛上, 各岛分别有其中的3-10种, 形态大小各异, 都与南美大陆的种类相似. 推测祖先地栖, 以种子为食.
\par
60年代, 分支生物地理学综合了分支系统学和板块构造学说的思想兴起, 隔离称为解释生物地理学格局的首要选择. 生物地理学基于扩散和分替两个理论分为两个学派.
\par
扩散强调物种本身具有起源中心, 物种由起源中心向外扩散, 地理格局不变, 忽视板块漂移.
\par
分替强调板块漂移引起的变化.
\par
生态生物地理学/历史生物地理学关注内容不同. 前者关注较短时间尺度/较小空间尺度的生态因素对生物分布的影响. 后者关注地质事件等历史因素的影响.
\par
生态学基础: 陆地上生物的分布主要受气候和土壤类型决定; 水生生物分布很大程度上受到温度/盐浓度/光线和压力的变化.
\par
物种分布的生物限制: 外界: 掠食者, 猎物; 内在: 习性, 生理.
\par
群落生态学: 物种不能独立存在, 都是一定区域内的群落或生态系统的有机组成部分. 群落中植物的分布由非生物因素所决定, 而植物的分布又决定了动物的分布.
\par
历史学基础:
\begin{cenum}
    \item 物种历史: 物种在时间和空间上的起源、发展和灭绝的过程.
    \item 冰川时期: 地球表面覆盖大规模冰川的地址时期. 地球历史上多次有显著变冷, 形成冰期. 至少数百万年的冰期谓大冰期. 7.2-5.8亿年前海洋全部结冰, 谓雪球事件.
    \begin{cenum}
        \item 冰期会改变海陆气候, 导致生物大灭绝, 动物被迫迁徙;
        \item 给冰川消融后新的生物类群在原地大繁荣的机会;
        \item 改变地形地貌, 形成新的隔离屏障.
    \end{cenum}
    \item 板块漂移: Wegener在1912年提出大陆漂移说. 2.5亿年前各物种存在于连接在一起的泛大陆上, 板块漂移使之分开;
    \begin{cenum}
        \item 大陆漂移有利于渐变的生物环境的形成, 促进了生物的渐进变化;
        \item 为生物进化提供了自然选择条件;
        \item 为生物进化提供了隔离条件;
        \item 为动植物登陆提供了更充分的场所;
        \item 导致世界气候的变化.
    \end{cenum}
    \item 迁移扩散: 物种流动超出原来种群范围外, 长距离穿越阻限的扩散.
    \begin{cenum}
        \item 扩散途径有通道(全部), 滤阻(部分), 险组(不可逾越).
        \item 长距离扩扩散能力要求耐饥, 善于迁徙, 适应力和繁殖能力.
        \item 间隔分布: 两个或多个相近的生物类群生活于相距很远的区域的分布现象. 原因在于阻限之形成.
    \end{cenum}
    \item 系统发育: 某一类群的形成和发展的过程. 系统发育学研究的是进化关系. 系统发育分析就是要推断这些进化关系.
    \begin{cenum}
        \item 进化树描述统一谱系的进化关系, 包括分子进化, 物种进化, 分子进化和物种进化的综合进化树.
    \end{cenum}
    \item 人类出于各种目的, 引入或消灭一些物种, 改变生态环境,从而改变了生物的分布格局. 加上人类污染环境, 改变和影响地理尽速, 对生物的生存, 繁衍和扩散造成影响.
\end{cenum}

\subsubsection{生物地理区划的依据} % (fold)
\label{ssub:生物地理区划的依据}

参考物种的选择标准: 自然分布的所有物种. 不适宜者如广泛分布的物种, 分布及其狭窄的物种, 分布信息不足的物种, 国外引入的物种, 广泛栽种的物种, 分类学有争议的物种.

% subsubsection 生物地理区划的依据 (end)

% subsection 生物地理学的概念和发展 (end)

\subsection{世界陆地生物地理区系} % (fold)
\label{sub:世界陆地生物地理区系}

\subsubsection{陆地生物群} % (fold)
\label{ssub:陆地生物群}

地带性生物群: 陆地上生物群分布在一定气候带的显域生境, 即主要受带气候支配, 主要是由气候, 尤其是水/热的组合状况决定.
\par
非地带性生物群: 不确定分布在哪个地带内, 而是分布在所有地带适宜条件下.

% subsubsection 陆地生物群 (end)

\subsubsection{世界陆地生物区系区} % (fold)
\label{ssub:世界陆地生物区系区}

澳洲界/新热带界/衣索比亚界/东洋界/古北界等.

\paragraph{全北区} % (fold)
\label{par:全北区}

包括北回归线以北的广大地区等.
\begin{cenum}
    \item 古北区: 种类相对贫乏, 特有种较多. 自然环境、栖息地类型、气候多样, 生态环境多样. 脊椎动物物种贫乏.
    \item 新北区: 气候多样, 特有科种较少, 物种最少(除了南极区)的一个地理区.
\end{cenum}

% paragraph 全北区 (end)

\paragraph{非洲热带区} % (fold)
\label{par:非洲热带区}

古热带区, 衣索比亚区. 有多样性且拥有丰富的特有类群. 大型哺乳动物远比其他任何地区丰富. 有大量的热带稀树草原有蹄动物群.

% paragraph 非洲热带区 (end)

\paragraph{新热带区} % (fold)
\label{par:新热带区}

包括整个中美, 南美大陆, 墨西哥及西印度群岛. 属于热带气候, 有大面积的热带雨林和草原, 包括全球最大的热带雨林亚马逊雨林. 种类极为繁多且特殊.

% paragraph 新热带区 (end)

\paragraph{东洋区} % (fold)
\label{par:东洋区}

覆盖东南亚等. 具有大陆区系特点, 气候温暖. 地处热带/亚热带, 降水丰富, 植被类型多样, 以热带和亚热带雨林为主.

% paragraph 东洋区 (end)

\paragraph{澳大利亚区} % (fold)
\label{par:澳大利亚区}

独立性最强. 由干旱地带, 潮湿地带和孤岛等景观组成, 气候较为干燥. 植物区系非常丰富. 物种较为原始, 保留中生代晚期特征. 有原兽亚纲和后兽亚纲.

% paragraph 澳大利亚区 (end)

\paragraph{南极区} % (fold)
\label{par:南极区}

陆栖生物面积最小一区. 海拔很高. 东部高原. 气候酷寒, 常有狂风暴雪. 大部分地面常年未被冰雪覆盖. 各岛屿上植物非常匮乏, 有很多沼泽和泥碳沼泽. 缺乏植物, 缺乏陆栖脊椎动物, 缺乏特有物种, 只有一些生活与海洋但也见于海岸的种类, 种类组成贫乏.

% paragraph 南极区 (end)

\begin{remark}
    最新的生物地理区划为11区.
\end{remark}

% subsubsection 世界陆地生物区系区 (end)

% subsection 世界陆地生物地理区系 (end)

\subsection{世界海洋生物地理区系} % (fold)
\label{sub:世界海洋生物地理区系}

\subsubsection{水域生物} % (fold)
\label{ssub:水域生物}

各种水体以及统一水体内各个部分条件并不完全一致, 因此出现了多种多样的生境, 生物与此相适应, 形成不同的生态类群. 通常分为
\begin{cenum}
    \item 漂浮生物: 生活在水体表面; 例如浮萍, 海蜗牛等.
    \item 浮游生物: 绿藻/蓝藻/水母/轮虫等.
    \item 自游生物: 鱼/虾/乌贼/章鱼/鲸/海豚/海龟/海蛇等.
    \item 底栖生物: 水生高等植物的附着生长的藻类. 动物如原生动物/海绵动物/腔肠动物/扁形动物等.
\end{cenum}

% subsubsection 水域生物 (end)

\subsubsection{海洋植物区系} % (fold)
\label{ssub:海洋植物区系}

海洋中的孢子植物主要是各种藻类. 海洋生物生态类型较为单一.

\begin{cenum}
    \item 北方海洋区: 有大量褐藻类;
    \item 热带海洋区: 有大量红藻类;
    \item 南方海洋区: 有褐藻类的巨藻属.
\end{cenum}

% subsubsection 海洋植物区系 (end)

\subsubsection{世界海洋动物区系区} % (fold)
\label{ssub:世界海洋动物区系区}

没有统一划分方案. 大多数等级较高的动物都是广泛分布的. 各级类群分布雷同, 以哪个类群为准有争议. 存在垂直带的分异, 分区更加复杂化.

\paragraph{北极冷水区} % (fold)
\label{par:北极冷水区}

动物种类和数量较为贫乏. 典型如北极鲸. 食肉有北极熊.

% paragraph 北极冷水区 (end)

\paragraph{北太平洋温水区} % (fold)
\label{par:北太平洋温水区}

动物丰富特殊, 多固有种. 例如海狗.

% paragraph 北太平洋温水区 (end)

\paragraph{北大西洋温水区} % (fold)
\label{par:北大西洋温水区}

类似北太平洋温水区.

% paragraph 北大西洋温水区 (end)

\paragraph{印度-太平洋暖水区} % (fold)
\label{par:印度_太平洋暖水区}

种类丰富, 靠近赤道地区, 海生哺乳动物贫乏. 特有种儒艮.

% paragraph 印度_太平洋暖水区 (end)

\paragraph{大西洋暖水区} % (fold)
\label{par:大西洋暖水区}

种类丰富, 但比前者贫乏.

% paragraph 大西洋暖水区 (end)

\paragraph{南方温水区} % (fold)
\label{par:南方温水区}

动物种类比热带海洋贫乏得多, 但个体数目多, 生物量大, 有大量浮游生物, 供养许多鲸群.

% paragraph 南方温水区 (end)

\paragraph{南极冷水区} % (fold)
\label{par:南极冷水区}

寒冷, 对生物生存十分不利, 因此动物区系非常贫乏, 缺乏许多世界性分布的集群. 代表性鸟类为企鹅.

% paragraph 南极冷水区 (end)

\paragraph{海陆区别} % (fold)
\label{par:海陆区别}

海洋中有丰富的高等级类群, 大多广泛世界性分布, 陆地则相反, 但陆地上生物种类比海洋丰富. 而海洋生物结构比较简单而原始, 这与海洋环境条件有关系.

% paragraph 海陆区别 (end)

% subsubsection 世界海洋动物区系区 (end)

% subsection 世界海洋生物地理区系 (end)

\begin{cenum}
    \item 生物分布格局的形成是物种扩散分化, 群落演替, 地质演化和其他生物因素共同作用的结果;
    \item 不同自然地理地带中栖息不同生物群. 世界陆地/海洋生物地理区系.
    \item 生物区系分析是了解区系组成/差异/形成原因及变化规律的重要手段.
\end{cenum}

% section 生物地理学 (end)

\end{document}
