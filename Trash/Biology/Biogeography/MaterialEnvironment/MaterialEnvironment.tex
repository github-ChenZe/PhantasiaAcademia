\documentclass{ctexart}

\usepackage{van-de-la-sehen}

\begin{document}

\section{物质环境} % (fold)
\label{sec:物质环境}

\subsection{生物对水环境的适应} % (fold)
\label{sub:生物对水环境的适应}

\subsubsection{水的分布格局} % (fold)
\label{ssub:水的分布格局}

世界水的分布格局, 谓一大二小(一个大循环: 海-陆, 两个小循环(海-海, 陆-陆)).
\par
水的三相点在$\SI{0.01}{\degree\celsius}$.
\begin{cenum}
    \item 气态水: 来自地面和水面的蒸发, 以及植物的蒸腾作用. 表示方法有四种:
    \begin{cenum}
        \item 水汽压($e$);
        \item 绝对湿度($a$);
        \item 相对湿度($R$);
        \item 饱和差$d$.
    \end{cenum}
    \emph{露点}谓水汽饱和时的温度.
\end{cenum}
降水形式多样, 例如雨, 雪, 爽, 露, 雾, 雹, 潵, 淞.
\begin{cenum}
    \item $\SI{10}{\milli\meter}$以下谓小雨;
    \item $\SI{25}{\milli\meter}$以下谓中雨;
    \item $\SI{50}{\milli\meter}$以下谓大雨;
    \item $\SI{100}{\milli\meter}$以下谓暴雨;
    \item $\SI{250}{\milli\meter}$以下谓大暴雨;
    \item $\SI{250}{\milli\meter}$以上谓特大暴雨.
\end{cenum}
\begin{cenum}
    \item $\SI{2.9}{\milli\meter}$以下谓小雪;
    \item $\SI{5.9}{\milli\meter}$以下谓中雪;
    \item $\SI{9.9}{\milli\meter}$以下谓大雪;
    \item $\SI{10}{\milli\meter}$以上谓暴雪.
\end{cenum}
雪的形成需要饱和水汽及温度, 以及凝结核.
\par
霜的形成与当时天气条件, 以及所附着的物体的属性有关. 霜将升华为水汽或融化为水.
\begin{cenum}
    \item 冬季, 低于$\SI{4}{\degree\celsius}$密度较小的水漂动在表面冰层下, 湖泊底部仍不结冰.
    \item 季风引起水的垂直移动, 将水的营养成分带到上层, 氧气带到深层.
    \item 温度变化最快的水层谓跃温层.
    \item 夏季发生温度分层现象, 阻止表面水层和湖下层间的混合.
\end{cenum}
河流入海口有较高的生物多样性.
\par
海洋鱼类的表皮防止水分渗出, 而淡水鱼类的表皮防止水从外部渗入. 例如鲨鱼中保留的尿素使其渗透压与海水的渗透压平衡.
\par
鱼鳃处动脉/静脉会和水交换氧气.

% subsubsection 水的分布格局 (end)

\subsubsection{动物对环境行为的适应} % (fold)
\label{ssub:动物对环境行为的适应}

例如甲虫通过在沙漠中倒立在雾中饮水.

% subsubsection 动物对环境行为的适应 (end)

\subsubsection{植物体的水分平衡} % (fold)
\label{ssub:植物体的水分平衡}

植物水分外逸: 吐水和蒸腾;\\
植物的气孔蒸腾与角质蒸腾;\\
根, 茎, 叶的水分平衡.\\
干旱对植物的影响和植物的抗旱性.

% subsubsection 植物体的水分平衡 (end)

% subsection 生物对水环境的适应 (end)

\subsection{大气环境对生物的影响} % (fold)
\label{sub:大气环境对生物的影响}

\subsubsection{大气环境的组成} % (fold)
\label{ssub:大气环境的组成}

(略)

% subsubsection 大气环境的组成 (end)

\subsubsection{Hadley环流圈} % (fold)
\label{ssub:hadley环流圈}

(略)

% subsubsection hadley环流圈 (end)

\subsubsection{风的类型} % (fold)
\label{ssub:风的类型}

台风和飓风都是北半球的热带气旋, 后者在大西洋或东太平洋发生. 中心风力达到十二级或以上的热带气旋, 前者为西太平洋上发生者.
\par
防风林以$50\%$之通过率为最佳状态.
\par
槟郎树的纤维结构使得其能保持竖直.
\begin{ex}
    干旱环境下的桉树被移植到广东, 死树干内部被水腐蚀, 失去受力能力. 
\end{ex}
\begin{ex}
    季节性的气流将害虫在南北之间移动.
\end{ex}
Gaia假说: 生命的总体会为了自己的需要而优化环境. 表面大气成分, 土壤和水体的情况, 是受生物多样性的活动状况而有所调节.
\begin{ex}[人类对高海拔的适应]
    血球比积: 红细胞压积上升.
\end{ex}
\begin{ex}[动物对低氧环境的适应]
    例如鸟内的气孔.
\end{ex}

% subsubsection 风的类型 (end)

% subsection 大气环境对生物的影响 (end)

\subsection{土壤环境中的科学问题} % (fold)
\label{sub:土壤环境中的科学问题}

土壤是由植物群落发育的. 形成1厘米的土壤需要800到1000万年.
土壤肥力有四要素(水, 肥, 气, 热).
土壤的三相系统由矿物质, 有机质和水气组成.
土壤有机质谓土壤中的动物和植物的新鲜组织及残体、腐烂分解产物.
\par
土壤中的团粒结构是理想的土壤类型.

% subsection 土壤环境中的科学问题 (end)

% section 物质环境 (end)

\end{document}
