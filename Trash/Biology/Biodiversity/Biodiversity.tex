\documentclass{ctexart}

\usepackage{van-de-la-sehen}

\begin{document}

\section{生物多样性} % (fold)
\label{sec:生物多样性}

现代的生物多样性的分叉呈树状.

\subsection{生物多样性是如何进化的} % (fold)
\label{sub:生物多样性是如何进化的}

生物进化的式样是多样化的: 上升式进化, 下降式进化, 特化/专门化等.
\par
进化的指导原则: 功能决定形态, 形态适应功能.
\begin{ex}
    虾/蟹的鳌的形成方式: 微小变化的积分.
\end{ex}
\begin{ex}
    白尾鹿的天然畸变个体钉角鹿, 受到自然选择的青睐而在种群中扩大传播, 因为它在逃跑时更有效.
\end{ex}
\begin{ex}
    头足纲(例如鱿鱼)的眼睛/内脏系统进化高级.
\end{ex}
昆虫的辐射式进化: 有利于种群突变/分布的因素:
\begin{cenum}
    \item 世代短, 环境作用频率高, 容易发生突变;
    \item 具变态, 营养生长与生殖分离;
    \item 体型小, 营养效率高;
    \item 性别分化多样;
    \item 可飞行;
    \item 发达的附肢和外骨骼;
    \item 习性多样;
    \item 具超冷的滞育现象, 可抵抗低温恶劣环境.
\end{cenum}
环境饰变: 环境对表型的可塑性有影响.
\par
属以上的大进化也是重要的形式.
\begin{cenum}
    \item 渐变是普遍存在的, 有分子遗传学的支持;
    \item 飞跃也是存在的, 没有渐变那样普遍. 分子生物学的机制是Hox基因. 作为进化飞跃的典型例子是啄木鸟.
\end{cenum}
\par
大小进化的关系: 间断与平衡.
\begin{cenum}
    \item 物种间缺乏过渡类型化石, 其实生命世界到处都是过渡.
    \item 分化中心与分布中心往往不在同一个地点.
    \item 高级分类群特征进化(高级分类单位是虚拟的).
\end{cenum}
高级分类群的进化: Hox基因家族.
\begin{cenum}
    \item 分子生物学研究具有身体分节的动物, 都由Hox基因家族控制体节发育.
    \item 与黑猩猩不同, 人的Hox基因可分成$4$个基因群集, 分别位于$2$, $7$, $12$, $17$号染色体. 这些基因可分成$13$个平行同源家族.
\end{cenum}

% subsection 生物多样性是如何进化的 (end)

\subsection{性选择在进化上的意义} % (fold)
\label{sub:性选择在进化上的意义}

性选择: 性选择为了种群(赢得雌性的喜爱), 自然选择为了个体生存.
\par
性选择是有代价的, 是耗能的. 进化会让性选择与自然选择达到平衡.
\begin{ex}
    白眼果蝇会被筛去.
\end{ex}
\begin{ex}
    大角鹿选择失控, 因大角更易获得青睐, 惟容易被树枝挂住.
\end{ex}
% subsection 性选择在进化上的意义 (end)

\subsection{共生与协同进化} % (fold)
\label{sub:共生与协同进化}

相互隔离的空间内, 同一种动物进化为两种不同的形态, 谓平行进化.
\begin{ex}
    白衣消化道中的原生动物和细菌帮助白蚁消化木屑是一种互利共生.
\end{ex}
互利共生谓$2$种生物个体联系在一起,  相互受益, 相互依存的关系.

% subsection 共生与协同进化 (end)

\subsection{生物多样性进化是加速的} % (fold)
\label{sub:生物多样性进化是加速的}

生物进化具有两种不同模式:
\begin{cenum}
    \item 硬进化是生命体物质内涵方面的进化.
    \item 软进化是生命体信息(基因, 心理, 文化)内涵方面的进化与改进. 同样需要经过自然选择.
\end{cenum}
生命进化具有三个信息系统:
\begin{cenum}
    \item 遗传信息系统;
    \item 心理信息系统;
    \item 文化信息系统.
\end{cenum}
进化的五种机制:
\begin{cenum}
    \item 突变-遗传机制(基因突变, 染色体畸变).
    \item 基因扩容机制(基因组扩增, 基因加倍, 沉默基因表达, 新基因起源); 适用于真核生物, 主要信息容量扩增, 形态, 结构, 功能复杂化, 实现垂直进化, 涉及到第一信息系统和第二信息系统的进化.
    \item 生殖重组机制(减数分裂, 基因重组); 适用于有性生殖的生物.
    \item 基因-心理转译机制. 先验心理是来源于祖先的经验积累(本能).
    \item 文化进化机制.
\end{cenum}
生物进化的不同历史时期, 进化模式, 信息系统和进化机制的数量是不同的, 越是高等的生物进化手段越多, 故生物进化是加速的.

% subsection 生物多样性进化是加速的 (end)

% section 生物多样性 (end)

\end{document}
