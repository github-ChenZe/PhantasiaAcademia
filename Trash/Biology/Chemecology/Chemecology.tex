\documentclass{ctexart}

\usepackage{van-de-la-sehen}

\begin{document}

\section{化学生态学} % (fold)
\label{sec:化学生态学}

\subsection{生物的信息化学物质} % (fold)
\label{sub:生物的信息化学物质}

信息化学物质是指同种或异种个体之间相互作用的化学次生物质, 能影响生物的行为、习性、繁殖、捕食、生长发育和生理活动等. 分为信息素和他感物质.
\par
信息素主要有性信息素, 聚集信息素, 警告信息素, 示踪信息素, 标记信息素等.
\par
他感物质分为利己素, 利他素, 拮抗素, 同利素.
\begin{cenum}
    \item 植物与植物间的信息联系:
    \begin{cenum}
        \item 自毒作用;
        \item 他感作用;
        \item 诱导作用(寄生生物与宿主之间).
    \end{cenum}
    \item 动物间的信息联系:
    \begin{cenum}
        \item 例如响蜜䴕与食蜜獾;
        \item 寄生吸虫对宿主蜗牛壳厚度进行条件;
        \item 寄生蠕虫影响妇女生殖率(蛔虫$\uparrow$, 钩虫$\downarrow$).
    \end{cenum}
    \item 植物与动物间的信息:
    \begin{cenum}
        \item 机械防御;
        \item 化学引诱;
        \item 色彩引诱;
        \item 味觉防御;
    \end{cenum}
\end{cenum}
化学信号可能包含性需求、追踪、报警和谋杀.
\begin{ex}
    Monomorium santschii的蚁后入侵M. salomonis巢穴后, 指使后者工蚁杀死蚁后.
\end{ex}

\subsubsection{植物的次生化学物质} % (fold)
\label{ssub:植物的次生化学物质}

植物体产生能够起到防御、刺激和吸引作用的各种化学物质. 区别在于基础代谢物质.
\par
植物的化学组成, 在时间上和空间上都是各不相同的, 是异质的. 植物具有物理、化学和发育上有效的抗性机制.
\par
次生化学物质如含氮化合物, 萜类, 酚类(而基本代谢产物如氨基酸、核苷酸、脂肪酸、糖等). 植物的次生物质相当多样.

\begin{ex}
    断肠草/鸡母珠等的次生代谢产物都包含剧毒化学物质.
\end{ex}
\begin{ex}
    大麻内部含大麻酚.
\end{ex}
\begin{ex}
    巧茶(Catha edulis)有兴奋物质卡西酮.
\end{ex}
\begin{ex}
    烟草属植物共$66$种, 原产美洲. 只有普通烟草(红花烟草)和黄花烟草被人们利用. 烟草具有相当毒性(尼古丁, 重金属, 放射性同位素). 蚂蝗会怕香烟和盐.
\end{ex}
\begin{ex}
    藜芦会含有剧毒物质环巴胺, 会导致羊的胎儿畸形.
\end{ex}

% subsubsection 植物的次生化学物质 (end)

% subsection 生物的信息化学物质 (end)

\subsection{昆虫与植物的关系} % (fold)
\label{sub:昆虫与植物的关系}

孢子植物/裸子植物与昆虫的关系简单; 这些植物是风媒植物或水媒植物.

\begin{ex}
    舞毒蛾在美国爆发后, 植物分泌单宁酸, 导致害虫死亡、食欲不振等.
\end{ex}
\begin{ex}
    阿拉斯加野兔大量死亡, 因为植物分泌有毒的萜烯.
\end{ex}

\subsubsection{昆虫对寄生植物的选择和利用} % (fold)
\label{ssub:昆虫对寄生植物的选择和利用}

\begin{cenum}
    \item 选择过程分为寻找-定向阶段; 降落着陆阶段; 接触评价阶段.
    \item 昆虫对植物毒素的反应策略为行为适应(避毒), 结构适应(储毒), 生理生化适应(解毒).
\end{cenum}
\begin{ex}
    昆虫取食在叶片上的部位不同, 避开毒. 例如热带雨林中的海芋.
\end{ex}

% subsubsection 昆虫对寄生植物的选择和利用 (end)

\subsubsection{传粉昆虫与花的关系} % (fold)
\label{ssub:传粉昆虫与花的关系}

\begin{cenum}
    \item 共生关系;
    \item 花的恒定性;
    \item 花的识别;
    \item 传粉的能量消耗;
    \item 植物的奖励策略等.
\end{cenum}

% subsubsection 传粉昆虫与花的关系 (end)

\subsubsection{次生物质在3级营养关系中的作用} % (fold)
\label{ssub:次生物质在3级营养关系中的作用}

三级营养关系包含寄主植物--植食性昆虫--天敌.
\begin{ex}
    杨树--蚜虫--土壤微生物; 胡杨--尺蠖--土壤微生物; 栎树--蛾幼虫--虫草菌等都是微生物--植物--昆虫关系的例子.
\end{ex}

% subsubsection 次生物质在3级营养关系中的作用 (end)

% subsection 昆虫与植物的关系 (end)

\subsection{他感作用与植物抗性} % (fold)
\label{sub:他感作用与植物抗性}

一种植物产生的化学物质, 释放到环境中, 对另一种植物产生了直接或间接的影响, 谓他感作用. 机制包含苯丙烷, 多聚乙酰, 类萜类, 甾类, 生物碱.

\begin{ex}
    常规抗性: 马铃薯的一种毛被蚜虫触及时, 会分泌粘液, 黏住蚜虫, 一触即断.
\end{ex}

植物的防御是有代价的. 直接代价为用于防御物质的合成、储存的能量消耗; 间接代价为用于防御的物质分配. 植物必须将有限的资源分配给生长、繁殖、防御. 常规防御是正常的耗能, 而诱导防御为机会性耗能.

% subsection 他感作用与植物抗性 (end)

% section 化学生态学 (end)

\end{document}
