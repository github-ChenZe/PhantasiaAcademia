\documentclass{ctexart}

\usepackage{van-de-la-sehen}

\begin{document}

\subsubsection*{配置} % (fold)
\label{ssub:配置}

2020年1月3日考试, 西区三教5楼, 16:00-17:00\\
助教 龚清跃 15656058497 qygong@mail.ustc.edu.cn\\
考试满分60, 平时成绩占40\%\\
有4次作业, 4次随堂考试, 作业使用给定纸张\\
\begin{pitfall}
    考试时间变更 2019.12.27(周五) 下午4点, 三教5楼.
\end{pitfall}

% subsubsection 配置 (end)

\section{生命的起源} % (fold)
\label{sec:生命的起源}

生命是一组碱基编码的信息, 各种代谢/调节路径选择的算法.
\par
\emph{模式生物}: 作为实验模型的特定研究对象. 对其研究的结论可适用于其它生物.
\begin{ex}
    逆转录病毒, 大肠杆菌, 酵母, 果蝇, 线虫, 斑马鱼, 小鼠, 拟蓝芥, 水稻...
\end{ex}
\begin{cenum}
    \item 越是高等的生物, 基因数目的蛋白质数量越多;
    \item 基因数多少, 不能简单对应生物体的复杂性;
    \item 越是高等的生物, 非编码DNA所占比例越大.
\end{cenum}
人类基因组中, 95\%的DNA不参加编码蛋白质.
\par
「垃圾DNA」并非垃圾, 而是非编码DNA.
\paragraph{生命的同源共祖} % (fold)
\label{par:生命的同源共祖}

因原核生物的基因水平转移到值起源复杂, 真核生物为单起源.
\begin{cenum}
    \item 生命起源于水体, 且为热起源;
    \item DNA是遗传物质, DNA复制使用模版-碱基配对机制, 使用三联体密码子把RNA翻译成蛋白;
    \item 将DNA转录成RNA使用有同源催化机制的RNA聚合酶, 使用rRNA, tRNA和核糖体蛋白的混合物来翻译蛋白质;
    \item ATP作为细胞内能量储存和合成DNA、RNA的能量来源;
    \item 细胞质被包在半透膜内. 营养和废物可以通过半透膜;
    \item \emph{生命}: 由物质能量和信息组成的依赖环境的开放系统. 是液晶态.
\end{cenum}

% paragraph 生命的同源共祖 (end)

\paragraph{生命起源的三个阶段} % (fold)
\label{par:生命起源的三个阶段}

元素演化, 化学演化, 生命起源.
\begin{cenum}
    \item 元素演化: 属于星球演化. 地球的位置(液态水).
    \item 化学演化: 无机物到有机物. 原始地球环境可以产生组成生物体的单糖, 脂肪, 氨基酸和核苷酸等分子单元, 但还没有出现真正的生命. cf. Grey和Miller的实验: 根据原始地球的还原大气设计了一套封闭循环试验装置, 模拟和验证了非生命有机分子在原始地球环境中组成. 惟1977年人们发现早起大气是中性的而非还原性的.
    \item 生物大分子到细胞的4个步骤:
    \begin{cenum}
        \item 生物大分子和催化作用形成;
        \item 有机物聚合成多聚体, 整合为多分子体系颗粒;
        \item 多聚体代谢与遗传体系形成. 遗传系统的起源: RNA起源假说:
        \begin{cenum}
            \item 实验显示, 试管中的RNA链可以自发地延伸和复制, 可以反转录合成RNA;
            \item 某些RNA具有像酶一样的催化活性;
            \item RNA可能是最早的遗传物质, 谓裸基因. 在早期的世界里可能遗传物质和酶无区分.
        \end{cenum}
        惟蛋白质可以知道氨基酸组装新的蛋白质, 问题又回到了原点. 最新观点为, 生命的开端并非一自我复制的分子, 而是一新陈代谢的网络.
        \item 细胞膜的出现(半透-活性): 现代生物的最早共同祖先, 即原核细胞.
    \end{cenum}
\end{cenum}
生命起源与生命进化是两种完全不同事件. 生命进化三要素: 基因突变/选择, 环境因素, 时间积累.

% paragraph 生命起源的三个阶段 (end)

\paragraph{生命起源的主要证据} % (fold)
\label{par:生命起源的主要证据}

原核细胞的起源: 通过研究微化石, 比较所有生命的共性, 设法重新制造一个细胞. 38-35亿年前地球缺乏氧气而存在还原性气体, 30亿年前地球上有原始汤.

% paragraph 生命起源的主要证据 (end)

\par
微小原核细胞复杂且进化慢

\paragraph{真核生物的起源} % (fold)
\label{par:真核生物的起源}

真核生物出现不超过20-15亿年前. 10-8亿年前真核细胞占优势.
\par
半自助细胞器的起源: 好氧细菌与线粒体, 蓝细菌与叶绿体, 在大小/膜的组成与膜蛋白的运转作用相似. 繁殖方式相似. 都有环状DNA. 中心粒也有环状DNA, 是半自主细胞器.

% paragraph 真核生物的起源 (end)

\paragraph{生物进化论发展} % (fold)
\label{par:生物进化论发展}

当今生物进化论的理论基础来自三个方面:
\begin{cenum}
    \item 拉马克进化学说;
    \item 达尔文进化论;
    \item 孟德尔-摩尔根遗传定律.
\end{cenum}

生物性状和特征变化往往是环境和遗传相互作用的结果.
\begin{ex}
    鹿科, 牛科, 猪科都是被猫科军备竞赛的产物.
\end{ex}

现代综合进化论: 想有一个基因库的群体是生物进化的基本单位; 物种形成和生物进化的机制包括\inlinehardlink{?}
\par
进化动力: 自然选择与生物适应. cf.寄生蛹实验.
\par
比较胚胎学证据: 高等物种的胚胎经过阶段与低等物种相似. 不同生物胚胎发育过程的变化揭示了一些不同的生物是由相同者进化而来的特征.
\begin{ex}
    各种类型的眼睛都是Pax6基因调控的, 而形态发育是Hox家族基因调控的.
\end{ex}
生殖隔离是新物种的形成条件.
\par
生物进化是五种进化机制的积累: 突变-遗传机制; 基因扩容机制(重要基因多份拷贝); 生殖重组机制; 基因-心理转译机制; 文化进化机制.

% paragraph 生物进化论发展 (end)

% section 生命的起源 (end)

\end{document}
