\documentclass{ctexart}

\usepackage{van-de-la-sehen}

\begin{document}

\section{生命的能量环境} % (fold)
\label{sec:生命的能量环境}

\begin{ex}[影响植物能量重新分配的实验]
    拟蓝芥每天人为触摸两次, 与对照组相比高度低. 经常抚摸的细胞壁增厚.
\end{ex}

\subsection{生命能量环境与代谢特点} % (fold)
\label{sub:生命能量环境与代谢特点}

环境是直接或间接影响某一特定生命体生存的一切事物的总和.
\par
资源与环境的区别: 在生态学中二者有时得到统一.
\par
生态因子: 环境中对生物生长发育繁殖行为和分布有直接或间接影响的环境要素(温度, 湿度, 食物, 氧气等). 生态环境是特定空间各种生态因子之和.
\par
生态因子分类: 气候因子, 土壤因子, 地形因子, 生物因子, 人为因子.
\par
生态因子三要素: 大小(数量), 质量, 性能.
\par
环境按照因子性质划分可分为能量环境和物质环境. 空间尺度与生态因子运动状态有关, 时间尺度与生态过程和进化有关.

\subsubsection{能量代谢特点} % (fold)
\label{ssub:能量代谢特点}

\begin{cenum}
    \item 生命能量本质为化学能, 传递和转化性质ATP, ADP.
    \item 生命获得能量途径为光和磷酸化, 氧化磷酸化, 无氧呼吸, 氧化磷酸化与解偶联.
    \item 环境温度直接影响生命能量的代谢速率.
    \item 绿色植物光和作用效率1\%. Co-P催化剂和Ralstonia eutropha(细菌)组成人工光和作用系统的效率为10\%.
\end{cenum}
\begin{remark}
    生命的每个层次都是由下一级的无序状态在能量作用下发展到上一级的有序状态. 生命系统是不断地由无序到有序, 按照负熵趋势进行发育演化. 而生命体的呼吸、死亡与分解, 是有序都无序, 正熵释放自由能的过程.
\end{remark}

% subsubsection 能量代谢特点 (end)

% subsection 生命能量环境与代谢特点 (end)

\subsection{光和温度的变化规律} % (fold)
\label{sub:光和温度的变化规律}

\subsubsection{光的变化规律} % (fold)
\label{ssub:光的变化规律}

光的变化规律(自学)以及太阳光对大气水分运动和海洋洋流的影响.

\paragraph{地表的增温过程} % (fold)
\label{par:地表的增温过程}

地表增温, 有
\begin{cenum}
    \item 太阳辐射$S$, 以及大气散射$S'$;
    \item 地面辐射$E_e$;
    \item 大气逆辐射$E_a$.
\end{cenum}
总的热量吸收为
\[ R = \pare{S+S'}\pare{1-\alpha} + E_a - E_e. \]
设$M$为地面与大气热交换, $L_e$为水液态与气态变换, $B$为表土与深土热交换. 在夏半年白天,
\[ R^+ > M^- + B^- + L_e^-, \]
从而温度上升. 冬半年夜晚,
\[ R^- > M^+ + B^+ + L_e^+, \]
从而温度下降.
\par
我国按候温划分季节. 即五日为一候, 取平均温度.

% paragraph 地表的增温过程 (end)

% subsubsection 光的变化规律 (end)

% subsection 光和温度的变化规律 (end)

\subsection{植物对光的适应规律} % (fold)
\label{sub:植物对光的适应规律}

\begin{cenum}
    \item 阳光既是资源又是环境;
    \item 阳光与温度紧密联系的两个生态因子;
    \item 植物对光的适应策略;
    \item 动物对光的适应策略, 维持体温, 反之信号, 醒觉亮度, 捕食与生物节律.
\end{cenum}
\begin{ex}
    地衣在树干上会选择有光的(南面)生长.
\end{ex}
\begin{ex}
    山的南坡和北坡植被带会产生差异. 东坡和西坡的植被带也会有区别, 例如气流在东坡上升时降雨, 而在西坡快速下降并吸水导致高处植被类型干燥而西坡同等植被之海拔低于东坡.
\end{ex}
\begin{ex}
    Dryas integrifolia的花瓣呈抛物型, 有利于聚集阳光. Novosievaersia glacias会根据温度而生长不同高度的部位.
\end{ex}
根据植物开花的光周期反应, 分为
\begin{cenum}
    \item 短日植物: 开花需要日照时间低于某值;
    \item 长日植物: 开花需要日照时间高于某值;
    \item 日中性植物: 开花不受日照时间影响.
\end{cenum}
植物开花还同时受到诱导周期数的影响. 红光的影响可以被远红光消除.
\begin{ex}
    充分利用光能, 农业上的连作套种.
\end{ex}

% subsection 植物对光的适应规律 (end)

\subsection{光对动物的信号作用} % (fold)
\label{sub:光对动物的信号作用}

长日照动物(春夏交配)/短日照动物(秋冬交配)/无反应.
\par
根据血液温度的高低, 分为温血动物和冷血动物, 根据血液温度的变换分为恒温动物和变温动物, 根据血液温度(能量)来源分为内温动物/外温动物/中性.
\par
冬眠/夏眠是对极端温度的主动适应. 恒温动物的休眠有真冬眠和假董秘嗯. 前者谓代谢降低, 体温高于环境$\SI{1}{\celsius}$到$\SI{2}{\celsius}$. 冰点时褐色脂肪团颤动产热激醒动物. 后者代谢不变.
\par
生物学零度谓生物代谢活动的最低温迪. 积温谓生物发育某阶段内逐天温度的累积值.
\begin{cenum}
    \item 活动积温谓大于冰点的积温;
    \item 有效积温谓大于生物学零度的积温.
\end{cenum}
同种生物在相同阶段内所需有效积温相同.
\par
温差波动越小, 该地区的树长得越高.
\par
物候与物候定律: 温度节律决定生物体时钟, 与光线无关. 物候: 生物长期适应原产地的气候条件, 形成了与此相适应的生长节律.
\begin{ex}
    随着海拔变化, 冬小麦物候会发生变化.
\end{ex}

% subsection 光对动物的信号作用 (end)

\subsection{Summary} % (fold)
\label{sub:summary}

\begin{cenum}
    \item 生物对体外能量环境的适应策略: 形态, 颜色, 行为, 繁殖, 生活史;
    \item 生物对体内能量环境的适应策略: 组织结构, 生理生化, 体制与寿命, 主动调节机制;
    \item 生物通过光和磷酸化, 动物通过氧化磷酸化获得能量; 生物从不浪费能量;
    \item 生物是负熵体.
\end{cenum}

% subsection summary (end)

% section 生命的能量环境 (end)

\end{document}
