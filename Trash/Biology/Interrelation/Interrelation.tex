\documentclass{ctexart}

\usepackage{van-de-la-sehen}

\begin{document}

\section{种内关系与种间关系} % (fold)
\label{sec:种内关系与种间关系}

\begin{ex}
    马达加斯加的彗星兰和长喙蛾协同进化.
\end{ex}

\subsection{种内关系的多样性} % (fold)
\label{sub:种内关系的多样性}

群落内部存在竞争/互助/协同等多种关系. 行为是体内外刺激所导致的生物体内生理情况和变化而引起的反映, 也是动植物和人类适应环境的产物.

\begin{cenum}
    \item 种内互助(负竞争)与利他行为;
    \item 种内竞争:
    \begin{cenum}
        \item 资源性竞争(具密度效应);
        \item 分摊竞争(资源平均分配, 同(甘共苦);
        \item 干扰性竞争(偷吃对方的卵);
        \item 自毒作用(豌豆, 西瓜, 西兰花);
    \end{cenum}
    \item 种内残杀(黄鳝, 性选择的雄性杀幼);
    \item 领域性与社会等级.
\end{cenum}
\begin{ex}
    有些细菌的质粒在必要时可制造并释放毒素, 不但可杀死宿主细胞, 同时自身死亡. 当宿主细胞破裂并释放毒素, 会杀死周围不含本质粒的细菌.
\end{ex}
\[ \text{亲缘关系系数} = \text{祖先数}\times \half \text{代距数}. \]
\begin{ex}
    鬣蜥主动放弃浅水的食物而潜入深海中吃海藻.
\end{ex}
\begin{ex}
    蜜罐蚁的工蚁分三种, 普通工蚁, 兵蚁和储蜜工蚁. 兵蚁无法自己进食, 需要工蚁喂养. 蜜罐蚁储存的蜜可以供多只工蚁喂养多日.
\end{ex}
\begin{ex}
    燕子具有较强的种间领域性.
\end{ex}
种间关系一共六种, 互利共生(++)/偏利关系(+0)/捕食关系(+-)/中性关系(00)/偏害关系(0-)/竞争(--).

\subsubsection{竞争关系} % (fold)
\label{ssub:竞争关系}

两种硅藻分别存在时密度都会上升, 但同时存在时会发生竞争导致一种被淘汰. 草履虫之间的竞争也有类似后果.
\paragraph{Gause假说} % (fold)
\label{par:gause假说}

Gause假说(竞争排斥原理): 亲缘关系接近的, 具有同样习性或生活方式者不可能长期在同一区域内共同生活.

% paragraph gauss假说 (end)

\paragraph{Lotka-Volterra竞争模型} % (fold)
\label{par:lotka_volterra竞争模型}

两个不同种群生活在一起, 竞争是必然的. 竞争系数的大小对于一个种群的竞争能力或竞争结果起重要作用.
\[ \+dtd{N_1} = r_1N_1\frac{K_1-N_1-\alpha N_2}{K_1},\quad \+dtd{N_2} = r_2N_2 \frac{K_2 - N_2 - \beta N_1}{K_2}. \]
两个物种的平衡线即$\displaystyle \+dtd{N_1} =0, \+dtd{N_2} = 0$处. 若$N_1,N_2>0$处无交点, 则两条直线在上方的物种获胜. 若有交点则可能共存或不明确.

% paragraph lotka_volterra竞争模型 (end)

\paragraph{种间基因的延伸作用} % (fold)
\label{par:种间基因的延伸作用}

蜗牛壳的厚度是受寄生的吸虫控制的. 又例如蛔虫寄生在中美洲蚂蚁体内.

% paragraph 种间基因的延伸作用 (end)

% subsubsection 竞争关系 (end)

\subsubsection{捕食关系} % (fold)
\label{ssub:捕食关系}

水中的真菌会有套环, 这是由线虫导致的.
\par
猎物种群在没有捕食者的情况下, 是按照指数增长的. 捕食者在没有猎物的情况下, 是按照指数下降的. 相遇后则两者之间的数量发生振荡.
\begin{ex}
    美洲兔和猞猁的数量按照十年的周期变化.
\end{ex}
捕食者与猎物的协同进化不是对称的.
\begin{cenum}
    \item 猎物的世代比捕食者要短;
    \item 捕食者的密度小于猎物的密度;
    \item 活命-饱餐理论表明两者压力不同;
    \item 捕食者多数有多种食物;
    \item 捕食者有避稀效应;
    \item 捕食中, 猎物的一般性防御出现较早.
\end{cenum}
\begin{ex}
    过度放牧和禁止放牧都会破坏草原植被.
\end{ex}
逃避捕食的策略之一为拟态. 例如没有毒的生物模仿剧毒的生物或是中等毒性的生物模仿有剧毒的生物以保护自己.
\begin{ex}
    雪蝶拟态起翅膀为跳蛛的腿降低了被捕食的几率.
\end{ex}

% subsubsection 捕食关系 (end)

\subsubsection{其它种间关系} % (fold)
\label{ssub:其它种间关系}

如互利共生. 或社会性寄生(一种生物自己不育雏, 把卵产在另一物种的巢内).

% subsubsection 其它种间关系 (end)

% subsection 种内关系的多样性 (end)

\subsection{种内与种间关系在群落中的统一} % (fold)
\label{sub:种内与种间关系在群落中的统一}

两个种群以上同时存在, 则种间关系上升为主要矛盾, 种内关系降为次要矛盾. 群落中, 种内和种间两种关系共存, 通过调整或改变生态位以共存.

% subsection 种内与种间关系在群落中的统一 (end)

% section 种内关系与种间关系 (end)

\end{document}
