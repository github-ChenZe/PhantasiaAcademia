\documentclass{ctexart}

\usepackage{van-de-la-sehen}

\begin{document}

\section{种群的遗传与进化} % (fold)
\label{sec:种群的遗传与进化}

大种群可提高繁殖成功率, 可有效对付捕食者.

\subsection{种群的概念与特征} % (fold)
\label{sub:种群的概念与特征}

种群谓特定时间内分布在一定空间范围内, 可自由交流基因的同一物种所有个体的集群. 它是介于生物个体与生物群落间的一个客观存在的生物学基本单位, 具有自己独特的结构组成, 生物学特征和功能.
\par
种群是物种进化的基本单位, 也是群落和生态系统的基本功能单位.
\begin{pitfall}
    物种不是生物学层次, 而是分类学单元/遗传繁殖单元. 种群是物种生存, 物种进化和种间关系的基本单元.
\end{pitfall}

\subsubsection{空间特征} % (fold)
\label{ssub:空间特征}

植物种群和少数动物具有个体的遗传单位和个体之上的构建单位(草莓, 竹子, 水螅).
\par
种群空间分布类型可分为均匀形, 集居形和随机形.

% subsubsection 空间特征 (end)

\subsubsection{数量特征} % (fold)
\label{ssub:数量特征}

种群的数量大小/数量变化/种群爆发/绝灭/环境要求的最小种群(灭绝点)/最大容量/最适宜种群大小/临界数量等.

% subsubsection 数量特征 (end)

\begin{ex}[特殊的自然种群 --- 超级有机体]
    白蚁(等翅目昆虫), 蚂蚁, 蜜蜂(膜翅目昆虫). 傍晚时分, 刚羽化的白蚁前赴后继地飞上高空, 沿途被无数捕食者捕获. 
\end{ex}
Allee定律表明, 每个种群都有一个大小最适合的种群数量, 过大或过小都不利于种群的发展. 过大容易失控, 过小交配机会太少.
\begin{ex}
    蜜蜂的分箱证明该定律.
\end{ex}
性比谓不同阶段的两性别比例.
\par
鱼类具有相当复杂的性别决定机制.
\begin{cenum}
    \item 性染色体决定: XX/XY型; ZZ/ZW型; XO/XX性; ZO/ZZ型; 复杂(WXY型, WY, WX, XX为雌性, YY和XY为雄性).
    \item 雌雄同体, 性逆转(黄鳝/石斑鱼).
    \item 体型大小, 大雄小雌(黄点拟矶塘鳢).
    \item 密度(叉尾斗鱼).
    \item pH(剑尾鱼).
    \item 温度决定(最多).
    \item 外源激素.
\end{cenum}

\subsubsection{生命表} % (fold)
\label{ssub:生命表}

生命表是描述死亡过程的有用工具, 能综合判断种群数量变化, 也能充分反映从出生到死亡的动态关系.
\par
动态生命表是根据观察一群同事出生的生物之死亡或存活动态过程所得. 静态是某特定时间对种群某些年龄结构个体抽样调查数据编制得到.

% subsubsection 生命表 (end)

% subsection 种群的概念与特征 (end)

\subsection{种群的增长与遗传} % (fold)
\label{sub:种群的增长与遗传}

繁殖方式:
\begin{cenum}
    \item 动物的繁殖: 营养繁殖, 无性繁殖, 有性生殖.
    \item 动物的繁殖: 营养繁殖, 孤雌生殖, 卵生, 卵胎生, 胎生.
\end{cenum}
指数增长是理想情形. 实际为Logistic增长: 最初起始阶段 --- 对数增长阶段 --- 减速阶段. 增长率正比于$N\pare{K-N}$, 故$N=K/2$时有最大增长率.

\subsubsection{种群数量的调节} % (fold)
\label{ssub:种群数量的调节}

种群数量的变动来自外因和内因. 外源性的调节理论如非密度制约的气候学说.

% subsubsection 种群数量的调节 (end)

\par
奠基者事件会导致遗传的多样性下降. 例如非洲白人羊角风发生率升高.

% subsection 种群的增长与遗传 (end)

\subsection{物种进化的基本单元} % (fold)
\label{sub:物种进化的基本单元}

\subsubsection{对环境因子的趋异适应 --- 生态型} % (fold)
\label{ssub:对环境因子的趋异适应_生态型}

生态型即一个物种的一群个体对某一特定生态因子发生基因型反映.
\par
自然选择发生作用需要表现型差异, 适合度差异和适合度遗传.
\begin{ex}
    稻的生态型有水分生态型(水稻/旱稻), 温度生态型(籼稻, 粳稻), 光照生态型(早稻, 中稻, 晚稻).
\end{ex}
生态型在分类中时亚种和品种, 具有遗传稳定性.

% subsubsection 对环境因子的趋异适应_生态型 (end)

% subsection 物种进化的基本单元 (end)

% section 种群的遗传与进化 (end)

\end{document}
