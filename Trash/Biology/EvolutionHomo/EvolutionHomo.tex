\documentclass{ctexart}

\usepackage{van-de-la-sehen}

\begin{document}

\section{人类的进化} % (fold)
\label{sec:人类的进化}

\subsection{人类的原始祖先} % (fold)
\label{sub:人类的原始祖先}

历史上, 人科至少有5个属, 18个种. 现在人类仅属于1个物种.

% subsection 人类的原始祖先 (end)

\subsection{人类的进化} % (fold)
\label{sub:人类的进化}

%人类是会制造工具的, 直立行走的猿.
人类进化史科分为四个阶段:
\begin{cenum}
    \item 早期猿人;
    \item 晚期猿人;
    \item 早期智人;
    \item 晚期智人.
\end{cenum}
猕猴和大猩猩具有大型颞区, 而人类的较小, 为咀嚼所要求.
\par
自高等爬行动物才出现大脑皮层, 褶皱从无到有.
\par
鸟类无腰椎, 黑猩猩有3--4节腰椎, 人类有5节腰椎.
\par
抓握反射拇指在食指和中指之间, 来源于猩猩.
\par
盲肠主要消化纤维素, 食草动物盲肠较大. 阑尾是有益菌的庇护所, 也可能是免疫系统的一部分.
\par
返祖现象, 例如尾椎生成尾巴, 或多毛基因被表达.
\par
人类和黑猩猩的蛋白质平均只有1\% 的差异. 超过6\% 的基因没有出现在任何黑猩猩基因中, 有1400新基因出现在人体内. 黑猩猩和人类中都有唾液淀粉酶基因. 黑猩猩中只有1份拷贝, 而人有2--15个.
\par
人类进化是由外因和内因共同决定的. 古猿来到地面: 地理分化. 人类躯干直立: 前后肢分工. 手的变化: 劳动与脑的互动加速手的进化. 语言的产生: 复杂劳动促进智慧和语言的发展. 思维发展与脑量增加: 语言促进思维发展, 劳动和语言促进大脑, 形成文化.

% subsection 人类的进化 (end)

\subsection{现代人如何进化而来} % (fold)
\label{sub:现代人如何进化而来}

最初的古人类可能出现于500--700万年前的非洲, 在大致100万年前走出非洲, 迁移到欧亚大陆. 现代人类在10万年前第二次走出非洲, 取代其他地区的古人类种. Linnaeus最初主张人类可分为高加索人(白人), 蒙古人(黄种人), 格罗尼人(黑人), 印第安人(红). 后有学者主张分为三种. 但实际上种族之间基因差异不明显, 小于个体间的差异.


% subsection 现代人如何进化而来 (end)

\subsection{进化中的缺陷} % (fold)
\label{sub:进化中的缺陷}

气管在食道以前, 容易被噎住和打嗝. 又例如喉返神经.
\par
个体发育会出现早期祖先的显著特征, 如腮和尾.

% subsection 进化中的缺陷 (end)

% section 人类的进化 (end)

\end{document}
