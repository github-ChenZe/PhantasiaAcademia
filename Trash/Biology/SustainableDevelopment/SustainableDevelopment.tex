\documentclass{ctexart}

\usepackage{van-de-la-sehen}

\begin{document}

\section{生态系统管理与可持续发展} % (fold)
\label{sec:生态系统管理与可持续发展}

\subsection{生态系统的功能} % (fold)
\label{sub:生态系统的功能}

生态系统是由生物群落与非生物环境相互作用而构成的集合体, 是生物学研究的最高的生命层次.
\par
生态系统是生态学的基本功能单位.
\par
种群是生态系统中生态作用的基本功能单位.
\par
生态系统的功能体现在能量流, 物质流和信息流运动方面.
\begin{citem}
    \item 能量流: 太阳能转化后沿食物链流动, 越流越细, 符合$10\%$定律;
    \begin{cenum}
        \item 生命系统是沿着负熵逐级进化的, 每一级生命自身的代谢和进化都是消耗能量的;
        \item 生命系统都是由下一级的无需状态在能量作用下进化到上一级的有序状态;
        \item 生命系统是不断由无需到有序的过程, 生命体的呼吸, 死亡与分解是由有序到无序, 按正熵释放自由能的过程.
    \end{cenum}
    \item 物质流: 生物库与环境库间循环, 分为完全循环与半循环;
    \begin{cenum}
        \item 物质是能量载体;
        \item 物质所含能量在化学键中, 绿色植物吸收并转化太阳能, 启动了食物链;
        \item 物质在环境和生物体之间不断地运动者, 并且是一种由运输戒指实现的往复式的运动方式;
        \item 生态系统是一个由光合营养, 捕食营养和吸收营养三种方式, 将物质和能量联系在一起的开放系统.
    \end{cenum}
    \item 信息流: 各生物种群间的信号联系.
\end{citem}

% subsection 生态系统的功能 (end)

\subsection{生态系统的管理} % (fold)
\label{sub:生态系统的管理}

\subsubsection{生态系统管理的意义} % (fold)
\label{ssub:生态系统管理的意义}

\begin{cenum}
    \item 生态系统具有发育和演替过程; 天然或人工生态系统;
    \item 物质在系统内是动态循环使用的, 能量不断地输入系统, 生态系统健康;
    \item 生态系统有稳定的阈值, 生态系统服务(服务对象与范围如生态系统内部服务和生物圈服务); 生态产品边界价值: 取决于生态系统服务价值的变化与生态系统服务供给的减少两者之间的相对关系;
    \item 根据生物个体的生长规律, 科学利用生物资源.
\end{cenum}
\begin{ex}
    生长对数期到缓慢生长期的转折点是捕获的最佳时间. 
\end{ex}

% subsubsection 生态系统管理的意义 (end)

\subsubsection{生态系统的管理模式} % (fold)
\label{ssub:生态系统的管理模式}

\begin{ex}
    桑基鱼塘系统.
\end{ex}
\begin{ex}
    日本的四维农业. 作物产品$\rightarrow$秸秆$\rightarrow$蘑菇$\rightarrow$牛饲料$\rightarrow$蚯蚓$\rightarrow$蚯蚓粪便作肥料.
\end{ex}
\begin{ex}
    新加坡的马雅生态农场.
\end{ex}
\begin{ex}
    英国洛桑庄园.
\end{ex}
\begin{ex}
    以梳齿式农田开发模式取代传统的围湖造田.
\end{ex}

% subsubsection 生态系统的管理模式 (end)

% subsection 生态系统的管理 (end)

\subsection{可持续发展及其面临的问题} % (fold)
\label{sub:可持续发展及其面临的问题}

可持续发展谓在不牺牲未来几代人的需要的情况下, 满足我们这代人的需要的发展.
\par
经济的可持续发展式基础, 生态的可持续发展是条件, 社会的可持续发展是目的.
\par
生产和生活中应当遵循3R原则, 减量化原则, 再使用原则, 再循环原则.

% subsection 可持续发展及其面临的问题 (end)

% section 生态系统管理与可持续发展 (end)

\end{document}
