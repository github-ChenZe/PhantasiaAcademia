\documentclass{ctexart}

\usepackage{van-de-la-sehen}

\begin{document}

\section{生物多样性} % (fold)
\label{sec:生物多样性}

\subsection{生物多样性的概念} % (fold)
\label{sub:生物多样性的概念}

生物多样性实质地球上一定范围内的生命有机体及其生存环境的多样化程度, 包括
\begin{cenum}
    \item 物种多样性;
    \item 遗传多样性;
    \item 生态系统多样性
\end{cenum}
物种多样性(条件)->(突变, 自然选择, 人工选育)->遗传多样性(结果)->生态系统多样性(基础)(隔离, 适应于进化)->Loop back.

% subsection 生物多样性的概念 (end)

\subsection{动物的物种多样性和遗传多样性} % (fold)
\label{sub:动物的物种多样性和遗传多样性}

物种多样性可分为区域物种多样性和群落物种多样性.
\par
形态学种谓一群形态结构相同或相似的生物个体;
生物学种谓能一群能成功相互交配的生物个体, 与其他物种发生生殖隔离.
动物命名法: 双名法: 属名+种本名(命名人姓氏), 三名法: 属名+种本名+亚种名.

\subsubsection{动物的分类} % (fold)
\label{ssub:动物的分类}

人为分类系统和自然分类系统.

% subsubsection 动物的分类 (end)

\subsubsection{物种的形成} % (fold)
\label{ssub:物种的形成}

基因突变, 重组, 扩增, 生殖隔离, 自然选择, 主动适应, 杂交. 异地物种形成: 不同环境的种群形成新物种的过程. 同地物种形成: 某湖中鱼的进化.
\begin{ex}
    生态位分化, 如马达加斯加的狐猴代替啄米鸟的生态位.
\end{ex}

% subsubsection 物种的形成 (end)

多样式指数是反映丰富度和均匀度的综合指标. Shannon-Wiener index即
\[ H = -\sum P_i \ln P_i, \]
其中$P_i$是物种$i$的$N_i / N$.

\paragraph{遗传多样性} % (fold)
\label{par:遗传多样性}

遗传多样性指存在于生物个体内, 某个物种内以及物种之间的遗传变异的总和.

% paragraph 遗传多样性 (end)

\paragraph{亚种} % (fold)
\label{par:亚种}

种内个体在地理和生态上充分隔离后, 形成的具有一定特征的群体. 亚种反映了物种对各地环境的适应, 可促进种的繁荣.

% paragraph 亚种 (end)

\paragraph{品种} % (fold)
\label{par:品种}

种内由于人工选择产生的具有新形态或新性状的个体总称. 例如蛋鸡, 肉型鸡.

% paragraph 品种 (end)

\begin{remark}
    人工选育的效应可能超过自然选择, 例如人对狗的驯化.
\end{remark}
\begin{ex}
    家猫也是驯化的结果. 驯化综合征表明驯养动物肾上腺减小, 脑组织体积变小, 温顺.
\end{ex}
\begin{ex}
    马与驴父杂交, 得驴骡. 驴与马父杂交, 得马骡.
\end{ex}

% subsection 动物的物种多样性和遗传多样性 (end)

\subsection{动物对环境的适应和进化} % (fold)
\label{sub:动物对环境的适应和进化}

前进进化谓生物的体制结构复杂化. 分歧进化谓物种分化过程. 两者非独立过程, 前进进化是物种分歧过程中由种间竞争导致的结果. 
\par
辐射适应谓动物为了扩大分布区而进行的进化. 动物的辐射适应如
\begin{cenum}
    \item 各大洲之间的动物, 地上, 地下, 空中都有与之对应的种类.
    \item 同一块大陆, 各类动物为了资源利用会错开生态位;
    \item 同一类动物错开食谱.
\end{cenum}
水生动物, 如棘皮动物也发生生态位分化(随时过滤/杂食/食腐).

\paragraph{趋同适应进化} % (fold)
\label{par:趋同适应进化}

趋同适应谓不同的生物活在相似环境中, 形成了相似的形态结构.
\begin{ex}
    兔形目(鼠兔科)和啮齿目有共同特征.
\end{ex}

% paragraph 趋同适应进化 (end)

\paragraph{趋异适应} % (fold)
\label{par:趋异适应}

趋异适应谓相同或相近的生物活在不同环境中形成了不同的形态结构.

% paragraph 趋异适应 (end)

\paragraph{协同进化} % (fold)
\label{par:协同进化}

如日本的琉球钝头蛇与蜗牛的军备竞赛. 这导致蜗牛右旋壳突变为左旋壳.

% paragraph 协同进化 (end)

\paragraph{行为和结构的协同适应} % (fold)
\label{par:行为和结构的协同适应}

功能决定形态, 形态适应功能. 动物, 植物的结构都反映这一规律. 犰狳的行为, 与鼠妇等自卫行为相同.

% paragraph 行为和结构的协同适应 (end)

\paragraph{生活史和行为多样性} % (fold)
\label{par:生活史和行为多样性}

十七年蝉即为一例. 在土壤中蛹吸允树根液, 滞留17年后羽化, 其生活史周期是17年. 此外还有十三年的种群.

% paragraph 生活史和行为多样性 (end)

\paragraph{行为多样性} % (fold)
\label{par:行为多样性}

行为的意义: 借助行为适应多变的环境达到生存和繁衍的目的. 本性为「吃」, 本能为「生」.
\begin{ex}
    拟态和保护色即为一例.
\end{ex}
\begin{ex}
    鸟类是世界上迁徙具体最长的动物. 会借助风力减少能耗. 
\end{ex}
\begin{ex}
    企鹅通过体表皮毛保温, 而鸵鸟需要散热.
\end{ex}
\begin{ex}
    哺乳行为并非哺乳动物特有. 蜘蛛幼虫会在母亲胸腹部吸出液体, 其蛋白质含量高于牛奶$4$倍.
\end{ex}

% paragraph 行为多样性 (end)

\subsubsection{繁殖习性的适应与进化} % (fold)
\label{ssub:繁殖习性的适应与进化}

水螅和水母无性生殖, 一些种类有明显的世代交替. 故生殖行为有多样性.

% subsubsection 繁殖习性的适应与进化 (end)

% subsection 动物对环境的适应和进化 (end)

\subsection{人类与动物的关系} % (fold)
\label{sub:人类与动物的关系}

模式生物: 对特定生物物种进行研究, 揭示普遍的生命现象, 谓该生物模式生物.

\subsubsection{人类行为对生物的影响} % (fold)
\label{ssub:人类行为对生物的影响}

例如生物灭绝.

% subsubsection 人类行为对生物的影响 (end)

% subsection 人类与动物的关系 (end)

\begin{cenum}
    \item 动物多样性有物种和遗传多样性, 离不开生态系统多样性.
    \item 除了基因突变, 重组, 隔离, 杂交还有各种适应是进化的主要动力.
    \item 种下单位有亚种, 品种.
\end{cenum}

% section 生物多样性 (end)

\end{document}
