\documentclass{ctexart}

\usepackage{van-de-la-sehen}

\begin{document}

\section{植物的群落结构与演替} % (fold)
\label{sec:植物的群落结构与演替}

\subsection{生物群落的概念} % (fold)
\label{sub:生物群落的概念}

生物群落谓在一定时间和空间范围内, 由各种生物种群组成的集合体.
\par
生物群落包括动物群落, 植物群落, 微生物群落.
\par
植物群落
\begin{cenum}
    \item 具一定的种类组成;
    \item 具形态结构, 营养结构, 生态结构, 季相;
    \item 形成群落环境(外部和内部);
    \item 种群间相互影响(竞争, 生态位);
    \item 具动态发育和生物量积累的特征;
    \item 具分布规律(水, 热)和特定范围(植被带);
    \item 是否有明显的边界?
\end{cenum}
关于植物群落是否为客观实体, 有机论学派认为「这是客观存在的, 有组织结构的」, 个体论学派认为「群落结构是松散的, 内部是连续变化的, 没有明显边界的, 是人为划分的非自然单元」.

% subsection 生物群落的概念 (end)

\subsection{植物群落的种类组成} % (fold)
\label{sub:植物群落的种类组成}

样地是在群落内部比较典型的自然地段, 样方是能够反映群落组成的最小面积.
\par
最小样方:
\begin{cenum}
    \item 森林乔木层10米x10米;
    \item 灌木层4米x4米;
    \item 草木层1米x1米;
    \item 北方针叶林400平方米;
    \item 雨林2500平方米.
\end{cenum}
\paragraph{曹氏样方法} % (fold)
\label{par:曹氏样方法}

以面积为横坐标, 样方内物种数为纵坐标, 获得一条曲线. 曲线转折处对应的面积即为最佳样方面积. 位于最佳样方面积上的冗余种可以忽略.

% paragraph 曹氏样方法 (end)

\begin{cenum}
    \item 优势种: 数量多, 体积大, 盖度大的种群;
    \item 建群种: 对群落的结构和环境起决定作用的种群, 或者谓关键种;
    \item 优势种不一定是建群种, 建群种不一定是优势种.
\end{cenum}

\paragraph{频度数与频度定律} % (fold)
\label{par:频度数与频度定律}

$F$为某个物种出现的样方数占群落总样方数的百分比. 频度定律谓
\[ A > B > C \ge D < E. \]
经过统计, $F$值$1\sim 20\%$为$A$, 依次递增, $81\%\sim 100\%$为$E$.

% paragraph 频度数与频度定律 (end)

\paragraph{物种多样性} % (fold)
\label{par:物种多样性}

生物多样性谓一定地域或空间范围内, 生物种类数量和环境的多样化程度.
\par
群落的物种多样性谓群落中物种的大小等.

% paragraph 物种多样性 (end)

% subsection 植物群落的种类组成 (end)

\subsection{植物群落的结构与影响因素} % (fold)
\label{sub:植物群落的结构与影响因素}

\subsubsection{植物群落的水平结构} % (fold)
\label{ssub:植物群落的水平结构}

局域空间异质性等.

% subsubsection 植物群落的水平结构 (end)

\subsubsection{植物群落的时间结构} % (fold)
\label{ssub:植物群落的时间结构}

季相等.

% subsubsection 植物群落的时间结构 (end)

\subsubsection{植物群落的生态结构} % (fold)
\label{ssub:植物群落的生态结构}

生活型: 趋同适应的结果.
\par
生活型的分类有高位芽植物, 地上芽植物, 地面芽植物, 地下芽植物, 一年生植物.
\par
生活型在岛屿上可能发生改变.

\paragraph{植物最佳叶面积的预测} % (fold)
\label{par:植物最佳叶面积的预测}

将不同叶面积的光合速率/耗水曲线对比, 收益最大者为最佳面积.

% paragraph 植物最佳叶面积的预测 (end)

% subsubsection 植物群落的生态结构 (end)

\subsubsection{影响群落结构的因素} % (fold)
\label{ssub:影响群落结构的因素}

\begin{cenum}
    \item 同资源种团: 等价种, 生态位近似, 利用资源的方式相同;
    \item 绞杀干扰作用(林窗间隙作用);
    \item 空间异质性;
    \item 竞争和捕食关系的影响.
\end{cenum}

% subsubsection 影响群落结构的因素 (end)

\subsubsection{群落的地带性分布} % (fold)
\label{ssub:群落的地带性分布}

沿经纬度的分布, 沿海拔的分布, 水体中的群落等. 群落对环境的影响如时间, 空间, 气候, 土壤.

% subsubsection 群落的地带性分布 (end)

\subsubsection{植被垂直地带性分布规律} % (fold)
\label{ssub:植被垂直地带性分布规律}

山体的自然环境条件, 垂直分布带谱, 水平分布和垂直分布的相关性.
\par
水域植被是非地带性的.

% subsubsection 植被垂直地带性分布规律 (end)

% subsection 植物群落的结构与影响因素 (end)

\subsection{植物群落的演替与分类} % (fold)
\label{sub:植物群落的演替与分类}

\subsubsection{演替} % (fold)
\label{ssub:演替}

群落演替是一个群落随着自身的发育终将被另一个群落所替代的过程.
\par
原生演替谓聪没有任何繁殖体的土壤基质中发育为一个群落的过程.
\par
一般模式为地衣-苔藓-蕨类或草本种子植物-木本植物.

\paragraph{次生演替} % (fold)
\label{par:次生演替}

次生演替: 从次生裸地上发育起来的植物群落演替过程.
\par
一般模式为蕨类植物或草本植物-木本植物.
\par
例外: 可以从阳性树种开始.

% paragraph 次生演替 (end)

% subsubsection 演替 (end)

\subsubsection{分类} % (fold)
\label{ssub:分类}

植被型: 高级单位, 如落叶阔叶林. 群系: 中级单位, 例如白桦群系. 群丛: 低级单位.

% subsubsection 分类 (end)

% subsection 植物群落的演替与分类 (end)

% section 植物的群落结构与演替 (end)

\end{document}
