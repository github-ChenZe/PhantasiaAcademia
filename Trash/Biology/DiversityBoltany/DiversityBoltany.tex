\documentclass{ctexart}

\usepackage{van-de-la-sehen}

\begin{document}

\section{植物多样性} % (fold)
\label{sec:植物多样性}

\begin{remark}
    眼虫和裸藻等同时具有动物和植物的特性.
\end{remark}

\subsection{植物界的新概念} % (fold)
\label{sub:植物界的新概念}

五界系统: 原核生物, 原生生物, 真菌, 植物, 动物.
\begin{definition}
    植物谓具有细胞壁和液泡, 并以孢子繁殖的自养生物, 以及种子繁殖的生物.
\end{definition}
植物细胞之三大特征谓细胞壁, 质体和液泡. 95\% 的植物有两种植物体(孢子体$2n$和配子体$n$).
\par
植物界分为有胚植物(高等植物, 苔藓/蕨类/裸子/被子)和无胚植物(低等植物, 藻类). 藻类/苔藓/蕨类属于孢子植物.
\par
地球上90\% 以上的生物量是植物的生物量. 包括物种/遗传/生境/习性/营养方式/繁殖方式多样性.
\par
地球上的植物中, 被子植物比重最大(29万/50万), 藻类次之(12万).
\par
生物分类七级, 界门纲目科属种. 科最具有识别性(e.g. 杜鹃花科, 50属, 1300种).
\par
双名法: 属名+种加词+命名人.
\begin{cenum}
    \item 属名第一个字母大写, 种加词全部小写;
    \item 属名不可重复使用, 种加词可以;
    \item 正式出版物中, 属名/种加词全部斜体.
\end{cenum}
例如苹果为\textit{Malus pumila} Mill. 银杏为\textit{Ginkgo biloba} L.
\par


% subsection 植物界的新概念 (end)

\subsection{植物的遗传多样性} % (fold)
\label{sub:植物的遗传多样性}

例如自然植物类群中的种下单位, 人工栽培条件下的品种, 太空育种等手段的品种.
\par
多倍体的特点是个体大, 直接加倍, 例如烟草/胜利有才/马铃薯/葡萄. 四倍体交二倍体得到三倍体.
\par
野生甘蓝不同类型的突变会得到甘蓝/羽叶甘蓝/榨菜/包心菜/西兰花/花椰菜等.
\par

\subsubsection{植物的生境多样性} % (fold)
\label{ssub:植物的生境多样性}

例如胡杨的叶形是由环境修饰的, 在分类学上无意义.

% subsubsection 植物的生境多样性 (end)

\subsubsection{植物主动适应环境} % (fold)
\label{ssub:植物主动适应环境}

适应干扰环境, 如卷丹百合的三重生殖策略.
\par
适应人工环境, 如玉米地杂草饭包草在花生地中与芝麻地中的叶片明显不同, cf. 植物的拟态.
\par
适应捕食环境, 如西番莲为了逃避捕食将改变叶片形态, 形成模仿虫卵的点.

% subsubsection 植物主动适应环境 (end)

\subsubsection{营养方式的多样性} % (fold)
\label{ssub:营养方式的多样性}

寄生和腐生是高级性状.
\par
桑寄生科的植物为了确保入侵宿主, 种子无种皮, 裸露胚乳包裹胚, 表面具有粘液.
\par
食虫植物是自养兼异养的(e.g. 猪笼草/捕蝇草/茅膏菜). 

% subsubsection 营养方式的多样性 (end)

\subsubsection{其他方面多样性} % (fold)
\label{ssub:其他方面多样性}

物种/遗传/生境/营养方式外, 还有分布/大小/习性/形态/结构/生活史/繁殖方式. 习性如草木/藤木/乔木, 次生代谢物质方面, 是资源植物和药用植物的基础.

% subsubsection 其他方面多样性 (end)

\subsubsection{菌物类多样性} % (fold)
\label{ssub:菌物类多样性}

菌物类具细胞壁(某阶段具细胞壁), 为异养生物(无叶绿素, 吸收营养).
\par
菌类有细菌门(具肽聚糖细胞壁), 黏菌门(营养体具动物行为, 生活史介于动物和植物之间), 真菌门.
\par
真菌亚门有鞭毛菌纲, 接合菌纲, 子囊菌纲, 担子菌纲, 半知菌类.
\par
e.g. 蛋巢菌, 减数分裂, 含8个孢子, 等雨滴打散孢子.

% subsubsection 菌物类多样性 (end)

\subsubsection{真菌的危害} % (fold)
\label{ssub:真菌的危害}

真菌在生态系统中发挥重要作用, 部分给人们的生产合生活带来益处, 如酵母/曲霉/青霉/食用菌, 惟部分亦对人类/生产有危害.

% subsubsection 真菌的危害 (end)

% subsection 植物的遗传多样性 (end)

\subsection{生物多样性的价值与保护} % (fold)
\label{sub:生物多样性的价值与保护}

保护途径: 迁地保护, 就地保护, 种质基因离体保护.
\par
生物多样性是进化的结果, 有生物对环境的主动适应, 也有环境对生物的选择压力和定向塑造作用.

% subsection 生物多样性的价值与保护 (end)

% section 植物多样性 (end)

\end{document}
