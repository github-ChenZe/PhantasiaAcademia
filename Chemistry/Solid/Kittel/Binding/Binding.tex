\documentclass[hidelinks]{article}

\usepackage[sensei=Nakahara,gakka=Geometry\ in\ Physics,section=Quantum,gakkabbr=QM]{styles/kurisuen}
\usepackage{sidenotes}
\usepackage{van-de-la-sehen-en}
\usepackage{van-de-environnement-en}
\usepackage{boite/van-de-boite-en}
\usepackage{van-de-abbreviation}
\usepackage{van-de-neko}
\usepackage{van-le-trompe-loeil}
\usepackage{cyanide/van-de-cyanide}
\setlength{\parindent}{0pt}
\usepackage{enumitem}
\newlist{citemize}{itemize}{3}
\setlist[citemize,1]{noitemsep,topsep=0pt,label={-},leftmargin=1em}

\usepackage{mathtools}
\usepackage{ragged2e}

\DeclarePairedDelimiter\abs{\lvert}{\rvert}%
\DeclarePairedDelimiter\norm{\lVert}{\rVert}%

% Swap the definition of \abs* and \norm*, so that \abs
% and \norm resizes the size of the brackets, and the 
% starred version does not.
\makeatletter
\let\oldabs\abs
\def\abs{\@ifstar{\oldabs}{\oldabs*}}
%
\let\oldnorm\norm
\def\norm{\@ifstar{\oldnorm}{\oldnorm*}}
\makeatother

\newcommand*{\Value}{\frac{1}{2}x^2}%

\usepackage{fancyhdr}
\usepackage{lastpage}

\fancypagestyle{plain}{%
\fancyhf{} % clear all header and footer fields
\fancyhead[R]{\smash{\raisebox{2.75em}{{\hspace{1cm}\color{lightgray}\textsf{\rightmark\quad Page \thepage/\pageref{LastPage}}}}}} %RO=right odd, RE=right even
\renewcommand{\headrulewidth}{0pt}
\renewcommand{\footrulewidth}{0pt}}
\pagestyle{plain}

\newtheorem*{experiment*}{Measurement}
\newtheorem{example}{Example}
\newtheorem{remark}{Remark}

\def\elementcell#1#2#3#4#5#6#7{%
    \draw node[draw, regular polygon, regular polygon sides=4, minimum height=2cm, draw=cyan, line width=0.4mm, fill=cyan!15!white, #1, inner sep=-2mm](#3) {\Large\textbf{\textsf{\color{cyan!50!black}#4}}};
    \draw (#3.corner 1) node[below left] {\footnotesize{\phantom{Hj}#5}};
    \draw (#3.corner 2) node[below right] {\small{\textsf{#6}}};
    \draw (#3.side 3) node[above] {\footnotesize #7};
    \draw (#3.corner 2) ++ (0,-0.4mm) node(nw#3) {};
    \tcbsetmacrotowidthofnode{\elementcellwidth}{#3}
    \node [fill=cyan, line width=0mm, rectangle, rounded corners=1.8mm, rectangle round south east=false, rectangle round south west=false, anchor=south west, minimum width=\elementcellwidth] at (nw#3) {\small\textsf{\color{white}#2}};
}

\DeclareSIUnit\Dq{Dq}
\usepackage{physics}
\usepackage{bbm}
\newtheorem{lemma}{Lemma}
\newtheorem{proposition}{Proposition}

\DeclareMathOperator{\Pfaffian}{Pf}
\DeclareMathOperator{\sign}{sign}

\usepackage[super]{nth}

\usetikzlibrary{decorations.pathmorphing}

\tikzset{snake it/.style={decorate, decoration=snake}}

\usepackage{stackengine}
\stackMath
\usepackage{scalerel}
\usepackage[outline]{contour}

\newlength\thisletterwidth
\newlength\gletterwidth
\newcommand{\leftrightharpoonup}[1]{%
{\ooalign{$\scriptstyle\leftharpoonup$\cr%\kern\dimexpr\thisletterwidth-\gletterwidth\relax
$\scriptstyle\rightharpoonup$\cr}}\relax%
}
\def\tensorb#1{\settowidth\thisletterwidth{$\mathbf{#1}$}\settowidth\gletterwidth{$\mathbf{g}$}\stackon[-0.1ex]{\mathbf{#1}}{\boldsymbol{\leftrightharpoonup{#1}}}  }
\def\onedot{$\mathsurround0pt\ldotp$}
\def\cddot{% two dots stacked vertically
  \mathbin{\vcenter{\baselineskip.67ex
    \hbox{\onedot}\hbox{\onedot}}%
  }}%

\begin{document}

\section{Crystal Binding and Elastic Constants} % (fold)
\label{sec:crystal_binding_and_elastic_constants}

\subsection{Types of Crystals} % (fold)
\label{sub:types_of_crystals}

\subsubsection{Definitions} % (fold)
\label{ssub:definitions}

\begin{termdef}{Cohesive Energy}
    The energy that must be added to the crystal to separate its components into \emph{neutral} free atoms at rest.
\end{termdef}
\begin{termdef}{Lattice Energy}
    The energy that must be added to the crystal to separate its components into free \emph{ions} at rest at infinite separation.
\end{termdef}

% subsubsection definitions (end)

\subsubsection{Crystals of Inert Gases} % (fold)
\label{ssub:crystals_of_inert_gases}

\begin{figure}[ht]
    \centering
    \begin{tikzpicture}
        \draw[snake it] (0,0) -- (2,0);
        \draw[snake it] (5,0) -- (7,0);
        \draw (0,0.3) -- (0,1);
        \draw (2,0.3) -- (2,1);
        \draw[-latex] (0,0.65) -- (2,0.65) node[midway, above] {$x_2$};
        \draw (5,0.3) -- (5,1);
        \draw (7,0.3) -- (7,1);
        \draw[-latex] (5,0.65) -- (7,0.65) node[midway, above] {$x_1$};
        \draw (0,-0.3) -- (0,-1);
        \draw (5,-0.3) -- (5,-1);
        \draw[-latex] (0,-0.65) -- (5,-0.65) node[midway, below] {$R$};
        \draw[fill=white] (0,0) circle (0.3);
        \draw[fill=white] (2,0) circle (0.3);
        \draw[fill=white] (5,0) circle (0.3);
        \draw[fill=white] (7,0) circle (0.3);
        \draw (0,0) node {$+$};
        \draw (2,0) node {$-$};
        \draw (5,0) node {$+$};
        \draw (7,0) node {$-$};
    \end{tikzpicture}
    \caption{Interacting harmonic oscillators.}
    \label{fig:interacting_harmonic_oscillators}
\end{figure}

\paragraph{Van der Waals-London Interaction} % (fold)
\label{par:van_der_waals_london_interaction}

For the system \begin{marginwarns}
    $\displaystyle \epsilon_0 = \rec{4\pi}$ hereinafter.
\end{marginwarns} in \cref{fig:interacting_harmonic_oscillators}, the unperturbated Hamiltonian is
\[ \+sH_0 = \rec{2m}p_1^2 + \half m\omega_0^2 x_1^2 + \rec{2m}p_2^2 + \half m\omega_0^2 x_2^2, \]
while the perturbation is given by
\[ \+cH_1 = \frac{e^2}{R} + \frac{e^2}{R+x_1 - x_2} - \frac{e^2}{R+x_1} - \frac{e^2}{R - x_2} \approx -\frac{2e^2 x_1x_2}{R^3}, \]
which may be diagonalized by introducing
\[ x_s = \rec{\sqrt{2}}\pare{x_1+x_2},\quad x_a = \rec{\sqrt{2}}\pare{x_1 - x_2},\quad p_s = \rec{\sqrt{2}}\pare{p_1+p_2},\quad \text{and}\quad p_a = \rec{\sqrt{2}}\pare{p_1 - p_2}, \]
whence the total Hamiltonian is written
\[ \+cH = \brac{\rec{2m}p_s^2 + \half\pare{m\omega_0^2 - \frac{2e^2}{R^3}}x_s^2} + \brac{\rec{2m}p_a^2 + \half \pare{m\omega_0^2+\frac{2e^2}{R^3}}x_a^2}, \]
which lowers the zero-point energy by introducing a difference
\[ \Delta U = -\hbar\omega_0 \cdot \rec{8}\pare{\frac{2e^2}{m\omega_0^2 R^3}}^2 = -\frac{A}{R^6}. \]
\vspace{-\baselineskip}
\begin{finaleq}{Van der Waals Potential}
    $\displaystyle \Delta U = -{A}/{R^6}$, where $A = \hbar\omega_0 \alpha^2$, where $\omega_0$ is of the strongest optical absorption line and $\alpha$ is the electronic polarizability.
\end{finaleq}

% paragraph van_der_waals_london_interaction (end)

\paragraph{Repulsive Interaction} % (fold)
\label{par:repulsive_interaction}

The repulsive potential is proportional to $R^{-12}$.
\begin{finaleq}{Lennard-Jones Potential}
    \[ U\pare{R} = 4\epsilon\brac{\pare{\frac{\sigma}{R}}^{12} - \pare{\frac{\sigma}{R}}^6}. \]
\end{finaleq}

% paragraph repulsive_interaction (end)

\paragraph{Equilibrium Lattice Constants} % (fold)
\label{par:equilibrium_lattice_constants}

With $p_{ij} R$ denoting the distance between the atoms at site $i$ and site $j$ where $R$ is the nearest-neighbor distance, the total potential energy is
\[ U\+_tot_\pare{R} = \half N\cdot {4\epsilon}\brac{{\sum_j}' \pare{\frac{\sigma}{p_{ij}R}}^{12} - {\sum_j}' \pare{\frac{\sigma}{p_{ij}R}}^6}. \]
The energy for the FCC structure is given by
\[ U\+_tot_\pare{R} = 2N\epsilon \brac{\num{12.13}\times \pare{\frac{\sigma}{R}}^{12} - \num{14.45}\pare{\frac{\sigma}{R}}^6}, \]
which yields the stationary value $R_0/\sigma = 1.09$, and the cohesive energy $U\+_tot_\pare{R_0} = -\num{2.15}\times 4N\epsilon$.
\begin{remark}
    The quantum correction to the kinetic energy should be taken into account, which is smaller when the atoms are heavier.
\end{remark}

% paragraph equilibrium_lattice_constants (end)

% subsubsection crystals_of_inert_gases (end)

\subsubsection{Ionic Crystals} % (fold)
\label{ssub:ionic_crystals}

For ionic crystals of the formula \ce{AB}, if the potential of interaction between ions could be written as the sum of the Coulomb potential and a repulsive potential of the form $\lambda e^{-r/\rho}$, and with the approximation that the exponential potential is neglectable for ions other than the nearest neighbour, the potential is written
\[ U_{ij} = \begin{cases}
    \displaystyle \lambda e^{-R/\rho} - \frac{q^2}{R}, & \mathrm{nearest-neighbour}, \\
    \displaystyle \pm \rec{p_{ij}} \frac{q^2}{R}, & \mathrm{otherwise},
\end{cases} \]
and the total potential energy is
\[ U\+_tot_ = N\pare{z\lambda e^{-R/\rho} - \frac{\alpha q^2}{R}}, \]
where $z$ is the coordination number of the ions, where \emph{$N$ is the total number of anions (or cations)} $\alpha$ is defined by
\begin{termdef}{The Madelung Constant}
    \[ \alpha = {\sum_j}' \frac{\pare{\pm}}{p_{ij}}. \]
\end{termdef}
The equilibrium \begin{margintips}
    Substitute $q^2/4\pi\epsilon_0$ for $q^2$ to yield the expression in the SI units.
\end{margintips} separation is given by demanding $\rd{U\+_tot_}/\rd{R} = 0$, whence we have
\[ R_0^2 e^{-R_0 / \rho} = \frac{\rho \alpha q^2}{z\lambda}, \]
and the total potential energy is thus
\[ U\+_tot_ = -\frac{N\alpha q^2}{R_0}\pare{1-\frac{\rho}{R_0}}. \]
\begin{table}[ht]
    \centering
    \begin{tabular}{cc}
        \hline
        Structure & $\alpha$ \\
        \hline
        \ce{NaCl} & \num{1.747595} \\
        \ce{CsCl} & \num{1.762675} \\
        (Cubic) \ce{ZnS} & \num{1.6381} \\
        \hline
    \end{tabular}
    \caption{The Madelung constants of some typical structures.}
    \label{table:madelung_typical}
\end{table}
\begin{sample}
    \begin{example}
        The Madelung constant of the one-dimensional chain of ions is $\alpha = 2\ln 2$.
    \end{example}
\end{sample}
The Madelung constants of some typical structures are listed in \cref{table:madelung_typical}.

% subsubsection ionic_crystals (end)

\subsubsection{Covalent Crystals} % (fold)
\label{ssub:covalent_crystals}

There is a continuous range of crystals between the ionic and the covalent limits.

% subsubsection covalent_crystals (end)

\subsubsection{Metals} % (fold)
\label{ssub:metals}

Metallic binding is characterized by the lowering of energy of the valence electrons in the metals as compared with the free atom.
\par
Metals tend to crystallize in relatively closed packed structures: hcp, fcc, bcc and some other closed related structures.
\par
In the transition metals there is aditional binding from inner electron shells. Transition metals and the metals immediately following them in the periodic table have large $d$-electron shells and are characterized by high-binding energy.

% subsubsection metals (end)

\subsubsection{Hydrogen Bonds} % (fold)
\label{ssub:hydrogen_bonds}

The hydrogen bond is largely ionic in character, being formed only between the most electronegative atoms, particularly \ce{F}, \ce{O}, and \ce{N}. The hydrogen bond connects only two atoms.

% subsubsection hydrogen_bonds (end)

\subsubsection{Atomic Radii} % (fold)
\label{ssub:atomic_radii}

Predictions of lattice constants may be done additively by assigning the self-consistent radii to various types of bonds:
\begin{cenum}
    \item one set for ionic crystals with the constituent ions 6-coordinated in inert gas closed shell configuration;
    \item another set for the ions in tetrahedrally-coordinated structures;
    \item and another set for 12-coordinated (close-packed) metals.
\end{cenum}

% subsubsection atomic_radii (end)

% subsection types_of_crystals (end)

\subsection{Analysis of Elastic Strains} % (fold)
\label{sub:analysis_of_elastic_strains}

If the crystal is subjected to a deformation
\[ \+vR\pare{\+vr} = u\pare{\+vr}\+ux + v\pare{\+vr}\+uy + w\pare{\+vr}\+uz, \]
with $\+vR\pare{0} = 0$ we may expand $\+vR\pare{\+vr}$ by the \gloss{strain components}
\begin{finale}
\vspace{-\baselineskip}
    \begin{gather*}
        e_{xx} = \+DxDu,\quad e_{yy} = \+DyDv,\quad e_{zz} = \+DzDw; \\
        e_{yz} = \+DzDv + \+DyDw,\quad e_{zx} = \+DzDu + \+DxDw,\quad e_{xy} = \+DyDu + \+DxDv.
    \end{gather*}
\end{finale}

\subsubsection{Dilation} % (fold)
\label{ssub:dilation}

With $V'$ denoting the volume after the deformation,
\[ \inlinefinaleq{\delta = \frac{V' - V}{V} = e_{xx} + e_{yy} + e_{zz}.} \]

% subsubsection dilation (end)

\subsubsection{Stress Components} % (fold)
\label{ssub:stress_components}

There are nine \gloss{stress components}: $X_x, Y_y, Z_z$, $X_y = Y_x$, $X_z = Z_x$, $Y_z = Z_y$, where $X_y$ is applied in the $x$ direction to a unit area of a plane whose normal lies in the $y$ direction.

% subsubsection stress_components (end)

\subsubsection{Elastic Compliance and Stiffness Constants} % (fold)
\label{ssub:elastic_compliance_and_stiffness_constants}

The strain components and the stress components are related by
\[ \+ve = \begin{pmatrix}
    e_{xx} \\ e_{yy} \\ e_{zz} \\ e_{yz} \\ e_{zx} \\ e_{xy}
\end{pmatrix} = \tensorb{S} \begin{pmatrix}
    X_x \\ Y_y \\ Z_z \\ Y_z \\ Z_x \\ X_y
\end{pmatrix},\quad \text{and} \quad \begin{pmatrix}
    X_x \\ Y_y \\ Z_z \\ Y_z \\ Z_x \\ X_y
\end{pmatrix} = \tensorb{C} \begin{pmatrix}
    e_{xx} \\ e_{yy} \\ e_{zz} \\ e_{yz} \\ e_{zx} \\ e_{xy}
\end{pmatrix} = \tensorb{C} \+ve. \]
The quantities $S_{11}, S_{12}, \cdots$ are called the \gloss[-\baselineskip]{elastic compliance constants} or elastic constants. The quantities $C_{11}, C_{12}, \cdots$ are called the \gloss{elastic stiffness constants} or moduli of elasticity.

% subsubsection elastic_compliance_and_stiffness_constants (end)

\subsubsection{Elastic Energy Density} % (fold)
\label{ssub:elastic_energy_density}

\begin{finaleq}{Elastic Energy Density}
    \[ U = \half \sum_{\lambda=1}^6 \sum_{\mu=1}^6 \tilde{C}_{\lambda\mu}e_\lambda e_\mu. \]
\end{finaleq}

% subsubsection elastic_energy_density (end)

\subsubsection{Elastic Stiffness Constants of Cubic Crystals} % (fold)
\label{ssub:elastic_stiffness_constants_of_cubic_crystals}

The elastic energy density of a cubic crystal is
\begin{align*}
    U &= \half C_{11} \pare{e_{xx}^2 + e_{yy}^2 + e_{zz}^2} + \half C_{44}\pare{e_{yz}^2 + e_{zx}^2 + e_{xy}^2} + \\
    &\phantom{=\ } C_{12} \pare{e_{yy}e_{zz} + e_{zz}e_{xx} + e_{xx}e_{yy}}.
\end{align*}
Therefore, the only independent stiffness constants in cubic crystals are $C_{11}$, $C_{12}$ and $C_{44}$.

% subsubsection elastic_stiffness_constants_of_cubic_crystals (end)

\subsubsection{Bulk Modulus and Compressibility} % (fold)
\label{ssub:bulk_modulus_and_compressibility}

For a cubic crystal, the \gloss{bulk modulus} is given by
\[ B = -V\+dVdp = \rec{3}\pare{C_{11} + 2C_{12}}, \]
where the relation
\[ U = \half B\delta^2 = \rec{6}\pare{C_{11} + 2C_{12}}\delta^2 \]
holds. The \gloss{compressibility} is defined by $K = 1/B$.

% subsubsection bulk_modulus_and_compressibility (end)

% subsection analysis_of_elastic_strains (end)

\subsection{Elastic Waves in Cubic Crystals} % (fold)
\label{sub:elastic_waves_in_cubic_crystals}

The equation of motion in a cubic crystal is
\begin{align}
    \label{eq:motion_u}
    \rho \+D{t^2}D{^2 u} &= C_{11} \+D{x^2}D{^2 u} + C_{44}\pare{\+D{y^2}D{^2u} + \+D{z^2}D{^2 u}} + \pare{C_{12} + C_{44}} \pare{\frac{\partial^2 v}{\partial x\partial y} + \frac{\partial^2 w}{\partial x \partial z}}, \\
    \label{eq:motion_v}
    \rho \+D{t^2}D{^2 v} &= C_{11} \+D{y^2}D{^2 v} + C_{44}\pare{\+D{x^2}D{^2v} + \+D{z^2}D{^2 v}} + \pare{C_{12} + C_{44}} \pare{\frac{\partial^2 u}{\partial x\partial y} + \frac{\partial^2 w}{\partial y \partial z}}, \\
    \label{eq:motion_w}
    \rho \+D{t^2}D{^2 w} &= C_{11} \+D{z^2}D{^2 w} + C_{44}\pare{\+D{x^2}D{^2w} + \+D{y^2}D{^2 w}} + \pare{C_{12} + C_{44}} \pare{\frac{\partial^2 u}{\partial x\partial z} + \frac{\partial^2 v}{\partial y \partial z}}.
\end{align}

% subsection elastic_waves_in_cubic_crystals (end)

\mathsubsubsection{Wave100}{$\brac{100}$ Direction}{Waves in the $\brac{100}$ Direction}{Waves in the [100] Direction} % (fold)
\label{ssub:waves_in_the_100_direction}

\paragraph{Longitudinal} % (fold)
\label{par:longitudinal}

Seeking solutions to equation \eqref{eq:motion_u} of the form
\[ u = u_0 \exp\brac{i\pare{Kx - \omega t}}, \]
we find
\[ \omega^2\rho = C_{11}K^2,\quad v_s = \sqrt{\frac{C_{11}}{\rho}}. \]

% paragraph longitudinal (end)

\paragraph{Transverse} % (fold)
\label{par:transverse}

Seeking solutions to equation \eqref{eq:motion_v} of the form
\[ v = v_0 \exp\brac{i\pare{Kx - \omega t}}, \]
we find
\[ \omega^2\rho = C_{44}K^2,\quad v_s = \sqrt{\frac{C_{44}}{\rho}}. \]

% paragraph transverse (end)

% subsubsection waves_in_the_100_direction (end)

\mathsubsubsection{Wave110}{$\brac{110}$ Direction}{Waves in the $\brac{110}$ Direction}{Waves in the [110] Direction} % (fold)
\label{ssub:waves_in_the_100_direction}

\paragraph{Out of the $xy$ plane} % (fold)
\label{par:shear_wave}

Seeking solutions to equation \eqref{eq:motion_w} of the form
\[ w = w_0 \exp\brac{i\pare{K_x x + K_y y - \omega t}}, \]
we find
\[ \omega^2\rho = C_{44}\pare{K_x^2 + K_y^2}, \]
independent of the exact direction.

% paragraph shear_wave (end)

\paragraph{In the $xy$ plane} % (fold)
\label{par:transverse_wave}

Seeking solutions to equation \eqref{eq:motion_u} and equation \eqref{eq:motion_v} of the form
\[ \begin{cases}
    u = u_0 \exp\brac{i\pare{K_x x + K_y y - \omega t}},
    v = v_0 \exp\brac{i\pare{K_x x + K_y y - \omega t}},
\end{cases} \]
where $\displaystyle K_x = K_y = \frac{K}{\sqrt{2}}$, we find
\[ \begin{cases}
    \displaystyle \omega^2\rho = \half \pare{C_{11} + C_{12} + 2C_{44}}K^2,\quad u=v,& \mathrm{longitudinal}, \\[.5em]
    \displaystyle \omega^2\rho = \half \pare{C_{11} - C_{12}}K^2,\quad u=-v,& \mathrm{transverse}.
\end{cases} \]

% paragraph transverse_wave (end)

% subsubsection waves_in_the_100_direction (end)

% section crystal_binding_and_elastic_constants (end)

\end{document}
